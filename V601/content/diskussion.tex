\section{Diskussion}
\label{sec:Diskussion}
Aus der ersten Messreihe wurde die Energie der Elektronen zu
$\SI{6,73}{\eV}$ bestimmt, daraus berechnet sich das
Kontaktpotential zu $\SI{2,27}{\V}$.
Hier kann kein konkreter Fehler angegeben werden, doch dadurch, dass der Wert durch
ablesen bestimmt wird ist diese Größe natürlich fehlerbehaftet. Besonders, da
da die Steigungsdreiecke per Hand eingezeichnet und abgelesen werden.
Anschließend wurde aus der Frank-Hertz-Kurve die erste Anregungsenergie
von Quecksilber ermittelt, diese
beträgt:
\begin{equation}
  E_1=\SI{5,01(10)}{\eV}.
\end{equation}
Umgerechnet mit $\lambda=\frac{c}{\nu}$ ergibt sich die Wellenlänge von
\begin{equation}
  \lambda=\SI{247(5)}{\nm}.
\end{equation}
Somit handelt es sich um Ultraviolette Strahlung und liegt nicht mehr im sichtbaren Bereich.
Wird die Anregungsenergie mit dem Theoriewert von
\begin{equation}
  E_{1\text{Theorie}}=\SI{4,9}{\eV}
\end{equation}
\cite{anregung}
verglichen, beträgt die Abweichung 2,2\%.

Im letzten Versuchsteil wurde die Ionisierungsenergie von Quecksilber
bestimmt. Diese beträgt
\begin{equation}
  U_{\text{Ionisierung}}=\SI{12,93}{\eV}.
\end{equation}
Dieser Wert wird nun mit dem Theoriewert von
\begin{equation}
  U_{\text{Io,Theorie}}=\SI{10,437}{\eV}
\end{equation}
\cite{ionisierung}
verglichen, es ergibt sich eine Abweichung von 23,88\%.
