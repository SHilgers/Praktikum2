\section{Auswertung}
\label{sec:Auswertung}

\subsection{Energiekalibration und Bestimmung der Vollenergienachweiswahrscheinlichkeit}
Zur Kalibration wird ein $\ce{^{152}Eu}$-Strahler verwendet, dessen Aktivität am 01.10.2000
%\begin{align*}
%  \SI{4130(60)}{\becquerel}
%\end{align*}
$\SI{4130(60)}{\becquerel} $ betrug. \\
Nach dem Gesetz des radioaktiven Zerfalls berechnet sich die Aktivität am Messtag (08.04.2019) durch

\begin{equation}
  \symup{A} (t) = \symup{A}(0)\cdot \symup{e}^{-\lambda t} \: ,
\end{equation}

wobei $\lambda=\SI{1.6244(19)e-9}{\per\second}$ \cite{lara} die Zerfallskonstante
von $\ce{^{152}Eu}$ bezeichnet.

Der Fehler ergibt sich hierbei nach der Gauß´schen Fehlerfortpflanzung
\begin{equation}
  \increment f = \sqrt{ \sum_{i=1}^N \left( \frac{\partial f}{\partial x_i}\right)^2
  \cdot (\increment x_i)^2  } \: ,
  \label{eqn:gaus}
\end{equation}
also gemäß
\begin{equation}
  \increment \symup{A} (t) = \sqrt{ (\symup{e}^{-\lambda t})^{2}\cdot (\increment \symup{A}(0))^2
   + (-t\cdot\symup{A}(0)\cdot \symup{e}^{-\lambda t})^2\cdot(\increment \lambda)^2}
\end{equation}
Die Anzahl der Tage vom 01.10.2000 bis zum 08.04.2019 beträgt 6763 Tage, was
584323200 Sekunden entspricht, sodass sich insgesamt der Wert $\SI{1599(29)}{\becquerel} $
für die Aktivität der Probe am Messtag ergibt. \\
Der abgedeckte Raumwinkel lässt sich aus dem gemessenen Abstand a der Probe
zum Detektor, wobei auch der Abstand von $\SI{1.5}{\centi\meter}$ zwischen Al-Haube und Detektor
berücksichtigt wird,
und dem angegebenen Radius r des Detektorvolumens bestimmen. Die entsprechenden Werte
betragen
\begin{align*}
  a &= \SI{8.8}{\centi\meter} \\
  r &= \SI{2.25}{\centi\meter} \: .
\end{align*}

Die Formel zur Berechnung des abgedeckten Raumwinkelanteils ergibt sich dabei
über geometrische Überlegungen zu
\begin{equation}
  \frac{\Omega}{4\pi}= \frac{1}{2}(1-\frac{a}{\sqrt{a^2+r^2}})   \: ,
\end{equation}
in diesem Fall also $\frac{\Omega}{4\pi}= 0.01558$.
Diese somit errechneten Werte sind später wichtig zur Bestimmung der Vollenergienachweiswahrscheinlichkeit. \\
Das gemessene Spektrum des kalibrierten $\ce{^{152}Eu}$-Strahlers ist in Abbildung
\ref{fig:plot1} dargestellt. Es sind jedoch nur die ersten 4000 Kanäle dargestellt,
da bei höheren Kanälen keine signifikanten Messwerte mehr zu sehen sind. Die Messwerte
reichen bis Kanalnummer 8191 und die Messzeit beträgt $\SI{3598}{\second}$.
\begin{figure}
  \centering
  \includegraphics[height=9cm]{Eu.pdf}
  \caption{Spektrum des $\ce{^{152}Eu}$-Strahlers}
  \label{fig:plot1}
\end{figure}

Um mit diesem die Energiekalibration durchzuführen, werden die Peaks des Spektrums
jeweils mit einer Gaußverteilung der Form
\begin{equation}
  \symup{g} (x) = a + b \cdot \symup{e}^{(\frac{x-z}{c})^2}
  \label{eqn:gausk}
\end{equation}
gefittet.

Die sich daraus ergebenen Parameter sind in Tabelle \ref{tab:tabe1} angegebenen.
\begin{table}[H]
  \centering
  \caption{Messwerte und Ergebniss der Bestimmung der Schallgeschwindigkeit}
  \label{tab:tabe1}
    \begin{tabular}{S||S S||S S||S|S}
    \toprule
    $ \text{Länge l des Zylinders [mm]} $ & $ U_{1} [\text{V}] $ &
    $ t_{1} [\mu\text{s}] $ & $ U_{2} [\text{V}] $ &
    $ t_{2} [\mu\text{s}] $ & $ \increment t [\mu\text{s}]$ &
    $ \text{c} [\text{m}/\text{s}]$\\
    \midrule
    31.0 & 1.335 \: & 24.0 & 1.096 \:  & 46.9 & 22.9 & 2707.42 \\
          \bottomrule
    \end{tabular}
  \end{table}


Die zentrale Lage der Peaks im Hinblick auf die Kanalnummer ist durch den Parameter
z gegeben. Diese Werte werden zusammen mit der jeweiligen relativen Höhe mit den theoretischen
Emissionslinien der Datenbank \cite{lara} verglichen und es wird jedem Peak eine Linie
zugeordent. Diese Zuordnung ist zusammen mit der jeweiligen relativen Emissionswahrscheinlichkeit P
in Tabelle \ref{tab:tabe2} dargestellt.
\begin{table}[H]
  \centering
  \caption{Zählrate und Energiemaximum bei variiertem Druck, Abstand a=2cm}
  \label{tab:tab2}
    \begin{tabular}{c c c c c}
    \toprule
    Druck $\rho$/\;mbar & Energiemaximum & Zählrate $N$ & Energie $E_{\alpha}$ & effektive Länge $x$/\;cm\\
    \midrule
    0 & 796 &131382  &4          & 0.0   \\
    50 & 775 &131464 &3.89 & 0.09 \\
    100 &756 &130732 &3.79 & 0.19\\
    150 &749 &129617 &3.76  &  0.29\\
    200 &749 &130444 &3.76  & 0.39\\
    250 &727 &129600 &3.65 & 0.49\\
    300 &722 &128936 &3.63 & 0.59\\
    350 &708 &128478 &3.56 & 0.69\\
    400 &696 &128122 &3.49 & 0.79\\
    450 &687 &127415 &3.45 & 0.89\\
    500 &674 &126608 &3.39 & 0.99\\
    550 &663 &126372 &3.33 &1.09\\
    600 &651 &124989 &3.27 & 1.18\\
    650 &634 &124942 &3.19 & 1.28\\
    700 &618 &124295 &3.11 &1.38\\
    750 &602 &123299 &3.03 & 1.48\\
    800 &584 &119958 &2.93 &1.58\\
    850 &566 &120673 &2.84 &1.68\\
    900 &548 &117907 &2.75 & 1.78\\
    950 &534 &116111 &2.68&   1.88\\
    1000 &499& 108630&2.51 & 1.07\\
    \bottomrule
    \end{tabular}
  \end{table}

Mit den Wertepaaren aus Kanalnummer und Linienenergie wird nun eine lineare Ausgleichsrechnung der
Form
\begin{equation}
  f(x) = a\cdot x +b
  \label{eqn:li}
\end{equation}
durchgeführt, woraus sich die Parameter
\begin{align}
  a &= \SI{0.403169(29)}{\kilo\electronvolt} \\
  b &= \SI{-3.034(60)}{\kilo\electronvolt}
\end{align}
ergeben. Die Wertepaare sind zusammen mit der resultierenden Gerade in Abbildung \ref{fig:plot3}
dargestellt. Die Fehler der Messwerte sind aufgrund ihrer sehr geringen relativen Größe dabei zu vernachlässigen.
\begin{figure}
  \centering
  \includegraphics[height=9cm]{plot3.pdf}
  \caption{Lineare Ausgleichsrechnung zur Energiekalibration}
  \label{fig:plot3}
\end{figure}
Die Energiekalibration erfolgt somit gemäß
\begin{equation}
  \symup{E}_{\gamma} (x) = \SI{0.403169}{\kilo\electronvolt}\cdot x - \SI{3.034}{\kilo\electronvolt}
  \label{eqn:gerade}
\end{equation}
wobei x die Kanalnummer bezeichnet.
Der dazugehörige Fehler ergibt mittels Gleichung \ref{eqn:gaus} durch
\begin{equation}
   \increment \symup{E}_{\gamma} (x) =\sqrt{ (\SI{0.000029}{\kilo\electronvolt}\cdot x)^2 +
   (\SI{0.060}{\kilo\electronvolt})^2}
   \label{eqn:fgerade}
\end{equation}
\\
Zur Bestimmung der Vollenergienachweiswahrscheinlichkeit wird zunächst die Gleichung \ref{eqn:gausk}
integriert, um den Inhalt der Peaks zu bestimmen, wobei der Untegrund $a$ vorher abgezogen wird.
Es ergibt sich somit ein Linieninhalt von
\begin{equation}
  \symup{I} =\int_{-\infty}^{\infty} c \cdot \symup{e}^{(\frac{x-z}{b})^2} \symup{d}x
  = c \cdot b \cdot \sqrt{\pi}
  \label{eqn:inh}
\end{equation}
in Abhängigkeit der Parameter b und c.
Der Fehler ergibt sich durch Gleichung \ref{eqn:gaus} über die Gleichung
\begin{equation}
  \increment \symup{I} = \sqrt{ (\increment c \cdot b \cdot \sqrt{\pi})^{2}
   + (c \cdot \increment b \cdot \sqrt{\pi}})^{2} \: .
     \label{eqn:inhf}
\end{equation}
Mit den Werten aus Tabelle \ref{tab:tabe1} lassen sich somit die einzelnen Linieninhalte berrechen,
welche in Tabelle \ref{tab:tabe3} angegeben sind.
\begin{table}[H]
  \centering
   \begin{tabular}{c c c}
    \toprule
     n& $\nu$/\; 1/s & $\nu_{Wechsel}$\\
    \midrule
    0,5 & 100.01& 50,0\\
    1 & 79.93 & 79.93\\
    2 & 23.93 & 47.86\\
    \bottomrule
  \end{tabular}
  \caption{Gemessene Frequenzen der Sägezahnspannung, sowie die Daraus resultierenden Frequenzen für die
  Wechselspannung.}
  \label{tab:tab3}
\end{table}

Zum Vergleich werden nun die Theoriewerte berrechnet, als Produkt
der Emissionswahrscheinlichkeiten P aus Tabelle \ref{tab:tabe2},
dem abgedeckten Raumwinkelanteil $\frac{\Omega}{4\pi}= 0.01558$, der errechneten Aktivität
$A =\SI{1599(29)}{\becquerel} $
und der Messzeit von $t = \SI{3598}{\second}$
\begin{equation}
  \symup{I}_{\text{theo}} = P\cdot \frac{\Omega}{4\pi} \cdot A \cdot t
  \label{eqn:itheo}
\end{equation}
mit dem Fehler über Gleichung \ref{eqn:gaus} von
\begin{equation}
  \increment \symup{I}_{\text{theo}} = \sqrt{ (\increment P\cdot \frac{\Omega}{4\pi} \cdot A \cdot t)^{2}
   + (P\cdot \frac{\Omega}{4\pi} \cdot \increment A \cdot t)^{2}} \: .
\end{equation}
Aus dem jeweiligen Quotienten
\begin{equation}
  \symup{Q} = \frac{\symup{I}}{\symup{I}_{\text{theo}}}
  \label{eqn:ven1}
\end{equation}
mit dem dazugehörigen Fehler
\begin{equation}
  \increment \symup{Q} = \sqrt{ (\frac{1}{\symup{I}_{\text{theo}} \cdot \increment \symup{I})^{2}
   + (\frac{\symup{I}}{\symup{I}_{\text{theo}}})^{2}}\cdot \increment \symup{I}_{\text{theo}})^{2}}
\end{equation}
ergibt sich somit jeweils die Nachweiswahrscheinlichkeit des Peaks, wie in Tabelle
\ref{tab:tabe4} dargestellt ist.
\begin{table}[H]
  \centering
   \begin{tabular}{c c c c}
    \toprule
    Nummer der Oberwelle & $ U_{\text Theorie,Rechteck}\: / \si{\volt} $ &
    $ U_{\text Theorie,Dreick}\: / \si{\volt} $ & $ U_{\text Theorie,Sägezahn}\: / \si{\volt} $ \\
    \midrule
    1 & 1145 & 182 & 573 \\
    2 & 0 & 0 & 286 \\
    3 & 573 & 20 & 191 \\
    4 & 0 & 0 & 143 \\
    5 & 229 & 7 & 115 \\
    6 & 0 & 0 & 96 \\
    7 & 164 & 4 & 82 \\
    8 & 0 & 0 & 72 \\
    9 & 127 & 2 & 64 \\
    10 & 0 & 0 & 57 \\
    \bottomrule
  \end{tabular}
  \caption{Eingestellte Schwingungsamplituden.}
  \label{tab:tabe4}
\end{table}

Da die Nachweiswahrscheinlichkeit im Allgemeinen energieabhängig ist, wird Q in
Abhängigkeit von $ \text{E}_{\gamma} $ dargestellt und mit einer Potenzfunktion der Form
\begin{equation}
  \symup{Q}(\text{E}_{\gamma}) = c\cdot (\text{E}_{\gamma}-a)^{d}
  \label{eqn:ven2}
\end{equation}
gefittet, wie in Abbildung \ref{fig:plot4} dargestellt ist.
Es ergeben sich hierbei die Parameter
\begin{align*}
  a &= -191 \pm 64 \\
  c &= 2474 \pm 3467 \\
  d &= -1.47 \pm 0.19 \: .
\end{align*}
 %Die Formel für die eigentliche Höhe eines
%gemessenen Peaks ergibt sich aus Gleichungen \ref{eqn:ven1} und \ref{eqn:ven2} durch Umstellen zu
%\begin{equation}
%  \symup{I}_{\text{theo}} = \frac{\symup{I}}{\symup{Q}}=\frac{\symup{I}}{c\cdot (\text{E}_{\gamma}-a)^{d}}
%  \label{eqn:ven}
%\end{equation}
%mit dem dazugehörigen Fehler
%\begin{equation}
%  \symup{I}_{\text{theo}} = \frac{\symup{I}}{\symup{Q}}=\frac{\symup{I}}{c\cdot (\text{E}_{\gamma}-a)^{d}}
%  NAMNAM
%  \label{eqn:ven}
%\end{equation}
\begin{figure}
  \centering
  \includegraphics[height=9cm]{plot4.pdf}
  \caption{Werte zur Bestimmung der Vollenergienachweiswahrscheinlichkeit sowie gefittete Potenzfunktion}
  \label{fig:plot4}
\end{figure}

\subsection{Untersuchung eines monochromatischen Gamma-Spektrums}
Zur Untersuchung des monochromatischen Gamma-Spektrums wird das aufgenommene Spektrum
zunächst durch die Gleichung \ref{eqn:gerade} kalibriert, wobei sich der Fehler über
\ref{eqn:fgerade} ergibt. Die so erhaltenen Werte sind in Abbildung \ref{fig:plot5}
dargestellt, wobei auf Fehlerbalken aufgrund der geringen Fehler verzichtet wird. Die
Messzeit beträgt $\SI{2593}{\second}$ und es wurden 8191 Kanäle gemessen, wobei nur die
ersten 2000 dargestellt sind, da bei höheren Kanalnummern keine signifikanten Messwerte
mehr zu erkennen sind.
\begin{figure}
  \centering
  \includegraphics[height=9cm]{Cs.pdf}
  \caption{Kalibriertes Spektrum des $\ce{^{137}Cs}$-Strahlers }
  \label{fig:plot5}
\end{figure}
Zur Bestimmung der Energie wird die Vollenergielinie mit der Gaußverteilung aus Gleichung
\ref{eqn:gausk} gefittet, wobei sich die Parameter
\begin{align*}
  a &= 5.7 \pm 4.2 \\
  b &= 1674 \pm 17 \\
  c &= \SI{1.227(15)}{\kilo\electronvolt}\\
  z &= \SI{661.5985(98)}{\kilo\electronvolt} \:
\end{align*}
ergeben.
Die Energie des Strahlers ist dabei durch den Parameter $z$ gegeben, also
\begin{align*}
  \symup{E}_{Cs} = \SI{661.5985(98)}{\kilo\electronvolt} \: .
\end{align*}
Der Theoriewert von $\ce{^{137}Cs}$ beträgt $\SI{661.657(3)}{\kilo\electronvolt}$ \cite{lara}. \\
Der Bereich um die Vollenergielinie ist in
Abbildung \ref{fig:plot6} dargestellt, woraus sich die Halbwertsbreite und die
Zehntelwertsbreite ablesen lässt, wobei eine Ungenauigkeit durch
Ablesefehler von etwa $\SI{0.2}{\kilo\electronvolt}$ angenommen wird.
Es ergibt sich hierdurch
\begin{align*}
  \symup{E}_{1/2} = \SI{662.6(2)}{\kilo\electronvolt} -\SI{660.6(2)}{\kilo\electronvolt}
  =\SI{2.0(3)}{\kilo\electronvolt}\\
  \symup{E}_{1/10} = \SI{663.3(2)}{\kilo\electronvolt} -\SI{659.6(2)}{\kilo\electronvolt}
  =\SI{3.7(3)}{\kilo\electronvolt} \:
\end{align*}
woraus sich ein Verhältniss von
\begin{equation}
  \frac{\symup{E}_{1/2}}{\symup{E}_{1/10}} =  0.54 \pm 0.09 \:
\end{equation}
ergibt.
\begin{figure}
  \centering
  \includegraphics[height=9cm]{Plot6.pdf}
  \caption{Vollenergielinie}
  \label{fig:plot6}
\end{figure}
Die Halbwertsbreite einer Gaußkurve gemäß Gleichung \ref{eqn:gausk} ist durch die
Formel
\begin{equation}
  \symup{E}_{1/2} = 2c\cdot\sqrt{ln2}
\end{equation}
mit dem Fehler
\begin{equation}
  \increment \symup{E}_{1/2} = 2\increment b\cdot\sqrt{ln2} \: ,
\end{equation}
gegeben und beträgt somit
\begin{align*}
  \symup{E}_{1/2,theo} = \SI{1.701(21)}{\kilo\electronvolt} \: .
\end{align*}
Die Zehntelwertsbreite ergibt sich analog über die Gleichung
\begin{equation}
  \symup{E}_{1/10} = 2c\cdot\sqrt{ln10}
\end{equation}
mit dem Fehler
\begin{equation}
  \increment \symup{E}_{1/10} = 2\increment b\cdot\sqrt{ln10} \: ,
\end{equation}
zu
\begin{align*}
  \symup{E}_{1/10, theo} = \SI{5.651(69)}{\kilo\electronvolt} \: .
\end{align*}
Das Verhältniss dieser beiden Größen ist unabhängig von den jeweiligen Parametern der
Kurve stets
\begin{equation}
  \frac{\symup{E}_{1/2}}{\symup{E}_{1/10}} = \sqrt{\frac{ln2}{ln10}}
  \approx 0.549 \: .
\end{equation}
Diese theoretisch erhaltenen Werte werden mit den abglesenen verglichen, wobei sich
die Abweichung über die Formel
\begin{equation}
  \frac{\lvert \text{Wert}_{\text{Theorie}}-\text{Wert}_{\text{Messung}}\rvert}{\text{Wert}_{\text{Theorie}}}
  \label{eqn:abw}
\end{equation}
berechnen lässt zu 1.64 \%. Diese Abweichung ist offensichtlich sehr gering und liegt innerhalb der
Ableseunsicherheit, was auf eine gute Beschreibung der Vollenergielinie
durch eine Gaußkurve schließen lässt. Die recht große Abweichung der Halb- und Zehntelwertsbreite
an sich lässt sich dadurch erklären, dass wohlmöglich eine falsche Maximalhöhe des Peaks angenommen
wurde.
Der Inhalt der Vollenergielinie ergibt sich durch Gleichungen \ref{eqn:inh}
und \ref{eqn:inhf} zu
\begin{align*}
  \symup{I}_{VEL} =  3641 \pm 58 \: .
\end{align*}
\\
Aus den Messwerten und Abbildung \ref{fig:plot5} lässt sich erkennen, dass die
Compton-Kante bei etwa $\SI{478(3)}{\kilo\electronvolt}$ liegt, da dort (Kanalnummer
1198) das Spektrum ein lokales
Maximum von 53 Counts animmt und anschließend abfällt, wobei die Ableseunsicherheit auf etwa
$\SI{3}{\kilo\electronvolt}$ geschätzt wird.
Aus Gleichung \ref{eqn:kante} ergibt sich der theoretische Wert zu $\SI{477.280(90)}{\kilo\electronvolt}$
und die Abweichung somit zu 0.15 \%.
Um den Inhalt des Comptonkontinuums zu bestimmen wird der Bereich zwischen $\SI{20}{\kilo\electronvolt}$
und $\SI{478}{\kilo\electronvolt}$ mit der Funktion aus Gleichung \ref{eqn:diffCompton} gefittet,
wobei der Term $a =\frac{3}{8}\sigma_{\text{Th}}\frac{1}{m_0 c^2 e^2}$ als Fitparameter
verwendet wird. Für diesen ergibt sich
\begin{align*}
  a =  4.297 \pm 0.056 \: .
\end{align*}
Dieser Wert wird nun verwendet, um die Funktion numerisch in dem Bereich zwischen $\SI{20}{\kilo\electronvolt}$
und $\SI{478}{\kilo\electronvolt}$
zu integrieren, wodurch sich der Inhalt des Comptonkontinuums zu
\begin{align*}
  \symup{I}_{Compton} =  16266 \pm 214
\end{align*}
ergibt.

Der Rückstreupeak wird erneut mit der Gleichung \ref{eqn:gausk} gefittet, wodurch sich
die Parameter
\begin{align*}
  a &= 65.8 \pm 2.0 \\
  b &= 36 \pm 28 \\
  c &= \SI{0.28(28)}{\kilo\electronvolt}\\
  z &= \SI{193.78(24)}{\kilo\electronvolt} \:
\end{align*}
ergeben, der Rückstreupeak liegt also bei $\SI{193.78(24)}{\kilo\electronvolt}$.
Der Theoriewert nach Gleichung \ref{eqn:Rückstreu} ergibt sich zu $\SI{184.3184(76)}{\kilo\electronvolt}$.
\\
Zur Bestimmung der Absorptionswahrscheinlichkeiten wird die Formel
\begin{equation}
  \symup{p} = 1-\symup{e}^{-\mu \cdot l}
\end{equation}
verwendet, wobei $l$ die Länge des Detektors, in diesem Fall also $\SI{3.9}{\centi\meter}$, bezeichnet
und $\mu$ den Extinkionskoeffizient, welcher für den Comptoneffekt etwa
$\mu_{c}=\SI{0.38}{\per\centi\meter}$ und für den Photoeffekt etwa
$\mu_{p}=\SI{0.008}{\per\centi\meter}$ beträgt.
Dies führt zu Absorptionswahrscheinlicheiten von
\begin{align*}
  \symup{p}_{c} = 0.7728 \\
  \symup{p}_{p} = 0.0307 \: ,
\end{align*}
sodass sich theoretisch ein Verhältniss zwischen den Inhalten des Photopeaks und des
Comptonkontinuums von
\begin{equation*}
  (\frac{\symup{I}_{VEL}}{\symup{I}_{Compton}})_{\text{theo}} = 1-\symup{e}^{-\mu \cdot l}
  = 0.03975
\end{equation*}
ergeben sollte. Das tatsächliche Verhältniss aus den Messdaten beträgt
\begin{equation*}
  \frac{\symup{I}_{VEL}}{\symup{I}_{Compton}}
  = 0.224 \pm 0.005
\end{equation*}
und weist somit eine Abweichung von 463.52 \% auf, gemäß Formel \ref{eqn:abw}.


\subsection{Aktivitätsbestimmung}
Bei dem dritten vermessenen Spektrum soll zunächst festgestellt werden, ob es sich bei der gemessenen
Probe um einen $\ce{^{133}Ba}$-Strahler oder einen $\ce{^{125}Sb}$-Strahler handelt. Zu diesem
Zweck wird das Spektrum zunächst gemäß Gleichung \ref{eqn:gerade} kalibriert, wie in Abbildung
\ref{fig:plot7} dargestellt ist. Die Messzeit beträgt $\SI{4378}{\second}$; von den 8191 verwendeten
Kanälen werden nur die ersten 2000 verwendet, da darüber hinaus keine signifikanten
Messwerte mehr zu erkennen sind.
\begin{figure}
  \centering
  \includegraphics[height=9cm]{Ba.pdf}
  \caption{Unbekanntes Spektrum}
  \label{fig:plot7}
\end{figure}
Die Peaks werden erneut mit Gleichung \ref{eqn:gausk} gefittet und die resultierenden
Parameter in Tabelle \ref{tab:tabe5} dargestellt.
\begin{table}[H]
  \centering
  \caption{Mechanischen Kompressorleistung zu den Zeiten $t_1$, $t_2$, $t_3$ und $t_4$.}
  \label{tab:tabe5}
    \begin{tabular}{S S}
    \toprule
    $ t  \: / \si{\second} $ & $ N_{\text{mech}} \: / \: \si{\watt}$ \\
    \midrule
    480 & 4.72 \pm 0.16 \\
    960 & 6.19 \pm 0.22 \\
    1500 & 6.67 \pm 0.26 \\
    1980 & 6.26 \pm 0.28 \\
      \bottomrule
    \end{tabular}
\end{table}

Durch einen Vergleich mit der Datenbank
\cite{lara} fällt auf, dass die Peaks mit denen von Barium übereinstimmen und es sich somit
bei der Probe offensichtlich um Barium handelt. Die Zuordnung zu den theoretischen Emissionslinien
und die entsprechenden Absorptionswahrscheinlickeiten sind in Tabelle \ref{tab:tabe6} angegeben.
\begin{table}[H]
  \centering
  \caption{Messwerte des Absorptionsspektrums von Zirkonium}
  \label{tab:tabe6}
    \begin{tabular}{S S}
    \toprule
    $ \text{Winkel} / ° $ & $ \text{Impulse pro s}$\\
    \midrule
    18.0 & 59.0 \\
    18.2 & 59.0 \\
    18.4 & 58.0 \\
    18.6 & 56.0 \\
    18.8 & 55.0 \\
    19.0 & 57.0 \\
    19.2 & 70.0 \\
    19.4 & 88.0 \\
    19.6 & 105.0 \\
    19.8 & 105.0 \\
    20.0 & 112.0 \\
    20.2 & 115.0 \\
    20.4 & 113.0 \\
    20.6 & 116.0 \\
    20.8 & 116.0 \\
    21.0 & 114.0 \\

          \bottomrule
        \end{tabular}
    \end{table}

Zur Bestimmung der Aktivität wird mittels Gleichung \ref{eqn:inh} zunächst der Inhalt der Peaks errechnet.
Aus Gleichung \ref{eqn:itheo} lässt sich ein linearer Zusammenhang zwischen diesem Inhalt und der Aktivität
erkennen, wobei der Proportionalitätsfaktor durch
\begin{equation}
  l = P\cdot \frac{\Omega}{4\pi}\cdot t \cdot Q
  \label{eqn:lin}
\end{equation}
gegeben ist. Zur Bestimmung der Aktivität wird also eine Lineare Ausgleichsrechnung
gemäß Formel \ref{eqn:li} mit Wertepaaren aus $l$ und $\symup{I}$ durchgeführt,
wobei der Fitparameter a die Aktivität angibt. Dabei wird die $\SI{81.0657}{\kilo\electronvolt}$-Linie
außer Acht gelassen, da bei so niedrigen Energien der Quotient $Q$ der Vollenergienachweiswahrscheinlichkeit
zu ungenau ist. Somit ergeben sich die Parameter
\begin{align*}
  a = 416.4 \pm 2.7 \\
  b = 35 \pm 14
\end{align*}
und damit eine Aktivität von $\SI{416.4(27)}{\becquerel} $.
Die Wertepaare und die Ausgleichsgerade sind in Abbildung \ref{fig:plot9} dargestellt.
\begin{figure}
  \centering
  \includegraphics[height=9cm]{Plot9.pdf}
  \caption{Lineare Regression zur Aktivitätsbestimmung von $\ce{^{133}Ba}$}
  \label{fig:plot9}
\end{figure}



\subsection{Nuklididentifikation}
Im letzten Versuchsteil wird eine Probe unbekannter Zusammensetzung untersucht, wobei die
Messzeit $\SI{4064}{\second}$ beträgt. Von den 8191 gemessenen Kanälen werden die ersten 6000 zur Auswertung
verwendet, die entsprechenden Messwerte werden gemäß Gleichung \ref{eqn:gerade} kalibriert und
sind in Abbildung \ref{fig:plot8} dargestellt.
\begin{figure}
  \centering
  \includegraphics[height=9cm]{Un.pdf}
  \caption{Unbekanntes Spektrum}
  \label{fig:plot8}
\end{figure}

Um die Energie der Peaks zu bestimmen und sie somit zuordnen zu können, werden sie
wieder an die Gaußkurve gemäß Gleichung \ref{eqn:gausk} angepasst. Die Fitparameter sind in Tabelle
\ref{tab:tabe7} angegeben.

\begin{table}[H]
  \centering
  \caption{Werte der zweiten Messreihe für Wert 16}
  \label{tab:tabe7}
    \begin{tabular}{S S S}
    \toprule
    $ \text{R}_{2} \: / \: \si{\ohm} $ & $\text{R}_{3} \: / \: \si{\ohm} $ &
    $\text{R}_{4} \: / \: \si{\ohm} $ \\
    \midrule
    500 & 178 & 328 \\
    664 & 178 & 328 \\
    1000 & 119 & 332 \\
    \bottomrule
    \end{tabular}
\end{table}

Die Linienenergien werden dann mit der Datenbank \cite{lara} abgeglichen und
passenden Nukliden zugeordent.
Diese Zuordnung ist in Tabelle \ref{tab:tabe8} zusammen mit den jeweiligen Emissionswahrscheinlichkeiten
dargestellt.
\begin{table}[H]
  \centering
  \caption{Werte der Messreihe die Wien-Robinson-brücke}
  \label{tab:tabe8}
    \begin{tabular}{S S S S}
    \toprule
    $ \nu \: / \: \si{\hertz} $ & $\text{U}_b \: / \: \si{\volt} $ &
    $\text{U}_s \: / \: \si{\volt} $ &
    $\frac{U_b}{U_s}$ \\
    \midrule
    20 & 0.120 & 3.08 & 0.039 \\
    50 & 0.248 & 4.56 & 0.054 \\
    100 & 0.320 & 4.64 & 0.069 \\
    150 & 0.264 & 4.56 & 0.058 \\
    200 & 0.136 & 4.50 & 0.030 \\
    220 & 0.072 & 4.48 & 0.016 \\
    230 & 0.032 & 4.48 & 0.007 \\
    240 & 0.024 & 4.48 & 0.005 \\
    242 & 0.016 & 4.48 & 0.004 \\
    250 & 0.040 & 4.48 & 0.009 \\
    265 & 0.080 & 4.48 & 0.018 \\
    300 & 0.298 & 4.56 & 0.065 \\
    500 & 0.704 & 4.56 & 0.154 \\
    1000 & 1.17 & 4.32 & 0.271 \\
    3000 & 1.41 & 4.28 & 0.330 \\
    10000 & 1.44 & 4.24 & 0.340 \\
    20000 & 1.44 & 4.24 & 0.340 \\
    30000 & 1.44 & 4.24 & 0.340 \\

    \bottomrule
    \end{tabular}
\end{table}

Alle identifizierten Nuklide gehören zu der Uran-Radium Zerfallsreihe, welche auch als 4n+2
Reihe bekannt ist. Es lässt sich also vermuten, dass die Probe ursprünglich zum Teil aus
$\ce{^{238}U}$ oder $\ce{^{234}Th}$ bestand, welches über  die Zeit zerfallen ist und somit die
radioaktiven Tochternuklide $\ce{^{226}Ra}$, $\ce{^{214}Pb}$ und $\ce{^{214}Bi}$ erklärt.
Alle weiteren Nuklide der Reihe haben keine, oder zumindest nur sehr schwache
Gammasignaturen, sodass diese nicht detektiert werden können.
%Für die Aktivitätsbestimmung sind nur bei $\ce{^{214}Pb}$ und $\ce{^{214}Bi}$ genügend
%Linien vorhanden um eine nicht allzu ungenaue Rechnung durchführen zu können. Das Vorgehen ist analog zur
%Bestimmung der Aktivität von $\ce{^{133}Ba}$ und die resultierenden Ausgleichsgeraden sind in den
%Abbildungen \ref{fig:plot11} und \ref{fig:plot12} dargestellt. Bei der Aktivitätsbestimmung
%von $\ce{^{214}Pb}$ wurde die $\SI{77.1088}{\kilo\electronvolt}$ Linie außer acht gelassen,
%da hier der Quotient $Q$ zu ungenau ist.
%Es ergeben sich die Fitparameter
%\begin{align*}
%  a_{\text{Pb}} &= 1106 \pm 17 \\
%  b_{\text{Pb}} &= -6\pm 69 \\
%  a_{\text{Bi}} &= 1442 \pm 89 \\
%  b_{\text{Bi}} &= 7 \pm 21 \\
%\end{align*}
%und somit Aktivitäten von $\SI{1106(17)}{\becquerel} $ für $\ce{^{214}Pb}$ und
%$\SI{1442(89)}{\becquerel} $ für $\ce{^{214}Bi}$.
%\begin{figure}
%  \centering
%  \includegraphics[height=9cm]{Plot11.pdf}
%  \caption{Lineare Regression zur Aktivitätsbestimmung von $\ce{^{214}Pb}$}
%  \label{fig:plot11}
%\end{figure}
%\begin{figure}
%  \centering
%  \includegraphics[height=9cm]{plot12.pdf}
%  \caption{Lineare Regression zur Aktivitätsbestimmung von $\ce{^{214}Bi}$}
%  \label{fig:plot12}
%\end{figure}
