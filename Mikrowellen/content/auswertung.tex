\section{Auswertung}
\label{sec:Auswertung}
\subsection{Untersuchung der Moden}

Die gemessenen Werte der Moden des Reflexklystrons sind in Tabelle \ref{tab:tab1} und das
dazugehörige Modendiagramm ist in Abbildung \ref{fig:Mode} dargestellt.
\begin{table}[H]
  \centering
  \caption{Messwerte und Ergebniss der Bestimmung der Schallgeschwindigkeit}
  \label{tab:tabe1}
    \begin{tabular}{S||S S||S S||S|S}
    \toprule
    $ \text{Länge l des Zylinders [mm]} $ & $ U_{1} [\text{V}] $ &
    $ t_{1} [\mu\text{s}] $ & $ U_{2} [\text{V}] $ &
    $ t_{2} [\mu\text{s}] $ & $ \increment t [\mu\text{s}]$ &
    $ \text{c} [\text{m}/\text{s}]$\\
    \midrule
    31.0 & 1.335 \: & 24.0 & 1.096 \:  & 46.9 & 22.9 & 2707.42 \\
          \bottomrule
    \end{tabular}
  \end{table}

\begin{figure}[H]
  \centering
  \includegraphics[height=9cm]{plot2.pdf}
  \caption{Messwerte der Moden mit einer Ausgleichsrechnung.}
  \label{fig:Mode}
\end{figure}
Die Messwerte werden mit einer Parabel der Form
\begin{equation}
  f(x)=-a\cdot(x-b)^2 +c
\end{equation}
gefittet.

\subsection{Bestimmung von Wellenlänge und Frequenz}

Die mit Hilfe des Frequenzmessers ermittelte Frequenz beträgt
\begin{equation}
  f=\SI{9009}{\MHz}.
\end{equation}

Die bei der Wellenlängenmessung gemessenen Sondenpositionen sind zusammen mit den daraus berechneten Abständen
und der Wellenlänge in Tabelle \ref{tab:tab4} dargestellt. Um Ungenauigkeiten beim Ablesen zu berücksichtigen wird ein
Fehler von $\pm$ 1 auf die letzte Nachkommastelle angenommen. Dabei entspricht die Wellenlänge dem doppelten
Sondenabstand.
\begin{table}[H]
  \centering
   \begin{tabular}{c c c c}
    \toprule
    Nummer der Oberwelle & $ U_{\text Theorie,Rechteck}\: / \si{\volt} $ &
    $ U_{\text Theorie,Dreick}\: / \si{\volt} $ & $ U_{\text Theorie,Sägezahn}\: / \si{\volt} $ \\
    \midrule
    1 & 1145 & 182 & 573 \\
    2 & 0 & 0 & 286 \\
    3 & 573 & 20 & 191 \\
    4 & 0 & 0 & 143 \\
    5 & 229 & 7 & 115 \\
    6 & 0 & 0 & 96 \\
    7 & 164 & 4 & 82 \\
    8 & 0 & 0 & 72 \\
    9 & 127 & 2 & 64 \\
    10 & 0 & 0 & 57 \\
    \bottomrule
  \end{tabular}
  \caption{Eingestellte Schwingungsamplituden.}
  \label{tab:tabe4}
\end{table}

Aus diesen Daten ergibt sich für die mittlere Wellenlänge im Hohlleiter zu
\begin{equation}
  \lambda_g = \SI{48.6(1)}{\mm}.
\end{equation}
Nach der Formel
\begin{equation}
  f=c\cdot\sqrt{\Big(\frac{1}{\lambda_g}\Big)^2 + \Big(\frac{1}{2a}\Big)^2}
\end{equation}
lässt sich aus dem Mittelwert für $\lambda_g$ die Frequenz mit
\begin{equation}
  f=\SI{9009(13)}{\MHz}
\end{equation}
berechnen. Der dazugehörige Fehler wurde über die Gaußsche Fehlerfortpflanzug
\begin{equation}
  \Delta f=\sqrt{\frac{c^2\cdot\Delta \lambda^2_g}{\Big(\frac{1}{\lambda^2_g}\Big)+\Big(\frac{\lambda^3_g}{4a^2}\Big)}}
\end{equation}
berechnet.
%Die Wellenlänge $\lambda_0$ im freien Raum wird über
%\begin{equation}
%  \lambda_0=\frac{1}{\sqrt{(\frac{1}{\lambda_g})^2} +(\frac{1}{2a})^2}
%\end{equation}
Mit der allgemeinen Beziehung $c=f\cdot\lambda_0$ lässt sich $\lambda_0$ zu
\begin{equation}
  \lambda_0=\SI{33.30(5)}{\mm}
\end{equation}
bestimmen. Um die Grenzwellenlänge zu berechnen wird Gleichung \ref{eqn:zsmhl} nach $\lambda_c$ umgestellt:
\begin{equation}
  \lambda_c=\frac{\lambda_0}{1-\Big(\frac{\lambda_0}{\lambda_g}\Big)^2}.
\end{equation}
Mit den bereits berechneten Wellenlängen $\lambda_g$ und $\lambda_0$ ergibt sich eine Grenzwellenlänge von
\begin{equation}
  \lambda_c=\SI{62.77(19)}{\mm}.
\end{equation}
Hier berechnet sich der Fehler über
\begin{equation}
  \Delta\lambda_c=\sqrt{\Bigg(\frac{\lambda^2_g\lambda^2_0 + \lambda^4_g}{(\lambda^2_0-\lambda^2_g)^2}\cdot\Delta\lambda_0\Bigg)^2+
  \Bigg(\frac{-2\lambda^3_0}{\Big(1-\frac{\lambda^2_0}{\lambda^2_g}\Big)^2\cdot\lambda^3_g}\cdot\Delta\lambda_g\Bigg)^2}
\end{equation}
Zudem lässt sich nach der Formel \ref{eqn:vphase} die Phasengeschwindigkeit zu
\begin{equation}
  v_\text{Ph}=\SI{4,74(3)e5}{\m\per\s}
\end{equation}
mit der Fehlerformel
\begin{equation}
  \Delta v_\text{Ph}=\sqrt{\Bigg(\frac{c\cdot\lambda_0}{\lambda^2_c(1-\frac{\lambda^2_0}{\lambda^2_c})^{(3/2)}}\cdot\Delta\lambda_0\Bigg)^2
  +\Bigg(-\frac{c\cdot\lambda^2_0}{(1-\frac{\lambda^2_0}{\lambda^2_c})^{(3/2)}\cdot\lambda^3_c} \cdot\Delta\lambda_c\Bigg)^2}
\end{equation}
berechnen.
\subsection{Bestimmung der Dämpfung}

Die gemessenen Werte der Dämpfungsbestimmung über die Methode der Leistungsverhältnisse sind
in Tabelle \ref{tab:tab2} dargestellt. Die Messwerte als auch die Theoriedaten sind in Abbildung \ref{fig:plot1}
direkt gegen die Mikrometereinstellung $d$ aufgetragen, da die Skala des SWR-Meters der Beziehung
$10\cdot\log\frac{V_0}{V}=10\cdot\log\frac{P_0}{P}$ folgt.

%Dabei gilt die Formel
%\begin{equation}
%  \Big(\frac{P_1}{P_2}\Big)\si{dB}=10\cdot\log\Big(\frac{P_1}{P_2}\Big).
%\end{equation}
\begin{table}[H]
  \centering
  \caption{Zählrate und Energiemaximum bei variiertem Druck, Abstand a=2cm}
  \label{tab:tab2}
    \begin{tabular}{c c c c c}
    \toprule
    Druck $\rho$/\;mbar & Energiemaximum & Zählrate $N$ & Energie $E_{\alpha}$ & effektive Länge $x$/\;cm\\
    \midrule
    0 & 796 &131382  &4          & 0.0   \\
    50 & 775 &131464 &3.89 & 0.09 \\
    100 &756 &130732 &3.79 & 0.19\\
    150 &749 &129617 &3.76  &  0.29\\
    200 &749 &130444 &3.76  & 0.39\\
    250 &727 &129600 &3.65 & 0.49\\
    300 &722 &128936 &3.63 & 0.59\\
    350 &708 &128478 &3.56 & 0.69\\
    400 &696 &128122 &3.49 & 0.79\\
    450 &687 &127415 &3.45 & 0.89\\
    500 &674 &126608 &3.39 & 0.99\\
    550 &663 &126372 &3.33 &1.09\\
    600 &651 &124989 &3.27 & 1.18\\
    650 &634 &124942 &3.19 & 1.28\\
    700 &618 &124295 &3.11 &1.38\\
    750 &602 &123299 &3.03 & 1.48\\
    800 &584 &119958 &2.93 &1.58\\
    850 &566 &120673 &2.84 &1.68\\
    900 &548 &117907 &2.75 & 1.78\\
    950 &534 &116111 &2.68&   1.88\\
    1000 &499& 108630&2.51 & 1.07\\
    \bottomrule
    \end{tabular}
  \end{table}


\begin{figure}
  \centering
  \includegraphics[height=9cm]{plot1.pdf}
  \caption{Messdaten und Theoriewerte der Dämpfung.}
  \label{fig:plot1}
\end{figure}

\subsection{Untersuchung des Welligkeitsverhältnisses}

Die Messwerte der direkten Methode der Stehwellenmessung sind in Tabelle \ref{tab:tab3} aufgetragen.
\begin{table}[H]
  \centering
   \begin{tabular}{c c c}
    \toprule
     n& $\nu$/\; 1/s & $\nu_{Wechsel}$\\
    \midrule
    0,5 & 100.01& 50,0\\
    1 & 79.93 & 79.93\\
    2 & 23.93 & 47.86\\
    \bottomrule
  \end{tabular}
  \caption{Gemessene Frequenzen der Sägezahnspannung, sowie die Daraus resultierenden Frequenzen für die
  Wechselspannung.}
  \label{tab:tab3}
\end{table}

Da sich im Laufe der Messung die Wellenlänge im Hohlleiter veränder haben könnte, werden zur Kontrolle nochmals
die Abstände der Minima gemessen und daraus die Wellenlänge berechnet. Die Abstände der Minima wurden mit
\begin{align*}
  x_1&=\SI{118.0(1)}{\mm}\:\:\text{und}\\
  x_2&=\SI{93.7(1)}{\mm}
\end{align*}
gemessen.
Daraus ergibt sich eine Wellenlänge von $\lambda_g=\SI{48.6(3)}{\mm}$.\\

Die bei der “3-dB-Methode“ gemessenen Werte lauten:
 \begin{align*}
   d_\text{1}&=\SI{119.1(1)}{\mm} \:\:\text{und}\\
   d_\text{2}&=\SI{117.0(1)}{\mm}
 \end{align*}
 Mit der Formel
 \begin{equation}
   S=\sqrt{1+\frac{1}{\sin^2\Big(\frac{\pi (d_1 - d_2)}{\lambda_g \cdot g}\Big)}}\sim\frac{\lambda_g}{\pi (d_1 - d_2)}
 \end{equation}
 und dem dazugehörigen Fehler
 \begin{equation}
   \Delta S=\sqrt{\Big(\frac{1}{\pi(d_1 - d_2)}\cdot\Delta\lambda_g\Big)^2+\Big(-\frac{\lambda_g}{\pi{(d_1 - d_2)^2}}\cdot\Delta d_1\Big)^2+
   \Big(\frac{\lambda_g}{\pi(d_1 - d_2)^2}\cdot\Delta d_2\Big)^2}
 \end{equation}
 ergibt sich ein Stehwellenverhältnis von $S=\SI{7.4(5)}{}$.\\

 Bei der Abschwächer-Methode wird das Stehwellenverhältnis über die Differenz der Abschwächeinstellungen bestimmt, wobei
 ein Ablesefehler von $\pm 1$ angenommen wird.
 Das Dämpfungsglied wird auf
 \begin{align*}
   A_1&=\SI{20(1)}{\dB} \:\:\text{und}\\
   A_2&=\SI{43(1)}{\dB}
 \end{align*}
 eingestellt. Das Stehwellenverhältnis wird mit der Formel
 \begin{equation}
   S=10\cdot\frac{A_2 - A_1}{20}
 \end{equation}
 und der Fehlerformel
 \begin{equation}
   \Delta S=\sqrt{\Big(\frac{-1}{2}\cdot\Delta A_1\Big)^2+\Big(\cdot{\frac{1}{2}\cdot\Delta A_2}\Big)^2  }
 \end{equation}
zu $S=\SI{11.5(7)}{}$ berechnet.
