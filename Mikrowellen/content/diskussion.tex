\section{Diskussion}
\label{sec:Diskussion}
Die Untersuchung der Moden des Reflexklystrons ergibt, dass sich die einzelnen Moden gut mit einer
nach unten geöffneten Parabel annähern lassen. Die Bestimmung der Wellenlänge liefert für
die mittlere Wellenlänge im Hohlleiter eine Wellenlänge von $\lambda_g=\SI{48,6(1)}{\mm}$ und
außerhalb der Hohlleiters eine Wellenlänge von $\lambda_0=\SI{33,30(5)}{\mm}$. Wie zu erwarten ist die
Wellenlänge im Hohlleiter größer, als im freien Raum. Zudem liegen diese Wellenlängen in der für
Mikrowellen zu erwartenden Größenordnung. Die daraus berechnete Frequenz $f=\SI{9009(13)}{\MHz}$
stimmt bis auf die geringe Abweichung des Fehler mit der direkt gemessenen Frequenz von $\SI{9009}{\MHz}$
überein. Aus diesen Daten wird die Grenzfrequenz zu $\lambda_c=\SI{62,77(19)}{\mm}$ berechnet.
Damit ergibt sich eine Phasengeschwindigkeit von $v_\text{Ph}=\SI{4,74(3)e5}{\m\per\s}$.

Der Vergleich der Messdaten und der Theoriewerte der Dämpfungsbestimmung zeigt, dass die Messwerte
konstant unterhalb der Theoriewerte liegen. Dies deutet auf einen systematischen Fehler hin, welcher durch einen
Wackelkontakt im Kabel des SWR-Meters erklärt werden könnte. Schon bei leichten Berührungen des Kabels
kam es zu großen Schwankungen des SWR-Wertes.

Für die Stehwellenverhältnisse werden je nach Messmethode unterschiedliche Werte ermittelt. Die Messwerte der
direkten Methode sind in Tabelle \ref{tab:tab3} dargestellt. Die “3-dB-Methode“ liefert ein Stehwellenverhältnis
von $S=\SI{7,4(5)}{}$, währen die Abschwächermethode ein Stehwellenverhältnis von $\SI{11,5(7)}{}$ liefert.
