\section{Auswertung}

\subsection{Eichung des Magneten}
Die gemessenen Werte zur Eichung des Magneten sind in Tabelle \ref{tab:tabe1} angegeben.
\begin{table}[H]
  \centering
  \caption{Messwerte und Ergebniss der Bestimmung der Schallgeschwindigkeit}
  \label{tab:tabe1}
    \begin{tabular}{S||S S||S S||S|S}
    \toprule
    $ \text{Länge l des Zylinders [mm]} $ & $ U_{1} [\text{V}] $ &
    $ t_{1} [\mu\text{s}] $ & $ U_{2} [\text{V}] $ &
    $ t_{2} [\mu\text{s}] $ & $ \increment t [\mu\text{s}]$ &
    $ \text{c} [\text{m}/\text{s}]$\\
    \midrule
    31.0 & 1.335 \: & 24.0 & 1.096 \:  & 46.9 & 22.9 & 2707.42 \\
          \bottomrule
    \end{tabular}
  \end{table}

Um eine Eichkurve zu bestimmen werden diese Werte mit einem Polynom dritten Grades
der Form
\begin{equation}
  B(I)=a\cdot I^3+b\cdot I^2 + c \cdot I +d
\end{equation}
gefittet.
Die resultierenden Parameter lauten
\begin{align*}
  a &= \SI{-0.087(7)}{\milli\tesla\per\cubic\ampere} \\
  b &= \SI{1.85(18)}{\milli\tesla\per\square\ampere}\\
  c &= \SI{48.7(15)}{\milli\tesla\per\ampere}\\
  d &= \SI{10(3)}{\milli\tesla}
\end{align*}
Die entsprechende Kurve ist zusammen mit den Messwerten in Abbildung \ref{fig:plot1}
dargestellt.
\begin{figure}
  \centering
  \includegraphics[height=9cm]{Plot1.pdf}
  \caption{Eichkurve und Messwerte des Magneten}
  \label{fig:plot1}
\end{figure}
Somit ergibt sich die Magnetfeldstärke bei gegebenem Strom über die Formel
\begin{equation}
  B(I)=\SI{-0.087}{\milli\tesla\per\cubic\ampere}\cdot I^3
  +\SI{1.85}{\milli\tesla\per\square\ampere}\cdot I^2
  +\SI{48.7}{\milli\tesla\per\ampere} \cdot I + \SI{10}{\milli\tesla} \; ,
  \label{eqn:bfeld}
\end{equation}
wobei der Fehler sich gemäß der Gaußschen Fehlerfortpflanzung
\begin{equation}
  \increment f = \sqrt{ \sum_{i=1}^N \left( \frac{\partial f}{\partial x_i}\right)^2
  \cdot (\increment x_i)^2  }
  \label{eqn:gaus}
\end{equation}
über
\begin{equation}
  \begin{split}
  \increment B(I) &= \sqrt{ \left( \increment a \cdot I^3 \right)^2 + \left( \increment b \cdot I^2 \right)^2
  + \left( \increment c \cdot I \right)^2 + \left( \increment d \right)^2} \\\
  &= \sqrt{ \left( \SI{0.007}{\milli\tesla\per\cubic\ampere} \cdot I^3 \right)^2 +
  \left(  \SI{0.18}{\milli\tesla\per\square\ampere} \cdot I^2 \right)^2
  + \left( \SI{1.5}{\milli\tesla\per\ampere} \cdot I \right)^2 +
  \left( \SI{3}{\milli\tesla} \right)^2}
  \label{eqn:fbfeld}
\end{split}
\end{equation}

\subsection{Auswertung der roten Linie}
Die rote Linie bei $\SI{643.8}{\nano\meter}$  wurde einmal ohne Magnetfeld gemessen und einmal
bei einem Magnetfeld bei einem Strom von $\SI{10}{\ampere}$. Nach Gleichungen
\ref{bfeld} und \ref{fbfeld} entspricht dies einer Magnetfeldstärke von $\SI{0.595(25)}{\tesla}$,
wobei sich der Theoriewert ???=1,0 aus Gleichung \ref{eqn:???} ergibt.
Die aufgenommnen Bilder sind in Abbildungen \ref{fig:rot1} und \ref{fig:rot2} angegeben.
\begin{figure}
\begin{subfigure}[c]{0.5\textwidth}

\includegraphics[width=0.8\textwidth]{Rot1.JPG}
\subcaption{\tiny Linien ohne Aufspaltung}
\label{fig:rot1}
\end{subfigure}
\begin{subfigure}[c]{0.5\textwidth}
\includegraphics[width=0.8\textwidth]{Rot2.JPG}
\subcaption{\tiny Linien mit Aufspaltung bei B=$\SI{0.595(25)}{\tesla}$}
\label{fig:rot2}
\end{subfigure}
\caption{Bilder der roten Cadmium Linie}
\end{figure}
Aus den Bildern werden die Abstände der Linien in Pixeln vermessen. Die daraus resultierenden Messwerte sind
in Tabelle \ref{tab:tabe2} angegeben. Die entsprechenden Wellenlängenänderungen lassen sich über
\begin{equation}
  \delta \lambda = \frac{\increment S}{2 \delta S} \increment \lambda_D
\end{equation}
berechnen, wobei $\increment \lambda_D$ für die rote Linie bei $\SI{643.8}{\nano\meter}$
einem Wert von $\SI{0.0489}{\nano\meter}$ entspricht.
\begin{table}[H]
  \centering
  \caption{Zählrate und Energiemaximum bei variiertem Druck, Abstand a=2cm}
  \label{tab:tab2}
    \begin{tabular}{c c c c c}
    \toprule
    Druck $\rho$/\;mbar & Energiemaximum & Zählrate $N$ & Energie $E_{\alpha}$ & effektive Länge $x$/\;cm\\
    \midrule
    0 & 796 &131382  &4          & 0.0   \\
    50 & 775 &131464 &3.89 & 0.09 \\
    100 &756 &130732 &3.79 & 0.19\\
    150 &749 &129617 &3.76  &  0.29\\
    200 &749 &130444 &3.76  & 0.39\\
    250 &727 &129600 &3.65 & 0.49\\
    300 &722 &128936 &3.63 & 0.59\\
    350 &708 &128478 &3.56 & 0.69\\
    400 &696 &128122 &3.49 & 0.79\\
    450 &687 &127415 &3.45 & 0.89\\
    500 &674 &126608 &3.39 & 0.99\\
    550 &663 &126372 &3.33 &1.09\\
    600 &651 &124989 &3.27 & 1.18\\
    650 &634 &124942 &3.19 & 1.28\\
    700 &618 &124295 &3.11 &1.38\\
    750 &602 &123299 &3.03 & 1.48\\
    800 &584 &119958 &2.93 &1.58\\
    850 &566 &120673 &2.84 &1.68\\
    900 &548 &117907 &2.75 & 1.78\\
    950 &534 &116111 &2.68&   1.88\\
    1000 &499& 108630&2.51 & 1.07\\
    \bottomrule
    \end{tabular}
  \end{table}

Durch die Gleichung
\begin{equation}
  \bar{x} = \frac{1}{N} \sum_{i=1}^{N} x_i \: \:
  \label{eqn:mit}
\end{equation}
\noindent lässt sich der Mittelwert bilden, wobei der dazugehörige Fehler sich durch
\begin{equation}
  \increment \bar{x} = \frac{1}{\sqrt{N}} \sqrt{ \frac{1}{N-1} \sum_{i=1}^N
  (x_i - \bar{x})^2}
  \label{eqn:mitf}
\end{equation}
ergibt. Somit ergibt sich für $\delta \lambda$ ein Mittelwert von
$ \delta \lambda = \SI{11.52(20)}{\pico\meter} $.
Durch Gleichung \ref{eqn:???} ergibt sich letztendlich für ???
\begin{align*}
??=1,00 \pm 0,04 \: .
\end{align*}

\subsection{Auswertung der blauen Linie}
Für die blaue Linie bei $\SI{480}{\nano\meter}$  wurde neben der Aufnahme ohne Magnetfeld einmal
die Aufspaltung bei einem Strom von $\SI{5.5}{\ampere}$ für die $\sigma$-Linien aufgenommen
und einmal bei einem Strom von $\SI{14}{\ampere}$ für die $\pi$-Linien. Nach Gleichungen
\ref{bfeld} und \ref{fbfeld} entspricht dies Magnetfeldstärken von $\SI{0.319(10)}{\tesla}$
und $\SI{0.82(5)}{\tesla}$ , mit dem theoretischen Werten ??=1..
Die aufgenommnen Bilder sind in Abbildungen \ref{fig:blau1},\ref{fig:blau2} und \ref{fig:blau3} angegeben.
\begin{figure}
\begin{subfigure}[c]{0.5\textwidth}

\includegraphics[width=0.8\textwidth]{Blau1.JPG}
\subcaption{\tiny Linien ohne Aufspaltung}
\label{fig:blau1}
\end{subfigure}
\begin{subfigure}[c]{0.5\textwidth}
\includegraphics[width=0.8\textwidth]{Blau2.JPG}
\subcaption{\tiny Linien mit Aufspaltung bei B=$\SI{0.319(10)}{\tesla}$}
\label{fig:blau2}
\end{subfigure}
\begin{subfigure}[c]{0.5\textwidth}
\includegraphics[width=0.8\textwidth]{Blau3.JPG}
\subcaption{\tiny Linien mit Aufspaltung bei B=$\SI{0.82(5)}{\tesla}$}
\label{fig:blau3}
\end{subfigure}
\caption{Bilder der blauen Cadmium Linie}
\end{figure}
Die gemessenen Abstände und die entsprechenden Wellenlängenänderungen mit dem Wert
$\increment \lambda_D = \SI{0.02695}{\nano\meter}$ der blauen Linie sind in Tabelle
\ref{tab:tabe3} angegeben.
Der Mittelwert der Wellenlängenänderung ergibt sich nach Gleichungen \ref{eqn:mit} und
\ref{eqn:mitf} zu $ \delta \lambda_{\sigma} = \SI{6.17(4)}{\pico\meter} $ für die $\sigma$-Linie
und zu $ \delta \lambda_{\pi} = \SI{5.01(5)}{\pico\meter} $ für die $\pi$-Linie.
Aus Gleichung \ref{eqn:????} ergibt sich somit
\begin{align}
  ???&=1,80 \pm 0,06 \; , \sigma-\text{Linie}\\
  ???&=0,57 \pm 0,032 \; , \pi-\text{Linie}
\end{align}
für die Werte von ??? der beiden Linien.
