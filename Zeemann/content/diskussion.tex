\section{Diskussion}
\label{sec:Diskussion}
Die experimentell bestimmten Werte sind zusammen mit den Theoriewerten aus
Gleichung \ref{eqn:gJ} und der relativem Abweichung,
die sich über
\begin{equation}
  \frac{\lvert \text{Wert}_{\text{Theorie}}-\text{Wert}_{\text{Messung}}\rvert}{\text{Wert}_{\text{Theorie}}}
  \label{eqn:abw}
\end{equation}
bestimmen lässt, in Tabelle \ref{tab:tabe4} angegeben. Für die $\sigma$-Linie ergeben sich hierbei
zwei unterschiedliche Theoriewerte, nämlich $g_J\,=\, 2$ und $g_J\,=\, 1,5$. Die Linien lassen sich jedoch experimentell nicht
auflösen, sodass als Theoriewert der Mittelwert zwischen beiden, also $g_J\,=\, 1,75 $ verwendet wird.
\begin{table}[H]
  \centering
   \begin{tabular}{c c c c}
    \toprule
    Nummer der Oberwelle & $ U_{\text Theorie,Rechteck}\: / \si{\volt} $ &
    $ U_{\text Theorie,Dreick}\: / \si{\volt} $ & $ U_{\text Theorie,Sägezahn}\: / \si{\volt} $ \\
    \midrule
    1 & 1145 & 182 & 573 \\
    2 & 0 & 0 & 286 \\
    3 & 573 & 20 & 191 \\
    4 & 0 & 0 & 143 \\
    5 & 229 & 7 & 115 \\
    6 & 0 & 0 & 96 \\
    7 & 164 & 4 & 82 \\
    8 & 0 & 0 & 72 \\
    9 & 127 & 2 & 64 \\
    10 & 0 & 0 & 57 \\
    \bottomrule
  \end{tabular}
  \caption{Eingestellte Schwingungsamplituden.}
  \label{tab:tabe4}
\end{table}

Der Theoriewert der roten Linie und der experimentell ermittelte Wert stimmen exakt überein, die Messung war somit
sehr genau. Auch bei der $\sigma$-Linie der blauen Linie ist die Abweichung nur sehr gering und der Theoriewert
liegt innerhalb eines Fehlerintervalls. Diese Genauigkeit war durch das hohe Auflösungsvermögen der Lummer-Gehrcke-Platten
möglich, welches nach Gleichung \ref{eqn:Auflösung} bei der roten Spektrallinie 209128,5 und bei der blauen
Spektrallinie 285458,1 beträgt, möglich.
Die Abweichung bei der $\pi$-Linie ist hingegen mit $14,00\%$ etwas größer und liegt zudem nur innerhalb des
dritten Fehlerintervalls. Dies kann daran liegen, dass die Aufspaltung bei dieser Linie am geringsten war und
somit am schwersten aufzulösen. Eine weitere Fehlerquelle ist das ablesen der Abstände über den Monitor in
Pixeln, was teilweise nicht ganz exakt möglich war. \\
Auch die Eichung des Magnetfeldes über eine Hall-Sonde ist fehleranfällig, da die Hall-Sonde nicht exakt an
der Stelle der Lampe positoniert werden kann und sie zudem auch sehr anfällig gegenüber leichten Verkippungen ist, da
stets nur das Magnetfeld senkrecht zu der Sonde gemessen wird.
Auch das Ablesen des Stroms war nicht sehr genau möglich. \\
Im Rahmen dieser Unsicherheiten und Fehlerquellen sind die Ergebnisse jedoch recht nah an den Theoriewerten, sodass
anscheinend keine weiteren systematischen Fehler aufgetreten sind.
