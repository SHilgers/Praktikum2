\section{Auswertung}
\subsection{Elektronen im elektrischen Feld}

\label{sec:Auswertung}
Für die Auswertung sind folgende Daten der Apperaturen gegeben:
\begin{align*}
  L=&\SI{14,3}{\cm} \\
  p_{x}=&\SI{1,9}{\cm}\\
  N=&20    \\
  R=&\SI{0,282}{\m}
\end{align*}

Mit der Empfindlichkeit einer Kathodenstrahlröhre wird das Verhältnis zwischen Ablenkung $D$
und angelegter Spannung $U_d$ bezeichnet. In einem $D\text{-}U_d$-Diagramm entspricht das genau der
Steigung. Um die Empfindlichkeit zu bestimmen wird die Ablenkung $D$ gegen die
Spannung $U_D$ aufgetragen, dies ist in Abbildung \ref{fig:plot1} zu sehen.
Die dazu verwendeten Messdaten sowie die berchneten Empfindlichkeiten
sind in Tabelle \ref{tab:tab1} zu finden. Die Ablenkung $D$ wurde in Kästcheneinheiten
des Beobachungsschirms abgelesen, dabei entspricht ein Kästchen $\SI{0,6}{\cm}$.\\
\\
\begin{table}[H]
  \centering
  \caption{Messwerte und Ergebniss der Bestimmung der Schallgeschwindigkeit}
  \label{tab:tabe1}
    \begin{tabular}{S||S S||S S||S|S}
    \toprule
    $ \text{Länge l des Zylinders [mm]} $ & $ U_{1} [\text{V}] $ &
    $ t_{1} [\mu\text{s}] $ & $ U_{2} [\text{V}] $ &
    $ t_{2} [\mu\text{s}] $ & $ \increment t [\mu\text{s}]$ &
    $ \text{c} [\text{m}/\text{s}]$\\
    \midrule
    31.0 & 1.335 \: & 24.0 & 1.096 \:  & 46.9 & 22.9 & 2707.42 \\
          \bottomrule
    \end{tabular}
  \end{table}

\begin{figure}
  \centering
  \includegraphics{plot1.pdf}
  \caption{Diagramm zur Bestimmung der Empfindlichkeit.}
  \label{fig:plot1}
\end{figure}

Um die Apparaturkonstante
\begin{equation}
  K=\frac{pL}{2d}
  \label{eqn:apper}
\end{equation}
aus den Messwerten zu bestimmen wird in einem weiteren Diagramm die Empfindlichkeit $a$
gegen den Kehrwert der Beschleunigungsspannung $\sfrac{1}{U_B}$ aufgetragen. Das Ergebnis
ist in Abbildung \ref{fig:plot3} zu sehen.

\begin{figure}
  \centering
  \includegraphics{plot3.pdf}
  \caption{Empfindlichkeit aufgetragen gegen $\frac{1}{U_B}$.}
  \label{fig:plot3}
\end{figure}

Die lineare Regression liefert folgende Ausgleichsgerade:
\begin{equation}
  \frac{D}{U_d}=\SI{-36,8(15)}{\per\V}+\SI{0,01(0)}{\cm\per\V}.
  \label{eqn:ausgleich}
\end{equation}

Die Steigung der Ausgleichsgerade gibt nach folgender Beziehung die Apperaturkonstante an:
\begin{equation}
  D=\frac{pL}{2d}\frac{U_d}{U_B} \iff \frac{D}{U_d}=\frac{Lp}{2d}\frac{1}{U_B}=a\cdot\frac{1}{U_B}.
\end{equation}

Als theoretischer Wert für die Apperaturkonstante ergibt sich nach \ref{eqn:apper}
\begin{equation*}
  K=\SI{35,75}{\cm}.
\end{equation*}

Nach der Formel
\begin{equation}
  \frac{\text{Theoriewert-Messwert}}{\text{Theoriewert}}\cdot 100
\end{equation}
berechnet sich die Abweichung zu 4,07\%, wobei die Beträge verwendet werden.


Aus den gemessenen Synchrosinationsfrequenzen in Tabelle \ref{tab:tab3} wird die Frequenz der
Sinusspannung ermittelt, denn stehende Bilder bilden sich nur für ein rationales Verhältnis
von $\nu_{\text{Säge}}$ zu $\nu_{\text{Wechsel}}$.
Es gilt die Formel
\begin{equation}
  n\nu_{\text{Säge}}=\nu_{\text{Wechsel}}
\end{equation}
und da $n$ ein rationales Verhältnis einnehmen muss, hier $\frac{n}{2}$,
kann daraus die Wechselspannung ermittelt werden. Die Werte dafür sind in Tabelle
\ref{tab:tab3} zu sehen.
\begin{table}[H]
  \centering
   \begin{tabular}{c c c}
    \toprule
     n& $\nu$/\; 1/s & $\nu_{Wechsel}$\\
    \midrule
    0,5 & 100.01& 50,0\\
    1 & 79.93 & 79.93\\
    2 & 23.93 & 47.86\\
    \bottomrule
  \end{tabular}
  \caption{Gemessene Frequenzen der Sägezahnspannung, sowie die Daraus resultierenden Frequenzen für die
  Wechselspannung.}
  \label{tab:tab3}
\end{table}


Aus diesen Werten wird der Mittelwert berechnet und es ergibt sich
$\nu_{Wechsel}=\SI{59,26(1820)}{\Hz}$.
Außerdem wurchde die maximale Auslenkung der stehenden Welle in y-Richtung ausgemessen, diese
beträgt 2,5\;cm.



\subsection{Elektronen im magnetischen Feld}

Nun soll die Ablenkung der Elektronen im magnetischen Feld untersucht werden.
Dazu sind in Tabelle \ref{tab:tab2} die Messwerte für den Spulenstrom $I$, das
dazugehörige magnetische Feld $B$, sowie die Ablenkung $D$ der Elektronen zu finden.
Dabei wird $B$ nach Formel \ref{eqn:helmholtz} berechnet.

\begin{table}[H]
  \centering
  \caption{Zählrate und Energiemaximum bei variiertem Druck, Abstand a=2cm}
  \label{tab:tab2}
    \begin{tabular}{c c c c c}
    \toprule
    Druck $\rho$/\;mbar & Energiemaximum & Zählrate $N$ & Energie $E_{\alpha}$ & effektive Länge $x$/\;cm\\
    \midrule
    0 & 796 &131382  &4          & 0.0   \\
    50 & 775 &131464 &3.89 & 0.09 \\
    100 &756 &130732 &3.79 & 0.19\\
    150 &749 &129617 &3.76  &  0.29\\
    200 &749 &130444 &3.76  & 0.39\\
    250 &727 &129600 &3.65 & 0.49\\
    300 &722 &128936 &3.63 & 0.59\\
    350 &708 &128478 &3.56 & 0.69\\
    400 &696 &128122 &3.49 & 0.79\\
    450 &687 &127415 &3.45 & 0.89\\
    500 &674 &126608 &3.39 & 0.99\\
    550 &663 &126372 &3.33 &1.09\\
    600 &651 &124989 &3.27 & 1.18\\
    650 &634 &124942 &3.19 & 1.28\\
    700 &618 &124295 &3.11 &1.38\\
    750 &602 &123299 &3.03 & 1.48\\
    800 &584 &119958 &2.93 &1.58\\
    850 &566 &120673 &2.84 &1.68\\
    900 &548 &117907 &2.75 & 1.78\\
    950 &534 &116111 &2.68&   1.88\\
    1000 &499& 108630&2.51 & 1.07\\
    \bottomrule
    \end{tabular}
  \end{table}


Die Ablenkung wurde für beide Beschleunigungsspannungen gegen das Magnetfeld aufgetragen.
Die Messwerte und die lineare Regression sind in den Abblildungen \ref{fig:plot4} und
\ref{fig:plot5} zu sehen.

\begin{figure}
  \centering
  \includegraphics{plot4.pdf}
  \caption{Ablenkung der Elektronen bei U=250\; V.}
  \label{fig:plot4}
\end{figure}

\begin{figure}
  \centering
  \includegraphics{plot5.pdf}
  \caption{Ablenkung der Elektronen bei U=350\; V.}
  \label{fig:plot5}
\end{figure}

Aus den Regressionen in Abbildung \ref{fig:plot4} und \ref{fig:plot5}
ergeben sich für $U_{B}=\SI{250}{V}$ und $U_{B}=\SI{350}{\V}$ folgende Steigungen $a$:
\begin{align*}
  a_{\text{250}}&=\SI{12846,32(18)}{\m\per\tesla}\\
  a_{\text{350}}&=\SI{10833,09(18)}{\m\per\tesla}
\end{align*}

Da $\frac{D}{D^{2}+L^{2}} $ gegen $B$ aufgetragen wurde, kann aus der Formel \ref{eqn:magnet} eine Beziehung für
die Steigung abgelesen werden. Durch umstellen nach $\frac{e_0}{m_e}$ kann die
spezifische Ladung berechnet werden.
\begin{align}
  &a=\frac{1}{8U_{B}}\cdot\frac{e_0}{m_e}\\
  \iff &\frac{e_0}{m_0}=a^{2}\cdot 8 U_{B}^2
\end{align}

Für die beiden Messreihen ergeben sich folgende Werte:
\begin{align*}
  \frac{e_0}{m_0}&=\SI{8,25(23)e+13}{\coulomb\per\kg}\\
  \frac{e_0}{m_0}&=\SI{1,15(4)e+14}{\coulomb\per\kg}.
\end{align*}

Aus den weiteren Messwerten wird die Stärke des Erdmagnetfeldes bestimmt. Dazu wurde der
Inklinationswinkel mit $\phi=65°$ und die angelegte Spannung mit
\begin{align*}
  I_1=\SI{0,16}{\A} \;\;\;\;\text{und}\;\;\;\; I_2=\SI{0,16}{\A}
\end{align*}
bestimmt.

Daraus ergibt sich folgender Wert für die horizontale Komponente des Erdmagnetfeldes:
\begin{equation*}
  B_1=\SI{1e-8}{\tesla}.
\end{equation*}


Unter der Berücksichtigung des Inklinationswinkels ergibt sich für das
Erdmagnetfeld
\begin{equation*}
  B_\text{Erde}=\SI{1,77}{\milli\tesla}.
\end{equation*}
