\section{Diskussion}
\label{sec:Diskussion}
Die Apparaturkonstante K, welche über das elektrische Feld bestimmt wird ist mit einer
betragsmäßigen Abweichung von 4,07\% sehr genau. Der Theoriewert wird mit
$\SI{35,75}{\cm}$ bestimmt und die Messung
ergibt $\SI{-36,8(15)}{\cm}$. Die unterschiedlichen Vorzeichen lassen sich über die
Polung der Kathodenstrahlröhre erklären.

Über die Erzeugung von stehenden Wellen wurde die Frequenz der Wechselspannung
zu
\begin{equation*}
  \nu_{Wechsel}=\SI{59,26(1820)}{\Hz}.
\end{equation*}
bestimmt. Dieser große Fehler lässt sich darüber erklären, dass schon bei der
Versuchsdurchführung keine exakt stehenden Wellen erreicht wurden, sondern immer
Flackern im Bild zu erkennen war. Auch weitere Versuche dieses Problem zu beheben
führten nicht zum Erfolg.

Aus dem zweiten Versuchsteil wird die spezifische Ladung der Elektronen mit
\begin{align*}
  \frac{e_0}{m_0}&=\SI{8,25(23)e+13}{\coulomb\per\kg}\\
  \frac{e_0}{m_0}&=\SI{1,15(4)e+14}{\coulomb\per\kg}
\end{align*}
bestimmt. Bei dem Vergleich mit dem Theoriewert
$\frac{e}{m_e}=\SI{1,75882e11}{\coulomb\per\kg}$ \cite{em}
wird deutlich, dass die Fehler der Messwerte deutlich größer sind als der Theoriewert.
%für die erste Messung eine Abweichung von 0,07 Fehlerintervallen, für die zweite Messung eine
%Abweichung von 0,04 Fehlerintervallen.
Diese großen Fehler lassen sich dadurch erklären, dass die Messung der Ablenkung der Elektronen
höchst ungenau ist, da der Referenzpunkt nicht makiert ist und die Verschiebung des
Elektronenstrahls sehr schlecht abgemessen werden kann. Zudem wird in der Rechnung das Magnetfeld
über den Strom berechnet. Die Kathodenstrahlröhre befindet sich aber in der Mitte des
Helmholzspulenpaares und da das Magnetfeld mit $B\propto\frac{1}{r^{2}}$ abfällt
ist dieser Wert fehlerbelastet.
