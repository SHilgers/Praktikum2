\section{Diskussion}
\label{sec:Diskussion}
Die Apperaturkonstante K, welche über das elektrische Feld bestimmt wird ist mit einer
betragsmäßigen Abweichung von 4,07\% sehr genau. Der theoriewert wird mit
$\SI{35,75}{\cm}$ bestimmt und die Messung
ergibt $\SI{-36,8(15)}{\cm}$. Die unterschiedlichen Vorzeichen lassen sich über die
Polung der Kathodenstrahlröhre erklären.

Über die Erzeugung von stehenden Wellen wurde die Frequenz der Wechselspannung
zu
\begin{equation*}
  \nu_{Wechsel}=\SI{21.02(92)}{\Hz}
\end{equation*}
bestimmt.

Aus dem zweiten Versuchsteil wird die spezifische Ladung der Elektronen mit
\begin{equation*}
  \frac{e}{m_e}=\SI{1.26(0)e6}{\coulomb\per\kg}
\end{equation*}
bestimmt. Der Vergleich mit dem Theoriewert
$\frac{e}{m_e}=\SI{1,75882e11}{\coulomb\kg}$ \cite{em}
ergibt eine Abweichung von 99,9\%.
Da der Fehler für beide Rechnungen so verschwindent klein ist, dass vernachlässigt werden
kann, aber die Abeichung zum Theoriewert doch so hoch ist, ist von einem
systematischen Fehler auzugenen. Zum einen ist die Messung der Ablenkung der Elektronen
höchst ungenau, das der Referenzpunkz nicht makiert ist und die die Verschiebung des
Elektronenstrahls sehr schlecht abgemessen werden kann. Zudem wird in der Rechnung das Magnetfeld
über den Strom berechnet. Die Kathodenstrahlröhre befindet sich aber in der Mitte des
Helmholzspulenpaares und da das Magnetfeld mit $B\propto\frac{1}{r^{2}}$ abfällt
ist dieser Wert fehlerbelastet.
