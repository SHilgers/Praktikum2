\begin{table}[H]
  \centering
   \begin{tabular}{c | c c c c c}
    \toprule
    \:\:\:\:\: & $U_B=\SI{180}{V}$& $U_B=\SI{220}{V}$& $U_B=\SI{270}{V}$& $U_B=\SI{370}{V}$&
    $U_B=\SI{500}{V}$\\
    \toprule
     Käschchen & $U$/\;V & $U$/\;V & $U$/\;V & $U$/\;V & $U$/\;V\\
    \midrule
    0.00 & 17.84 & 21.20 & 26.40 & - & -\\
    0.63 & 14.35 & 17.81 & 21.9 & 29.90 & -\\
    1.25 & 11.38 & 14.12 & 17.28 & 23.30 & 34.50\\
    1.88 & 8.24 & 10.33 & 12.54 & 17.35 & 25.20\\
    2.50 & 4.86 & 6.71 & 7.99 & 10.62 & 16.14\\
    3.13 & 1.80 & 2.94 & 3.17 & 4.20 & 6.93\\
    3.75 & -1.41 & -1.16 & -1.66 & -2.42 & -2.80\\
    4.38 & -5.18 & -5.16 & -7.04 & -9.25 & -13.15\\
    5.00 & -8.40 & -9.32 & -11.82 & -15.98 & -23.10\\
    \midrule
    $\frac{D}{U_d}\;\frac{cm}{V}$ &  -0.19 & -0.16 & -0.13 & -0.095 & -0.07 \\
    \bottomrule
  \end{tabular}
  \caption{Gemessenen Spannung in Abhängigkeit der Anzahl der Kästchen und der Beschleunigungsspannung, sowie
  die entsprechende Empfindlichkeit.}
  \label{tab:tab1}
\end{table}
