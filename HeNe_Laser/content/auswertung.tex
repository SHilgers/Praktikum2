\section{Auswertung}
\label{sec:Auswertung}


\subsection{Überprüfung der Stabilitätsbedingung}

\subsubsection{Zwei konkave Spiegel}
Die gemessenen Werte für die Anordnung mit zwei konkaven Spiegeln mit einem
Krümmungsradius von jeweils $r_i = \SI{1400}{\milli\meter}$ sind in Tabelle \ref{tab:tabe1}
angegeben. Der theoretische Verlauf entspricht nach Gleichungen \ref{eqn:stab1}
und \ref{eqn:stab2} einer Parabel,
sodass die Werte an eine Funktion der Form
\begin{equation}
  I(d)= I_0\Bigl(\frac{d^2}{r^2}-\frac{2d}{r}+1\Bigr)
\end{equation}
angepasst werden. Es werden jedoch nur die Messwerte bis $\SI{118}{\centi\meter}$
bei dem Fit berücksichtigt, da bei höheren Distanzen offensichtlich ein systematischer
Fehler in der Messung vorliegt.
Es ergibt sich hierdurch
\begin{align*}
  I_0 &= \SI{570(50)}{\micro\ampere} \\
  r &=\SI{998(11)}{\milli\meter}
\end{align*}
für die Parameter. Die Messwerte sind zusammen mit der
entsprechenden Kurve in Abbildung \ref{fig:plot1} dargestellt.
\begin{figure}
  \centering
  \includegraphics[height=9cm]{Plot1.pdf}
  \caption{Untersuchung der Stabilität mit zwei konkaven Spiegeln}
  \label{fig:plot1}
\end{figure}

  \subsubsection{Ein konkaver und ein planarer Spiegel}

In Tabelle \ref{tab:tabe2} sind die Messwerte der Anordnung mit einem planaren und einem
konkaven Spiegel mit $r_i = \SI{1400}{\milli\meter}$ angegeben.
Aus Gleichungen \ref{eqn:stab1}
und \ref{eqn:stab2} ergibt sich die Form der Stabilitätskurve zu einer Geraden, der
Fit wird also mit einer Funktion der Form
\begin{equation}
  I(d)= I_0\Bigl(1-\frac{d}{r}\Bigr)
\end{equation}
durchgeführt, wobei $r$ und $I_0$ die Fitparameter sind. Für diese ergibt sich hierdurch
\begin{align*}
  I_0 &= \SI{5.16(4)}{\micro\ampere} \\
  r &=\SI{1180(50)}{\milli\meter} \: .
\end{align*}
Die entsprechende Kurve ist zusammen mit den Messwerten in Abbildung \ref{fig:plot2}
dargestellt.

\begin{minipage}{0.5\textwidth}
\begin{table}[H]
  \centering
  \caption{Messwerte der Wärmepumpe}
  \label{tab:tabe1}
    \begin{tabular}{S S S S S S}
    \toprule
    $ t  \: / \si{\second} $ & $ p_a \: / \si{\bar} $ & $ p_b \: / \si{\bar} $ &
    $ T_1 \: / \si{\kelvin} $ & $ T_2 \: / \si{\kelvin} $ & $ P \: / \: \si{\watt} $\\
    \midrule
    0 & 5.0 & 5.0 & 293.65 & 293.65 & 0 \\
    60 & 4.7 & 6.0 & 294.15 & 293.55 & 115 \\
    120 & 4.4 & 6.4 & 295.15 & 293.15 & 118 \\
    180 & 4.5 & 6.9 & 296.35 & 291.95 & 122 \\
    240 & 4.6 & 7.0 & 297.55 & 290.95 & 125 \\
    300 & 4.6 & 7.0 & 298.85 & 289.95 & 125 \\
    360 & 4.5 & 7.2 & 300.05 & 289.15 & 123 \\
    420 & 4.4 & 7.4 & 301.15 & 288.45 & 123 \\
    480 & 4.3 & 7.8 & 302.35 & 287.65 & 122 \\
    540 & 4.2 & 8.0 & 303.55 & 286.95 & 122 \\
    600 & 4.2 & 8.1 & 304.65 & 286.25 & 121 \\
    660 & 4.1 & 8.3 & 305.75 & 285.55 & 121 \\
    720 & 4.0 & 8.5 & 306.75 & 284.95 & 121 \\
    780 & 4.0 & 8.8 & 307.75 & 284.35 & 121 \\
    840 & 3.9 & 9.0 & 308.75 & 283.75 & 121 \\
    900 & 3.8 & 9.1 & 309.65 & 283.15 & 121 \\
    960 & 3.8 & 9.2 & 310.55 & 282.55 & 122 \\
    1020 & 3.8 & 9.5 & 311.45 & 282.05 & 122 \\
    1080 & 3.7 & 9.8 & 312.25 & 281.55 & 122 \\
    1140 & 3.7 & 10.0 & 313.05 & 281.15 & 122 \\
    1200 & 3.7 & 10.0 & 313.9 & 280.65 & 122 \\
    1260 & 3.6 & 10.2 & 314.65 & 280.25 & 123 \\
    1320 & 3.6 & 10.3 & 315.35 & 279.85 & 123 \\
    1380 & 3.6 & 10.6 & 316.15 & 279.45 & 124 \\
    1440 & 3.6 & 10.8 & 316.85 & 279.15 & 124 \\
    1500 & 3.6 & 11.0 & 317.55 & 278.75 & 124 \\
    1560 & 3.6 & 11.1 & 318.25 & 278.55 & 124 \\
    1620 & 3.6 & 11.2 & 318.95 & 278.25 & 125 \\
    1680 & 3.5 & 11.4 & 319.55 & 277.95 & 125 \\
    1740 & 3.5 & 11.5 & 320.15 & 277.65 & 125 \\
    1800 & 3.5 & 11.7 & 320.75 & 277.45 & 125 \\
    1860 & 3.5 & 11.9 & 321.35 & 277.25 & 125 \\
    1920 & 3.5 & 12.0 & 321.95 & 277.05 & 125 \\
    1980 & 3.5 & 12.1 & 322.45 & 276.95 & 125 \\








      \bottomrule
    \end{tabular}
\end{table}

\end{minipage}
\begin{minipage}{0.5\textwidth}
\begin{table}[H]
  \centering
  \caption{Wertetabelle für $\alpha$ und $C_V$.}
  \label{tab:tab2}
    \begin{tabular}{S S S S S}
    \toprule
    $ T\: \text{in}\: \si{\K} $ & $ {\alpha \cdot 10^{-6} \: \text{in}\: \si {\per\K}} $ &
    $ C_V \: \text{in}\: \si{\J\per\K\mol} $\\
    \midrule %Cv, a *10-6, Cv
    %0 & 1 & 1\\
    88.60\pm0.24 & 9.56\pm0.06 & 14.17\pm8.13  \\ %&3.6 & 318.97\pm0.85\\
    93.81\pm0.24 & 10.10\pm0.06 & 17.58\pm10.03 \\ %& 4.7 & 440.90\pm1.11\\
    99.74\pm0.24 & 10.66\pm0.05 & 15.52\pm8.84 \\ %& 5.1 & 508.68\pm1.21\\
    104.74\pm0.24 & 11.07\pm0.05 & 18.44\pm10.52 \\ %& 4.6 & 481.79\pm1.09\\
    110.94\pm0.24 &  11.54\pm0.05 & 14.86\pm8.45 \\ %& 5.3 & 587.97\pm1.27\\
    115.96\pm0.24 & 11.89\pm0.05 & 18.49\pm10.52 \\ %& 4.6 & 533.41\pm1.10\\
    121.47\pm0.24 &  12.22\pm0.05 & 16.83\pm9.57 \\ %& 4.9 & 595.21\pm1.17\\
    126.99\pm0.24 & 12.53\pm0.04 & 16.79\pm9.54 \\ %& 4.9 & 622.29\pm1.18\\
    131.58\pm0.24 & 12.77\pm0.04 & 20.42\pm11.62 \\ %& 4.2 & 552.62\pm1.01\\
    136.65\pm0.24 & 13.02\pm0.04 & 18.40\pm10.47 \\ %& 4.6 & 628.57\pm1.11\\
    141.49\pm0.24 & 13.24\pm0.04 & 19.28\pm10.97 \\ %& 4.4 & 622.54\pm1.07\\
    146.34\pm0.24 & 13.44\pm0.04 & 19.24\pm10.95 \\ %& 4.4 & 643.88\pm1.07\\
    150.95\pm0.24 & 13.62\pm0.04 & 20.22\pm11.52 \\ %& 4.3 & 649.11\pm1.05\\
    155.34\pm0.24 & 13.79\pm0.04 & 21.31\pm12.14 \\ %& 4.1 & 636.88\pm0.98\\
    159.97\pm0.24 & 13.95\pm0.04 & 20.12\pm11.47 \\ %& 4.3 & 687.89\pm1.05\\
    164.62\pm0.24 & 14.10\pm0.04 & 20.18\pm11.51 \\ %& 4.3 & 707.87\pm1.06\\
    168.79\pm0.25 & 14.23\pm0.04 & 22.54\pm12.86 \\ %& 3.9 & 658.27\pm0.95\\
    173.45\pm0.25 &  14.37\pm0.04 & 20.08\pm11.46 \\ %& 4.3 & 745.84\pm1.06\\
    178.13\pm0.25 &  14.50\pm0.04 & 20.04\pm11.44 \\ %& 4.3 & 765.94\pm1.06\\
    182.56\pm0.25 &  14.62\pm0.04 & 21.11\pm12.06\\
    192.70\pm0.25 &  14.87\pm0.04 & 18.41\pm10.47\\
    200.15\pm0.25 &  15.04\pm0.04 & 25.19\pm14.28\\
    208.87\pm0.25 &  15.23\pm0.04 & 21.43\pm12.18\\
    217.12\pm0.25 &  15.38\pm0.04 & 22.65\pm12.88\\
    225.15\pm0.25 &  15.53\pm0.03 & 23.27\pm13.24\\
    232.70\pm0.25 &  15.70\pm0.03 & 24.75\pm14.08\\
    240.53\pm0.25 &  15.74\pm0.03 & 23.84\pm13.58\\
    248.39\pm0.25 &  15.89\pm0.03 & 23.74\pm13.53& \\
    256.01\pm0.25 &  15.97\pm0.03 & 24.46\pm13.94 \\
    263.41\pm0.26 &  16.01\pm0.03 & 25.22\pm14.38 \\
    271.08\pm0.26 &  16.18\pm0.03 & 24.26\pm13.86 \\
    278.52\pm0.26 &  16.27\pm0.03 & 25.03\pm14.29&\\
    285.98\pm0.26 &  16.35\pm0.03 & 24.92\pm14.25 \\
    293.21\pm0.26 &  16.42\pm0.03 & 25.74\pm14.72 \\
    300.98\pm0.26 &  16.50\pm0.03 & 23.87\pm13.68 \\
    308.51\pm0.26 &  16.57\pm0.03 & 24.63\pm14.12\\



      \bottomrule
    \end{tabular}
\end{table}

\end{minipage}
\begin{figure}
  \centering
  \includegraphics[height=9cm]{Plot2.pdf}
  \caption{Untersuchung der Stabilität mit einem konkaven und einem planaren Spiegel}
  \label{fig:plot2}
\end{figure}



\subsection{Vermessen der ersten beiden transversalen Moden}
Die Messwerte der $\text{T}_{00}$ Mode sind in Tabelle \ref{tab:tabe3} angegeben.
\begin{table}
  \centering
  \caption{Messwerte für den ersten Doppelspalt.}
   \begin{tabular}{S S| S S | S S}
    \toprule
    $x/\; \si{\mm}$& $A/\;\si{\nA}$ &
    $x/\; \si{\mm}$& $A/\;\si{\nA}$ &
    $x/\; \si{\mm}$& $A/\;\si{\nA}$ \\
    \midrule

    15.0& 4.6& 23.0& 25.0& 29.5& 6.0\\
    15.5& 4.2& 23.5& 30.0& 30.0& 5.3\\
    16.0& 4.0& 24.0& 35.0& 30.5& 4.9\\
    16.5& 4.0& 24.25& 36.0& 31.0& 4.7\\
    17.0& 4.4& 24.5& 37.0& 31.5& 4.4\\
    17.5& 5.5& 24.75& 38.0& 32.0& 4.2\\
    18.0& 6.6& 25.00& 37.0& 32.5& 3.8\\
    18.5& 7.7& 25.25& 36.0& 33.0& 3.6\\
    19.0& 8.2& 25.5& 36.0& 33.5& 3.2\\
    19.5& 8.4& 26.0& 33.0& 34.0& 3.2\\
    20.0& 8.4& 26.5& 28.5& 34.5& 3.2\\
    20.25& 8.4& 27.0& 23.0& 35.0& 3.3\\
    20.5& 8,7& 27.5& 18.0& 35.5& 3.4\\
    21.0& 9.8& 28.0& 13.5& 36.0& 3.5\\
    21.5& 12.0& 28.5& 10.0\\
    22.0& 15.0& 29.0& 7.8\\
    22.5& 20.0& 29.25& 6.7\\


   \bottomrule
  \end{tabular}
  \label{tab:tabelle3}
\end{table}

Diese werden
mit einer Gaußkurve der Form
\begin{equation}
  \symup{I}(x) =I_0\cdot \symup{e}^{-\Bigl(\frac{(x-x_0)^2}{c^2}\Bigr)}
  \label{eqn:gausk}
\end{equation}
gefittet, wodurch sich für die Fitparameter
\begin{align*}
  I_0 &= \SI{688(5)}{\micro\ampere} \\
  x_0 &= \SI{21.84(5)}{\milli\meter} \\
  c &= \SI{9.58(8)}{\milli\meter}
\end{align*}
ergeben. Die entsprechende Kurve ist zusammen mit den Messwerten in Abbildung \ref{fig:plot3}
dargestellt.
\begin{figure}
  \centering
  \includegraphics[height=9cm]{Plot3.pdf}
  \caption{Ausgleichskurve und Messwerte der $\text{T}_{00}$ Mode}
  \label{fig:plot3}
\end{figure}
Für die $\text{T}_{01}$ Mode ist die theoretische Kurve der Intensität entlang der x-Achse nicht mehr
durch eine Gaußkurve gegeben, sondern durch eine Funktion der Form
\begin{equation}
  \symup{I}(x) =I_0\cdot(x-x_0)^2 \cdot \symup{e}^{-2\Bigl(\frac{(x-x_0)^2}{c^2}\Bigr)} \: .
  \label{eqn:gausk}
\end{equation}
Hierbei ergeben sich die Parameter
\begin{align*}
  I_0 = \SI{3.8(7)}{\micro\ampere} \\
  x_0 = \SI{24.3(6)}{\milli\meter} \\
  c = \SI{13.4(8)}{\milli\meter} \: .
\end{align*}
Die Messwerte sind in Tabelle \ref{tab:tabe4} angegeben und in Abbildung \ref{fig:plot4}
zusammen mit der Ausgleichskurve graphisch dargestellt.
\begin{table}[H]
  \centering
   \begin{tabular}{c c c c}
    \toprule
    Nummer der Oberwelle & $ U_{\text Theorie,Rechteck}\: / \si{\volt} $ &
    $ U_{\text Theorie,Dreick}\: / \si{\volt} $ & $ U_{\text Theorie,Sägezahn}\: / \si{\volt} $ \\
    \midrule
    1 & 1145 & 182 & 573 \\
    2 & 0 & 0 & 286 \\
    3 & 573 & 20 & 191 \\
    4 & 0 & 0 & 143 \\
    5 & 229 & 7 & 115 \\
    6 & 0 & 0 & 96 \\
    7 & 164 & 4 & 82 \\
    8 & 0 & 0 & 72 \\
    9 & 127 & 2 & 64 \\
    10 & 0 & 0 & 57 \\
    \bottomrule
  \end{tabular}
  \caption{Eingestellte Schwingungsamplituden.}
  \label{tab:tabe4}
\end{table}

\begin{figure}[H]
  \centering
  \includegraphics[height=9cm]{Plot4.pdf}
  \caption{Ausgleichskurve und Messwerte der $\text{T}_{01}$Mode}
  \label{fig:plot4}
\end{figure}

\subsection{Messung der Polarisation}
Die Messwerte zur Untersuchung der Polarisation sind in Tabelle \ref{tab:tabe5} angegeben.
\begin{table}[H]
  \centering
  \caption{Bohrung 1 und 2, Vergleich der Sonden mit 1\;MHz und 2\;MHz.}
  \label{tab:tab5}
    \begin{tabular}{c c c c}
    \toprule
    Bohrung & $S_{\text{2\;MHz}}$/\;mm & $S_{\text{ 1\;MHz}}$/\;mm\\
    \midrule
    1 & 1,82 & 2,12\\
    2 & 1,83 & 1,97\\
    \bottomrule
    \end{tabular}
  \end{table}

Theoretisch ergibt sich bei vollständig polarisiertem Licht eine Verteilung gemäß
\begin{equation}
  I(\theta) = I_0 \cdot \text{cos}^2(\theta-\theta_0) \:,
\end{equation}
sodass die Messwerte mit dieser Funktion gefittet werden. Dabei ergeben sich die Parameter zu
\begin{align*}
  I_0= \SI{99.1(16)}{\micro\ampere} \\
  \theta_0 = 20,036° \pm 0,014°
\end{align*}
Die daraus resultierende Funktion ist zusammen mit den Messwerten in Abbildung \ref{fig:plot4}
angegeben.
\begin{figure}
  \centering
  \includegraphics[height=9cm]{Plot5.pdf}
  \caption{Messwerte und Ausgleichsrechnung zur Untersuchung der Polarisation}
  \label{fig:plot5}
\end{figure}

\subsection{Bestimmung der Wellenlänge}
Für die Bestimmung der Wellenlänge wurde einmal ein Gitter mit 100 Linien pro Millimeter
verwendet und eines mit 80 Linien pro Millimeter.
Aus der Bedingung für konstruktive Interferenz am Gitter
\begin{equation}
  \increment s = k \cdot \lambda
\end{equation}
mit dem Gangunterschied $\increment s$ lässt sich die Formel
\begin{equation}
  \lambda = \frac{b \cdot d_n}{n \cdot \sqrt{a^2+d_n^2}}
\end{equation}
herleiten, wobei $n$ die Ordnung des Maximums bezeichnet, $d_n$ den Abstand des
Maximums zur optischen Achse, $a$ den Abstand zwischen Schirm und Gitter und $b$ die
Gitterkonstante. Der Abstand $a$ beträgt bei dieser Messung $\SI{160(2)}{\centi\meter}$.
Bei der Berechnung der Wellenlänge muss zudem auf die Gaußsche Fehlerfortpflanzung
\begin{equation}
  \increment f = \sqrt{ \sum_{i=1}^N \left( \frac{\partial f}{\partial x_i}\right)^2
  \cdot (\increment x_i)^2  }
  \label{eqn:gaus}
\end{equation}
geachtet werden, da die gemessenen Werte mit einer Ableseunsicherheit behaftet sind.
In diesem Fall ergibt sich der Fehler also über
\begin{equation}
  \increment \lambda = \sqrt{\frac{b^2 \cdot d^2_n}{n^2 \cdot (a^2+d_n^2)^3}\cdot (\increment a)^2
  + \frac{b^2 \cdot a^4}{n^2 \cdot (a^2+d_n^2)^3}\cdot (\increment d_n)^2 }
\end{equation}
Die gemessenen Werte lauten zusammen mit der jeweiligen Ableseunsicherheit und den resultierenden
Wellenlängen:
\begin{table}[H]
  \centering
  \caption{Werte der Anpassungsschicht}
  \label{tab:tabe6}
    \begin{tabular}{S S S }
    \toprule
    $ \text{Zylinder} $ & $ \increment t [\mu\text{s}] $ &
    $ l_a \text{[mm]}$\\
    \midrule
    1 & 0.54 & 0.81 \\
    2 & 0.40 & 0.59 \\
    3 & 0.76 & 1.12 \\
    \text{1+2} & 0.49 & 0.73 \\
    4 & 0.70 & 1.03 \\
    \text{1+3} & 0.90 & 1.33 \\
    5 & 1.25 & 1.85 \\
    \text{1+4} & 0.69 & 1.02 \\
    6 & 0.44 & 0.66 \\

          \bottomrule
    \end{tabular}
  \end{table}

Durch die Gleichung
\begin{equation}
  \bar{x} = \frac{1}{N} \sum_{i=1}^{N} x_i \: \:
  \label{eqn:mit}
\end{equation}
\noindent lässt sich der Mittelwert bilden, wobei der dazugehörige Fehler sich durch
\begin{equation}
  \increment \bar{x} = \frac{1}{\sqrt{N}} \sqrt{ \frac{1}{N-1} \sum_{i=1}^N
  (x_i - \bar{x})^2}
  \label{eqn:mitf}
\end{equation}
ergibt. Der somit gemittelte Wert beträgt $\SI{634.0(9)}{\nano\meter}$.
Es ergibt sich jedoch auch ein Fehler durch die Gaußsche Fehlerfortpflanzung gemäß Gleichung
\ref{eqn:gaus} über
\begin{equation}
  \increment \lambda = \sqrt{\frac{1}{N^2} \sum_{i=1}^{N} \increment \lambda_i} \:,
\end{equation}
welcher mit $\pm 7$ deutlich größer und somit der signifikante Fehler ist. Der experimentelle Wert der
Wellenlänge beträgt somit $\SI{634(7)}{\nano\meter}$.
