\section{Auswertung}
\label{sec:Auswertung}



\subsection{Überprüfung der Stabilitätsbedingung}

\subsubsection{Zwei konkave Spiegel}
Die gemessenen Werte für die Anordnung mit zwei konkaven Spiegeln mit einem
Krümmungsradius von jeweils $r_i = \SI{1400}{\milli\meter}$ sind in Tabelle \ref{tab:tabe1}
angegeben. Der theoretische Verlauf entspricht nach Gleichungen \ref{eqn:stab1}
und \ref{stab2} einer Parabel,
sodass die Werte an eine Funktion der Form
\begin{equation}
  I(d)= I_0(\frac{d^2}{r^2}-\frac{2d}{r}+1)
\end{equation}
angepasst werden. Es werden jedoch nur die Messwerte bis $\SI{118}{\centi\meter}$
bei dem Fit berücksichtigt, da bei höheren Distanzen offensichtlich ein systematischer
Fehler in der Messung vorliegt.
Es ergibt sich hierdurch
\begin{align*}
  I_0= \SI{570(50)}{\micro\ampere} \\
  r=\SI{998(11)}{\milli\meter}
\end{align*}
für die Parameter. Die Messwerte sind zusammen mit der
entsprechenden Kurve in Abbildung \ref{fig:plot1} dargestellt.
\begin{figure}
  \centering
  \includegraphics[height=9cm]{Plot1.pdf}
  \caption{Untersuchung der Stabilität mit zwei konkaven Spiegeln}
  \label{fig:plot3}
\end{figure}

  \subsubsection{Ein konkaver und ein planarer Spiegel}

In Tabelle \ref{tab:tabe2} sind die Messwerte der Anordnung mit einem planaren und einem
konkaven Spiegel mit $r_i = \SI{1400}{\milli\meter}$ angegeben.
Aus Gleichungen \ref{eqn:stab1}
und \ref{stab2} ergibt sich die Form der Stabilitätskurve zu einer Geraden, der
Fit wird also mit einer Funktion der Form
\begin{equation}
  I(d)= I_0(1-\frac{d}{r})
\end{equation}
durchgeführt, wobei $r$ und $I_0$ die Fitparameter sind. Für diese ergibt sich hierdurch
\begin{align*}
  I_0= \SI{5.16(4)}{\micro\ampere} \\
  r=\SI{1180(50)}{\milli\meter} \: .
\end{align*}


\begin{minipage}{0.5\textwidth}
\begin{table}[H]
  \centering
  \caption{Messwerte und Ergebniss der Bestimmung der Schallgeschwindigkeit}
  \label{tab:tabe1}
    \begin{tabular}{S||S S||S S||S|S}
    \toprule
    $ \text{Länge l des Zylinders [mm]} $ & $ U_{1} [\text{V}] $ &
    $ t_{1} [\mu\text{s}] $ & $ U_{2} [\text{V}] $ &
    $ t_{2} [\mu\text{s}] $ & $ \increment t [\mu\text{s}]$ &
    $ \text{c} [\text{m}/\text{s}]$\\
    \midrule
    31.0 & 1.335 \: & 24.0 & 1.096 \:  & 46.9 & 22.9 & 2707.42 \\
          \bottomrule
    \end{tabular}
  \end{table}

\end{minipage}
\begin{minipage}{0.2\textwidth}
\begin{table}[H]
  \centering
  \caption{Zählrate und Energiemaximum bei variiertem Druck, Abstand a=2cm}
  \label{tab:tab2}
    \begin{tabular}{c c c c c}
    \toprule
    Druck $\rho$/\;mbar & Energiemaximum & Zählrate $N$ & Energie $E_{\alpha}$ & effektive Länge $x$/\;cm\\
    \midrule
    0 & 796 &131382  &4          & 0.0   \\
    50 & 775 &131464 &3.89 & 0.09 \\
    100 &756 &130732 &3.79 & 0.19\\
    150 &749 &129617 &3.76  &  0.29\\
    200 &749 &130444 &3.76  & 0.39\\
    250 &727 &129600 &3.65 & 0.49\\
    300 &722 &128936 &3.63 & 0.59\\
    350 &708 &128478 &3.56 & 0.69\\
    400 &696 &128122 &3.49 & 0.79\\
    450 &687 &127415 &3.45 & 0.89\\
    500 &674 &126608 &3.39 & 0.99\\
    550 &663 &126372 &3.33 &1.09\\
    600 &651 &124989 &3.27 & 1.18\\
    650 &634 &124942 &3.19 & 1.28\\
    700 &618 &124295 &3.11 &1.38\\
    750 &602 &123299 &3.03 & 1.48\\
    800 &584 &119958 &2.93 &1.58\\
    850 &566 &120673 &2.84 &1.68\\
    900 &548 &117907 &2.75 & 1.78\\
    950 &534 &116111 &2.68&   1.88\\
    1000 &499& 108630&2.51 & 1.07\\
    \bottomrule
    \end{tabular}
  \end{table}

\end{minipage}


\subsection{Vermessen der ersten beiden transversalen Moden}
Die Messwerte der $\text{T}_{00}$ sind in Tabelle \ref{tab:tabe3} angegeben. Diese werden
mit einer Gaußkurve der Form
\begin{equation}
  \symup{I}(x) =I_0\cdot \symup{e}^{-(\frac{(x-x_0)^2}{c^2})}
  \label{eqn:gausk}
\end{equation}
gefittet, wodurch sich für die Fitparameter
\begin{align*}
  I_0 = \SI{688(5)}{\micro\ampere}
  x_0 = \SI{21.84(5)}{\milli\meter}
  c = \SI{9.58(8)}{\milli\meter}
\end{align*}
ergeben. Die entsprechende Kurve ist zusammen mit den Messwerten in Abbildung \ref{fig:plot3}
dargestellt.
\begin{figure}
  \centering
  \includegraphics[height=9cm]{Plot3.pdf}
  \caption{Ausgleichskurve und Messwerte der $\text{T}_{00}$Mode}
  \label{fig:plot3}
\end{figure}
Für die $\text{T}_{01}$ ist die theoretische Kurve der Intensität entlang der x-Achse nicht mehr
durch eine Gaußkurve gegeben, sondern durch eine Funktion der Form
\begin{equation}
  \symup{I}(x) =I_0\cdot(x-x_0)^2 \cdot \symup{e}^{-2(\frac{(x-x_0)^2}{c^2})} \: .
  \label{eqn:gausk}
\end{equation}
Hierbei ergeben sich die Parameter
\begin{align*}
  I_0 = \SI{3.8(7)}{\micro\ampere} \\
  x_0 = \SI{24.3(6)}{\milli\meter} \\
  c = \SI{13.4(8)}{\milli\meter} \: .
\end{align*}
Die Ausgleichskurve und die Messwerte sind in Abbildung \ref{fig:plot4} graphisch dargestellt.
\begin{figure}
  \centering
  \includegraphics[height=9cm]{Plot3.pdf}
  \caption{Ausgleichskurve und Messwerte der $\text{T}_{01}$Mode}
  \label{fig:plot4}
\end{figure}

\subsection{Messung der Polarisation}
Die Messwerte zur Untersuchung der Polarisation sind in Tabelle \ref{tab:tabe5} angegeben.
Theoretisch ergibt sich bei vollständig polarisiertem Licht eine Verteilung gemäß
\begin{equation}
  I(\theta) = I_0 \cdot \text{cos}^2(\theta-\theta_0) \:,
\end{equation}
sodass die Messwerte mit dieser Funktion gefittet werden. Dabei ergeben sich die Parameter zu
\begin{align*}
  I_0= \SI{99.1(16)}{\micro\ampere}
  \theta_0 = 20,036° \pm 0,014°
\end{align*}
Die daraus resultierende Funktion ist zusammen mit den Messwerten in Abbildung \ref{fig:plot4}
angegeben.
\begin{figure}
  \centering
  \includegraphics[height=9cm]{Plot4.pdf}
  \caption{Messwerte und Ausgleichsrechnung zur Untersuchung der Polarisation}
  \label{fig:plot4}
\end{figure}

\subsection{Bestimmung der Wellenlänge}
Für die Bestimmung der Wellenlänge wurde einmal ein Gitter mit 100 Linien pro Millimeter
verwendet und eines mit 80 Linien pro Millimeter.
Aus der Bedingung für konstruktive Interferenz am Gitter
\begin{equation}
  \increment s = k \cdot \lambda
\end{equation}
mit dem Gangunterschied $\increment s$ lässt sich die Formel
\begin{equation}
  \lambda = \frac{b \cdot d_i}{i \cdot \sqrt{a^2+d_i^2}}
\end{equation}
herleiten, wobei $i$ die Ordnung des Maximums bezeichnet, $d_i$ den Abstand des
Maximums zur optischen Achse, $a$ den Abstand zwischen Schirm und Gitter und $b$ die
Gitterkonstante. Der Abstand $a$ beträgt bei dieser Messung $\SI{160(2)}{\centi\meter}$.
Bei der Berechnung der Wellenlänge muss zudem auf die Gaußsche Fehlerfortpflanzung
\begin{equation}
  \increment f = \sqrt{ \sum_{i=1}^N \left( \frac{\partial f}{\partial x_i}\right)^2
  \cdot (\increment x_i)^2  }
  \label{eqn:gaus}
\end{equation}
geachtet werden, da die gemessenen Werte mit einer Ableseunsicherheit behaftet sind.
In diesem Fall ergibt sich der Fehler also über
\begin{equation}
  \increment \lambda =
\end{equation}
Die gemessenen Werte lauten zusammen mit der jeweiligen Ableseunsicherheit und den resultierenden
Wellenlängen:
\begin{table}[H]
  \centering
  \caption{Messwerte des Absorptionsspektrums von Zirkonium}
  \label{tab:tabe6}
    \begin{tabular}{S S}
    \toprule
    $ \text{Winkel} / ° $ & $ \text{Impulse pro s}$\\
    \midrule
    18.0 & 59.0 \\
    18.2 & 59.0 \\
    18.4 & 58.0 \\
    18.6 & 56.0 \\
    18.8 & 55.0 \\
    19.0 & 57.0 \\
    19.2 & 70.0 \\
    19.4 & 88.0 \\
    19.6 & 105.0 \\
    19.8 & 105.0 \\
    20.0 & 112.0 \\
    20.2 & 115.0 \\
    20.4 & 113.0 \\
    20.6 & 116.0 \\
    20.8 & 116.0 \\
    21.0 & 114.0 \\

          \bottomrule
        \end{tabular}
    \end{table}

Durch die Gleichung
\begin{equation}
  \bar{x} = \frac{1}{N} \sum_{i=1}^{N} x_i \: \:
  \label{eqn:mit}
\end{equation}
\noindent lässt sich der Mittelwert bilden, wobei der dazugehörige Fehler sich durch
\begin{equation}
  \increment \bar{x} = \frac{1}{\sqrt{N}} \sqrt{ \frac{1}{N-1} \sum_{i=1}^N
  (x_i - \bar{x})^2}
  \label{eqn:mitf}
\end{equation}
ergibt. Der Fehler durch die Gaußsche Fehlerfortpflanzung ergibt sich gemäß Gleichung
\ref{eqn:gaus} durch
\begin{equation}
  \increment \lambda =
\end{equation}
