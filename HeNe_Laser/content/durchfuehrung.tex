\section{Durchführung}
\label{sec:Durchführung}
Der Versuchsaufbau setzt sich aus dem Laserrohr mit einer Länge von $l=\SI{408}{\mm}$ und einem
Durchmesser von $d=\SI{1.1}{\mm}$, zwei Resonatorspiegeln, einem Justierlaser
und einer Photodiode zusammen. Außerdem stehen für die einzelnen Durchführungsteile verschiedene
optische Elemente zur Verfügung. Alle Elemente befinden sich auf einer optischen Bank und lassen
sich verschieben. Um eine Besetzungsinversion zu erzeugen wird an die Elektroden des Laserrohres ein
Strom von $\SI{6.5}{nA}$ angelegt.

Um die Stabilitätsmessung durchzuführen werden die Spiegel zunächst mit Hilfe des Justierlasers
justiert. Bei allen Messungen wird als auskoppelder Spiegel ein konkaver Spiegel mit einem Krümmungsradius
von $\SI{1400}{\mm}$ und einem Transmissionsgrad von 1.5-1.8\% verwendet.
Für den Resonator mit zwei konkaven Spiegeln wird zusätzlich ein Spiegel mit einem Krümmungsradius von $\SI{1400}{\mm}$ verwendet,
die Resonatorlänge wird auf 70\;cm eingestellt
und in 2\;cm Schritten bis auf 1,30\;cm erhöht. Nach jedem Schritt wird auf die maximale Laserleistung nachjustiert.

Für den Resonator aus planpallelem und konkavem Spiegel wird ähnlich vorgegangen, es wird bei 60\;cm gestartet und die
Resonatorlänge in 1\;cm Schritten erhöht, bis der Laser verschwindet.\\
Um die $\text{TEM}_{00}$-Grundmode zu vermessen, wird eine Sammellinse zwischen Resonatorspiegel und Photodiode
positioniert, anschließend wird die Intensitätsverteilung gemessen, indem der Laserbereich mit der Photodiode
abgefahren wird. Um die $\text{TEM}_{01}$-Mode zu stabilisieren, wird ein Wolframdraht zwischen Laserrohr und
Resonatorspiegel platziert und justiert, bis die $\text{TEM}_{01}$-Mode stabil auf einem optischen Schirm zu sehen ist.
Die Intensitätsverteilung wird äquivalent zur Grundmode vermessen.
Um die Polarisation zu bestimmen, wird ein Polarisator in den Strahlengang des Lasers gestellt, dieser wird von
0-360° in 10° Schritten verstellt und die Intensität wird dabei an der Photodiode abgelesen.
Zur Bestimmung der Wellenlänge der Lasers werden nacheinander zwei Gitter in den Strahlengang gestellt.
An dem entstehenden Beugungsbild werden die Abstände der ersten und zweiten Nebenmaxima zum Hauptmaximum vermessen,
außerdem wird der Abstand vom Gitter zum optischen Schirm vermessen.
