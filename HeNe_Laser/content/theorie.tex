\section{Zielsetzung}
In diesem Versuch soll die Funktionsweise eines HeNe-Lasers untersucht werden. Außerdem werden einige
charakteristische Eigenschaften wie z.B. die Stabilitätsbedingung, die Polarisation und die
Wellenlänge des Lasers bestimmt.

\section{Theorie}
\label{sec:Theorie}
\subsection{Die Funktionsweise}
Jeder Laser besteht aus drei Komponenten, dem aktiven Lasermedium, der Pumpquelle und dem
Resonator. Dieser prinzipielle Aufbau ist in Abbildung \ref{fig:aufbau} zu sehen.
Mit diesen Komponenten ist es möglich monochromatisches Licht hoher Intensität
und Kohärenz zu emittieren.

\begin{figure}[H]
  \centering
  \includegraphics[width=9cm]{Aufbau3.png}
  \caption{Prinzipelle Aufbau eines Lasers \cite{Bachelor}.}
  \label{fif:aufbau}
\end{figure}

Über das Lasermedium wird das Strahlungsspektrum bestimmt, dies kann vom UV-Bereich bis hin
zum IR-Bereich reichen. Die Pumpquelle wird benötigt, um eine Besetzungsinversion und damit
induzierte Emission zu erzeugen. Der Resonator dient zur optischen Rückkoplung, wodurch das
Licht erneut duch das Lasermedium geschickt wird.

In einem Atom mit zwei Energieniiveaus, dem Grundzustand und einem angeregten Zustand gibt es
drei verschiedene Wechselwirkungen von Photonen. Ein Photon kann absorbiert werden, wenn es die
Energie des Übegangs hat. Wechselt ein Photon spontan vom angergten Zustand in den Grundzustand
handelt es sich um spontane Emission. Dieser Vorgang kann auch durch ein einfallendes Photon
stimmuliert werden, dann spricht man von stimmulierter Emission. Das ausgelöste und stimmulierte Photon
haben dabei die gleiche Energie, Phase und Ausbreitungsrichtung.

Bei einem Laser wird das Lasermaterial so manipuliert, dass die Wechselwirkung des Strahlungsfeldes
mit dem Lasermaterial zu einer Verstärkung des einfallenden Lichts führt. Dazu muss die stimmulierte
Emission der spontanen überwiegen, denn nur dann ist eine hohe Kohärenz und Verstärkung der Strahlung möglich.
Die Voraussetzung für stimmulierte Emission ist eine Besetzungsinversion, diese liegt vor, wenn der angeregte Zustand
eine höhere Besetzungszahl aufweist als der Grundzustand. Aufgrund der Boltzmann-Verteilung ist der Grundzustand im
thermischen Gleichgewicht jedoch stärker besetzt. Um trotzdem eine Besetzungsinversion zu erreichen wird dem
Lasermedium über die Pumpquelle permanent Energie von außen hinzugeführt, dies kann über über Elektronenstoß
oder optische Anregung erolgen.
Da die Verstärkung exponentiell mit dem Laufweg im Lasermedium ansteigt wird ein optischer Resonator werdendet, um den
Laufweg zu maximieren. Der Resonator besteht aus zwei Spiegeln, in dessen Mitte das Laserrohr steht, der Laserstrahl wird an
den Spiegel reflektiert und durchläuft das Lasermedium somit mehrfach. Einer der Spiegel ist teildurchlässig, um den Lasestrahl
auszukoppeln. Für den Aufbau des Resonators gibt es verschiedene Möglichkeiten, er kann aus planparallelen Spiegeln, spährischen
Spiegeln oder einer Mischung daraus bestehen. Je nach verwendeter Spiegelart spricht man von planparallelem oder spährischem
Resonator. Ziel ist es, so einen selbsterregten Oszillator zu erzeugen. Dafür müssen die Verluste durch den Resonator möglichst gering
gehalten werden, das ist bei einem konfokalen Resonator der Fall, bei dem die Spiegelbrennpunkte zusammenfallen.

Ein stabiler Resonator liegt vor, wenn die Verluste geringer sind als als die Verstärkung durch induzierte Emission.
Dies ist erfüllt, wenn die Stabilitätsbedingung erfüllt ist:
\begin{equation}
  0< g_1 \cdot g_2 <1.
\end{equation}
Dabei ist der Resonatorparameter
\begin{equation}
  g_i = 1- \frac{L}{r_i}
\end{equation}
von der Länge des Resonators $L$ und der Krümmung der Resonatorspiegel $r_i$ abhängig.

\subsection{Der HeNe-Laser}
Das Laserrohr des Helium-Neon-Lasers ist mit einem Helium-Neon Gasgemisch im Verhältnis 5:1 gefüllt,
Helium wird als Pumpquelle verwedet und Neon als Lasermedium. Das Laserrohr ist mit Elektroden versehen,
zwischen denen eine Gasentladung stattfindet. Diese Gasentladung bringt die Heliumatome durch Elektronenstoß in einen
langlebigen, metastabilen angeregten Zustand, über Stöße 2. Art wird die Energie der angeregten Heliumatome an die
Neonatome abgegeben:
\begin{equation}
  He^* + Ne \rightarrow He + Ne^* + \Delta E.
\end{equation}
Dabei sind $He^*$ und $Ne^*$ angeregte Zustände. Dieser Vorgang, welcher in Abbildung \ref{fig:entladung} zu sehen ist,
wird auch als Pumpvorgang bezeichnet.

\begin{figure}[H]
  \centering
  \includegraphics[height=7cm]{Entladung.png}
  \caption{Niveauschema des HeNe-Lasers \cite{Springer2}.}
  \label{fig:entladung}
\end{figure}

Die vom HeNe-Laser bekannte rote Laserlinie bei 633\;nm wird durch den Übergang $3s_2 \rightarrow 2p_4$
erzeugt.
