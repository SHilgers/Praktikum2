\section{Diskussion}
\label{sec:Diskussion}
Die relativen Abweichungen der experimentell bestimmten Werten von den theoretischen lässt sich über
die Formel
\begin{equation}
  \frac{\lvert \text{Wert}_{\text{Theorie}}-\text{Wert}_{\text{Messung}}\rvert}{\text{Wert}_{\text{Theorie}}}
  \label{eqn:abw}
\end{equation}
berechnen.
Für den Krümmungsradius $r$ des konkaven Spiegels liefert der Fit des ersten Teils der Auswertung
$r=\SI{998(11)}{\milli\meter}$ für die Anordnung mit 2 konkaven Spiegeln,
bzw. $r=\SI{1180(50)}{\milli\meter}$ für die Anordnung mit einem planaren.
Dies entspricht Abweichungen von 28,71\% bzw. 15,71\% vom Theoriewert $r=\SI{1400}{\milli\meter}$. Im ersten Fall
entspricht dies ca. 37 Fehlerintervallen und im zweiten Fall etwa 4, es ist also in jedem Fall ein systematischer
Fehler in der Messung vorhanden. Dies liegt daran, dass die Nachjustierung nur sehr schwer möglich war und somit zu
einem großen Verrauschen der Messwerte führt. Auch das korrekte Positionieren des Photometers war nicht immer exakt möglich.
\\
Die $\text{T}_{00}$ Mode und die $\text{T}_{01}$ Mode entsprechen in ihrer
Intensitätsverteilung recht gut dem theoretisch erwarteten Verlauf, insbesondere die $\text{T}_{00}$ Mode lässt sich sehr gut
mit der Gaußfunktion fitten. Die resultierenden Parameter sind dementsprechend nur mit einem geringen relativen Fehler versehen.
Bei der $\text{T}_{01}$ ist lediglich eine Asymmetrie bezüglich des Knotens zu erkennen, die vermutlich aus einer nicht ganz exakten
Positionierung des Wolframdrahtes herrührt.
\\
Auch der Verlauf der Polarisationsmessung folgt abgesehen von einigen leicht abweichenden Werten dem theoretischen Verlauf. Die realtiven Fehler
der Fitparameter betragen bei $I_0$ nur 1,61\% und bei $\theta_0$ lediglich 0,07\%.
\\
Bei der Messung der Wellenlänge weicht der experimentell erhaltene Wert von $\SI{634(7)}{\nano\meter}$ um 0,16\% vom Theoriewert $\lambda=\SI{633}{\nano\meter}$ ab, welcher
innerhalb eines Fehlerintervalls liegt. Diese Messung scheint somit sehr exakt zu sein.
