\section{Diskussion}
\label{sec:Diskussion}
Die errechnete Schallgeschwindigkeit von $ \SI{2707.42}{\meter\per\second}$ weißt eine
relative Abweichung von 0,83 \% im Vergleich zum Theoriewert von $ \SI{2730}{\meter\per\second}$
aus der Vorbereitung auf, was für eine recht genaue Messung spricht. Diese Abweichung wurde
durch die Formel
\begin{equation*}
  \frac{\lvert \text{Wert}_{\text{Theorie}}-\text{Wert}_{\text{Messung}}\rvert}{\text{Wert}_{\text{Theorie}}}
  \label{eqn:abw}
\end{equation*}
berechnet.\\
Die Ergebnisse der beiden verschiedenen Schallgeschwindigkeitsbestimmungen befinden sich
zusammen mit der relativen Abweichung vom Theoriewert von $\SI{2730}{\meter\per\second}$ in
Tabelle \ref{tab:tabe7}.
\begin{table}[H]
  \centering
  \caption{Werte der zweiten Messreihe für Wert 16}
  \label{tab:tabe7}
    \begin{tabular}{S S S}
    \toprule
    $ \text{R}_{2} \: / \: \si{\ohm} $ & $\text{R}_{3} \: / \: \si{\ohm} $ &
    $\text{R}_{4} \: / \: \si{\ohm} $ \\
    \midrule
    500 & 178 & 328 \\
    664 & 178 & 328 \\
    1000 & 119 & 332 \\
    \bottomrule
    \end{tabular}
\end{table}

Die Abweichungen können dadurch erklärt
werden, dass die Messung durch die Anpassungsschicht aus destilliertem Wasser leicht verfälscht wird.
Die mit dem Durchschallungsverfahren gemessenen Schallgeschwindigkeiten sind stets etwas
kleiner als die mit dem Impuls-Echo verfahren gemessenen Schallgeschwindigkeiten und weisen
einen größeren relativen Fehler auf. Dies ist dadurch zu eklären, dass bei dem Durchschallungsverfahren
der Zylinder vom Schall nur einmal durchlaufen wird, anstatt zweimal wie beim Impuls-Echo-Verfahren.
Dadurch ist die durchlaufene Strecke und die gemessene Zeit deutlich kürzer und der
systematische Fehler durch die Anpassungsschicht fällt stärker ins Gewicht.
