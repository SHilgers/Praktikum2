\section{Auswertung}
\label{sec:Auswertung}

\subsection{Acrylblock A-Scan}
Die mit der Schieblehre gemessenen Maße des Acrylblocks und der Bohrungen sind in
Tabelle \ref{tab:tab0} zu finden.
\begin{table}[H]
  \centering
  \caption{Maße des Blocks, gemessen mit der Schieblehre.}
  \label{tab:tab0}
    \begin{tabular}{c c c c }
    \toprule
    Bohrung & $s_{\text{oben}}$/\;mm & $s_{\text{unten}}$/\;mm & Größe $S$/\;mm\\
    \midrule
    1 & 19,00 & 60,00 & -\\
    2 & 17,00 & 61,00 & -\\
    3 & 61,00 & 13,00 & 6,00\\
    4& 54,00  & 22,00 & 5,00\\
    5 & 42,00 & 30,00 & 4,00\\
    6 & 39,00 & 39,00 & 3,00\\
    7& 31,00 & 47,00 & 3,00\\
    8 & 22,00 & 55,00 & 3,00\\
    9 & 15,00 & 63,00 & 3,00\\
    10 & 7,00 & 71,00 & 3,00\\
    11 & 55,00 & 15,00 & 10,00\\
    \bottomrule
    Höhe des Blocks:& $h=\SI{80,00}{\mm}$
    \end{tabular}
  \end{table}


In Tabelle \ref{tab:tab1} sind die Messwerte für die Tiefe der Störstellen
von beiden Richtungen (Block in normaler Position $s_{\text{oben}}$ und
umgedreht $s_{\text{unten}}$), sowie der Durchmesser der Störstellen
dargestellt. Hier wird die 2\;MHz Sonde verwendet.
\begin{table}[H]
  \centering
  \caption{Messwerte und Ergebniss der Bestimmung der Schallgeschwindigkeit}
  \label{tab:tabe1}
    \begin{tabular}{S||S S||S S||S|S}
    \toprule
    $ \text{Länge l des Zylinders [mm]} $ & $ U_{1} [\text{V}] $ &
    $ t_{1} [\mu\text{s}] $ & $ U_{2} [\text{V}] $ &
    $ t_{2} [\mu\text{s}] $ & $ \increment t [\mu\text{s}]$ &
    $ \text{c} [\text{m}/\text{s}]$\\
    \midrule
    31.0 & 1.335 \: & 24.0 & 1.096 \:  & 46.9 & 22.9 & 2707.42 \\
          \bottomrule
    \end{tabular}
  \end{table}

Die Höhe des Blocks wird mit $h=\SI{81,53}{\mm}$ gemessen.
Der Vergleich mit Wert der mit der Schieblehre gemessen wurde ergibt sich eine Differenz von
$\Delta h =\SI{1,53}{\mm}$. Diese Differenz ist auf die Schutzschicht der Sonden zurückzuführen,
da diese zweimal durchlaufen wird ergibt sich eine Dicke von $h_{\text{Schutz}}=\SI{0,765}{\mm}$.
Von den gemessenen Werten müssen in der Rechnung also $\SI{1,53}{\mm}$ abgezogen werden, damit die
Schutzschicht die Werte für die Lage der Bohrungen nicht verfälscht.

Die Größe der Störstellen berechnet sich nach
\begin{equation}
  S=h-s_{\text{oben}} -s_{\text{unten}},
  \label{eqn:größe}
\end{equation}
diese ist ebenfalls in Tabelle \ref{tab:tab1} zu sehen.

Die Bohrungen 1 und 2 werden zusätzlich mit der 1\;MHz Sonde vermessen, die
Ergebnisse sind in Tabelle \ref{tab:tab2} zu sehen. Hier wird die Höhe des
Blocks mit $h=\SI{83,00}{\mm}$ gemessen, die Differenz beträgt
$\Delta h =\SI{3,00}{\mm}$. Also ist die Schutzschicht bei dieser Sonde
$h_{\text{Schutz}}=\SI{1,00}{\mm}$ dick.
\begin{table}[H]
  \centering
  \caption{Zählrate und Energiemaximum bei variiertem Druck, Abstand a=2cm}
  \label{tab:tab2}
    \begin{tabular}{c c c c c}
    \toprule
    Druck $\rho$/\;mbar & Energiemaximum & Zählrate $N$ & Energie $E_{\alpha}$ & effektive Länge $x$/\;cm\\
    \midrule
    0 & 796 &131382  &4          & 0.0   \\
    50 & 775 &131464 &3.89 & 0.09 \\
    100 &756 &130732 &3.79 & 0.19\\
    150 &749 &129617 &3.76  &  0.29\\
    200 &749 &130444 &3.76  & 0.39\\
    250 &727 &129600 &3.65 & 0.49\\
    300 &722 &128936 &3.63 & 0.59\\
    350 &708 &128478 &3.56 & 0.69\\
    400 &696 &128122 &3.49 & 0.79\\
    450 &687 &127415 &3.45 & 0.89\\
    500 &674 &126608 &3.39 & 0.99\\
    550 &663 &126372 &3.33 &1.09\\
    600 &651 &124989 &3.27 & 1.18\\
    650 &634 &124942 &3.19 & 1.28\\
    700 &618 &124295 &3.11 &1.38\\
    750 &602 &123299 &3.03 & 1.48\\
    800 &584 &119958 &2.93 &1.58\\
    850 &566 &120673 &2.84 &1.68\\
    900 &548 &117907 &2.75 & 1.78\\
    950 &534 &116111 &2.68&   1.88\\
    1000 &499& 108630&2.51 & 1.07\\
    \bottomrule
    \end{tabular}
  \end{table}


Außerdem wurde die Grafik des A-Scans der 1\;MHz und der 2\;MHz Sonde abgespeichert,
sie sind in Abbildung \ref{fig:Ascan1},\ref{fig:Ascan1u} und \ref{fig:Ascan2},\ref{fig:Ascan2u} zu sehen.


\begin{figure}[H]
  \centering
  \includegraphics[width=16cm]{Ascanklein.jpg}
  \caption{A-Scan der 1. und 2. Bohrung mit der 1\;MHz Sonde.}
  \label{fig:Ascan1}
\end{figure}

\begin{figure}[H]
  \centering
  \includegraphics[width=16cm]{Ascankleinumgedreht.jpg}
  \caption{A-Scan der 1. und 2. Bohrung mit der 1\;MHz Sonde, umgedreht.}
  \label{fig:Ascan1u}
\end{figure}

\begin{figure}[H]
  \centering
  \includegraphics[width=16cm]{Ascan2MHz.jpg}
  \caption{A-Scan der 1. und 2. Bohrung mit der 2\;MHz Sonde.}
  \label{fig:Ascan2}
\end{figure}

\begin{figure}[H]
  \centering
  \includegraphics[width=16cm]{Ascanumgedreht2MHz.jpg}
  \caption{A-Scan der 1. und 2. Bohrung mit der 2\;MHz Sonde.}
  \label{fig:Ascan2u}
\end{figure}

Ein Vergleich der Abbildungen \ref{fig:Ascan1}, \ref{fig:Ascan1u} und \ref{fig:Ascan2},\ref{fig:Ascan2u} fällt auf, dass
die Auflösung der 2\;MHz Sonde höher ist als die der 1\;MHz Sonde. Um eine noch bessere
Auflösung zu erhalten muss eine Sonde mit noch höherer Frequenz verwendet werden z.B. 4\;MHz.

\subsection{Acrylblock B-Scan}
Der B-Scan des Acrylblocks liefert folgende Bilder:

\begin{figure}[H]
  \centering
  \includegraphics[height=6cm]{bscan12.jpg}
  \caption{B-Scan des Acrylblocks.}
  \label{bscan1}
\end{figure}

\begin{figure}[H]
  \centering
  \includegraphics[height=6cm]{bscan22.jpg}
  \caption{B-Scan des Acrylblocks,umgedreht.}
  \label{bscan2}
\end{figure}

Mit Hilfe der Curser-Funktion wurden sie Tiefen $s_{\text{oben}}$ und $s_{\text{unten}}$
vermessen, die Ergebnisse sind in Tabelle \ref{tab:tab3} zu sehen.
\begin{table}[H]
  \centering
   \begin{tabular}{c c c}
    \toprule
     n& $\nu$/\; 1/s & $\nu_{Wechsel}$\\
    \midrule
    0,5 & 100.01& 50,0\\
    1 & 79.93 & 79.93\\
    2 & 23.93 & 47.86\\
    \bottomrule
  \end{tabular}
  \caption{Gemessene Frequenzen der Sägezahnspannung, sowie die Daraus resultierenden Frequenzen für die
  Wechselspannung.}
  \label{tab:tab3}
\end{table}

Für die Bohrung Nr.10 kann kein Wert berechnet werden, da für diese Bohrung kein
Signal gemessen werden kann. Grund dafür ist, dass eine andere Bohrung
im Weg liegt.

Der Vergleich mit den berechneten Werten aus dem A-Scan zeigt, dass
die Ergebnisse des B-Scans deutlich größere Werte ergeben als die des A-Scans.


\subsection{Herzmodell TM-Scan}
Aus dem A-Scan ergeben sich eine Wasserhöhe von $h=\SI{31,04}{\mm}$ und der
Durchmesser des Zylinders beträgt $d=\SI{49,00}{\mm}$
Im Abbildung \ref{fig:Herz} ist der TM-Scan der Herzmodells zu sehen, es werden
10 Ausschläge in einem Zeitintervall von $t=\SI{18,39}{\s}$ gemessen. Damit ergibt sich eine
Frequenz von
\begin{equation}
  \nu_{\text{Herz}}=\SI{1,84}{\Hz}.
\end{equation}

\begin{figure}[H]
  \centering
  \includegraphics[height=6cm]{Herz.jpg}
  \caption{TM-Scan des Herzmodells.}
  \label{fig:Herz}
\end{figure}

Nach der Formel $V=\pi h r^{2}$ wird das Volumen im Ruhezustand und mit
aufgeblasener Membran berechnet
\begin{align*}
  V_{\text{Ruhe}}= \pi\cdot\SI{24,5}{\mm}^{2}\cdot\SI{31,04}{\mm}=\SI{5,8e-5}{\m^{3}}\\
  V_{\text{oben}}= \pi\cdot\SI{24,5}{\mm}^{2}\cdot\SI{24,86}{\mm}=\SI{4,6e-5}{\m^{3}}.
\end{align*}

Mit der Formel
\begin{equation}
  V_{\text{Herz}}=(V_{\text{Ruhe}}-V_{\text{oben}})\cdot\nu_{\text{Herz}}
\end{equation}
ergibt sich für das Herzvolumen
\begin{equation}
  V_{\text{Herz}}=\SI{2,21e-5}{\m^{3}\per\s},
\end{equation}
das entspricht $\SI{0,022}{\liter\per\s}$.
