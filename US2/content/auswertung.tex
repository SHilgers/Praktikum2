\section{Auswertung}
\label{sec:Auswertung}

\subsection{Acrylblock A-Scan}

Aus den in Tabelle \ref{tab:tab1} gegebenen Messwerten für die Schalllaufzeit
kann nach Formel \ref{eqn:??} die Tiefe der Störstelle $s$ berechent werden.
Die Ergebnisse sind ebenfalls in Tabelle \ref{tab:tab1} zu sehen.
Die Schallgeschwindigkeit $c$ für Acryl wird Tabelle \ref{tab:tab??}
entnommen.
\begin{table}[H]
  \centering
  \caption{Messwerte und Ergebniss der Bestimmung der Schallgeschwindigkeit}
  \label{tab:tabe1}
    \begin{tabular}{S||S S||S S||S|S}
    \toprule
    $ \text{Länge l des Zylinders [mm]} $ & $ U_{1} [\text{V}] $ &
    $ t_{1} [\mu\text{s}] $ & $ U_{2} [\text{V}] $ &
    $ t_{2} [\mu\text{s}] $ & $ \increment t [\mu\text{s}]$ &
    $ \text{c} [\text{m}/\text{s}]$\\
    \midrule
    31.0 & 1.335 \: & 24.0 & 1.096 \:  & 46.9 & 22.9 & 2707.42 \\
          \bottomrule
    \end{tabular}
  \end{table}

Für die Messwerte der zweiten Messreihe, nachdem der Zylinder umgedreht wurde,
wird äquivalent vorgegangen. Die Schalllaufzeiten und dazugehörigen Tiefen $s$
sind in Tabelle \ref{tab:tab2} dargestellt.
\begin{table}[H]
  \centering
  \caption{Zählrate und Energiemaximum bei variiertem Druck, Abstand a=2cm}
  \label{tab:tab2}
    \begin{tabular}{c c c c c}
    \toprule
    Druck $\rho$/\;mbar & Energiemaximum & Zählrate $N$ & Energie $E_{\alpha}$ & effektive Länge $x$/\;cm\\
    \midrule
    0 & 796 &131382  &4          & 0.0   \\
    50 & 775 &131464 &3.89 & 0.09 \\
    100 &756 &130732 &3.79 & 0.19\\
    150 &749 &129617 &3.76  &  0.29\\
    200 &749 &130444 &3.76  & 0.39\\
    250 &727 &129600 &3.65 & 0.49\\
    300 &722 &128936 &3.63 & 0.59\\
    350 &708 &128478 &3.56 & 0.69\\
    400 &696 &128122 &3.49 & 0.79\\
    450 &687 &127415 &3.45 & 0.89\\
    500 &674 &126608 &3.39 & 0.99\\
    550 &663 &126372 &3.33 &1.09\\
    600 &651 &124989 &3.27 & 1.18\\
    650 &634 &124942 &3.19 & 1.28\\
    700 &618 &124295 &3.11 &1.38\\
    750 &602 &123299 &3.03 & 1.48\\
    800 &584 &119958 &2.93 &1.58\\
    850 &566 &120673 &2.84 &1.68\\
    900 &548 &117907 &2.75 & 1.78\\
    950 &534 &116111 &2.68&   1.88\\
    1000 &499& 108630&2.51 & 1.07\\
    \bottomrule
    \end{tabular}
  \end{table}


Die Ergebnisse für die Tiefen der Störstelle werden nach der Formel
\begin{equation}
  \bar{x} = \frac{1}{N} \sum_{i=1}^{N} x_i \: \:
  \label{eqn:mit}
\end{equation}
gemittelt. Der dazugehörige Fehler ergibt sich durch
\begin{equation}
  \increment \bar{x} = \frac{1}{\sqrt{N}} \sqrt{ \frac{1}{N-1} \sum_{i=1}^N
  (x_i - \bar{x})^2}.
  \label{eqn:mitf}
\end{equation}

Es ergeben sich folgende Werte:
\begin{align*}
  s_1&=\SI{}{\m}\\
  s_2&=\SI{}{\m}\\
\end{align*}

AUs der Höhe des Zylinders $h=\SI{}{\m}$, die mit der Schieblehre gemessen wurde und den
berechneten Tiefen $s$ kann die Größe der Störstelle $S$ mit
\begin{equation}
  S=h-s_1 -s_2
\end{equation}
berechnet werden.
Es ergibt sich
\begin{equation}
  S=\SI{}{\m}.
\end{equation}

%bessere AUflösung?
%Laufzeitkorrektur durch Schutzschicht aus Sonde









\subsection{Acrylblock B-Scan}
Der B-Scan des Acrylblocks liefert folgende Bilder:

\begin{figure}[H]
  \centering
  \includegraphics{/path/to/figure}
  \caption{}
  \label{}
\end{figure}

Aus ihnen wird die Größe der Störstelle abgelesen:
\begin{align*}
  S&=\SI{}{\m}\\
  S&=\SI{}{\m}\\
\end{align*}

Der Vergleich mit den berechneten Werten aus dem A-Scan zeigt, dass






\subsection{Herzmodell TM-Scan}
Aus dem A-Scan ergeben sich die in Tabelle \ref{tab:tab3} zu sehen Laufzeiten
für das Echo.
\begin{table}[H]
  \centering
   \begin{tabular}{c c c}
    \toprule
     n& $\nu$/\; 1/s & $\nu_{Wechsel}$\\
    \midrule
    0,5 & 100.01& 50,0\\
    1 & 79.93 & 79.93\\
    2 & 23.93 & 47.86\\
    \bottomrule
  \end{tabular}
  \caption{Gemessene Frequenzen der Sägezahnspannung, sowie die Daraus resultierenden Frequenzen für die
  Wechselspannung.}
  \label{tab:tab3}
\end{table}

