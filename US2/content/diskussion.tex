\section{Diskussion}
\label{sec:Diskussion}

Die Ergebnisse für die Größe der Störstellen sind in Tabelle \ref{tab:tab4} zusammengefasst.
Dort ist auch die Abweichung der gemessenen Größen mit den Messwerten der
Schieblehre zu finden.
\begin{table}[H]
  \centering
   \begin{tabular}{c c c c}
    \toprule
    Nummer der Oberwelle & $ U_{\text Theorie,Rechteck}\: / \si{\volt} $ &
    $ U_{\text Theorie,Dreick}\: / \si{\volt} $ & $ U_{\text Theorie,Sägezahn}\: / \si{\volt} $ \\
    \midrule
    1 & 1145 & 182 & 573 \\
    2 & 0 & 0 & 286 \\
    3 & 573 & 20 & 191 \\
    4 & 0 & 0 & 143 \\
    5 & 229 & 7 & 115 \\
    6 & 0 & 0 & 96 \\
    7 & 164 & 4 & 82 \\
    8 & 0 & 0 & 72 \\
    9 & 127 & 2 & 64 \\
    10 & 0 & 0 & 57 \\
    \bottomrule
  \end{tabular}
  \caption{Eingestellte Schwingungsamplituden.}
  \label{tab:tabe4}
\end{table}


Es ist zu erkennen, dass der A-Scan eine geringere Abweichung aufweißt als der
B-Scan.
Die Bohrungen wurden sowohl mit einer 1\;MHz Sonde als auch einer 2\;MHz Sonde
ausgemessen, die Ergebnisse stehen in Tabelle \ref{tab:tab5}.
\begin{table}[H]
  \centering
  \caption{Mechanischen Kompressorleistung zu den Zeiten $t_1$, $t_2$, $t_3$ und $t_4$.}
  \label{tab:tabe5}
    \begin{tabular}{S S}
    \toprule
    $ t  \: / \si{\second} $ & $ N_{\text{mech}} \: / \: \si{\watt}$ \\
    \midrule
    480 & 4.72 \pm 0.16 \\
    960 & 6.19 \pm 0.22 \\
    1500 & 6.67 \pm 0.26 \\
    1980 & 6.26 \pm 0.28 \\
      \bottomrule
    \end{tabular}
\end{table}

Diese Ergebnisse können nicht mit Messwerten der Schieblehre verglichen werden, da
diese Bohrungen zu klein sind um sie zu vermessen.

Aus den Messwerten des TM-Scans wurde die Herzfrequenz des Modells zu
\begin{equation}
  \nu_{\text{Herz}}=\SI{1,84}{\Hz}
\end{equation}
bestimmt, daraus konnte das Herzvolumen
\begin{equation}
  HZV=\SI{2,21e-5}{\m^{3}\per\s}
\end{equation}
berechnet werden. Das entspricht $\SI{0,022}{\liter\per\s}$.
