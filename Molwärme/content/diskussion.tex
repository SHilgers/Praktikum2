\section{Diskussion}
\label{sec:Diskussion}
%\theta_Lit=345K
Der theoretisch zu erwartende Verlauf von $C_V \sim T^3$ für kleine
Temperaturen ist in den experimentellen Daten nur schwer zu erkennen, siehe dazu Abbildung \ref{fig:Cv}.
Für hohe Temperaturen wird eine Annäherung an die Dulong-Petit Konstante von
$C_m=\SI{24,9}{J\per\mol\K}$ erwartet. Diese Annäherung lässt sich auch in den
Messwerten erkennen.\\

Werden die beiden berechneten Deye-Temperaturen mit dem Literaturwert
$\theta_{D_\text{Lit}}=\SI{345}{\K}$ \cite{chemie} verglichen ergibt sich zum theoretisch berechneten
Wert eine Abweichung von 3,7\;\%. Die Abweichung des experimentell bestimmten
Wertes von dem Literaturwert beträgt 68,4\;\%.
Der theoretisch berechnete Wert weicht nur gering vom Literaturwert ab, der experimentelle
Wert zeigt eine große Abweichung. Untereinander weisen die experimentell
und theoretisch berechneten Debye-Temperaturen eine Abweichung von 75.01\;\% auf.\\
Eine mögliche Ursache für die Abweichung der experimentellen Daten ist ein systematischer Fehler.
Während der Messung war es schwierig die beiden Pt-100-Widerstände und damit die Temperatur
von Probe und Zylinder aneinander anzupassen. Durch die unterschiedlichen Temperaturen kommt
es zu Wärmestrahlung und die tatsächlich aufgenommene Wärmemenge der Probe kann nicht mehr bestimmt werden.
Außerdem kann Konvektion auftreten und die Messung verfälschen, falls der Rezipient nicht
perfekt evakuiert wird oder nicht ganz dicht ist.
