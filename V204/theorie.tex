\section{Theorie}
\subsection{Zielsetzung}
Mit Hilfe des Michelson-Interferometers ist es möglich die Wellenlänge von Licht
zu bestimmen, sowie die Bechungsindizes von Gasen. Hier soll die Wällenlänge
eines Lasers ausgemessen werden und der Brechungsindize von Luft bestimmt werden.

\subsection{Kohärenz und Interferenz von Licht}
Da Licht eine elektromagnetische Welle ist, kann es durch die Maxwell-Gleichungen
beschrieben werden. Charakteristisch für eine Welle ist, dass sie sowohl zeitlich als
auch räumlich periodisch ist. So lässt sich die zeitlich und räumlich periodische
elektrische Feldstärke $\vec{E}$ für den einfachsten Fall (einer ebenen Welle) durch
\begin{equation}
  \vec{E}(x,t)=E_{0}\cos{kx-\ommega t -\delta}
  \label{eqn:efeld}
\end{equation}
beschreiben.
Dabei beschreibt $k=\frac{2\pi}{\lamda}$ die Wellenzahl, $\lamda$ ist die Wellenlänge,
$\ommega$ ist die Kreisfrequenz und mit $\delta$ wird eine beliebige Phasenverschiebung bezeichnet.\\

Für elektromagnetische Wellen gilt das Superpositionsprinzip: Die in einem Punkt P
ankommenden elektromagnetischen Wellen überlagern sich.\\
Die elektrische Feldstärke von Licht lässt sich aufgrung der hohen Frequenz (ca. $10^{15} \SI{Hz}$) nicht direkt messen,
deshalb wird volgende Relation zwischen Intensität $I$ und elektrischem Feld $\vec{E}$ verwendet:
\begin{equation}
  I=const|\vec{E}^{2}|
  \label{eqn:intensität}.
\end{equation}
Aus dem Superposituonsprinzip und Gleichung \ref{eqn:intensität} folgt für die
Gesamtintensität:
\begin{equation}
  I_{ges}=2const\vec{E_0}^{2}(1+\cos(\delta_2-\delta_1))
  \label{eqn:iges}.
\end{equation}
An diese Gleichung wird deulich, dass zur Summe der Intensitäten $2const\vec{E_0}^2$ noch
ein Interferenzterm $2const\vec{E_0}^{2}\cos(\delta_2-\delta_1)$ hinzukommt. Dieser Interferenzterm
kann werte zwischen $+2const\vec{E_0}^{2}$ und  $-2const\vec{E_0}^{2}$ annehmen. Insbesondere
verschwindet er, wenn $\delta_2-\delta_1$ ein ungerades Vielfaches von $\pi$ ist.

Für Interferen müssen bestimmte EIgenschaften erfüllt sein, so ist mit Licht aus
natürlichen Quellen (z.B. Sonne) nur schwer Interferenz zu beobachen. Das liegt daran, dass
die Phasendifferenz $\delta_2-\delta_1$ bei natürlichen Quellen beliebiege positive als
auch negative Werte annimmt, also nicht konstant ist. Der Grund dafür liegt in der
Entstehung des Lichts: Wenn Licht emittiert wird wechselt ein angeretes ELektron auf ein
niedrigeres Energieniveau, dabei wird Energie in Form von Licht ausgesendet. Dieser ausgesendete
Wellenzug hat eine endliche Länge. Da die Emission ein satistisch verteilter Vorgang ist
kann es aus verschiedenen Lichtquellen oder Licht von verschieden Orten einer Quelle nicht zu
einem konstanten Phasenunterschied kommen.\\

Dieses Licht ohne konstanten Phasenunterschied wird als inköhärentes Licht bezeichnet.
Dementsprechend besitzt köhärentes Licht einen konstanten Phasenunterschied $\ommega$, auch
$k$ und $\ommega$ sind konstant.
Köhärentes Licht ist interferenzfähig und wird beispielsweise von Lasern erzeugt.
Unter bestimmten Voraussetzungen interferiert auch Licht aus konventionellen Quellen,
dazu wird das Licht einer Quelle (hier ein Atom oder Molekül) in zwei Strahlen aufgeteilt.
Dies erfolgt entweder über einen Strahlteile oder über eine Doppelblende wie in
Abbildung \ref{fig:blende} zu sehen. Anschlieschend werden die beiden Strahlen wieder in
einem Punkt zusammengeführt. Da die Lichtstrahlen dann unterschiedlich lange Wege zurückgelegt haben
kommt es zu einer Phasenverschiebung. So kann, muss es aber nich zu Interferenz kommen.
\begin{figure}[H]
  \centering
  \includegraphics{Unbekannt.JPG}
  \caption{Anordnung zum spalten eines konventionellen Lichtstrahls um Interferenzerscheinungen zu erzeugen}
  \label{fig:blende}.
\end{figure}
Eine wichtige Größe in diesem Zsammenhang ist die Kohärenzlänge $l$, die Emission eines 
Lichtzuges dauert eine endliche Zeitspanne an, dementsprechend hat der Lichtzug auch eine
endliche Länge. Ist der Wegunterschied größer als die Länge des Lichtzges kann es nicht zu
Interferenz kommen, da die Lichtstrahlen zu unterschiedlichen Zeitpunkten am gleichen
Punkt P eintreffen. Die kohärenzlänge $l$ ist genau die Länge, an der keine Interferenz mehr zu
beobachten ist. Des W eiteren gilt folgender Zusammenhang zwischen der Kohärenzlänge $l$, der Anzahl an
beobachten Interferenzmaxima $N$ sowie der Wellenlänge $\lamda$:
\begin{equation}
  l=N\lamda
  \label{eqn:interf}
\end{equation}

\subsection{Das Michelson-Interferometer}









\label{sec:Theorie}

%\cite{sample}
