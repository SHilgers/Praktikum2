\section{Diskussion}
Mit Hilfe der statischen Methode ist zu erkennen, dass Aluminum die Wärme am besten leitet, da
es von den vorliegenden Materialen die großte Wärmeleitfähigkeit $\kappa$ besitzt.
An den Ergebnissen des schmalen und breiten Messingstabes wird deutlich, das das der Wärmestrom
proportional zur Querschnittsfläche ist, weshalb der breite Stab die Wärme besser leitet.
Die Ergebnisse des Edelstahlstabes stimmen nicht mit der Theorie überein, denn diese
liegen sehr nahe an den Werten des Aluminiumstabes. Zu erwarten wäre, dass
er aufgrund der geringen Wärmeleitfähigkeut die Wärme am schlechtesten leitet.
Ein Vergleich mit Theoriewerten ist hier nicht möglich, da die Wärmeströme
von der einstastellten Spannung abhängen.
\\
Die Ergebnisse der dynamischen Methode werden mit den nachgeschlagenen Theoriewerten verglichen, um
die relative Abweichung zu ermitteln.
\begin{align*}
  &\kappa_{\text{Messing}} =\SI{93,98(1074)}{\W\per\meter\kelvin}   &\kappa_{\text{Messing,Teorie}} =\SI{120}{\W\per\meter\kelvin}\\
  &\implies \text{relative Abweichung}= 21,63\%\\
  &\kappa_{\text{Aluminium}}=\SI{591,98(1712)}{\W\per\meter\kelvin}  &\kappa_{\text{Messing,Teorie}} =\SI{237}{\W\per\meter\kelvin}\\
  &\implies \text{relative Abweichung}= 150,37\%\\
  &\kappa_{\text{Edelstahl}}=\SI{14,31(02)}{\W\per\meter\kelvin}   &\kappa_{\text{Messing,Teorie}} =\SI{15}{\W\per\meter\kelvin}\\
  &\implies \text{relative Abweichung}= 4,6\%
\end{align*}
Die Wete für Messing und Aluminium sind recht genau und liegen im Rahmen der Messungenauigkeit, die
zumeinen durch das ablesen der Werte aus den Diagrammen zu erklären ist. Eine weitere Erklärung sind
lie leichten Zeitverzögerungen beim umschalen des Peltier-Elements.
Die Abweichung von Aluminium ist sehr hoch und damit nicht nur mit der Messungenauigkeit zu erklären.
Eine mögliche Ursache könnte könnte nicht voll funktionsfähige Messapparatur sein, worauf der
Assistent schon zu Beginn des Versuches higewiesen hat. Diese Annahme wird dadurch unterstützt, dass
sowohl bei der statischen als auch der Dynamischen Methonde die Werte des Aluminiumstabes nicht zur Theorie passen.


\label{sec:Diskussion}
