\section{Diskussion}
Mit Hilfe der statischen Methode ist zu erkennen, dass Aluminum die Wärme am besten leitet, da
es von den vorliegenden Materialen die großte Wärmeleitfähigkeit $\kappa$ besitzt.
An den Ergebnissen des schmalen und breiten Messingstabes wird deutlich, dass der Wärmestrom
proportional zur Querschnittsfläche ist, weshalb der breite Stab die Wärme besser leitet.
Edelstahl leitet die Wärme am schlechtesten, dieses Ergebnis war aufgrund der
geringen Wärmeleitfähigkeit $\kappa$ zu erwarten.
Ein Vergleich mit Theoriewerten ist hier nicht möglich, da die Wärmeströme
von der einstastellten Spannung abhängen.
\\
Aus dem Graphen \ref{fig:diff} der Temperaturdifferenzen lässt sich ablesen, dass
die Temperaturdifferenz von Edelstahl als auch von Messing am Ende des Graphen einen
Wert $>0$ annimmt. Das bedeutet, dass das Termoelement nahe dem Peltier-Element eine größere Temperatur
hat als das fern vom Peltier-Element. Da der Edelstahl Graph über dem Messing Graphen verläuft
kann auf eine höhere Wärmeletfähigkeit von Messing geschlossen werden, da die Wärme das
ferne Termoelement schneller erreicht und daher die Temperaturdiffernz geringer ist.
Auf diese Weise sind auch die Maxima zu Beginn der Messung zu erklären, da zu diesem Zeitpunkt die
Temperaturdifferenz besonders hoch ist. Die Temperatur steigt zu Beginn am nahen Termoelement sehr stark an,
doch das ferne Termoelement erwärmt sich nur langsam.
Diese Ergebnisse stimmen mit den Erwartungen aus der Theorie überein: Edelstahl leitet aufgrund
der geringen Wärmeletfähigkeit die Wärme schlechter als Messing.
\\
Die Ergebnisse der dynamischen Methode werden mit den nachgeschlagenen Theoriewerten verglichen, um
die relative Abweichung zu ermitteln.
\begin{align*}
  &\kappa_{\text{Messing}} =\SI{93,98(1074)}{\W\per\meter\kelvin}   &\kappa_{\text{Messing,Teorie}} =\SI{120}{\W\per\meter\kelvin}\\
  &\implies \text{relative Abweichung}= 21,63\%\\
  &\kappa_{\text{Aluminium}}=\SI{295,99(1345)}{\W\per\meter\kelvin}  &\kappa_{\text{Aluminium,Teorie}} =\SI{237}{\W\per\meter\kelvin}\\
  &\implies \text{relative Abweichung}= 24,89\%\\
  &\kappa_{\text{Edelstahl}}=\SI{14,31(02)}{\W\per\meter\kelvin}   &\kappa_{\text{Edelstahl,Teorie}} =\SI{15}{\W\per\meter\kelvin}\\
  &\implies \text{relative Abweichung}= 4,6\%
\end{align*}
Der Messwert für Edelstahl weißt den geringsten Fehler auf, aber auch die Werte
für Messing und Aluminium liegen im Rahmen der Messungenauigkeit. Diese ist
zum einen durch das ablesen der Werte aus den Diagrammen zu erklären, eine weitere Erklärung sind
die leichten Zeitverzögerungen beim umschalen des Peltier-Elements.



\label{sec:Diskussion}
