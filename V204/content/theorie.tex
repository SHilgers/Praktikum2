\section{Zielsetzung}
Bei dem Versuch soll die Wärmeleitfähigkeit von Messing, Aluminium und Edelstahl
bestimmt werden.

\section{Theorie}
Liegt in einem Körper eine Temperaturdifferenz vor, kommt es laut dem zweiten Hauptsatz der Thermodynamik
zu einem Wärmetransport entlang des Temperaturgefälles. In einem festen Körper,
wie die in dem Versuch zu untersuchenden Stäbe geschieht dies über Wärmeleitung
mittels Phononen und freien Elektronen. Alternativ kann der Wärmetransport in anderen
Materialien auch über Konvektion und Wärmestrahlung erfolgen.

\noindent Herrscht zwischen den zwei Enden eines Stabes eine Temperaturdifferenz, so fließt in
der Zeit dt die Wärmemenge
\begin{equation}
  dQ= - \kappa A \frac{\partial T}{\partial x} dt \: \: ,
  \label{eqn:warm}
\end{equation}
wobei $ \kappa $ die Wärmeleitfähigkeit, welche abhängig vom Material ist, und A
die Querschnittsfläche bezeichnet.
Das negative Vorzeichen zeigt an, dass die Wärme stets von kälteren zu wärmeren
Bereichen, also in Richtung abnehmender Temperatur fließt.

\noindent Mithilfe der Wärmestromdichte
\begin{equation}
  j_w = - \kappa \frac{\partial T}{\partial x}
  \label{eqn:wstrom}
\end{equation}
lässt sich somit durch die Kontinuitätsgleich die Wärmeleitungsgleichung
\begin{equation}
  \frac{\partial T}{\partial t} = \frac{\kappa}{\rho c} \frac{\partial ^2 T}{\partial x^2}
  \label{eqn:wleit}
\end{equation}
aufstellen. Hierbei gibt $ \rho $ die Dichte des Materials und c die spezifische Wärme
an.
Durch den Term $ \sigma_T =\frac{\kappa}{\rho c} $ wird die “Geschwindigkeit” angegeben, mit
der sich der Temperaturunterschied ausgleicht. Dieser wird als
Temperaturleitfähigkeit bezeichnet.
\\
Findet eine periodische Erwärmung und Abkühlung eines Stabes mit einer großen Länge statt,
ergibt sich hierdurch eine Temperaturwelle mit zeitlicher und räumlicher Abhängigkeit
in dem Stab. Diese Welle besitzt dann die Form
\begin{equation}
  T(x,t) = T_{\text{max}} e^{-\sqrt{\frac{\omega \rho c}{2 \kappa}}x}
  \cos (\omega t - \sqrt{\frac{\omega \rho c}{2 \kappa}}x)  \: \: .
  \label{eqn:welle}
\end{equation}
Die Phasengeschwindigkeit lässt sich dabei über
\begin{equation}
   v = \frac{\omega}{k} = \omega / \sqrt{\frac{\omega \rho c}{2 \kappa}}
   = \sqrt{\frac{2 \kappa \omega}{\rho c}}
   \label{eqn:vphase}
 \end{equation}
berechnen
\\
Wird die Welle an 2 verschiedenen Stellen $x_{\text{nah}}$ und $x_{\text{fern}}$ gemessen,
kann man durch das Amplitudenverhältniss $A_{\text{nah}}$ und $A_{\text{fern}}$
die Dämpfung bestimmen.
Bei einer Welle der Periode $T^* $ und der Phase $ \phi$, lässt sich durch die Gleichungen
$ \omega = \frac{2\pi}{T^*} $ und $ \phi = \frac{2\pi \Delta t}{T^*} $ schlussendlich
die Formel
\begin{equation}
  \kappa = \frac{\rho c (\Delta x)^2}{2 \Delta t \: ln(A_{\text{nah}}/A_{\text{fern}})}
  \label{eqn:leitf}
\end{equation}
für die Wärmeleitfähigkeit aufstellen, wobei $\Delta x$ den räumlichen Abstand der
beiden Messpunkte $x_{\text{nah}}$ und $x_{\text{fern}}$ und $\Delta t $ die Phasendifferenz
der Welle zwischen den beiden Messpunkten angibt.
