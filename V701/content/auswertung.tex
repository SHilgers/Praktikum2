\section{Auswertung}
\label{sec:Auswertung}

\subsection{Bestimmung der Reichweite}

Die bei $\SI{0}{\milli\bar}$ gemessene Position des Maximums entspricht einer
Energie von ca. $\SI{4}{\MeV}$. Unter der Annahme einer linearen
Energieskala können mit diesem Startwert die anderen Energien $E_{\alpha}$
berechnet werden, die Ergebnisse sind in Tabelle \ref{tab:tab1} zu sehen.
(Ab $\SI{850}{\milli\bar}$ ist die Messung des Energiemaximuns nicht möglich, deshalb kann
für diese Werte auch keine Energie berechnet werden.)
Dort sind auch die nach Formel \ref{eqn:compare_rho_temp} berechneten
effektiven Längen eingetragen, für die Berechneung wird der bei der
ersten Messung eingestelle Abstand von $x_0=\SI{2,5}{\cm}$ verwendet.

\begin{table}[H]
  \centering
  \caption{Messwerte und Ergebniss der Bestimmung der Schallgeschwindigkeit}
  \label{tab:tabe1}
    \begin{tabular}{S||S S||S S||S|S}
    \toprule
    $ \text{Länge l des Zylinders [mm]} $ & $ U_{1} [\text{V}] $ &
    $ t_{1} [\mu\text{s}] $ & $ U_{2} [\text{V}] $ &
    $ t_{2} [\mu\text{s}] $ & $ \increment t [\mu\text{s}]$ &
    $ \text{c} [\text{m}/\text{s}]$\\
    \midrule
    31.0 & 1.335 \: & 24.0 & 1.096 \:  & 46.9 & 22.9 & 2707.42 \\
          \bottomrule
    \end{tabular}
  \end{table}


Wird die Zählrate gegen die effektive Länge aufgetragen, so ergibt sich
Abbildung \ref{fig:plot1}.

\begin{figure}[H]
  \centering
  \includegraphics[height=8cm]{plot1.pdf}
  \caption{Zählrate $N$ aufgetragen gegen die effektive Länge $x$.}
  \label{fig:plot1}
\end{figure}

Die mittlere Reichweite der $\alpha$-Teilchen wird bestimmt, indem
der lineare Teil der Funktion gefittet wird, anschließend wird der Schnittpunkt
der Ausgleichsgerade mit $\sfrac{N}{2}$ berechnet. So ergibt sich
der Schnittpunkt:
\begin{equation}
  R_m=\frac{\sfrac{N}{2}-b}{m},
  \label{eqn:mittel}
\end{equation}
woraus sich die mittlere Reichweite von $\SI{1,93(23)}{\cm}$ ergibt.
Der Fehler wird mit der Gauß'schen Fehlerfortpflanzung
\begin{equation}
  \increment f = \sqrt{ \sum_{i=1}^N \left( \frac{\partial f}{\partial x_i}\right)^2
  \cdot (\increment x_i)^2  } \: .
  \label{eqn:gaus}
\end{equation}
berechnet, für diesen Fall ergibt sich
\begin{equation}
  \Delta R_m=\sqrt{\Bigl(\frac{-1}{m}\Bigr)^2 \Bigl(\Delta b\Bigr)^2+ \Bigl(\frac{\sfrac{N}{2}+b}{m^2}\Bigr)^2
  \Bigl(\Delta m\Bigr)^2}.
\end{equation}

Aus Gleichung \ref{eqn:kleine} ergibt sich somit eine Energie von
\begin{equation*}
  E_{\alpha}=\SI{13.0(31)}{\MeV}.
\end{equation*}
Der zugehörige Fehler berechnet sich nach der Formel:
\begin{equation}
  \Delta E_{\alpha}=\sqrt{\Biggl(\frac{2}{3\Bigl(\sqrt[3]{\frac{R_m}{3,1}}^2\Bigr)}\Biggl)^2\cdot \Bigl(\Delta R_m\Bigr)^2}
\end{equation}

In Abbildung \ref{fig:plot13} wird die Energie gegen die effektive Länge aufgetragen,
aus der linearen Ausgleichsgeraden wird die Ableitung $\sfrac{dE}{dx}$ bestimmt, die
den Energieverlust $\sfrac{-dE}{dx}$ darstellt.
Es ergibt sich ein Energieverlust von:
\begin{equation*}
  \frac{-dE}{dx}=\SI{0,49(2)}{\MeV}.
\end{equation*}

\begin{figure}
  \centering
  \includegraphics[height=8cm]{plot13.pdf}
  \caption{Energie $E$ aufgetragen gegen die effektive Länge $x$.}
  \label{fig:plot13}
\end{figure}

Für die zweite Messreihe mit Abstand $a=\SI{2}{\cm}$, dessen Messwerte in Tabelle \ref{tab:tab2} zu
sehen sind, wird ebenfalls die Zählrate $N$ gegen die effektive Länge $x$ aufgetragen.
An den Messwerten ist zu erkennen, das die Werte für die Zählrate deutlich langsamer
abfallen als dies bei der ersten Messreihe der Fall ist.
\begin{table}[H]
  \centering
  \caption{Zählrate und Energiemaximum bei variiertem Druck, Abstand a=2cm}
  \label{tab:tab2}
    \begin{tabular}{c c c c c}
    \toprule
    Druck $\rho$/\;mbar & Energiemaximum & Zählrate $N$ & Energie $E_{\alpha}$ & effektive Länge $x$/\;cm\\
    \midrule
    0 & 796 &131382  &4          & 0.0   \\
    50 & 775 &131464 &3.89 & 0.09 \\
    100 &756 &130732 &3.79 & 0.19\\
    150 &749 &129617 &3.76  &  0.29\\
    200 &749 &130444 &3.76  & 0.39\\
    250 &727 &129600 &3.65 & 0.49\\
    300 &722 &128936 &3.63 & 0.59\\
    350 &708 &128478 &3.56 & 0.69\\
    400 &696 &128122 &3.49 & 0.79\\
    450 &687 &127415 &3.45 & 0.89\\
    500 &674 &126608 &3.39 & 0.99\\
    550 &663 &126372 &3.33 &1.09\\
    600 &651 &124989 &3.27 & 1.18\\
    650 &634 &124942 &3.19 & 1.28\\
    700 &618 &124295 &3.11 &1.38\\
    750 &602 &123299 &3.03 & 1.48\\
    800 &584 &119958 &2.93 &1.58\\
    850 &566 &120673 &2.84 &1.68\\
    900 &548 &117907 &2.75 & 1.78\\
    950 &534 &116111 &2.68&   1.88\\
    1000 &499& 108630&2.51 & 1.07\\
    \bottomrule
    \end{tabular}
  \end{table}

Wie in Abbildung \ref{fig:plot2} zu sehen überschneiden sich die Messwerte nicht mit
der $\sfrac{N}{2}$-Linie. Daher kann die mittlere Reichweite und somit auch
die Energie nicht bestimmt werden.

\begin{figure}[H]
  \centering
  \includegraphics[height=8cm]{plot2.pdf}
  \caption{Zählrate $N$ aufgetragen gegen die effektive Länge $x$.}
  \label{fig:plot2}
\end{figure}


\subsection{Statistik des radioaktiven Zerfalls}

Die Messergebnisse des zweiten Versuchsteils sind in Tabelle \ref{tab:tab3}
dargestellt und werden in Abbildung \ref{fig:plot4} in einem Histogramm
veranschaulicht, welches normiert wird.
\begin{table}[H]
  \centering
   \begin{tabular}{c c c}
    \toprule
     n& $\nu$/\; 1/s & $\nu_{Wechsel}$\\
    \midrule
    0,5 & 100.01& 50,0\\
    1 & 79.93 & 79.93\\
    2 & 23.93 & 47.86\\
    \bottomrule
  \end{tabular}
  \caption{Gemessene Frequenzen der Sägezahnspannung, sowie die Daraus resultierenden Frequenzen für die
  Wechselspannung.}
  \label{tab:tab3}
\end{table}


\begin{figure}[H]
  \centering
  \includegraphics[height=8cm]{plot4.pdf}
  \caption{Histogramm der Zählraten mit Gauß- und Poissonverteilung.}
  \label{fig:plot4}
\end{figure}

Es werden sowohl die Gauß-, als auch die Poissonverteilung eingezeichnet, um
diese mit den Messwerten vergleichen zu können.
Da die Poissonverteilung von dem Mittelwert $\bar{N}$ der Messwerte und die
Gaußverteilung von dem Mittelwert $\bar{N}$ und der Varianz $\sigma^{2}$ abhängen, werden diese
mit Hilfe der allgemeinen Formeln berechnet. Diese lauten für den Mittelwert
\begin{equation}
  \bar{x} = \frac{1}{N} \sum_{i=1}^{N} x_i \: \:
  \label{eqn:mit}
\end{equation}
und für die Standardabweichung
\begin{equation}
  \sigma=\increment \bar{x} = \frac{1}{\sqrt{N}} \sqrt{ \frac{1}{N-1} \sum_{i=1}^N
  (x_i - \bar{x})^2}.
  \label{eqn:mitf}
\end{equation}
Dabei ist die Varianz das Quadrat der Standardabweichung $\sigma$.
Für die aufgenommenen Messwerte ergeben sich mit den Formeln \ref{eqn:mit} und
\ref{eqn:mitf} folgende Werte:
\begin{align*}
  \bar{N}&=\SI{1015.52}{}\\
  \sigma^{2}&=\SI{10.88}{}.\\
\end{align*}
