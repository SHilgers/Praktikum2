\section{Auswertung}
\label{sec:Auswertung}

Die bei $\SI{0}{\milli\bar}$ gemessene Position des Maximums entspricht einer
Energie von ca. $\SI{4}{\MeV}$. Unter der Annahme einer linearen
Energieskala können mit diesem Startwert die anderen Energien $E_{\alpha}$
berechnet werden, die Ergebnisse sind in Tabelle \ref{tab:tab1} zu sehen.
Dort sind auch die nach Formel \ref{eqn:??} berechneten
effektiven Längen eingetragen, für die Berechneung wird der bei der
ersten Messung eingestelle Abstand von $x_0=\SI{2,5}{\cm}$ verwendet.

\begin{table}[H]
  \centering
  \caption{Messwerte und Ergebniss der Bestimmung der Schallgeschwindigkeit}
  \label{tab:tabe1}
    \begin{tabular}{S||S S||S S||S|S}
    \toprule
    $ \text{Länge l des Zylinders [mm]} $ & $ U_{1} [\text{V}] $ &
    $ t_{1} [\mu\text{s}] $ & $ U_{2} [\text{V}] $ &
    $ t_{2} [\mu\text{s}] $ & $ \increment t [\mu\text{s}]$ &
    $ \text{c} [\text{m}/\text{s}]$\\
    \midrule
    31.0 & 1.335 \: & 24.0 & 1.096 \:  & 46.9 & 22.9 & 2707.42 \\
          \bottomrule
    \end{tabular}
  \end{table}


Wird die Zählrate gegen die effektive Länge aufgetragen, so ergibt sich
Abbildung \ref{fig:plot1}.

\begin{figure}[H]
  \centering
  \includegraphics{plot1.pdf}
  \caption{}
  \label{fig:plot1}
\end{figure}

Die mittlere Reichweite der $\alpha$-Teilchen wird bestimmt, indem
der lineare Teil der Funktion gefittet wird, anschließend wird der
Geradenschnittpunkt mit $\sfrac{N}{2}$ gleichgesetzt. Durch umstellen
ergibt sich für die mittlere Reichweite die Formel:
\begin{equation}
  R_m=\frac{\sfrac{N}{2}-b}{m},
  \label{eqn:mittel}
\end{equation}
woraus sich die mittlere Reichweite von $\SI{1,93(23)}{\cm}$ ergibt.
Aus Gleichung \ref{eqn:??} ergibt sich somit eine Energie von
\begin{equation*}
  E_{\alpha}=\SI{0,122(29)}{\MeV}.
\end{equation*}

In Abbildung \ref{fig:plot2} wird die Energie gegen die effektive Länge aufgetragen,
aus der linearen Ausgleichsgeraden wird die Ableitung $\sfrac{dE}{dx}$ bestimmt, die
den Energieverlust $\sfrac{-dE}{dx}$ darstellt.
Es ergibt sich ein Energieverlust von:
\begin{equation*}
  \frac{-dE}{dx}=\SI{0,49(2)}{\MeV}.
\end{equation*}

Für die zweite Messreihe, dessen Messwerte in Tabelle \ref{tab:tab2} zu
sehen sind, wird ebenfalls die Zählrate $N$ gegen die effektive Länge $x$ aufgetragen.
An den Messwerten ist zu erkennen, das die Werte für die Zählrate deutlich langsamer
abfallen als das bei der ersten Messreihe der Fall ist.
Wie in Abbildung \ref{fig:plot2} zu sehen überschneiden sich die Messwerte nicht mit
der $\sfrac{N}{2}$-Linie. Daher kann die mittlere Reichweite und somit auch
die Energie nicht bestimmt werden.

\begin{figure}[H]
  \centering
  \includegraphics{plot2.pdf}
  \caption{}
  \label{fig:plot2}
\end{figure}


Die Messergebnisse des zweiten Versuchsteils sind in Tabelle \ref{tab:tab3}
dargestellt und werden in Abbildung \ref{fig:plot4} in einem Histogramm
veranschaulicht.

\begin{figure}[H]
  \centering
  \includegraphics{plot4.pdf}
  \caption{}
  \label{fig:plot4}
\end{figure}

Es werden sowohl die Gauß-, als auch die Poissonverteilung eingezeichnet, um
diese mit den Messwerten vergleichen zu können.
Da die Poissonverteilung von dem Mittelwert der Messwerte und die
Gaußverteilung von dem Mittelwert $\bar{N}$ und der Varianz $\sigma^{2}$ abhängen werden diese ermittelt.
Dabei ist die VArianz das Quadrat der Standardabweichung $\sigma$.
\begin{align*}
  \bar{N}&=\SI{1015.52}{}\\
  \sigma^{2}&=\SI{10.88}{}\\
\end{align*}
