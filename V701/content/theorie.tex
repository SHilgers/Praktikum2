\section{Zielsetzung}
Bei diesem Versuch soll die Reichweite in Luft und die Energie von
$\alpha$-Strahlung bestimmt werden.
Zudem wird die Statistik des Radioaktiven Zerfalls untersucht.


\section{Theorie}
\label{sec:Theorie}
Beim Durchlaufen von Materie verlieren $\alpha$-Teilchen Energie, beispielsweise
durch Ionisationsprozesse, Dissoziation oder Anregung von Molekülen oder Rutherford
Streuung, wobei letzteres einen eher kleinen Anteil liefert. \\
Bei kleineren Geschwindigkeiten ist der Wirkungsquerschnitt der Strahlung höher,
sodass die Wechselwirkungswahrscheinlichkeit zunimmt. Im Allgemeinen ist der
Energieverlust $-\text{d}E_{\alpha}/\text{d}x$ abhängig von der Dichte des durchlaufenen
Materials und der Energie der $\alpha$-Strahlung.
Ist die Energie groß genug, sodass keine Ladungsaustauschprozesse stattfinden,
lässt sich der Energieverlust durch die
Formel
\begin{equation}
  -\frac{\text{d}E_{\alpha}}{\text{d}x}=\frac{z^2 e^4}{4\pi\epsilon_0m_e}\cdot
  \frac{n\text{Z}}{v^2}\log{\left(\frac{2m_e v^2}{I}\right)}
  \label{eqn:bb}
\end{equation}
beschreiben, welche auch als Bethe-Bloch-Gleichung bezeichnet wird, wobei $z$ die
Ladung und $v$ die Geschwindigkeit der $\alpha$-Teilchen bezeichnet. Z ist zudem
die Ordnungszahl des durchlaufenen Gases, $n$ die Teilchendichte und $I$ gibt
die Ionisationsenergie des Gases an.\\
Die Gleichung
\begin{equation}
  R = \int_0^{E_{\alpha}} \frac{\text{d}E_{\alpha}}{-\text{d}E_{\alpha}/\text{d}x}
  \label{eqn:weg}
\end{equation}
gibt die Strecke bis zur vollständigen Abbremsung eines $\alpha$-Teilchens an,
was auch als Reichweite $R$ bezeichnet wird.\\
Wird eine bestimmte Energie unterschritten, verliert die Bethe-Bloch-Gleichung
(\ref{eqn:bb}) aufgrund von Ladungsaustauschprozessen ihre Gültigkeit, sodass hier
die mittlere Reichweite aus empirisch bestimmten Kurven ermittelt werden muss.
Dabei kann die Reichweite in Luft bei Energien $E_{\alpha} \leq \SI{2.5}{\mega\electronvolt}$
durch die Relation
\begin{equation}
  R_m = 3,1 \cdot E_{\alpha}^{3/2}
  \label{eqn:kleine}
\end{equation}
angenähert werden, wobei $R_m$ in der Einheit $\si{\milli\meter}$ und $E_{\alpha}$
in $\si{\mega\electronvolt}$ angegeben wird.\\
Sind die Temperatur und das Volumen konstant, ergibt sich eine Proportionaliät
zum Druck $p$ gemäß der Form
\begin{equation}
  x = x_0 \frac{p}{p_0} \:.
  \label{eqn:compare_rho_temp}
\end{equation}
Somit kann die Reichweite bestimmt werden, indem die Absorption in Abhängigkeit vom
Druck gemessen wird. Der feste Abstand zwischen dem Detektor und der Quelle wird dabei als
$x_0$ bezeichnet und $p_0=\SI{1013}{\milli\bar}$ ist der Normaldruck, sodass sich $x$ als effektive
Länge ergibt.
