\section{Diskussion}
\label{sec:Diskussion}
Für den Abstand $a=\SI{2,5}{\cm}$ zwischen Quelle und Detektor ergibt sich eine
mittlere Reichweite von
\begin{equation}
 R_m=\SI{1,93(23)}{\cm},
\end{equation}
dies entspricht einer Energie von
\begin{equation}
 E_{\alpha}=\SI{13.0(31)}{\MeV}.
\end{equation}
Die Reichweite von $\alpha$-Strahlung ist natürlich von ihrer Energie abhängig, doch im
Allgemein überschreitet die Reichweite die Größenordnung von wenigen Zentimetern nicht.
Die berechnete mittlere Reichweite bestätigt dies.

Des weiteren wird der Energieverlust zu
\begin{equation}
  \frac{-dE}{dx}=\SI{0,49(2)}{\MeV}
\end{equation}
ermittelt.

Ein Vergleich mit der zweiten Messung für den Abstand $a=\SI{2}{\cm}$ ist nicht
möglich, da die mittlere Reichweite und die daraus resultierende
Energie nicht ermittelt werden können, da es keinen Schnittpunk der Gerade bei
$\sfrac{N}{2}$ mit den Messwerten gibt.
\\
\\
Im zweiten Versuchsteil wir die Statistik des radioaktiven Zerfalls untersucht.
Dazu wird ein Histogramm der Messwerte erstellt und mit der
Gauß- und der Poissonverteilung untersucht. Dazu werden der Mittelwert der Zählrate
$\bar{N}$ und die Varianz $\sigma^2$ berechnet:
  \begin{align*}
    \bar{N}&=\SI{1015.52}{}\\
    \sigma^{2}&=\SI{10.88}{}.\\
  \end{align*}

Es fällt auf, dass die Poissonverteilung
deutlich näher an den Messwerten liegt als die Gaußverteilung. Dies war auch zu erwarten, denn
die Poissonverteilung beschreibt die Wahrscheinlichkeit für das Eintreffen seltener
Ereignisse. Dazu zählt der radioaktive Zerfall eines Atoms innerhalb der der Messzeit.
