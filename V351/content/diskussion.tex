\section{Diskussion}
Die bei der Fourieranalyse ermittelte Steigung für die Rechteckspannung
weicht 20,9\% von dem Theoriewert ab, dass entspricht 7,5 Fehlerintervallen.
Die Abweichung der Sägezahnspannung ist mit 2,2\% sehr gering, dass ist auch
an den Fehlerintervallen zu erkannen, denn die Steigung weicht um 0,5 Fehlerintervalle
von dem Theoriewert ab. Eine sehr hohe Abweichung tritt bei der Dreieckspannung
auf, sie beträgt 82,3\%, beziehungsweise 137,2 Fehlerintervalle.
Da die Abweichung bei der Rechteckspannung mit 7,5 Fehlerintervallen recht hoch
ist, lässt sich vermuten, dass ein systematischer Fehler vorliegt, der die Messung
verfälscht.
%Einer Erklärung für die Abweichungen ist die Annahmer über gerade
%und ungerade Funktionen, sie führt zu systematischen Fehlern.
Für die sehr hohe Abweichung der Dreickspannung ist es sehr wahrscheinlich, dass  ein
systhematischer Fehler vorliegt. Eine weitere Erklärung können die wenigen Messwerte
sein, denn im Vergleich zur Rechteck- und Sägezahnspannung werden nur die Hälfte der
Messwerte ermittelt, da die Spannungsamplitude für die Dreickspannung mit $\frac{1}{n²}$ abfällt.
Daduch können die wenigen und sehr kleinen Spannungsamplituden nur schwer bestimmt werden.

%Die bei der Fourier-Analyse ermittelten Steigungen für die Reckteck- und
%Sägezahnspannung liegen mit 20,9\% und 2,2\% Abweichung nah an den Theoriewerten.
%Diese kleinen Abweichungen lassen sich durch fehlerhaftes oder ungenaues messen
%erklären.
%Die Abweichung von 82,3\% die sich für die Dreickspannung ergibt lässt ich zum einen
%ebenfalls durch fehlerhaftes oder ungenaues messen erklären, dass kann bei dieser
%großen Ungenauigkeit jedoch nicht die einzige Ursache sein.
%Eine mögliche Erklärung sind die wenigen Messwerte, denn im Vergleich zur Rechteck- und
%Sägezahnspannung werden nur die Hälfte der Messwerte ermittelt, da die
%Spannungsamplitude für die Dreickspannung mit $\frac{1}{n²}$ abfällt. Daduch
%können die wenigen und sehr kleinen Spannungsamplituden nur schwer bestimmt werden.

Durch die Fourier-Synthese kann die Dreickspannung gut approximiert werden, denn
es treten nur sehr kleine Ungenauigkeiten auf. Das lässt sich durch den Abfall
der Spannungsamplituden mit $\frac{1}{n²}$ erklären, denn dadurch lässt sich die
Dreickspannung schon durch wenige Spannungsamplituden gut approximieren.
Zur Approximation der Rechteck- und Sägezahnspannung sind mehr
Spannungsamplituden notwendig, da diese aber nicht exakt eingestellt werden können,
ergibt sich auch für die approximierte Spannung ein größerer Fehler.
Wie schon in der Auswertung erwähnt ist der Fehler an den unstetigen Punkten
besonders gut zu erkennen.

\label{sec:Diskussion}
