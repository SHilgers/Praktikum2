\section{Auswertung}
\label{sec:Auswertung}
\subsection{Fourier-Analyse}
Für die Fourier-Analyse sind die Amplituden der Oberwellen ausschlaggebend.
Die Messwerte der Amplituden, die auf die erste Oberwelle normierten Amplituden
und die zugehörigen Freqenzen und sind in den folgenden Tabellen aufgeführt.
\begin{table}[H]
  \centering
  \caption{Messwerte und Ergebniss der Bestimmung der Schallgeschwindigkeit}
  \label{tab:tabe1}
    \begin{tabular}{S||S S||S S||S|S}
    \toprule
    $ \text{Länge l des Zylinders [mm]} $ & $ U_{1} [\text{V}] $ &
    $ t_{1} [\mu\text{s}] $ & $ U_{2} [\text{V}] $ &
    $ t_{2} [\mu\text{s}] $ & $ \increment t [\mu\text{s}]$ &
    $ \text{c} [\text{m}/\text{s}]$\\
    \midrule
    31.0 & 1.335 \: & 24.0 & 1.096 \:  & 46.9 & 22.9 & 2707.42 \\
          \bottomrule
    \end{tabular}
  \end{table}

\begin{table}[H]
  \centering
  \caption{Zählrate und Energiemaximum bei variiertem Druck, Abstand a=2cm}
  \label{tab:tab2}
    \begin{tabular}{c c c c c}
    \toprule
    Druck $\rho$/\;mbar & Energiemaximum & Zählrate $N$ & Energie $E_{\alpha}$ & effektive Länge $x$/\;cm\\
    \midrule
    0 & 796 &131382  &4          & 0.0   \\
    50 & 775 &131464 &3.89 & 0.09 \\
    100 &756 &130732 &3.79 & 0.19\\
    150 &749 &129617 &3.76  &  0.29\\
    200 &749 &130444 &3.76  & 0.39\\
    250 &727 &129600 &3.65 & 0.49\\
    300 &722 &128936 &3.63 & 0.59\\
    350 &708 &128478 &3.56 & 0.69\\
    400 &696 &128122 &3.49 & 0.79\\
    450 &687 &127415 &3.45 & 0.89\\
    500 &674 &126608 &3.39 & 0.99\\
    550 &663 &126372 &3.33 &1.09\\
    600 &651 &124989 &3.27 & 1.18\\
    650 &634 &124942 &3.19 & 1.28\\
    700 &618 &124295 &3.11 &1.38\\
    750 &602 &123299 &3.03 & 1.48\\
    800 &584 &119958 &2.93 &1.58\\
    850 &566 &120673 &2.84 &1.68\\
    900 &548 &117907 &2.75 & 1.78\\
    950 &534 &116111 &2.68&   1.88\\
    1000 &499& 108630&2.51 & 1.07\\
    \bottomrule
    \end{tabular}
  \end{table}

\begin{table}[H]
  \centering
   \begin{tabular}{c c c}
    \toprule
     n& $\nu$/\; 1/s & $\nu_{Wechsel}$\\
    \midrule
    0,5 & 100.01& 50,0\\
    1 & 79.93 & 79.93\\
    2 & 23.93 & 47.86\\
    \bottomrule
  \end{tabular}
  \caption{Gemessene Frequenzen der Sägezahnspannung, sowie die Daraus resultierenden Frequenzen für die
  Wechselspannung.}
  \label{tab:tab3}
\end{table}

Die Amplituden werden nun doppeltlogarithmisch gegen die Anzahl der Oberwellen
aufgetragen, anschließend wird eine Ausgleichsgerade und deren Steigung $a$ berechnet.
An den in der Vorbereitung berechneten Fourier-Koeffizienten
kann die erwartete Steigung der Ausgleichsgerade abgelesen werden, da diese
entweder mit $\frac{1}{n}$ oder $\frac{1}{n²}$ abfallen. Dieser
Zusamenhang wird in folgender Rechnung deutlich:
\begin{align*}
  y &= B*x^{-a}\\
    &= \ln(B)+\ln(x^{-a})\\
    &= \ln(B)-a\ln(x).
\end{align*}
Diese Formel ist äquivalent zur allgemeinen Geradengleichung $y=mx+v$.
Somit sind folgende Steigungen zu erwarten: \\

\begin{align*}
  a_{\text{Rechteckspannung}} = -1 \\
  a_{\text{Dreieckspannung}} = -2  \\
  a_{\text{Sägezahnspannung}} = -1
\end{align*}

\begin{figure}[H]
  \centering
  \includegraphics{plot1.pdf}
  \caption{Überprüfung der Steigung für die Rechteckspannung.}
  \label{fig:plot1}
\end{figure}
Für die Steigung der Ausgleichsgerade der Rechteckspannung ergibt sich
\begin{align*}
 a_{\text{Rechteckspannung}}= -0.791 \pm 0.028.
\end{align*}
Die Abweichung der Steigung von den Theoriewerten wird durch
\begin{equation}
  \Biggl\lvert \: \frac{a_{\text{Theorie}}-a_{\text{Messung}}}{a_{\text{Theorie}}} \Biggr\rvert
  \label{eqn:abw}
\end{equation}
berechnet.
Somit ergibt sich eine Abweichung von 20,9\% von den Theoriewerten.

 \begin{figure}[H]
  \centering
  \includegraphics{plot2.pdf}
  \caption{Überprüfung der Steigung für die Dreieckspannung.}
  \label{fig:plot2}
\end{figure}
Die Steigung der Ausgleichsgerade beträgt
\begin{align*}
  a_{\text{Dreieckspannung}}= -0.354 \pm 0.012.
\end{align*}
Für die Dreieckspannung ergibt sich mit \eqref{eqn:abw} eine Abweichung von
82,3\%.
\begin{figure}[H]
  \centering
  \includegraphics{plot3.pdf}
  \caption{Überprüfung der Steigung für die Sägezahnspannung.}
  \label{fig:plot3}
\end{figure}
Die Steigung der Sägezahnspannung beträgt
\begin{align*}
  a_{\text{Sägezahnspannung}}= -1.022 \pm 0.041.
\end{align*}
Nach \eqref{eqn:abw} ergibt sich eine Abweichung von 2,2\%.

\subsection{Fourier-Synthese}
Bei der Fourier-Synthese werden die Spannungen aus einer endlichen Anzahl von
Spannungsanplituden zusammengesetzt, so ergeben sich die folgenden Spannungsbilder.
\begin{figure}[H]
  \centering
  \includegraphics{recht.png}
  \caption{Synthetisierte Rechteckspannung.}
  \label{fig:Rechteck}
\end{figure}

\begin{figure}[H]
  \centering
  \includegraphics{drei.png}
  \caption{Synthetisierte Dreieckspannung.}
  \label{fig:Dreieck}
\end{figure}

\begin{figure}
  \centering
  \includegraphics{sag.png}
  \caption{Synthetisierte SäSägezahnspannung.}
  \label{fig:Säge}
\end{figure}

Die Dreieckspannung \ref{fig:Dreieck} lässt sich trotz der wenigen Spannungsamplituden
gut approximieren. Die Rechteckspannung \ref{Rechteck} und auch die
Sägezahnspannung \ref{Säge} dagegen zeigen größere Abweichungen, besongers
an den unstetigen Punkten.

%\begin{figure}
%  \centering
%  \includegraphics{plot.pdf}
%  \caption{Plot.}
%  \label{fig:plot}
%\end{figure}
