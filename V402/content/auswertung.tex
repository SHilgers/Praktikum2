\section{Auswertung}
\label{sec:Auswertung}
\subsection{Bestimmung des Winkels zwischen den brechenden Oberflächen}

Die Messwerte zur Bestimmung des Winkels $\varphi$ zwischen den brechenden Oberflächen befinden sich
in Tabelle \ref{tab:tabe1},
zusammen mit den durch Gleichung \ref{eqn:phi} errechneten Werten.
\begin{table}[H]
  \centering
  \caption{Messwerte und Ergebniss der Bestimmung der Schallgeschwindigkeit}
  \label{tab:tabe1}
    \begin{tabular}{S||S S||S S||S|S}
    \toprule
    $ \text{Länge l des Zylinders [mm]} $ & $ U_{1} [\text{V}] $ &
    $ t_{1} [\mu\text{s}] $ & $ U_{2} [\text{V}] $ &
    $ t_{2} [\mu\text{s}] $ & $ \increment t [\mu\text{s}]$ &
    $ \text{c} [\text{m}/\text{s}]$\\
    \midrule
    31.0 & 1.335 \: & 24.0 & 1.096 \:  & 46.9 & 22.9 & 2707.42 \\
          \bottomrule
    \end{tabular}
  \end{table}


Durch die Gleichung
\begin{equation}
  \bar{x} = \frac{1}{N} \sum_{i=1}^{N} x_i \: \:
  \label{eqn:mit}
\end{equation}
\noindent lässt sich der Mittelwert bilden, wobei der dazugehörige Fehler sich durch
\begin{equation}
  \increment \bar{x} = \frac{1}{\sqrt{N}} \sqrt{ \frac{1}{N-1} \sum_{i=1}^N
  (x_i - \bar{x})^2}
  \label{eqn:mitf}
\end{equation}
ergibt. Somit ergibt sich insgesamt:
\begin{align*}
  \varphi = 60,01° \: \pm \: 0,02 °
\end{align*}
\subsection{Bestimmung der Brechungsinidizes}
Die gemessenen Wertepaare aus $\Omega_{l}$ und $\Omega_{r}$ lassen sich in Tabelle \ref{tab:tabe2}
ablesen. Aus Gleichung \ref{eqn:eta} lässt sich hieraus der Winkel $\eta$ berechnen, aus welchem sich
wiederum durch Gleichung \ref{eqn:brechungsindex} der jeweilige Brechungsindex bestimmen lässt.
Die Ergebnisse sind ebenfalls in Tabelle \ref{tab:tabe2} aufgeführt.
\begin{table}[H]
  \centering
  \caption{Zählrate und Energiemaximum bei variiertem Druck, Abstand a=2cm}
  \label{tab:tab2}
    \begin{tabular}{c c c c c}
    \toprule
    Druck $\rho$/\;mbar & Energiemaximum & Zählrate $N$ & Energie $E_{\alpha}$ & effektive Länge $x$/\;cm\\
    \midrule
    0 & 796 &131382  &4          & 0.0   \\
    50 & 775 &131464 &3.89 & 0.09 \\
    100 &756 &130732 &3.79 & 0.19\\
    150 &749 &129617 &3.76  &  0.29\\
    200 &749 &130444 &3.76  & 0.39\\
    250 &727 &129600 &3.65 & 0.49\\
    300 &722 &128936 &3.63 & 0.59\\
    350 &708 &128478 &3.56 & 0.69\\
    400 &696 &128122 &3.49 & 0.79\\
    450 &687 &127415 &3.45 & 0.89\\
    500 &674 &126608 &3.39 & 0.99\\
    550 &663 &126372 &3.33 &1.09\\
    600 &651 &124989 &3.27 & 1.18\\
    650 &634 &124942 &3.19 & 1.28\\
    700 &618 &124295 &3.11 &1.38\\
    750 &602 &123299 &3.03 & 1.48\\
    800 &584 &119958 &2.93 &1.58\\
    850 &566 &120673 &2.84 &1.68\\
    900 &548 &117907 &2.75 & 1.78\\
    950 &534 &116111 &2.68&   1.88\\
    1000 &499& 108630&2.51 & 1.07\\
    \bottomrule
    \end{tabular}
  \end{table}

Der Fehler von n ergibt sich hierbei durch die Gauß´sche Fehlerfortpflanzung
\begin{equation}
  \increment f = \sqrt{ \sum_{i=1}^N \left( \frac{\partial f}{\partial x_i}\right)^2
  \cdot (\increment x_i)^2  } \: ,
  \label{eqn:gaus}
\end{equation}
in diesem Fall also
\begin{equation}
  \increment n = \sqrt{ \left( \frac{ \sin(-\frac{\eta}{2})}{\sin^2(\frac{\varphi}{2})} \right)^2
  \cdot (\increment \varphi)^2 } \: .
\end{equation}

\subsection{Bestimmung der Dispersionskurve}
Die Wertepaare aus $n^2$ und $\lambda$ werden zunächst in der Abbildung \ref{fig:plot1}
graphisch dargestellt.
\begin{figure}[H]
  \centering
  \includegraphics{plot1.pdf}
  \caption{Wertepaare zur Ermittlung der Dispersionskurve}
  \label{fig:plot1}
\end{figure}

Nun muss zwischen den beiden möglichen Dispersionskurven unterschieden werden.
Dazu werden zunächst die optimalen Parameter $\text{A}_0$ und $\text{A}_2$, sowie
$\text{A}'_0$ und $\text{A}'_2$
bestimmt, indem eine lineare Ausgleichsrechnung der Form
\begin{equation}
  y = a\cdot x +b
  \label{eqn:linear}
\end{equation}
durchgeführt wird. Für die Gleichung \ref{eqn:nquadrat1} werden dazu Wertepaare aus $\frac{1}{\lambda^2}$
und $n^2$ verwendet (Dispersionskurve 1), für die Gleichung \ref{eqn:nquadrat2} Wertepaare aus $\lambda^2$
und $n^2$ (Dispersionskurve~2).
Hieraus ergeben sich die Parameter:
\begin{align*}
  \text{A}_0 &= 2,93 \pm 0,01 \\
  \text{A}_2 &= \SI{57182(1500)}{\nano\meter\squared} \\
  \text{A}'_0 &= 3,38 \pm 0,03 \\
  \text{A}'_2 &= \SI{-8.0(11)e-7}{\per\nano\meter\squared}
\end{align*}
Durch diese Parameter lassen sich die jeweiligen Dispersionskurven erstellen, welche in Abbildung
\ref{fig:plot2} graphisch dargestellt sind.
\begin{figure}[H]
  \centering
  \includegraphics{plot2.pdf}
  \caption{Wertepaare und Dispersionskurven}
  \label{fig:plot2}
\end{figure}
Hieran lässt sich bereits erkennen, dass die Dispersionskurve 1 die Wertepaare genauer
approximiert und somit vermutlich geeigneter ist.
Um eine genauere Aussage treffen zu können, werden zudem die Abweichungsquadrate gemäß
Gleichung \ref{eqn:abweichungquadrat} bestimmt, woraus sich für die jeweiligen Abweichungsquadrate
die Werte
\begin{align*}
  \text{s}_{\text{n}}^2 &= 0,000023 \\
  \text{s}_{\text{n}}'^2 &= 0,000622
\end{align*}
ergeben.
Auch hieran lässt sich erkennen, dass $\text{s}_{\text{n}}'^2$, also das Abweichungsquadrat
von Dispersionskurve 2 deutlich größer ist (Faktor 27) als $\text{s}_{\text{n}}^2$.
Dispersionskurve 1 approximiert die Kurve also deutlich besser, sodass sich also
der Brechungsindex durch die Formel
\begin{equation}
  n(\lambda) = \sqrt{2,93 + \frac{\SI{57182}{\nano\meter\squared}}{{\lambda}^2}}
  \label{eqn:brech}
\end{equation}
annähern lässt, wobei sich der Fehler nach Gleichung \ref{eqn:gaus} über
\begin{equation}
    \increment n(\lambda) = \sqrt{\frac{1}{4(\text{A}_0+\frac{\text{A}_2}{{\lambda}^2})}
    \cdot (\increment \text{A}_0)^2
    + \frac{1}{4(\text{A}_0 {\lambda}^4 +\text{A}_2{\lambda}^2)}
    \cdot (\increment \text{A}_2)^2}
\end{equation}
ergibt, also über
\begin{equation}
    \increment n(\lambda) = \sqrt{\frac{0,0001}{11,72+\frac{\SI{228728}{\nano\meter\squared}}{{\lambda}^2}}
    + \frac{\SI{2250000}{\nano\meter^4}}{11,72 {\lambda}^4 +\SI{228728}{\nano\meter\squared}{\lambda}^2}} \: .
\end{equation}

\subsection{Abbesche Zahl}
Zur Bestimmung der Abbeschen Zahl wird Gleichung \ref{eqn:abbel} verwendet.
Hierzu müssen zunächst die Brechungsindizes der Fraunhofer Linien bestimmt werden,
wozu Gleichung \ref{eqn:brech} genutzt wird. Die Ergebnisse befinden sich in Tabelle
\ref{tab:tabe3}
\begin{table}[H]
  \centering
   \begin{tabular}{c c c}
    \toprule
     n& $\nu$/\; 1/s & $\nu_{Wechsel}$\\
    \midrule
    0,5 & 100.01& 50,0\\
    1 & 79.93 & 79.93\\
    2 & 23.93 & 47.86\\
    \bottomrule
  \end{tabular}
  \caption{Gemessene Frequenzen der Sägezahnspannung, sowie die Daraus resultierenden Frequenzen für die
  Wechselspannung.}
  \label{tab:tab3}
\end{table}

Hieraus ergibt sich die Abbesche Zahl zu
\begin{align}
  \nu = 24,6 \pm 0,6 \: ,
\end{align}
wobei sich der Fehler nach Gleichung \ref{eqn:gaus} über
\begin{equation}
  \increment \nu = \sqrt{\left(\frac{1}{n_F-n_C} \right)^2 \cdot (\increment n_D)^2
  + \left(\frac{n_D-1}{(n_F-n_C)^2} \right)^2 \cdot (\increment n_F)^2
  + \left(\frac{n_D-1}{(n_F-n_C)^2} \right)^2 \cdot (\increment n_C)^2}
\end{equation}
berechnen lässt.

\subsection{Auflösungsvermögen}
Das Auflösungsvermögen lässt sich aus der Gleichung \ref{eqn:auflösungsver}
bestimmen, wobei $\text{b}=\SI{3}{\centi\meter}$ die Basislänge des Prismas
bezeichnet.
Der Fehler ergibt sich nach \ref{eqn:gaus} durch die Gleichung
\begin{equation}
  \increment A = \frac{b}{\lambda^3}\cdot \sqrt{\left( \frac{1}{\sqrt{\text{A}_0
  +\frac{\text{A}_2}{\lambda^2}}}-\frac{\text{A}_2}{2}\frac{1}{\lambda^2
  \sqrt{\text{A}_0 + \frac{\text{A}_2}{\lambda^2}}^3}\right)^2 \cdot (\increment \text{A}_2)^2
  + \left(\frac{\text{A}_2}{2
  \sqrt{\text{A}_0 + \frac{\text{A}_2}{\lambda^2}}^3}\right)^2 \cdot (\increment \text{A}_0)^2} \: .
\end{equation}
Hiermit wird das Auflösungsvermögen für die Fraunhofer Linien bestimmt, wobei die
Ergebnisse in Tabelle \ref{tab:tabe4} angegeben sind.
\begin{table}[H]
  \centering
   \begin{tabular}{c c c c}
    \toprule
    Nummer der Oberwelle & $ U_{\text Theorie,Rechteck}\: / \si{\volt} $ &
    $ U_{\text Theorie,Dreick}\: / \si{\volt} $ & $ U_{\text Theorie,Sägezahn}\: / \si{\volt} $ \\
    \midrule
    1 & 1145 & 182 & 573 \\
    2 & 0 & 0 & 286 \\
    3 & 573 & 20 & 191 \\
    4 & 0 & 0 & 143 \\
    5 & 229 & 7 & 115 \\
    6 & 0 & 0 & 96 \\
    7 & 164 & 4 & 82 \\
    8 & 0 & 0 & 72 \\
    9 & 127 & 2 & 64 \\
    10 & 0 & 0 & 57 \\
    \bottomrule
  \end{tabular}
  \caption{Eingestellte Schwingungsamplituden.}
  \label{tab:tabe4}
\end{table}


\subsection{Bestimmung der nächsten Absorptionsstelle}
Aus den Gleichungen \ref{eqn:nquadrat1} und \ref{eqn:nquadratpotenz1} folgt, dass die nächste Absorptionsstelle sich durch die
Gleichung
\begin{equation}
  \lambda_1 = \sqrt{\frac{\text{A}_2}{\text{A}_0-1}}
\end{equation}
gegeben ist, mit dem Fehler
\begin{equation}
  \increment \lambda_1 =\sqrt{\frac{1}{4(\text{A}_2(\text{A}_0-1))} \cdot (\increment
  \text{A}_2)^2 + \frac{\text{A}_2}{4((\text{A}_0-1)^3)} \cdot (\increment
  \text{A}_0)^2  } \: ,
\end{equation}
sodass diese sich zu
\begin{align*}
    \lambda_1 = \SI{172(2)}{\nano\meter}
\end{align*}
ergibt und im ultravioletten Bereich liegt.
