\section{Diskussion}
\label{sec:Diskussion}
In dem Versuch wurde der Winkel zwischen den brechenden Oberflächen zu
\begin{align*}
  \varphi = 60,01° \: \pm \: 0,02 °
\end{align*}
bestimmt. Da ein komplett symmetrisches Prisma verwendet wurde, beträgt der
theoretische Wert also 60,00°
Durch die Formel
\begin{equation*}
  \frac{\lvert \text{Wert}_{\text{Theorie}}-\text{Wert}_{\text{Messung}}\rvert}{\text{Wert}_{\text{Theorie}}}
  \label{eqn:abw}
\end{equation*}
lässt sich die relative Abweichung zu 0,0167\% bestimmen, was für eine sehr genaue
Messung spricht. \\
Das Abweichungsquadrat
\begin{align*}
  \text{s}_{\text{n}}^2 &= 0,00010
\end{align*}
der gewählten Funktion ist ebenfalls sehr klein, sodass auch dies für die Genauigkeit der
Messung spricht. \\
Der Theoriewert der Abbeschen Zahl für SF14 beträgt 26,53 \cite{Abbe}.
Dies bedeutet also eine relative Abweichung von 21,45\%
zu dem errechnete Wert von 20,84. Dies ist eine recht große Abweichung im Vergleich
zu den vorherigen Auswertungsteilen, was auf einen systematischen Fehler hindeutet,
beispielsweise das Vernachlässigen von Termen mit $\lambda^4$.
