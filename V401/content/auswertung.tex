\section{Auswertung}
\label{sec:Auswertung}
\subsection{Bestimmung der Wellenlänge}
Zur Berechnung der Wellenlänge muss zunächst die gemessene Verschiebung
$\Delta d = 3 \si{\milli\meter}$
mit dem Kehrwert $\frac{1}{\text{Ü}}$ der Hebelübersetzung $\text{Ü}=5.017$ multipliziert werden,
um somit die reale Verschiebung des Spiegels $\Delta d_{\text{real}} \approx 0.598$ zu erhalten.
Aus dieser wird zusammen mit den gezählten Interferenzmaxima $z$ durch die Formel
\ref{eqn???} die jeweilige Wellenlänge $\lambda$ berechnet. Diese Ergebnisse sind
zusammen mit den Messwerten für z in Tabelle \ref{tab:tabe1} abzulesen.
\begin{table}[H]
  \centering
  \caption{Messwerte und Ergebniss der Bestimmung der Schallgeschwindigkeit}
  \label{tab:tabe1}
    \begin{tabular}{S||S S||S S||S|S}
    \toprule
    $ \text{Länge l des Zylinders [mm]} $ & $ U_{1} [\text{V}] $ &
    $ t_{1} [\mu\text{s}] $ & $ U_{2} [\text{V}] $ &
    $ t_{2} [\mu\text{s}] $ & $ \increment t [\mu\text{s}]$ &
    $ \text{c} [\text{m}/\text{s}]$\\
    \midrule
    31.0 & 1.335 \: & 24.0 & 1.096 \:  & 46.9 & 22.9 & 2707.42 \\
          \bottomrule
    \end{tabular}
  \end{table}

Die einzelnen Ergebnisse werden nun durch die Gleichung
\begin{equation}
  \bar{x} = \frac{1}{N} \sum_{i=1}^{N} x_i \: \:
  \label{eqn:mit}
\end{equation}
\noindent gemittelt, wobei der dazugehörige Fehler sich durch
\begin{equation}
  \increment \bar{x} = \frac{1}{\sqrt{N}} \sqrt{ \frac{1}{N-1} \sum_{i=1}^N
  (x_i - \bar{x})^2}
  \label{eqn:mitf}
\end{equation}
ergibt.
Somit ergibt sich insgesamt eine Wellenlänge von
\begin{equation*}
  \lambda= \SI{613.46(72)e-9}{\meter}
\end{equation*}
\subsection{Bestimmung des Brechungsindex von Luft}
Der Brechungsindex von Luft wird durch Formel \ref{eqn} berechnet, wobei die für
den Normaldruck $p_0$, die Normaltemperatur $T_0$, die Umgebungstemperatur T und die
Länge der evakuierten Zelle b folgende Werte verwendet werden:
\begin{align*}
  p_0 &= \SI{1.0132}{\bar} \\
  T_0 &= \SI{273.15}{\kelvin} \\
  T &= \SI{295.15}{\kelvin} \\
  b &=  50 \cdot 10^{-3} \si{\meter} \: .\\
\end{align*}
Der Kammerdruck wurde jeweils um etwa $\SI{0.8}{\bar}$ im Vergleich zum Normaldruck
gesenkt, sodass dieser
ungefähr $ p = \SI{0.2132}{\bar}$ beträgt.
Da die Wellenlänge des ersten Auswertungsteils bereits Fehlerbelastet ist, muss
hierbei die Gauß´sche Fehlerfortpflanzung
\begin{equation}
  \increment f = \sqrt{ \sum_{i=1}^N \left( \frac{\partial f}{\partial x_i}\right)^2
  \cdot (\increment x_i)^2  }
  \label{eqn:gaus}
\end{equation}
beachtet werden, die in diesem Fall
\begin{equation}
  ???????
\end{equation}
lautet.
Die Messwerte sind zusammen mit den jeweils errechneten Brechungsindizes in Tabelle
\ref{tab:tabe2} aufgeführt.
\begin{table}[H]
  \centering
  \caption{Zählrate und Energiemaximum bei variiertem Druck, Abstand a=2cm}
  \label{tab:tab2}
    \begin{tabular}{c c c c c}
    \toprule
    Druck $\rho$/\;mbar & Energiemaximum & Zählrate $N$ & Energie $E_{\alpha}$ & effektive Länge $x$/\;cm\\
    \midrule
    0 & 796 &131382  &4          & 0.0   \\
    50 & 775 &131464 &3.89 & 0.09 \\
    100 &756 &130732 &3.79 & 0.19\\
    150 &749 &129617 &3.76  &  0.29\\
    200 &749 &130444 &3.76  & 0.39\\
    250 &727 &129600 &3.65 & 0.49\\
    300 &722 &128936 &3.63 & 0.59\\
    350 &708 &128478 &3.56 & 0.69\\
    400 &696 &128122 &3.49 & 0.79\\
    450 &687 &127415 &3.45 & 0.89\\
    500 &674 &126608 &3.39 & 0.99\\
    550 &663 &126372 &3.33 &1.09\\
    600 &651 &124989 &3.27 & 1.18\\
    650 &634 &124942 &3.19 & 1.28\\
    700 &618 &124295 &3.11 &1.38\\
    750 &602 &123299 &3.03 & 1.48\\
    800 &584 &119958 &2.93 &1.58\\
    850 &566 &120673 &2.84 &1.68\\
    900 &548 &117907 &2.75 & 1.78\\
    950 &534 &116111 &2.68&   1.88\\
    1000 &499& 108630&2.51 & 1.07\\
    \bottomrule
    \end{tabular}
  \end{table}

Durch Mittelung durch die Formeln \ref{eqn:mit} und \ref{eqn:mitf} ergibt sich somit insgesamt
ein Brechungsindex von
\begin{equation*}
  n = 1,00028712 \pm 0,00000034
\end{equation*}
