\section{Auswertung}
\label{sec:Auswertung}
\subsection{Bestimmung der Güteziffer}
In Tabelle \ref{tab:tabe1} sind die gemessen Werte der Wärmepumpe aufgeführt.
\begin{table}[H]
  \centering
  \caption{Messwerte und Ergebniss der Bestimmung der Schallgeschwindigkeit}
  \label{tab:tabe1}
    \begin{tabular}{S||S S||S S||S|S}
    \toprule
    $ \text{Länge l des Zylinders [mm]} $ & $ U_{1} [\text{V}] $ &
    $ t_{1} [\mu\text{s}] $ & $ U_{2} [\text{V}] $ &
    $ t_{2} [\mu\text{s}] $ & $ \increment t [\mu\text{s}]$ &
    $ \text{c} [\text{m}/\text{s}]$\\
    \midrule
    31.0 & 1.335 \: & 24.0 & 1.096 \:  & 46.9 & 22.9 & 2707.42 \\
          \bottomrule
    \end{tabular}
  \end{table}

\noindent Die Temperaturverläufe von $ T_1 $ (warmes Reservoir) und $ T_2 $ (kaltes
Reservoir) sind in Abbildung \ref{fig:plot1} in Abhängigkeit der Zeit dargestellt.
\begin{figure}[H]
  \centering
  \includegraphics{plot1.pdf}
  \caption{Messwerte und Ausgleichsfunktionen der Temperaturverläufe}
  \label{fig:plot1}
\end{figure}
Die Augleichsfunktionen wurde durch eine Ausgleichsrechung mit der Funktion
\begin{equation}
  T(t) = at^2 +bt +c
\end{equation}
erstellt, wobei sich bei dem wärmeren Reservoir die Parameter
\begin{align*}
  a_w &= (-33,3 \: \pm \: 1,0) \cdot 10^{-7} \: \si{\kelvin\per\second\squared}\\
  b_w &= (21,50 \: \pm \: 0,21) \cdot 10^{-3} \: \si{\kelvin\per\second}\\
  c_w &= (292.882 \: \pm \: 0,091) \: \si{\kelvin} \\
\end{align*}
und bei dem kälteren Reservoir die Parameter
\begin{align*}
  a_k &= (33,7 \: \pm \: 1,0) \cdot 10^{-7} \: \si{\kelvin\per\second\squared}\\
  b_k &= (-15,44 \: \pm \: 0,21) \cdot 10^{-3} \: \si{\kelvin\per\second}\\
  c_k &= (294.349 \: \pm \: 0,088) \: \si{\kelvin}\\
\end{align*}
ergeben.

Um den Differentialquotienten zu bilden, wird die Funktion T(t) nun nach t abgeleitet, sodass
sich
\begin{equation}
  \frac{d T(t)}{d t} = 2at +b
\end{equation}
ergibt.
Werden nun die entsprechenden Parameter, sowie die vier beliebig gewählten Zeiten
$t_1 = \SI{480}{\second}$, $t_2 = \SI{960}{\second}$, $t_3 = \SI{1500}{\second}$
und $t_4 = \SI{1980}{\second}$ eingesetzt, ergeben sich die jeweiligen Differentialquotienten.
\begin{table}[H]
  \centering
  \caption{Zählrate und Energiemaximum bei variiertem Druck, Abstand a=2cm}
  \label{tab:tab2}
    \begin{tabular}{c c c c c}
    \toprule
    Druck $\rho$/\;mbar & Energiemaximum & Zählrate $N$ & Energie $E_{\alpha}$ & effektive Länge $x$/\;cm\\
    \midrule
    0 & 796 &131382  &4          & 0.0   \\
    50 & 775 &131464 &3.89 & 0.09 \\
    100 &756 &130732 &3.79 & 0.19\\
    150 &749 &129617 &3.76  &  0.29\\
    200 &749 &130444 &3.76  & 0.39\\
    250 &727 &129600 &3.65 & 0.49\\
    300 &722 &128936 &3.63 & 0.59\\
    350 &708 &128478 &3.56 & 0.69\\
    400 &696 &128122 &3.49 & 0.79\\
    450 &687 &127415 &3.45 & 0.89\\
    500 &674 &126608 &3.39 & 0.99\\
    550 &663 &126372 &3.33 &1.09\\
    600 &651 &124989 &3.27 & 1.18\\
    650 &634 &124942 &3.19 & 1.28\\
    700 &618 &124295 &3.11 &1.38\\
    750 &602 &123299 &3.03 & 1.48\\
    800 &584 &119958 &2.93 &1.58\\
    850 &566 &120673 &2.84 &1.68\\
    900 &548 &117907 &2.75 & 1.78\\
    950 &534 &116111 &2.68&   1.88\\
    1000 &499& 108630&2.51 & 1.07\\
    \bottomrule
    \end{tabular}
  \end{table}

Die Fehler berechnen sich hierbei aus den Fehlern der Ausgleichsrechnung durch die Gauß´sche
Fehlerfortpflanzung
\begin{equation}
  \increment f = \sqrt{ \sum_{i=1}^N \left( \frac{\partial f}{\partial x_i}\right)^2
  \cdot (\increment x_i)^2  } \: .
  \label{eqn:gaus}
\end{equation}
Also in diesem Fall:
\begin{equation}
  \increment \frac{d T(t)}{d t} = \sqrt{(2t)^2 \cdot (\increment a)^2 + (\increment b)^2 } \: .
  \label{eqn:f1}
\end{equation}
\\
\\
\noindent Um die Güteziffer zu bestimmen, wird Gleichungen \ref{eqn:güte3} verwendet,
wobei die spezifischen Wärmekapazität von Wasser $\SI{4183}{\joule\per\kilo\gram\per\kelvin}$ \cite{chemie}
beträgt, die Masse des Wassers $\SI{4}{\kilo\gram}$ und die Wärmekapazität des Kupfers
$\SI{750}{\joule\per\kelvin}$. Die entsprechende Leistung kann in der Tabelle \ref{tab:tabe1}
abgelesen werden.
Die Fehler berechnen sich hierbei nach Gleichung \ref{eqn:gaus} durch
\begin{equation}
  \increment v_{\text{real}} = \sqrt{\left(\frac{m_1c_w +m_2c_k}{N}\right)^2 \cdot \left(\increment \frac{d T(t)_1}{d t}\right)^2} \: .
  \label{eqn:f2}
\end{equation}
Die Theoriewerte für die ideale Wärmepumpe berechnen sich dabei durch die Gleichung
\ref{eqn:güte2} (die Temperatur muss zunächst noch in Kelvin umgerechnet werden) und die Abweichungen durch
\begin{equation*}
  \frac{\lvert \text{Wert}_{\text{Messung}}-\text{Wert}_{\text{Theorie}}\rvert}{\text{Wert}_{\text{Theorie}}} \: .
\end{equation*}
Somit ergibt sich insgesamt für die jeweils entsprechenden Differentialquotienten:
\begin{table}[H]
  \centering
   \begin{tabular}{c c c}
    \toprule
     n& $\nu$/\; 1/s & $\nu_{Wechsel}$\\
    \midrule
    0,5 & 100.01& 50,0\\
    1 & 79.93 & 79.93\\
    2 & 23.93 & 47.86\\
    \bottomrule
  \end{tabular}
  \caption{Gemessene Frequenzen der Sägezahnspannung, sowie die Daraus resultierenden Frequenzen für die
  Wechselspannung.}
  \label{tab:tab3}
\end{table}


\subsection{Bestimmung des Massendurchsatzes}
Um den Massendurchsatz zu bestimmen, wird zunächst die Verdampfungswärme L des
verwendeten Gases bestimmt.
Hierzu wird zunächst der Logarithmus des Verhältnisses $\frac{p_b}{p_0}$ gebildet
(Annahme: $p_0 \approx \SI{1}{\bar}$), was daraufhin gegen den Kehrwert der Temperatur $T_1$
aufgetragen wird. Dann wird eine lineare Ausgleichsrechnung der Form $ f(x) =ax +b$ durchgeführt, wie in
Abbildung \ref{fig:plot2} zu sehen ist.
\begin{figure}[H]
  \centering
  \includegraphics{plot2.pdf}
  \caption{Dampfdruckkurve mit linearer Regression}
  \label{fig:plot2}
\end{figure}
Hieraus ergeben sich die Parameter
\begin{align*}
  a= (-2340 &\pm 69) \: \si{\kelvin} \\
  b= (9,77 &\pm 0,22) \\
\end{align*}
und für die Verdampfungswärme ergibt sich
\begin{equation*}
  L= -a*R = \SI{19.46(58)e3}{\joule\per\mol}
\end{equation*}
wobei R= $\SI{8.3144}{\joule\per\mol\per\kelvin}$ \cite{chemie2} die allgemeine Gaskonstante ist.
Der Fehler berechnet sich durch
\begin{equation}
  \increment L = \sqrt{\left(-R\right)^2 \cdot \left(\increment a\right)^2} \: .
  \label{eqn:f2}
\end{equation}
Aus Gleichung \ref{eqn:masse} und einer molaren Masse von $\SI{120.91}{\gram\per\mol}$
\cite{chemie3} folgt:
\begin{table}[H]
  \centering
   \begin{tabular}{c c c c}
    \toprule
    Nummer der Oberwelle & $ U_{\text Theorie,Rechteck}\: / \si{\volt} $ &
    $ U_{\text Theorie,Dreick}\: / \si{\volt} $ & $ U_{\text Theorie,Sägezahn}\: / \si{\volt} $ \\
    \midrule
    1 & 1145 & 182 & 573 \\
    2 & 0 & 0 & 286 \\
    3 & 573 & 20 & 191 \\
    4 & 0 & 0 & 143 \\
    5 & 229 & 7 & 115 \\
    6 & 0 & 0 & 96 \\
    7 & 164 & 4 & 82 \\
    8 & 0 & 0 & 72 \\
    9 & 127 & 2 & 64 \\
    10 & 0 & 0 & 57 \\
    \bottomrule
  \end{tabular}
  \caption{Eingestellte Schwingungsamplituden.}
  \label{tab:tabe4}
\end{table}

Die Fehler berechenen sich hierbei durch:
\begin{equation}
  \increment \frac{dm}{dt} = \sqrt{\left(\frac{1}{L^2}(m_2c_w+m_kc_k) \frac{d T_2}{dt} \right)^2 \cdot \left(\increment L\right)^2
  + \left(\frac{1}{L}(m_2c_w+m_kc_k)\right)^2 \cdot \left(\increment \frac{d T_2}{dt} \right)^2}} \: .
  \label{eqn:f3}
\end{equation}
\subsection{Bestimmung der mechanischen Kompressorleistung}
Die Werte zur Berechnung der mechanischen Kompressorleistung lauten:
\begin{align*}
  \kappa &= 1,14 \\
  \rho_0 &= \SI{5.51}{\gram\per\liter} \\
  T_0 &= \SI{20.5}{\celsius}
\end{align*}
Zunächst muss hierzu noch die Dichte $\rho$ bestimmt werden, wobei mit der idealen Gasgleichung
die Gleichung
\begin{equation}
  pV =nRT \iff \frac{pV}{T}{nR}
\end{equation}
folgt. Aus $n_1R =n_2R$ folgt nun
\begin{equation}
  \frac{p_0 V_0}{T_0} = \frac{p_2 V_2}{T_2}
\end{equation}
und mit $\rho = \frac{m}{V}$, $\rho_2=\rho$ und $p_2 =p_a$ schlussendlich:
\begin{equation}
  \frac{p_0}{T_0\rho_0} = \frac{p_a}{T_2 \rho} \iff \rho = \frac{\rho_0 T_0 p_a}{T_2p_0}
\end{equation}
Wird dies in Gleichung \ref{eqn:Nmech}, sowie die Werte für $p_a$ und $p_b$ aus Tabelle
\ref{tab:tabe1} eingesetzt, ergibt sich somit:
\begin{table}[H]
  \centering
  \caption{Mechanischen Kompressorleistung zu den Zeiten $t_1$, $t_2$, $t_3$ und $t_4$.}
  \label{tab:tabe5}
    \begin{tabular}{S S}
    \toprule
    $ t  \: / \si{\second} $ & $ N_{\text{mech}} \: / \: \si{\watt}$ \\
    \midrule
    480 & 4.72 \pm 0.16 \\
    960 & 6.19 \pm 0.22 \\
    1500 & 6.67 \pm 0.26 \\
    1980 & 6.26 \pm 0.28 \\
      \bottomrule
    \end{tabular}
\end{table}

Die Fehlerrechnung erfolgt hierbei durch \ref{eqn:gaus} mit
\begin{equation}
  \increment N_{\text{mech}} = \sqrt{\left(\frac{1}{\kappa-1}\Biggl(p_{b}\sqrt[\kappa]{\frac{p_{a}}{p_{b}}}-p_{a}\Biggr)
  \frac{\rho_0 T_0 p_a}{T_2p_0}\right)^2 \cdot \left(\increment \frac{dm}{dt} \right)^2} \: .
  \label{eqn:f4}
\end{equation}
