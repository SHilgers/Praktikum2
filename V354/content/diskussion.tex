\section{Diskussion}
Im ersten Auswertungsteil weicht der erechnete Wert des effektiven Widerstands um
etwa $\SI{78.4}{\ohm}$ von dem erwarteten Wert ($R_1 = \SI{48.1 (1)}{\ohm}$) ab.
Dies liegt zum einen daran, dass der Innenwiderstand des Generators beim Erwartungswert nicht beachtet
wird, welcher etwa im Bereich von $ \SI{50}{\ohm}$ liegt.
Dieser wird daher auch in den folgenden Rechnungen berücksichtigt. Zum anderen
erhöhen auch die nicht angegeben Innenwiderstände der einzelnen Bauteile (Spule, Kondensator)
den effektiven Widerstand, sodass dieser höher ist als der des eingebauten Widerstands.
\\
\noindent Bei der Auswertung des Aperiodischen Grenzfalls ergibt sich eine prozentuale
Abweichung von etwa 30,93 \%, welche durch die Formel
\begin{equation*}
  \frac{\lvert \text{Wert}_{\text{Theorie}}-\text{Wert}_{\text{Messung}}\rvert}{\text{Wert}_{\text{Theorie}}}
\end{equation*}
berechnet wurde. Diese Abweichung liegt deutlich außerhalb der Fehlerintervalle
und ist somit vermutlich durch einen systematischen Fehler zu erklären, beispielsweise
das erneute Missachten der Widerstände von Spule und Kondensator. Auch ist das
Ablesen und Einstellen nur ungenau möglich.
\\
\noindent Die prozentuale Abweichung der Güte liegt bei etwa 14,18 \% und ist erneut außerhalb
der Fehlerintervalle; sie ist vermutlich auf die bereits genannten Gründe in der vorhergegangenen
Diskussion zurückzuführen.
Die Abweichung der Breite ist mit ca. 3,50 \% hingegen recht gering. \\
\noindent Bei der Messung der Phasenverschiebung ergeben sich relative Abweichungen von
\begin{align*}
  \nu_{res} : 1,15 \% \\
  \nu_1 : 2,73 \% \\
  \nu_2 : 3,16 \% \\
\end{align*}
Somit scheint auch diese Messung recht genau gewesen zu seien.
