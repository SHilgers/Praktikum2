
\section{Auswertung}
\subsection{Bestimmung des Dämpfungswiderstands}
Die Werte des im Versuch verwendeten Schwingkreises (Gerät 2) lauten
\begin{align*}
  L &= \SI{10.11(3)}{\milli\henry} \\
  C &= \SI{2.098(6)}{\nano\farad} \\
  R_1 &= \SI{48.1 (1)}{\ohm} \\
  R_2 &= \SI{509.5(5)}{\ohm} \\
\end{align*}
\noindent Der Spannungsverlauf, der zur Messung verwendet wurde ist in Abbildung \ref{fig:fig1}
mit eingezeicneter Einhüllender zu sehen.
\begin{figure}[H]
  \centering
  \includegraphics[height=7cm]{Schwingung.JPG}
  \caption{Schwingungsverlauf des Oszilloskops mit Einhüllender}
  \label{fig:fig1}
\end{figure}

\noindent Die sich hieraus ergebenden Wertepaare aus Kondensatorspannung $\text{U}_c$
und Zeit t befinden sich, getrennt nach Minima und Maxima, in Tabelle \ref{tab:tabe1}

\begin{table}[H]
  \centering
  \caption{Messwerte und Ergebniss der Bestimmung der Schallgeschwindigkeit}
  \label{tab:tabe1}
    \begin{tabular}{S||S S||S S||S|S}
    \toprule
    $ \text{Länge l des Zylinders [mm]} $ & $ U_{1} [\text{V}] $ &
    $ t_{1} [\mu\text{s}] $ & $ U_{2} [\text{V}] $ &
    $ t_{2} [\mu\text{s}] $ & $ \increment t [\mu\text{s}]$ &
    $ \text{c} [\text{m}/\text{s}]$\\
    \midrule
    31.0 & 1.335 \: & 24.0 & 1.096 \:  & 46.9 & 22.9 & 2707.42 \\
          \bottomrule
    \end{tabular}
  \end{table}


\noindent In Abbildung \ref{fig:plot1} sind die Messwerte zusammen mit der jeweils erechnete Ausgleichsfunktion
zu sehen, welche sich durch eine Ausgleichsrechnung mt der Funktion
\begin{equation}
  A(t)= A_0 \cdot \exp{(-2\pi \cdot \mu \cdot t)}
\end{equation}
ergibt.
\begin{figure}[H]
  \centering
  \includegraphics{plot1.pdf}
  \caption{Messwerte und Ausgleichsfunktionen der ersten Messung}
  \label{fig:plot1}
\end{figure}

\noindent Hieraus ergeben sich bei den Minima die Parameter
\begin{align*}
  A_0 &= \SI{-171.45(71)}{\volt} \\
  \mu&= \SI{1011.7(66)}{1\per\second}
\end{align*}
und bei den Maxima
\begin{align*}
  A_0 &= \SI{169.3(16)}{\volt} \\
  \mu&= \SI{980(13)}{1\per\second}
\end{align*}
sodass sich für $ \mu $ ein gemittelter Wert von
\begin{align*}
  \mu &= \frac{\mu_{\text{min}}+\mu_{\text{max}}}{2} = \SI{996(7)}{1\per\second}
\end{align*}
ergibt, wobei sich der Fehler durch
\begin{equation}
  \Delta \mu = \sqrt{(\frac{1}{2}\Delta \mu_{min})^2 + (\frac{1}{2}\Delta \mu_{max})^2}
\end{equation}
berechnet.
Hieraus ergibt sich durch die Gleichungen \ref{eqn:lös} und \ref{eqn:tex}
\begin{align*}
  R&= \SI{126.5(10)}{\ohm} \\
  T&= \SI{159.9(12)}{\micro\second} \: .
\end{align*}
Die Fehler berechnen sich hierbei durch die Gauß´sche Fehlerfortpflanzung
\begin{equation}
  \increment f = \sqrt{ \sum_{i=1}^N \left( \frac{\partial f}{\partial x_i}\right)^2
  \cdot (\increment x_i)^2  } \: .
  \label{eqn:gaus}
\end{equation}
Also:
\begin{align}
  \Delta R &=4 \pi \sqrt{(\mu \Delta L )^2 +(L \Delta \mu )^2} \\
  \Delta T &= \sqrt{\left(\frac{1}{2\pi\mu} \Delta \mu \right)^2}
\end{align}
\subsection{Bestimmung des Dämpfungswiderstands mit dem Aperiodischen Grenzfall}

Die gemessenen Werte des Widerstands $R_{ap}$ beim aperiodischen Grenzfall lauten
$\SI{3360}{\ohm}$, $\SI{3330}{\ohm}$ und $\SI{3370}{\ohm}$ , sodass sich hierbei ein Mittelwert
von $\SI{3353(17)}{\ohm}$ ergibt.
\noindent Theoretisch berechnet ergibt sich durch die Formel \ref{eqn:bed3}
ein Wert von
\begin{equation*}
  \text{R}_{ap}= 2 \cdot \sqrt{\frac{L}{C}} = \SI{4390(9)}{\ohm}
\end{equation*}


\subsection{Bestimmung der Frequenzabhängigkeit der Kondensatorspannung}

Die Messwerte zur Bestimmung der Frequenzabhängigkeit befinden sich in Tabelle \ref{tab:tabe2}.

\begin{table}[H]
  \centering
  \caption{Zählrate und Energiemaximum bei variiertem Druck, Abstand a=2cm}
  \label{tab:tab2}
    \begin{tabular}{c c c c c}
    \toprule
    Druck $\rho$/\;mbar & Energiemaximum & Zählrate $N$ & Energie $E_{\alpha}$ & effektive Länge $x$/\;cm\\
    \midrule
    0 & 796 &131382  &4          & 0.0   \\
    50 & 775 &131464 &3.89 & 0.09 \\
    100 &756 &130732 &3.79 & 0.19\\
    150 &749 &129617 &3.76  &  0.29\\
    200 &749 &130444 &3.76  & 0.39\\
    250 &727 &129600 &3.65 & 0.49\\
    300 &722 &128936 &3.63 & 0.59\\
    350 &708 &128478 &3.56 & 0.69\\
    400 &696 &128122 &3.49 & 0.79\\
    450 &687 &127415 &3.45 & 0.89\\
    500 &674 &126608 &3.39 & 0.99\\
    550 &663 &126372 &3.33 &1.09\\
    600 &651 &124989 &3.27 & 1.18\\
    650 &634 &124942 &3.19 & 1.28\\
    700 &618 &124295 &3.11 &1.38\\
    750 &602 &123299 &3.03 & 1.48\\
    800 &584 &119958 &2.93 &1.58\\
    850 &566 &120673 &2.84 &1.68\\
    900 &548 &117907 &2.75 & 1.78\\
    950 &534 &116111 &2.68&   1.88\\
    1000 &499& 108630&2.51 & 1.07\\
    \bottomrule
    \end{tabular}
  \end{table}

\noindent In Abbildung \ref{fig:plot2} ist das Verhältniss von $ \frac{U}{U_c} $ in einem
halblogarithmischen Diagramm gegen die Frequenz $ \nu $ aufgetragen.

\begin{figure}[H]
  \centering
  \includegraphics{plot2.pdf}
  \caption{Spannungsverhältniss in Abhängigkeit der Frequenz.}
  \label{fig:plot2}
\end{figure}
\noindent Um die Güte zu bestimmen, wird der Bereich um die Resonanzfrequenz
in Abbildung \ref{fig:plot3}
zudem linear dargestellt.

\begin{figure}[H]
  \centering
  \includegraphics{plot3.pdf}
  \caption{Spannungsverhältniss in Abhängigkeit der Frequenz}
  \label{fig:plot3}
\end{figure}

\noindent Hierraus lässt sich die Resonanzüberhöhung $\text{q} = 4.4791 $ ablesen. Aus den Werten des
Schaltkreises lässt sich durch die Gleichung
\begin{equation}
  q= \frac{\omega_0}{\omega_+-\omega_-}
\end{equation}
eine theoretische Resonanzüberhöhung von
\begin{equation*}
  \text{q}_{theo} = 3.923 \pm 0.009
\end{equation*}
\noindent berechnen. \\
\noindent Die abgelesenen Werte der Resonanzkurve lauten
\begin{align*}
  \nu_- =\SI{29000}{\hertz} \\ %\implies \omega_- = \SI{182212.37}{\hertz} \\
  \nu_+ =\SI{37500}{\hertz} %\implies \omega_+ = \SI{235619.45}{\hertz}
\end{align*}
woraus sich die Breite
\begin{equation*}
  \nu_+ - \nu_- = \SI{8500}{\hertz}
\end{equation*}
ergibt. Theoretisch lässt sich durch Gleichung \ref{eqn:gute}
eine Breite von $\SI{8808(27)}{\hertz} $ erechnen.
\subsection{Bestimmung der Frequenzabhängigkeit der Phasenverschiebung}

Die Messwerte zur Bestimmung der Frequnezabhängigkeit der Phasenverschiebung,
sowie die daraus erechnete Phase befinden sich in Tabelle \ref{tab:tabe3}.
\begin{table}[H]
  \centering
   \begin{tabular}{c c c}
    \toprule
     n& $\nu$/\; 1/s & $\nu_{Wechsel}$\\
    \midrule
    0,5 & 100.01& 50,0\\
    1 & 79.93 & 79.93\\
    2 & 23.93 & 47.86\\
    \bottomrule
  \end{tabular}
  \caption{Gemessene Frequenzen der Sägezahnspannung, sowie die Daraus resultierenden Frequenzen für die
  Wechselspannung.}
  \label{tab:tab3}
\end{table}


\noindent In Abbildung \ref{fig:phase} ist die Phasenverschiebung gegen die Frequenz
in einem halblogarithmischen Diagramm aufgetragen.

\begin{figure}[H]
  \centering
  \includegraphics{plot4.pdf}
  \caption{Phasenverschiebung in Abhängigkeit der Frequenz}
  \label{fig:phase}
\end{figure}

Zudem ist in Abbildung \ref{fig:plot4} der Bereich um die Resonanzfrequenz linear
dargestellt.

\begin{figure}[H]
  \centering
  \includegraphics{plot5.pdf}
  \caption{Phasenverschiebung in Abhängigkeit der Frequenz}
  \label{fig:plot4}
\end{figure}

\noindent Hieraus lassen sich die Resonanzfrequenz
\begin{align*}
  \nu_{res}= \SI{33600}{\hertz}
\end{align*}
sowie die beiden Frequenzen
\begin{align*}
  \nu_1 = \SI{29600}{\hertz} \\
  \nu_2 = \SI{38000}{\hertz}
\end{align*}
bei welchen die Phasenverschiebung gerade $\phi = \frac{\pi}{4}$ bzw. $\phi= \frac{3 \cdot \pi}{4}$
ist, ablesen. \\
\noindent Die nach Gleichungen \ref{eqn:res} und
\begin{equation}
  \omega_{1,2} = \pm \frac{R}{2L} + \sqrt{\frac{R^2}{4L^2} + \frac{1}{LC}}
\end{equation}
berechneten Theoriewerte lauten
\begin{align*}
  \nu_{res}= \SI{33990(70)}{\hertz} \\
  \nu_1 = \SI{30430(60)}{\hertz} \\
  \nu_2 = \SI{39240(80)}{\hertz}
\end{align*}
||||||| merged common ancestors
\section{Auswertung}
\subsection{Bestimmung des Dämpfungswiderstands}
Die Werte des im Versuch verwendeten Schwingkreises (Gerät 2) lauten
\begin{align*}
  L &= \SI{10.11(3)}{\milli\henry} \\
  C &= \SI{2.098(6)}{\nano\farad} \\
  R_1 &= \SI{48.1 (1)}{\ohm} \\
  R_2 &= \SI{509.5(5)}{\ohm} \\
\end{align*}
\noindent Der Spannungsverlauf, der zur Messung verwendet wurde ist in Abbildung \ref{fig:fig1}
mit eingezeicneter Einhüllender zu sehen.
\begin{figure}[H]
  \centering
  \includegraphics[height=7cm]{Schwingung.JPG}
  \caption{Schwingungsverlauf des Oszilloskops mit Einhüllender}
  \label{fig:fig1}
\end{figure}

\noindent Die sich hieraus ergebenden Wertepaare aus Kondensatorspannung $\text{U}_c$
und Zeit t befinden sich, getrennt nach Minima und Maxima, in Tabelle \ref{tab:tabe1}

\begin{table}[H]
  \centering
  \caption{Messwerte und Ergebniss der Bestimmung der Schallgeschwindigkeit}
  \label{tab:tabe1}
    \begin{tabular}{S||S S||S S||S|S}
    \toprule
    $ \text{Länge l des Zylinders [mm]} $ & $ U_{1} [\text{V}] $ &
    $ t_{1} [\mu\text{s}] $ & $ U_{2} [\text{V}] $ &
    $ t_{2} [\mu\text{s}] $ & $ \increment t [\mu\text{s}]$ &
    $ \text{c} [\text{m}/\text{s}]$\\
    \midrule
    31.0 & 1.335 \: & 24.0 & 1.096 \:  & 46.9 & 22.9 & 2707.42 \\
          \bottomrule
    \end{tabular}
  \end{table}


\noindent In Abbildung \ref{fig:plot1} sind die Messwerte zusammen mit der jeweils erechnete Ausgleichsfunktion
zu sehen, welche sich durch eine Ausgleichsrechnung mt der Funktion
\begin{equation}
  A(t)= A_0 \cdot \exp{-2\pi \cdot \mu \cdot t}
\end{equation}
ergibt.
\begin{figure}[H]
  \centering
  \includegraphics{plot1.pdf}
  \caption{Messwerte und Ausgleichsfunktionen der ersten Messung}
  \label{fig:plot1}
\end{figure}

\noindent Hieraus ergeben sich bei den Minima die Parameter
\begin{align*}
  A_0 &= \SI{-171.45(71)}{\volt} \\
  \mu&= \SI{1011.7(66)}{1\per\second}
\end{align*}
und bei den Maxima
\begin{align*}
  A_0 &= \SI{169.3(16)}{\volt} \\
  \mu&= \SI{980(13)}{1\per\second}
\end{align*}
sodass sich für $ \mu $ insgesamt ein Wert von
\begin{align*}
  \mu &= \SI{996(7)}{1\per\second}
\end{align*}
ergibt.
Hieraus ergibt sich durch die Gleichungen \ref{eqn:eqn:lös} und \ref{eqn:tex}
\begin{align*}
  R= \SI{126.5(10)}{\ohm} \\
  T= \SI{159.9(12)}{\micro\second} \: .
\end{align*}
Die Fehler berechnen sich hierbei durch die Gauß´sche Fehlerfortpflanzung
\begin{equation}
  \increment f = \sqrt{ \sum_{i=1}^N \left( \frac{\partial f}{\partial x_i}\right)^2
  \cdot (\increment x_i)^2  } \: .
  \label{eqn:gaus}
\end{equation}
\subsection{Bestimmung des Dämpfungswiderstands mit dem Aperiodischen Grenzfall}

Die gemessenen Werte des Widerstands $R_{ap}$ beim aperiodischen Grenzfall lauten
$\SI{3360}{\ohm}$, $\SI{3330}{\ohm}$ und $\SI{3370}{\ohm}$ , sodass sich hierbei ein Mittelwert
von $\SI{3353(17)}{\ohm}$ ergibt.
\noindent Theoretisch berechnet ergibt sich durch die Formel \ref{eqn:bed3}
ein Wert von
\begin{equation*}
  \text{R}_{ap}= 2 \cdot \sqrt{\frac{L}{C}} = \SI{4390(9)}{\ohm}
\end{equation*}


\subsection{Bestimmung der Frequenzabhängigkeit der Kondensatorspannung}

Die Messwerte zur Bestimmung der Frequenzabhängigkeit befinden sich in Tabelle \ref{tab:tabe2}.

\begin{table}[H]
  \centering
  \caption{Zählrate und Energiemaximum bei variiertem Druck, Abstand a=2cm}
  \label{tab:tab2}
    \begin{tabular}{c c c c c}
    \toprule
    Druck $\rho$/\;mbar & Energiemaximum & Zählrate $N$ & Energie $E_{\alpha}$ & effektive Länge $x$/\;cm\\
    \midrule
    0 & 796 &131382  &4          & 0.0   \\
    50 & 775 &131464 &3.89 & 0.09 \\
    100 &756 &130732 &3.79 & 0.19\\
    150 &749 &129617 &3.76  &  0.29\\
    200 &749 &130444 &3.76  & 0.39\\
    250 &727 &129600 &3.65 & 0.49\\
    300 &722 &128936 &3.63 & 0.59\\
    350 &708 &128478 &3.56 & 0.69\\
    400 &696 &128122 &3.49 & 0.79\\
    450 &687 &127415 &3.45 & 0.89\\
    500 &674 &126608 &3.39 & 0.99\\
    550 &663 &126372 &3.33 &1.09\\
    600 &651 &124989 &3.27 & 1.18\\
    650 &634 &124942 &3.19 & 1.28\\
    700 &618 &124295 &3.11 &1.38\\
    750 &602 &123299 &3.03 & 1.48\\
    800 &584 &119958 &2.93 &1.58\\
    850 &566 &120673 &2.84 &1.68\\
    900 &548 &117907 &2.75 & 1.78\\
    950 &534 &116111 &2.68&   1.88\\
    1000 &499& 108630&2.51 & 1.07\\
    \bottomrule
    \end{tabular}
  \end{table}

\noindent In Abbildung \ref{fig:plot2} ist das Verhältniss von $ \frac{U}{U_c} $ in einem
halblogarithmischen Diagramm gegen die Frequenz $ \nu $ aufgetragen.

\begin{figure}[H]
  \centering
  \includegraphics{plot2.pdf}
  \caption{Spannungsverhältniss in Abhängigkeit der Frequenz.}
  \label{fig:plot2}
\end{figure}
\noindent Um die Güte zu bestimmen, wird der Bereich um die Resonanzfrequenz
in Abbildung \ref{fig:plot3}
zudem linear dargestellt.

\begin{figure}[H]
  \centering
  \includegraphics{plot3.pdf}
  \caption{Spannungsverhältniss in Abhängigkeit der Frequenz}
  \label{fig:plot3}
\end{figure}

\noindent Hierraus lässt sich die Resonanzüberhöhung $\text{q} = 4.4791 $ ablesen. Aus den Werten des
Schaltkreises lässt sich durch die Gleichung
\begin{equation}
  q= \frac{\omega_0}{\omega_+-\omega_-}
\end{equation}
eine theoretische Resonanzüberhöhung von
\begin{equation*}
  \text{q}_{theo} = 3.923 \pm 0.009
\end{equation*}
\noindent berechnen. \\
\noindent Die abgelesenen Werte der Resonanzkurve lauten
\begin{align*}
  \nu_- =\SI{29000}{\hertz} \\ %\implies \omega_- = \SI{182212.37}{\hertz} \\
  \nu_+ =\SI{37500}{\hertz} %\implies \omega_+ = \SI{235619.45}{\hertz}
\end{align*}
woraus sich die Breite
\begin{equation*}
  \nu_+ - \nu_- = \SI{8500}{\hertz}
\end{equation*}
ergibt. Theoretisch lässt sich durch Gleichung \ref{eqn:gute}
eine Breite von $\SI{8808(27)}{\hertz} $ erechnen.
\subsection{Bestimmung der Frequenzabhängigkeit der Phasenverschiebung}

Die Messwerte zur Bestimmung der Frequnezabhängigkeit der Phasenverschiebung,
sowie die daraus erechnete Phase befinden sich in Tabelle \ref{tab:tabe3}.
\begin{table}[H]
  \centering
   \begin{tabular}{c c c}
    \toprule
     n& $\nu$/\; 1/s & $\nu_{Wechsel}$\\
    \midrule
    0,5 & 100.01& 50,0\\
    1 & 79.93 & 79.93\\
    2 & 23.93 & 47.86\\
    \bottomrule
  \end{tabular}
  \caption{Gemessene Frequenzen der Sägezahnspannung, sowie die Daraus resultierenden Frequenzen für die
  Wechselspannung.}
  \label{tab:tab3}
\end{table}


\noindent In Abbildung \ref{fig:phase} ist die Phasenverschiebung gegen die Frequenz
in einem halblogarithmischen Diagramm aufgetragen.

\begin{figure}[H]
  \centering
  \includegraphics{plot4.pdf}
  \caption{Phasenverschiebung in Abhängigkeit der Frequenz}
  \label{fig:phase}
\end{figure}

Zudem ist in Abbildung \ref{fig:plot4} der Bereich um die Resonanzfrequenz linear
dargestellt.

\begin{figure}[H]
  \centering
  \includegraphics{plot5.pdf}
  \caption{Phasenverschiebung in Abhängigkeit der Frequenz}
  \label{fig:plot4}
\end{figure}

\noindent Hieraus lassen sich die Resonanzfrequenz
\begin{align*}
  \nu_{res}= \SI{33600}{\hertz}
\end{align*}
sowie die beiden Frequenzen
\begin{align*}
  \nu_1 = \SI{29600}{\hertz} \\
  \nu_2 = \SI{38000}{\hertz}
\end{align*}
bei welchen die Phasenverschiebung gerade $\phi = \frac{\pi}{4}$ bzw. $\phi= \frac{3 \cdot \pi}{4}$
ist, ablesen. \\
\noindent Die nach Gleichungen \ref{eqn:res} und
\begin{equation}
  \omega_{1,2} = \pm \frac{R}{2L} + \sqrt{\frac{R^2}{4L^2} + \frac{1}{LC}}
\end{equation}
berechneten Theoriewerte lauten
\begin{align*}
  \nu_{res}= \SI{33990(70)}{\hertz} \\
  \nu_1 = \SI{30430(60)}{\hertz} \\
  \nu_2 = \SI{39240(80)}{\hertz}
\end{align*}
=======
\section{Auswertung}
\subsection{Bestimmung des Dämpfungswiderstands}
Die Werte des im Versuch verwendeten Schwingkreises (Gerät 2) lauten
\begin{align*}
  L &= \SI{10.11(3)}{\milli\henry} \\
  C &= \SI{2.098(6)}{\nano\farad} \\
  R_1 &= \SI{48.1 (1)}{\ohm} \\
  R_2 &= \SI{509.5(5)}{\ohm} \\
\end{align*}
\noindent Der Spannungsverlauf, der zur Messung verwendet wurde ist in Abbildung \ref{fig:fig1}
mit eingezeicneter Einhüllender zu sehen.
\begin{figure}[H]
  \centering
  \includegraphics[height=7cm]{Schwingung.JPG}
  \caption{Schwingungsverlauf des Oszilloskops mit Einhüllender}
  \label{fig:fig1}
\end{figure}

\noindent Die sich hieraus ergebenden Wertepaare aus Kondensatorspannung $\text{U}_c$
und Zeit t befinden sich, getrennt nach Minima und Maxima, in Tabelle \ref{tab:tabe1}

\begin{table}[H]
  \centering
  \caption{Messwerte und Ergebniss der Bestimmung der Schallgeschwindigkeit}
  \label{tab:tabe1}
    \begin{tabular}{S||S S||S S||S|S}
    \toprule
    $ \text{Länge l des Zylinders [mm]} $ & $ U_{1} [\text{V}] $ &
    $ t_{1} [\mu\text{s}] $ & $ U_{2} [\text{V}] $ &
    $ t_{2} [\mu\text{s}] $ & $ \increment t [\mu\text{s}]$ &
    $ \text{c} [\text{m}/\text{s}]$\\
    \midrule
    31.0 & 1.335 \: & 24.0 & 1.096 \:  & 46.9 & 22.9 & 2707.42 \\
          \bottomrule
    \end{tabular}
  \end{table}


\noindent In Abbildung \ref{fig:plot1} sind die Messwerte zusammen mit der jeweils erechnete Ausgleichsfunktion
zu sehen, welche sich durch eine Ausgleichsrechnung mt der Funktion
\begin{equation}
  A(t)= A_0 \cdot \exp{-2\pi \cdot \mu \cdot t}
\end{equation}
ergibt.
\begin{figure}[H]
  \centering
  \includegraphics{plot1.pdf}
  \caption{Messwerte und Ausgleichsfunktionen der ersten Messung}
  \label{fig:plot1}
\end{figure}

\noindent Hieraus ergeben sich bei den Minima die Parameter
\begin{align*}
  A_0 &= \SI{-171.45(71)}{\volt} \\
  \mu&= \SI{1011.7(66)}{1\per\second}
\end{align*}
und bei den Maxima
\begin{align*}
  A_0 &= \SI{169.3(16)}{\volt} \\
  \mu&= \SI{980(13)}{1\per\second}
\end{align*}
sodass sich für $ \mu $ insgesamt ein Wert von
\begin{align*}
  \mu &= \SI{996(7)}{1\per\second}
\end{align*}
ergibt.
Hieraus ergibt sich durch die Gleichungen \ref{eqn:eqn:lös} und \ref{eqn:tex}
\begin{align*}
  R= \SI{126.5(10)}{\ohm} \\
  T= \SI{159.9(12)}{\micro\second} \: .
\end{align*}
Die Fehler berechnen sich hierbei durch die Gauß´sche Fhlerfortpflanzung
\begin{equation}
  \increment f = \sqrt{ \sum_{i=1}^N \left( \frac{\partial f}{\partial x_i}\right)^2
  \cdot (\increment x_i)^2  } \: .
  \label{eqn:gaus}
\end{equation}
\subsection{Bestimmung des Dämpfungswiderstands mit dem Aperiodischen Grenzfall}

Die gemessenen Werte des Widerstands $R_{ap}$ beim aperiodischen Grenzfall lauten
$\SI{3360}{\ohm}$, $\SI{3330}{\ohm}$ und $\SI{3370}{\ohm}$ , sodass sich hierbei ein Mittelwert
von $\SI{3353(17)}{\ohm}$ ergibt.
\noindent Theoretisch berechnet ergibt sich durch die Formel \ref{eqn:bed3}
ein Wert von
\begin{equation*}
  \text{R}_{ap}= 2 \cdot \sqrt{\frac{L}{C}} = \SI{4390(9)}{\ohm}
\end{equation*}


\subsection{Bestimmung der Frequenzabhängigkeit der Kondensatorspannung}

Die Messwerte zur Bestimmung der Frequenzabhängigkeit befinden sich in Tabelle \ref{tab:tabe2}.

\begin{table}[H]
  \centering
  \caption{Zählrate und Energiemaximum bei variiertem Druck, Abstand a=2cm}
  \label{tab:tab2}
    \begin{tabular}{c c c c c}
    \toprule
    Druck $\rho$/\;mbar & Energiemaximum & Zählrate $N$ & Energie $E_{\alpha}$ & effektive Länge $x$/\;cm\\
    \midrule
    0 & 796 &131382  &4          & 0.0   \\
    50 & 775 &131464 &3.89 & 0.09 \\
    100 &756 &130732 &3.79 & 0.19\\
    150 &749 &129617 &3.76  &  0.29\\
    200 &749 &130444 &3.76  & 0.39\\
    250 &727 &129600 &3.65 & 0.49\\
    300 &722 &128936 &3.63 & 0.59\\
    350 &708 &128478 &3.56 & 0.69\\
    400 &696 &128122 &3.49 & 0.79\\
    450 &687 &127415 &3.45 & 0.89\\
    500 &674 &126608 &3.39 & 0.99\\
    550 &663 &126372 &3.33 &1.09\\
    600 &651 &124989 &3.27 & 1.18\\
    650 &634 &124942 &3.19 & 1.28\\
    700 &618 &124295 &3.11 &1.38\\
    750 &602 &123299 &3.03 & 1.48\\
    800 &584 &119958 &2.93 &1.58\\
    850 &566 &120673 &2.84 &1.68\\
    900 &548 &117907 &2.75 & 1.78\\
    950 &534 &116111 &2.68&   1.88\\
    1000 &499& 108630&2.51 & 1.07\\
    \bottomrule
    \end{tabular}
  \end{table}

\noindent In Abbildung \ref{fig:plot2} ist das Verhältniss von $ \frac{U}{U_c} $ in einem
halblogarithmischen Diagramm gegen die Frequenz $ \nu $ aufgetragen.

\begin{figure}[H]
  \centering
  \includegraphics{plot2.pdf}
  \caption{Spannungsverhältniss in Abhängigkeit der Frequenz.}
  \label{fig:plot2}
\end{figure}
\noindent Um die Güte zu bestimmen, wird der Bereich um die Resonanzfrequenz
in Abbildung \ref{fig:plot3}
zudem linear dargestellt.

\begin{figure}[H]
  \centering
  \includegraphics{plot3.pdf}
  \caption{Spannungsverhältniss in Abhängigkeit der Frequenz}
  \label{fig:plot3}
\end{figure}

\noindent Hierraus lässt sich die Resonanzüberhöhung $\text{q} = 4.4791 $ ablesen. Aus den Werten des
Schaltkreises lässt sich durch die Gleichung
\begin{equation}
  q= \frac{\omega_0}{\omega_+-\omega_-}
\end{equation}
eine theoretische Resonanzüberhöhung von
\begin{equation*}
  \text{q}_{theo} = 3.923 \pm 0.009
\end{equation*}
\noindent berechnen. \\
\noindent Die abgelesenen Werte der Resonanzkurve lauten
\begin{align*}
  \nu_- =\SI{29000}{\hertz} \\ %\implies \omega_- = \SI{182212.37}{\hertz} \\
  \nu_+ =\SI{37500}{\hertz} %\implies \omega_+ = \SI{235619.45}{\hertz}
\end{align*}
woraus sich die Breite
\begin{equation*}
  \nu_+ - \nu_- = \SI{8500}{\hertz}
\end{equation*}
ergibt. Theoretisch lässt sich durch Gleichung \ref{eqn:gute}
eine Breite von $\SI{8808(27)}{\hertz} $ erechnen.
\subsection{Bestimmung der Frequenzabhängigkeit der Phasenverschiebung}

Die Messwerte zur Bestimmung der Frequnezabhängigkeit der Phasenverschiebung,
sowie die daraus erechnete Phase befinden sich in Tabelle \ref{tab:tabe3}.
\begin{table}[H]
  \centering
   \begin{tabular}{c c c}
    \toprule
     n& $\nu$/\; 1/s & $\nu_{Wechsel}$\\
    \midrule
    0,5 & 100.01& 50,0\\
    1 & 79.93 & 79.93\\
    2 & 23.93 & 47.86\\
    \bottomrule
  \end{tabular}
  \caption{Gemessene Frequenzen der Sägezahnspannung, sowie die Daraus resultierenden Frequenzen für die
  Wechselspannung.}
  \label{tab:tab3}
\end{table}


\noindent In Abbildung \ref{fig:phase} ist die Phasenverschiebung gegen die Frequenz
in einem halblogarithmischen Diagramm aufgetragen.

\begin{figure}[H]
  \centering
  \includegraphics{plot4.pdf}
  \caption{Phasenverschiebung in Abhängigkeit der Frequenz}
  \label{fig:phase}
\end{figure}

Zudem ist in Abbildung \ref{fig:plot4} der Bereich um die Resonanzfrequenz linear
dargestellt.

\begin{figure}[H]
  \centering
  \includegraphics{plot5.pdf}
  \caption{Phasenverschiebung in Abhängigkeit der Frequenz}
  \label{fig:plot4}
\end{figure}

\noindent Hieraus lassen sich die Resonanzfrequenz
\begin{align*}
  \nu_{res}= \SI{33600}{\hertz}
\end{align*}
sowie die beiden Frequenzen
\begin{align*}
  \nu_1 = \SI{29600}{\hertz} \\
  \nu_2 = \SI{38000}{\hertz}
\end{align*}
bei welchen die Phasenverschiebung gerade $\phi = \frac{\pi}{4}$ bzw. $\phi= \frac{3 \cdot \pi}{4}$
ist, ablesen. \\
\noindent Die nach Gleichungen \ref{eqn:res} und
\begin{equation}
  \omega_{1,2} = \pm \frac{R}{2L} + \sqrt{\frac{R^2}{4L^2} + \frac{1}{LC}}
\end{equation}
berechneten Theoriewerte lauten
\begin{align*}
  \nu_{res}= \SI{33990(70)}{\hertz} \\
  \nu_1 = \SI{30430(60)}{\hertz} \\
  \nu_2 = \SI{39240(80)}{\hertz}
\end{align*}
>>>>>>> michelson
