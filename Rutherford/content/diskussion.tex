\section{Diskussion}
\label{sec:Diskussion}
Die Abweichung des experimentell bestimmten Wertes der Dicke der Goldfolie von dem
Theoriewert von $\symup{dx}_{\text{theo}}=\SI{2}{\micro\meter}$
lässt sich gemäß der Gleichung \ref{eqn:abw}

zu 62,5\% bestimmen. Der Theoriewert liegt innerhalb von 3 Standardabweichungen,
sodass von einem systematischen Fehler bei der Messung auszugehen ist. Ein Grund für diese
Abweichungen könnten die Schwierigkeiten beim Ablesen des Oszilloskops sein, da es zum Teil
schwierig war eine mittlere Pulshöhe festzulegen, durch das hohe schwanken der
Pulse. Zudem war es auch nicht möglich, den Druck über den jeweiligen Messprozess mithilfe
des Feindosierventils konstant zu halten, sodass es auch hier zu Abweichungen kommen kann.
\\
Bei der Messung des differentiellen Wirkungsquerschnitts ist zunächst an der Abbildung
\ref{fig:plot3} zu erkennen, dass die Ausrichtung des Detektors nicht exakt war, sondern das
Maximum, welches bei 0° liegen sollte bei etwa 2° liegt. Zudem lässt sich erkennen,
dass die experimentellen Werte bei kleinen Winkeln unter den Theoriewerten liegen, wohingegen
sie bei großen Winkeln oberhalb der Theoriewerte liegen. Ein wichtiger Grund hierfür ist, dass die Alpha-Strahlung
nicht vollständig kollimiert ist, das heißt sie trifft nicht als paralleler Strahl bei 0° auf den Detektor,
sondern öffnet sich Strahlförmig hinter der Blende. Somit gibt es keinen scharfen Peak bei 0°,
sondern eine breitere, gleicgförmigere Verteilung als erwartet. Zudem wurde auch der Einfluss
von Mehrfachstreuung bei der Bestimmung des Wirkungsquerschnitts nicht berücksichtigt. \\
Bei der Untersuchung der Z-Abhängigkeit ergab sich eine Abweichung von der erwarteten
Potenz von 40\%, der Wert lag jedoch noch innerhalb einer Standardabweichungen, sodass diese Messung recht
genau war im Verhältniss zu den Messwerten der vorherigen Auswertungsteilen. Aufgrund der geringen Datenmenge mit nur 3
Messwerten ist es hier jedoch schwierig allgemeine Aussagen zu treffen. 
