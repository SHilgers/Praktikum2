\section{Auswertung}
\label{sec:Auswertung}
\subsection{Bestimmung der Goldfoliendicke mittels Energieverlustmessung}
Die gemessenen Werte zur Bestimmung der Foliendicke sind in Tabelle \ref{tab:tabe1} angegeben.
\begin{table}[H]
  \centering
  \caption{Messwerte und Ergebniss der Bestimmung der Schallgeschwindigkeit}
  \label{tab:tabe1}
    \begin{tabular}{S||S S||S S||S|S}
    \toprule
    $ \text{Länge l des Zylinders [mm]} $ & $ U_{1} [\text{V}] $ &
    $ t_{1} [\mu\text{s}] $ & $ U_{2} [\text{V}] $ &
    $ t_{2} [\mu\text{s}] $ & $ \increment t [\mu\text{s}]$ &
    $ \text{c} [\text{m}/\text{s}]$\\
    \midrule
    31.0 & 1.335 \: & 24.0 & 1.096 \:  & 46.9 & 22.9 & 2707.42 \\
          \bottomrule
    \end{tabular}
  \end{table}

Die ersten 6 Messwerte ohne Folie zeigen eine deutliche Abweichung von dem erwarteten linearen
Verhalten und werden somit in der weiteren Auswertung nicht berücksichtigt, da hier offensichtlich ein
systematischer Fehler in der Messung besteht.
Die verbleibenden Werten werden an eine lineare Ausgleichsgerade
der Form
\begin{equation*}
  U(p)= a\cdot p+b
\end{equation*}
angepasst, woraus sich die Parameter
\begin{align*}
  a_0 &= \SI{-58(2)e-5}{\volt\per\pascal} \\
  a_{\text{Folie}} &= \SI{-62.0(11)e-5}{\volt\per\pascal} \\
  b_0 &= \SI{19.6(5)}{\volt} \\
  b_{\text{Folie}} &= \SI{14.87(16)}{\volt} \\
\end{align*}
ergeben.
Die Ausgleichsgrade ist zusammen mit den Messwerten in Abbildung \ref{fig:plot1} dargestellt.
\begin{figure}
  \centering
  \includegraphics[height=9cm]{Plot1.pdf}
  \caption{Messwerte zur Bestimmung der Foliendicke sowie Ausgleichsgraden}
  \label{fig:plot1}
\end{figure}
Mittels Extrapolation wird der y-Achsenabschnitt ermittelt, welcher durch den Parameter~$b$
gegeben ist und der Impulshöhe im Vakuum entspricht. \\
Zur Bestimmung des Energieverlustes durch die Goldfolie wird die Differenz zwischen
der Vakuumimpulshöhe mit Folie und ohne Folie
\begin{equation}
  b_{\text{diff}} = b_0 - b_{\text{Folie}}
\end{equation}
gebildet, wobei sich der Fehler gemäß der Gaußschen Fehlerfortpflanzung
\begin{equation}
  \increment f = \sqrt{ \sum_{i=1}^N \left( \frac{\partial f}{\partial x_i}\right)^2
  \cdot (\increment x_i)^2  }
  \label{eqn:gaus}
\end{equation}
über
\begin{equation}
    \increment b_{\text{diff}} = \sqrt{(\increment b_0)^2 + (\increment b_{\text{Folie}})^2}
\end{equation}
ergibt.
Hierdurch ergibt sich ein Wert von
\begin{align*}
  b_{\text{diff}} = \SI{4.8(5)}{\volt} \: .
\end{align*}
Die Impulshöhe ${b_0\,=\,\SI{19.6(5)}{\volt}}$ im Vakuum entspricht der Energie von
${E_0\,=\,\SI{5.638}{\mega\electronvolt}}$ \cite{online3}, wobei zwischen Energie und Impulshöhe ein linearer Zusammenhang
besteht. Somit lässt sich aus diesen Werten die Energiedifferenz
\begin{equation}
  E_{\text{diff}} =   b_{\text{diff}} \frac{E_0}{b_0}
\end{equation}
mit dem Fehler
\begin{equation}
  \increment E_{\text{diff}} =\sqrt{(\frac{E_0}{b_0} \increment b_{\text{diff}})^2
  + (b_{\text{diff}} \frac{E_0}{b_0^2} \increment b_0)^2}
\end{equation}
zu
\begin{align*}
  E_{\text{diff}} = \SI{1.37(11)}{\mega\electronvolt} \: .
\end{align*}
Die materialspezifischen Werte von Gold lauten \cite{online1}:
\begin{align*}
  \symup{Z} &= 79 \\
  \symup{m}_{\symup{atomar}} &= 196,97 \symup{u} \\
  \rho &= \SI{19.282}{\g\per\cubic\centi\metre} \\
  \symup{I} &= \SI{9.225}{\electronvolt}
\end{align*}
mit der atomaren Masseneinheit \cite{online4}
\begin{align*}
  \symup{u}= \SI{1.660539e-27}{\kilo\gram}
\end{align*}
Hieraus lässt sich mit der Formel
\begin{equation}
  N=\frac{\rho}{\symup{m}_{\symup{atomar}}}
\end{equation}
die Teilchenzahldichte zu
\begin{align*}
  N=\SI{5.895e28}{\per\cubic\metre}
\end{align*}
bestimmen.

Mit der Masse
\begin{align*}
  \symup{m}_{\alpha}=\SI{6.64e-27}{\kilo\gram}
\end{align*}
der Alphateilchen \cite{online2}
lässt sich die Geschwindigkeit $v$ über die Formel
\begin{equation}
  v =\sqrt{\frac{2 E_0}{\symup{m}_{\alpha}}}
\end{equation}
zu
\begin{align*}
  v =0,055 \cdot \symup{c}
\end{align*}
berechnen mit der Vakuumlichtgeschwindigkeit \cite{online4}
\begin{align*}
  \symup{c}=\SI{299792458}{\metre\per\second} \:.
\end{align*}
Durch Umstellen der Gleichung \eqref{eqn:BetheBloch}
ergibt sich somit insgesamt eine Foliendicke von
\begin{align*}
  \symup{dx}=\SI{0.75(6)}{\micro\metre} \: ,
\end{align*}
wobei sich der Fehler gemäß Gleichung \eqref{eqn:gaus} durch
\begin{equation}
  \increment dx=\increment dE \frac{m_e v^2 (4\pi \epsilon_0)^2}{\ln\Bigg(\frac{2m_e v²}{I}\Bigg)4\pi e^4 z²NZ}
\end{equation}
berechnen lässt.
%mit der Elektronenmasse \cite{online4}
%\begin{align*}
%  \symup{m}_{\symup{e}}= \SI{9.11e-31}{\kilo\gram} \: ,
%\end{align*}
%der elektrischen Ladung
%\begin{align*}
%  \symup{e
%\end{align*}






\subsection{Differentieller Streuquerschnitt}
Die gemessenen Werte zur Bestimmung des differentiellen Wirkungsquerschnitts sind in
Tabelle \ref{tab:tabe2} angegeben, wobei die Messzeit jeweils 6 Minuten beträgt. Der
Fehler auf die Anzahl der gemessenen Counts $N$ ergibt sich gemäß der Poisson-Verteilung durch
\begin{align*}
  \increment N = \sqrt{N} \:.
\end{align*}
\begin{table}[H]
  \centering
  \caption{Zählrate und Energiemaximum bei variiertem Druck, Abstand a=2cm}
  \label{tab:tab2}
    \begin{tabular}{c c c c c}
    \toprule
    Druck $\rho$/\;mbar & Energiemaximum & Zählrate $N$ & Energie $E_{\alpha}$ & effektive Länge $x$/\;cm\\
    \midrule
    0 & 796 &131382  &4          & 0.0   \\
    50 & 775 &131464 &3.89 & 0.09 \\
    100 &756 &130732 &3.79 & 0.19\\
    150 &749 &129617 &3.76  &  0.29\\
    200 &749 &130444 &3.76  & 0.39\\
    250 &727 &129600 &3.65 & 0.49\\
    300 &722 &128936 &3.63 & 0.59\\
    350 &708 &128478 &3.56 & 0.69\\
    400 &696 &128122 &3.49 & 0.79\\
    450 &687 &127415 &3.45 & 0.89\\
    500 &674 &126608 &3.39 & 0.99\\
    550 &663 &126372 &3.33 &1.09\\
    600 &651 &124989 &3.27 & 1.18\\
    650 &634 &124942 &3.19 & 1.28\\
    700 &618 &124295 &3.11 &1.38\\
    750 &602 &123299 &3.03 & 1.48\\
    800 &584 &119958 &2.93 &1.58\\
    850 &566 &120673 &2.84 &1.68\\
    900 &548 &117907 &2.75 & 1.78\\
    950 &534 &116111 &2.68&   1.88\\
    1000 &499& 108630&2.51 & 1.07\\
    \bottomrule
    \end{tabular}
  \end{table}

Die Counts pro Sekunde sind in Abhängigkeit des Winkel in Abbildung \ref{fig:plot2}
dargestellt.
\begin{figure}
  \centering
  \includegraphics[height=9cm]{Plot2.pdf}
  \caption{Messwerte zur Messung der Winkelabhängigkeit}
  \label{fig:plot2}
\end{figure}
Zur Bestimmung des Wirkungsquerschnitts wird der Raumwinkelanteil $\symup{\Omega}$ benötigt,
welcher gegeben ist durch das Verhältniss der abgedeckten Fläche zu der Oberfläche einer Kugel, deren Radius
durch den Abstand zur Quelle gegeben ist, also
\begin{equation}
  \symup{\Omega}= \frac{A}{4\pi r^2} \:.
\end{equation}
Die Fläche ist hier durch die aktive Detektorfläche mit ${A=\SI{1}{\centi\metre\squared}}$ \cite{online5}
gegeben und der Abstand zur Quelle beträgt ${r=\SI{10.1}{\centi\metre}}$.
Somit ergibt sich
\begin{align*}
  \symup{\Omega}= 7,8\cdot10^{-4} \: .
\end{align*}
Die Zerfallskonstante von $\ce{^{241}Am}$ \cite{online3} beträgt
\begin{align*}
  \lambda_{\text{Am}}=\SI{50.77(7)e-12}{\per\second} \: ,
\end{align*}
sodass sich mit der Formel
\begin{equation}
  \symup{A(t)}=\symup{e}^{-\lambda_{\text{Am}}t}\cdot \symup{A}_0
\end{equation}
eine Aktivität von
\begin{align*}
  \symup{A}=\SI{317}{\kilo\becquerel}
\end{align*}
am Messtag ergibt, wobei der Fehler aufgrund der Unsicherheit auf $\lambda_{\text{Am}}$
vernachlässigbar ist. Der Zeitraum $t$ ist dabei die Zeit zwischen
Oktober 1994 und dem Messtag, also ca. 25 Jahre und die ursprüngliche Aktivität
betrug ${\symup{A}_0=\SI{330}{\kilo\becquerel}}$ \cite{skript}.

%Der Raumwinkel, der durch einen rechteckigen Spalt abgedeckt ist, ist über die Formel
%\begin{equation}
%  \symup{\Omega}=4\text{arctan}(\frac{x\cdot y}{2r\sqrt{4r^2+x^2+y^2}})
%\end{equation}
%gegeben \cite{online6} mit den Spaltabmessungen $x$ und $y$, sowie dem Abstand zur Quelle
%$r$. Mit den Abmessungen ${x=\SI{2}{\milli\metre}}$, ${y=\SI{10}{\milli\metre}}$ und
%${r=\SI{39}{\milli\metre}}$ \cite{skript} ergibt sich somit
%\begin{align*}
%  \symup{\Omega}= 0.013 \: .
%\end{align*}
Somit lässt sich der differentielle Wirkungsquerschnitt gemäß der Formel
\begin{equation}
  \frac{d\sigma}{d\Omega}= \frac{A_{\text{exp}}}{dx\cdot \symup{A} N \cdot \symup{\Omega}^2}
  \label{eqn:Wq}
\end{equation}
berechnen. Dabei ist $A_{\text{exp}}$ die gemessene Aktivität, also die gemessenen Counts
pro Sekunde. Für die Foliendicke $dx$ wird der Theoriewert $\SI{2}{\micro\metre}$ verwendet,
$\symup{A}=\SI{317}{\kilo\becquerel}$ ist die Aktivität der Quelle und N die Teilchenzahldichte.
Der Fehler ergibt sich über die Gauß´sche Fehlerfortpflanzung \eqref{eqn:gaus} durch
\begin{equation}
   \increment \frac{d\sigma}{d\Omega}= \frac{1}{dx\cdot \symup{A} N \cdot \symup{\Omega}^2}\increment A_{\text{exp}} \: .
   \label{eqn:dWq}
\end{equation}
Die entsprechenden Theoriewerte lassen sich durch die Formel \eqref{eqn:Rutherford} berechnen und sind in Tabelle
\ref{tab:tabe4} angegeben, zusammen mit den experimentell ermittelten Werten und den jeweiligen Abweichungen,
welche sich über die Gleichung
\begin{equation}
  \frac{\lvert \text{Wert}_{\text{Theorie}}-\text{Wert}_{\text{Messung}}\rvert}{\text{Wert}_{\text{Theorie}}}
  \label{eqn:abw}
\end{equation}
berechnen lassen.
\begin{table}[H]
  \centering
   \begin{tabular}{c c c c}
    \toprule
    Nummer der Oberwelle & $ U_{\text Theorie,Rechteck}\: / \si{\volt} $ &
    $ U_{\text Theorie,Dreick}\: / \si{\volt} $ & $ U_{\text Theorie,Sägezahn}\: / \si{\volt} $ \\
    \midrule
    1 & 1145 & 182 & 573 \\
    2 & 0 & 0 & 286 \\
    3 & 573 & 20 & 191 \\
    4 & 0 & 0 & 143 \\
    5 & 229 & 7 & 115 \\
    6 & 0 & 0 & 96 \\
    7 & 164 & 4 & 82 \\
    8 & 0 & 0 & 72 \\
    9 & 127 & 2 & 64 \\
    10 & 0 & 0 & 57 \\
    \bottomrule
  \end{tabular}
  \caption{Eingestellte Schwingungsamplituden.}
  \label{tab:tabe4}
\end{table}

Die beiden unterschiedlich bestimmten Wirkungsquerschnitte sind zudem in Abbildung \ref{fig:plot3}
in Abhängigkeit des Winkels dargestellt. Bei einem Winkel von 0° lässt sich kein theoretischer Wert angeben,
da hier der Faktor $\frac{1}{\symup{sin(\phi)}}$ divergiert.
\begin{figure}
  \centering
  \includegraphics[height=9cm]{Plot3.pdf}
  \caption{Vergleich zwischen experimentell bestimmten Wirkungsquerschnitt und dem jeweiligen Theoriewert}
  \label{fig:plot3}
\end{figure}


\subsection{Z-Abhängigkeit}
Die Werte zur Bestimmung der Z-Abhängigkeit wurden mit einer Messzeit von $\SI{360}{\second}$
bei einem Winkel von 7° gemessen und sind in Tabelle
\ref{tab:tabe3} zusammen mit den materialspezifischen Werten angegeben.
\begin{table}[H]
  \centering
   \begin{tabular}{c c c}
    \toprule
     n& $\nu$/\; 1/s & $\nu_{Wechsel}$\\
    \midrule
    0,5 & 100.01& 50,0\\
    1 & 79.93 & 79.93\\
    2 & 23.93 & 47.86\\
    \bottomrule
  \end{tabular}
  \caption{Gemessene Frequenzen der Sägezahnspannung, sowie die Daraus resultierenden Frequenzen für die
  Wechselspannung.}
  \label{tab:tab3}
\end{table}

Die Fehler ergeben sich auch hier über das $\sqrt{\text{N}}$-Gesetz der Poisson-Verteilung.
Analog zum Vorgehen im vorherigen Auswertungsteil, lassen sich für die unterschiedlichen
Materialien jeweils der experimentell bestimmte Wirkungsquerschnitt über Gleichungen \eqref{eqn:Wq}
und \eqref{eqn:dWq} bestimmen, sowie der theoretische Wert durch Gleichung \eqref{eqn:Rutherford}.
Die somit berechneten Werte sind zusammen mit der jeweiligen Abweichung gemäß
Gleichung \eqref{eqn:abw} in Tabelle \ref{tab:tabe5}
angegeben.\\
\begin{table}[H]
  \centering
  \caption{Mechanischen Kompressorleistung zu den Zeiten $t_1$, $t_2$, $t_3$ und $t_4$.}
  \label{tab:tabe5}
    \begin{tabular}{S S}
    \toprule
    $ t  \: / \si{\second} $ & $ N_{\text{mech}} \: / \: \si{\watt}$ \\
    \midrule
    480 & 4.72 \pm 0.16 \\
    960 & 6.19 \pm 0.22 \\
    1500 & 6.67 \pm 0.26 \\
    1980 & 6.26 \pm 0.28 \\
      \bottomrule
    \end{tabular}
\end{table}

Die Werte sind doppellogarithmisch in Abhängigkeit der Kernladungszahl Z in Abbildung \ref{fig:plot5}
dargestellt.
Nach Gleichung \eqref{eqn:Rutherford} ist ein quadratisches Anwachsen des Wirkungsquerschnittes
mit der Kernladungszahl zu erwarten, also ${\frac{d\sigma}{d\Omega} \sim \symup{Z}^2}$.
Doppellogarithmisch ergibt sich daraus eine Gerade mit Steigung 2, wie es auch
an den theoretischen Werten zu erkennen ist. Die experimentellen Werte steigen zwar auch mit
wachsender Kernladungszahl, jedoch ist keine Gerade zu erkennen, da insbesondere der Wert für Gold
zu niedrig ist.
Durch eine lineare Ausgleichsrechnung der doppellogarithmischen Werte ergeben sich die Parameter
\begin{align*}
  a =  1,2 \pm 0,9 \\
  b =  -56 \pm 3 \: ,
\end{align*}
wie in Abbildung \ref{fig:plot5} dargestellt ist und somit ergibt sich eine Abweichung von 40\% vom Theoriewert 2 der Steigung a. Wird der Wert
von Gold vernachlässigt und nur die Werte von Aluminium und Bismut mit einer Gerade verbunden
ergibt sich eine Steigung von 1,73 und somit sinkt die Abweichung auf 13,5\%.
\begin{figure}
  \centering
  \includegraphics[height=9cm]{Plot5.pdf}
  \caption{Z-Abhängigkeit des Wirkungsquerschnittes}
  \label{fig:plot5}
\end{figure}
