\begin{table}[H]
  \centering
  \caption{Vergleich zwischen experimentell bestimmten und theoretisch berechnetem Wirkungsquerschnitt}
  \label{tab:tabe4}
    \begin{tabular}{c c c c c}
    \toprule \\
    $\text{Winkel/}°$ & $\frac{d\sigma}{d\Omega}_{\text{exp}}/\SI{e-24}{\metre\squared}$
    & $\frac{d\sigma}{d\Omega}_{\text{theo}}/\SI{e-24}{\metre\squared}$ & Abweichung \\
    \hline
    -12 & 1,2 \pm 0,4 & 0,8 & 43,4 \% \\
    -11 & 3,4 \pm 0,6 & 1,2 & 183,7 \% \\
    -10 & 2,6 \pm 0,6 & 1,8 & 45,5 \% \\
    -9 & 5,1 \pm 0,8 & 2,7 & 91,1 \% \\
    -8 & 8,1 \pm 1,0 & 4,3 & 87,7 \% \\
    -7 & 16,5 \pm 1,4 & 7,3 & 125,2 \% \\
    -6 & 21,6 \pm 1,6 & 13,6 & 59,4 \% \\
    -5 & 30,4 \pm 1,9 & 28,1 & 8,2 \% \\
    -4 & 48,9 \pm 2,4 & 68,6 & 28,7 \% \\
    -3 & 62,5 \pm 2,8 & 216,7 & 71,1 \% \\
    -2 & 83 \pm 3 & 1097 & 92,5 \% \\
    -1 & 137 \pm 4 & 17542 & 99,2 \% \\
    0 & 178 \pm 5 & / & / \\
    1 & 188 \pm 5 & 17542 & 98,9 \% \\
    2 & 197 \pm 5 & 1097 & 82,0 \% \\
    3 & 181 \pm 5 & 217 & 16,6 \% \\
    4 & 151 \pm 4 & 69 & 120,5 \% \\
    5 & 129 \pm 4 & 28 & 359,1 \% \\
    6 & 91 \pm 3 & 14 & 573,0 \% \\
    7 & 69,2 \pm 2,9 & 7,3 & 845,7 \% \\
    8 & 53,0 \pm 2,5 & 4,3 & 1134,0 \% \\
    9 & 33,0 \pm 2,0 & 2,7 & 1128,6 \% \\
    10 & 22,2 \pm 1,6 & 1,8 & 1161,1 \% \\
    11 & 14,3 \pm 1,3 & 1,2 & 1085,6 \% \\
    12 & 9,3 \pm 1,1 & 0,9 & 989,5 \% \\
    14 & 3,4 \pm 0,6 & 0,5 & 641,7 \% \\
    16 & 2,1 \pm 0,5 & 0,2 & 665,9 \% \\


          \bottomrule
        \end{tabular}
\end{table}
