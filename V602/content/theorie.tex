\section{Theorie}
\label{sec:Theorie}
\subsection{Zielsetzung}
In diesem Versuch wird mit Hilfe von Röntgenstrahlung (ca. $\SI{10}{\eV}$ bis $\SI{100}{\eV}$)
das Emissionsspektrum einer
Cu-Röntgenröhre bestimmt. Außerdem werden die Absoptionsspektren verschiedener
Materialien untersucht.

\subsection{Grundlagen}
Wenn Elektronen in einer evakuierten Röhre aus einer Glühkathode ausgelöst und
dann zu Anode beschleunigt werden, wo sie auf das Anodenmaterial treffen, entsteht
Röntgenstrahlung. Diese besteht aus dem kontinuierlichen Bremsspektrum und dem
charakteristischen Spektum des Anodenmaterials.

\subsection{Emission von Röntgenstrahlung}

Im Coulombfeld des Atomkerns wird das Elektron abgebremst, dabei wird ein
Photon (Röntgenquant) ausgesendet, dass so entstehende Spekrum wird als Bremsspektrum
bezeichnet, da das Elektron sowohl einen Teil seiner Energie, als auch
seine gesamte Engie abgeben kann. Deshalb ist das Bremsspektrum ein kontinuierliches
Spektrum, wie in Abbildung \ref{fig:kont} zu sehen. Es hat die maximalen Energie bzw. die minimale Wellenlänge:
\begin{equation}
  \lambda_{min}=\frac{h\cdot c}{e_0 U}
  \label{lamdbamin}
\end{equation}
die bei vollständiger Abbremsung des Elektrons entsteht. Also wird die gesamte
kinetische Energie $E_{kin}=E_{0} \cdot U$ in die  Strahlungsenergie $E=h\cdot \nu$
umgewandelt.
\begin{figure}
  \centering
  \includegraphics[height=4cm]{kont.png}
  \caption{Kontinuierliches Bremsspektrum.}
  \label{fig:kont}
  \cite{skript}
\end{figure}

Trifft ein beschleunigtes Elektron genau auf ein Hüllenelektron der Anode,
wird dieses Hüllenelektron ausgelöst. Nun kann ein Elektron einer höheren Schale den
freien Platz besetzten, bei dem Übergang wird Röntgenstrahlung der Energie
\begin{equation}
  h \nu=E_n -E_m
\end{equation}
freigesetzt. Da die Energiezustände der Hüllenelektronen quantisiert sind, ist auch die
Energie der Röntgenstrahlng quantisiert. Diese Strahlung ist charakteristisch für das
Anodenmatrial und wird dementsprechend charakteristische Strahung genannt.
Die einzelnen charakteristischen Linien werden mit $K_{\alpha}, K_{\beta}, L_{\alpha}...$
bezeichnet, wobei $K$,$L$,$M$...die Schale angibt auf dem das Elektron endet, während
der Index angibt von welcher Schale das Elektron kommt. Das charakteristische Spektrum
ist dem Bremsspektrum überlagert, dies ist gut in Abbildung \ref{fig:kont2} zu sehen.
\begin{figure}
  \centering
  \includegraphics[height=5cm]{kont2.png}
  \caption{Bremsspektrum überlagert mit kontinuierlichem Spektrum.}
  \label{fig:kont2}
  \cite{skript}
\end{figure}
\\
Bei größeren Atomen wird die Coulobanziehung des Kerns durch die anderen Hüllenelektronen
abgeschiermt, deshalb wir eine effektive Kernladungszahl $z_{eff}=z-\sigma$
eingeführt. Damit folgt für die Bindungsenergie eines Hüllenelektrons:
\begin{equation}
  E_n= -R_{\infty}{z_{eff}}^{2}\cdot \frac{1}{n²},
  \label{Ebindung}
\end{equation}
wobei $\sigma$ die Abschirmkonstante ist, die sich für jedes Elektron unterscheidet.
$R_{\infty}=\SI{13.6}{\eV}$ ist die Rydbergenergie.
Aus \ref{Ebindung} lässt sich für die Energie $E_{K_{\alpha}}$ der $K_{\alpha}$-Linie
die Formel
\begin{equation}
  E_{K_{\alpha}}=R_{\infty}(z-\sigma_{1})^{2}\cdot\frac{1}{1²}-R_{\infty}(z-\sigma_{2})^{2}\cdot\frac{1}{2²}
\end{equation}
herleiten.
Jede charakteristische Linien lässt sich in eine Feinstrucktur unterteilen, da
äußere Elektronen aufgrund des Bahndrehimpulses und des Elektronenspins unterschiedlich
viel Energie besitzen. Diese Feinstrucktur wird in diesem Versuch jedoch nicht genauer
untersucht.
%In diesem Versuch wird eine Kupferanode verwendet, somit
%können die $Cu_{K_{\alpha}}$ und $Cu_{K_{\beta}}$ -Linien beobachtet werden. Diese sind der
%kontinuierlichen Bremsstrahlung überlagert. Dies ist beispielsweise in Abbildung
%\ref{fig:spektrum} dargestellt.

\subsection{Absorption von Röntgenstrahlung}\\
Für den Fall das die Energie der Röntgenstrahlung unter $\SI{1}{\mega\eV}$ liegt
sind der Photoeffekt und der Comptoneffekt die dominierenden Effekte.
Mit zunehmender Energie nimmt der Absorptionskoeffizient so lange ab, bis
er plötzlich sprunghaft ansteigt. Dies geschieht genau dann, wenn die Enrgie der
Röntgenstrahlung gerade größer ist als die Bindungsenergie eines Hüllenelektrons
der inneren Schale ist.
Dieser sprunghafte Anstieg wird als Absoptionskante bezeichnet, diese hat die Lage von
\begin{equation}
  h\nu_{\text{abs}}=E_n - E_{\infty}.
\end{equation}
Die Absoptionskanten werden je nach Schale aus der das Elektron stammt mit
K-, L-, M-,... Absorptionskanten bezeichnet. Diese lassen sich noch weiter in
eine Feinstrucktur unterteilen, da die Feinstrucktur in diesem Versuch nicht untersucht wird,
wird nicht näher darauf eingegangen.

Um die Energie und damit auch die Wellenlänge der Röntgenstrahlung zu bestimmen wird die
Bragg-Bedingung ausgenutzt. Hierbei fällt die Röntgenstrahlung auf einen Kristall und wird an den
Atomen des Kristalls gebeugt, dadurch kommt es zu Interferez. Der Ort an dem konstruktive
Interferenz beobachtet wird, wird Glanzwinkel $\theta$ genannt.
Mathematisch ausgedrückt hat die Bragg-Bedingung folgende Gestallt:
\begin{equation}
  2d\sin(\theta)=n\lambda.
  \label{bragg}
\end{equation}

\begin{figure}
  \centering
  \includegraphics[height=4cm]{bragg.png}
  \caption{Röntgenstrahlung wird an den Atomen des Kristallgitters gebeugt.}
  \label{fig:bragg}
  \cite{skript}
\end{figure}

\section{Vorbereitung}

Zur Vorbereitung auf diesen Versuch wurden Energiewerte sowie zu erwartende Glanzwinkel verschiedener
Metalle ermittelt, um den Messbereich genauer wählen zu können.
Für die Überprüfung der Absorption wurden folgende Daten für Kupfer ermittelt:
\begin{align*}
  Cu-K_{\alpha}-Linie:& \;\;\;E_K=\SI{8,10}{\keV}\;\;\;\;&\theta &=22,7°\\
  Cu-K_{\beta}-Linie:& \;\;\;E_K=\SI{8,92}{\keV}\;\;\;\;&\theta &=20,15°.
\end{align*}
In Tabelle \ref{tab:tabvor} sind Daten zu finden, die für die Absopionsmessung
vorbereitet wurden. Es sind hier nur die Metalle aufgelistet, die auch im Versuch
untersucht werden.
\begin{table}[H]
  \centering
   \begin{tabular}{c c c c c}
    \toprule
     & Z & $E_{K}$/\;keV & $\theta_{K}$ /\;° & $\sigma_{K}$ \\
    \midrule
    Zn & 30 & 9,65 & 18,6 & 3,56\\
    Sr & 38 & 16,1 & 11,0 & 3,59\\
    Br & 35 & 13,5 & 13,2 & 3,49\\
    Zr & 40 & 18,0 & 9,8 & 3,63\\
    Au & 79 & 12-14 & 14,9-12,7 & 3,74 \\
    \bottomrule
  \end{tabular}
  \caption{Zur Vorbereitung ermittelte Werte der einzelnen Metalle.}
  \label{tab:tabvor}
\end{table}

%Z & $E_{K}$/\; keV & $\theta_{K}$/\; ° & $\sigma_{K}$

