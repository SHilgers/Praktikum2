\section{Auswertung}
\subsection{Überprüfung der Bragg Bedingung}
Die Messwerte zur Überprüfung der Bragg Bedingung befinden sich in Tabelle \ref{tab:tab1}
\begin{table}[H]
  \centering
  \caption{Messwerte und Ergebniss der Bestimmung der Schallgeschwindigkeit}
  \label{tab:tabe1}
    \begin{tabular}{S||S S||S S||S|S}
    \toprule
    $ \text{Länge l des Zylinders [mm]} $ & $ U_{1} [\text{V}] $ &
    $ t_{1} [\mu\text{s}] $ & $ U_{2} [\text{V}] $ &
    $ t_{2} [\mu\text{s}] $ & $ \increment t [\mu\text{s}]$ &
    $ \text{c} [\text{m}/\text{s}]$\\
    \midrule
    31.0 & 1.335 \: & 24.0 & 1.096 \:  & 46.9 & 22.9 & 2707.42 \\
          \bottomrule
    \end{tabular}
  \end{table}


Hieraus lässt sich ein Maximum von 140 Impulsen pro s bei einem Winkel von 28,2°
ablesen.

\subsection{Emissionsspektrum der Cu-Röntgenröhre}
Die Messwerte des Emissionsspektrums sind in Tabelle \ref{tab:tabe2} abzulesen. In
der Abbildung \ref{fig:plot1} sind sie zudem grafisch dargestellt.
\begin{figure}[H]
  \centering
  \includegraphics{plot1.pdf}
  \caption{Messwerte des Emissionsspektrums}
  \label{fig:plot1}
\end{figure}
Der Grenzwinkel beträgt hierbei etwa 10°, woraus sich nach Formel \ref{eqn:???} eine minimale
Wellenlänge von $ \SI{35.106}{\pico\meter}$ und eine maximale Energie von $\SI{5.658}{\femto\joule}$
bzw. $\SI{35.317}{\kilo \electronvolt}$ berechen lässt. Die Theoriewerte ergeben sich
aus Gleichung \ref{eqn:???} zu einer minimalen Wellenlänge von $ \SI{35.424}{\pico\meter}$
und einer maximalen Energie von $\SI{5.608}{\femto\joule}$
bzw. $\SI{35}{\kilo \electronvolt}$
Durch die Formel
\begin{equation*}
  \frac{\lvert \text{Wert}_{\text{Theorie}}-\text{Wert}_{\text{Messung}}\rvert}{\text{Wert}_{\text{Theorie}}}
  \label{eqn:abw}
\end{equation*}
ergibt sich somit eine relative Abweichung von $0.9 \%$.

Aus Abbildung \ref{fig:plot1} lässt sich für die Halbwertsbreite der K$\alpha$-Linie
ein Wert von etwa $\Delta \theta$= 0,7° (zwischen 39,70° und 40,40°) ablesen und für die K$\beta$-Linie ein
Wert von etwa $\Delta \theta$= 0,8° (zwischen 44,15° und 44,95°).
Hieraus lässt sich durch $\Delta$E = $\text{E}_1$ - $\text{E}_2$
eine Auflösung von $\SI{0.15}{\kilo\electronvolt}$ (aus K$\alpha$) bzw. $\SI{0.14}{\kilo\electronvolt}$
(aus K$\beta$) berechnen.
Durch die Gleichung
\begin{equation}
  \bar{x} = \frac{1}{N} \sum_{i=1}^{N} x_i \: \:
  \label{eqn:mit}
\end{equation}
\noindent lässt sich der Mittelwert bilden, wobei der dazugehörige Fehler sich durch
\begin{equation}
  \increment \bar{x} = \frac{1}{\sqrt{N}} \sqrt{ \frac{1}{N-1} \sum_{i=1}^N
  (x_i - \bar{x})^2}
  \label{eqn:mitf}
\end{equation}
ergibt.
Somit ergibt sich also insgesamt eine Auflösung von $\SI{0.145(5)}{\kilo\electronvolt}$.


Die Energiedifferenz zwischen der K$\alpha$-Linie und der K$\beta$-Linie beträgt etwa $\SI{0.835}{\kilo\electronvolt}$.
Aus der Gleichung \ref{eqn:???} ergibt sich hieraus eine Abschirmkonstante von



\subsection{Absorptionsspektren}
\subsubsection{Brom}
Die Messwerte des Absorptionsspektrums von Brom sind in Tabelle \ref{tab:tab3} aufgeführt und
in Abbildung \ref{fig:brom} graphisch dargestellt.
\begin{table}[H]
  \centering
   \begin{tabular}{c c c}
    \toprule
     n& $\nu$/\; 1/s & $\nu_{Wechsel}$\\
    \midrule
    0,5 & 100.01& 50,0\\
    1 & 79.93 & 79.93\\
    2 & 23.93 & 47.86\\
    \bottomrule
  \end{tabular}
  \caption{Gemessene Frequenzen der Sägezahnspannung, sowie die Daraus resultierenden Frequenzen für die
  Wechselspannung.}
  \label{tab:tab3}
\end{table}

\begin{figure}[H]
  \centering
  \includegraphics{Brom.pdf}
  \caption{Absorptionsspektrum von Brom}
  \label{fig:brom}
\end{figure}
Die K-Kante liegt hierbei bei einem Winkel von ungefähr 26,05°, woraus sich nach Formel
\ref{eqn:???} eine Absorptionsenergie von $\SI{13.657}{\kilo}{\electronvolt}$. Durch Gleichung
\ref{eqn:abw} ergibt sich somit eine prozentuale Abweichung von
zum Theoriewert aus der Vorbereitung.


Aus Gleichung \ref{eqn:???} lässt sich aus dieser Absorptionsenergie eine Abschirmkonstante
von ???? berechnen.

\subsubsection{Strontium}
Die Messwerte des Absorptionsspektrums von Strontium sind in Tabelle \ref{tab:tab4} aufgeführt und
in Abbildung \ref{fig:Strontium} graphisch dargestellt.
\begin{table}[H]
  \centering
   \begin{tabular}{c c c c}
    \toprule
    Nummer der Oberwelle & $ U_{\text Theorie,Rechteck}\: / \si{\volt} $ &
    $ U_{\text Theorie,Dreick}\: / \si{\volt} $ & $ U_{\text Theorie,Sägezahn}\: / \si{\volt} $ \\
    \midrule
    1 & 1145 & 182 & 573 \\
    2 & 0 & 0 & 286 \\
    3 & 573 & 20 & 191 \\
    4 & 0 & 0 & 143 \\
    5 & 229 & 7 & 115 \\
    6 & 0 & 0 & 96 \\
    7 & 164 & 4 & 82 \\
    8 & 0 & 0 & 72 \\
    9 & 127 & 2 & 64 \\
    10 & 0 & 0 & 57 \\
    \bottomrule
  \end{tabular}
  \caption{Eingestellte Schwingungsamplituden.}
  \label{tab:tabe4}
\end{table}

\begin{figure}[H]
  \centering
  \includegraphics{Strontium.pdf}
  \caption{Absorptionsspektrum von Strontium}
  \label{fig:Strontium}
\end{figure}
Die K-Kante liegt hierbei bei einem Winkel von etwa 21,70°. Nach analoger Rechnung zu Brom
ergibt sich für Strontium somit eine Absorptionsenergie von $\SI{13.657}{\kilo}{\electronvolt}$
(relative Abweichung: ) und eine Abschirmkonstante von (relative Abweichung: ).

\subsubsection{Zink}
Die Messwerte des Absorptionsspektrums von Zink sind in Tabelle \ref{tab:tab5} aufgeführt und
in Abbildung \ref{fig:Zink} graphisch dargestellt.
\begin{table}[H]
  \centering
  \caption{Mechanischen Kompressorleistung zu den Zeiten $t_1$, $t_2$, $t_3$ und $t_4$.}
  \label{tab:tabe5}
    \begin{tabular}{S S}
    \toprule
    $ t  \: / \si{\second} $ & $ N_{\text{mech}} \: / \: \si{\watt}$ \\
    \midrule
    480 & 4.72 \pm 0.16 \\
    960 & 6.19 \pm 0.22 \\
    1500 & 6.67 \pm 0.26 \\
    1980 & 6.26 \pm 0.28 \\
      \bottomrule
    \end{tabular}
\end{table}

\begin{figure}[H]
  \centering
  \includegraphics{Zink.pdf}
  \caption{Absorptionsspektrum von Zink}
  \label{fig:Zink}
\end{figure}
Die K-Kante liegt hierbei bei einem Winkel von etwa 36,75°. Nach analoger Rechnung
ergibt sich für Zink somit eine Absorptionsenergie von $\SI{13.657}{\kilo}{\electronvolt}$
(relative Abweichung: ) und eine Abschirmkonstante von (relative Abweichung: ).

\subsubsection{Zirkonium}
Die Messwerte des Absorptionsspektrums von Zirkonium sind in Tabelle \ref{tab:tab6} aufgeführt und
in Abbildung \ref{fig:Zirkonium} graphisch dargestellt.
\begin{table}[H]
  \centering
  \caption{Messwerte des Absorptionsspektrums von Zirkonium}
  \label{tab:tabe6}
    \begin{tabular}{S S}
    \toprule
    $ \text{Winkel} / ° $ & $ \text{Impulse pro s}$\\
    \midrule
    18.0 & 59.0 \\
    18.2 & 59.0 \\
    18.4 & 58.0 \\
    18.6 & 56.0 \\
    18.8 & 55.0 \\
    19.0 & 57.0 \\
    19.2 & 70.0 \\
    19.4 & 88.0 \\
    19.6 & 105.0 \\
    19.8 & 105.0 \\
    20.0 & 112.0 \\
    20.2 & 115.0 \\
    20.4 & 113.0 \\
    20.6 & 116.0 \\
    20.8 & 116.0 \\
    21.0 & 114.0 \\

          \bottomrule
        \end{tabular}
    \end{table}

\begin{figure}[H]
  \centering
  \includegraphics{Zirkonium.pdf}
  \caption{Absorptionsspektrum von Zirkonium}
  \label{fig:Zirkonium}
\end{figure}
Die K-Kante liegt hierbei bei einem Winkel von etwa 19,35°. Nach analoger Rechnung
ergibt sich für Zirkonium somit eine Absorptionsenergie von $\SI{13.657}{\kilo}{\electronvolt}$
(relative Abweichung: ) und eine Abschirmkonstante von (relative Abweichung: ).
