\section{Auswertung}
\subsection{Überprüfung der Bragg Bedingung}
Die Messwerte zur Überprüfung der Bragg Bedingung befinden sich in Tabelle \ref{tab:tabe1}
\begin{table}[H]
  \centering
  \caption{Messwerte und Ergebniss der Bestimmung der Schallgeschwindigkeit}
  \label{tab:tabe1}
    \begin{tabular}{S||S S||S S||S|S}
    \toprule
    $ \text{Länge l des Zylinders [mm]} $ & $ U_{1} [\text{V}] $ &
    $ t_{1} [\mu\text{s}] $ & $ U_{2} [\text{V}] $ &
    $ t_{2} [\mu\text{s}] $ & $ \increment t [\mu\text{s}]$ &
    $ \text{c} [\text{m}/\text{s}]$\\
    \midrule
    31.0 & 1.335 \: & 24.0 & 1.096 \:  & 46.9 & 22.9 & 2707.42 \\
          \bottomrule
    \end{tabular}
  \end{table}


Hieraus lässt sich ein Maximum von 140 Impulsen pro s bei einem Winkel von 28,2°
ablesen.

\subsection{Emissionsspektrum der Cu-Röntgenröhre}
Die Messwerte des Emissionsspektrums sind in Tabelle \ref{tab:tabe2} abzulesen. In
Abbildung \ref{fig:plot1} sind sie zudem graphisch dargestellt.
\begin{table}[H]
  \centering
  \caption{Zählrate und Energiemaximum bei variiertem Druck, Abstand a=2cm}
  \label{tab:tab2}
    \begin{tabular}{c c c c c}
    \toprule
    Druck $\rho$/\;mbar & Energiemaximum & Zählrate $N$ & Energie $E_{\alpha}$ & effektive Länge $x$/\;cm\\
    \midrule
    0 & 796 &131382  &4          & 0.0   \\
    50 & 775 &131464 &3.89 & 0.09 \\
    100 &756 &130732 &3.79 & 0.19\\
    150 &749 &129617 &3.76  &  0.29\\
    200 &749 &130444 &3.76  & 0.39\\
    250 &727 &129600 &3.65 & 0.49\\
    300 &722 &128936 &3.63 & 0.59\\
    350 &708 &128478 &3.56 & 0.69\\
    400 &696 &128122 &3.49 & 0.79\\
    450 &687 &127415 &3.45 & 0.89\\
    500 &674 &126608 &3.39 & 0.99\\
    550 &663 &126372 &3.33 &1.09\\
    600 &651 &124989 &3.27 & 1.18\\
    650 &634 &124942 &3.19 & 1.28\\
    700 &618 &124295 &3.11 &1.38\\
    750 &602 &123299 &3.03 & 1.48\\
    800 &584 &119958 &2.93 &1.58\\
    850 &566 &120673 &2.84 &1.68\\
    900 &548 &117907 &2.75 & 1.78\\
    950 &534 &116111 &2.68&   1.88\\
    1000 &499& 108630&2.51 & 1.07\\
    \bottomrule
    \end{tabular}
  \end{table}


\begin{table}[H]
  \centering
    \begin{tabular}{S S |S S}
    \toprule
    $ \text{Winkel} / ° $ & $ \text{Impulse pro s}$ & $ \text{Winkel} / ° $ & $ \text{Impulse pro s}$ \\
    \midrule
    22.8 & 172.0 & 45.2 & 333.0 \\
    23.2 & 155.0 & 45.5 & 113.0 \\
    23.6 & 167.0 & 46.0 & 80.0 \\
    24.0 & 159.0 & 46.4 & 66.0 \\
    24.4 & 173.0 & 46.8 & 64.0 \\
    24.8 & 156.0 & 47.2 & 59.0 \\
    25.2 & 154.0 & 47.6 & 60.0 \\
    25.6 & 154.0 & 48.0 & 54.0 \\
    26.0 & 157.0 & 48.4 & 58.0 \\
    26.4 & 129.0 & 48.8 & 50.0 \\
    26.8 & 122.0 & 49.2 & 46.0 \\
    27.2 & 122.0 & 49.6 & 42.0 \\
    27.6 & 128.0 & 50.0 & 47.0 \\
    28.0 & 116.0 & 50.4 & 39.0 \\
    28.4 & 114.0 & 50.8 & 43.0 \\
    28.8 & 113.0 & 51.2 & 41.0 \\
    29.2 & 109.0 & 51.6 & 37.0 \\
    29.6 & 111.0 & 52.0 & 34.0 \\
    30.0 & 111.0 & & \\
          \bottomrule
        \end{tabular}
    \end{table}

\begin{figure}[H]
  \centering
  \includegraphics{plot1.pdf}
  \caption{Messwerte des Emissionsspektrums}
  \label{fig:plot1}
\end{figure}
Der Grenzwinkel beträgt hierbei etwa 10°, woraus sich nach Formel \ref{eqn:bragg} eine minimale
Wellenlänge von $ \SI{35.106}{\pico\meter}$ und eine maximale Energie von $\SI{5.658}{\femto\joule}$
bzw. $\SI{35.317}{\kilo \electronvolt}$ berechen lässt. Die Theoriewerte ergeben sich
aus Gleichung \ref{eqn:lamdbamin} zu einer minimalen Wellenlänge von $ \SI{35.424}{\pico\meter}$
und einer maximalen Energie von $\SI{5.608}{\femto\joule}$
bzw. $\SI{35}{\kilo \electronvolt}$
Durch die Formel
\begin{equation*}
  \frac{\lvert \text{Wert}_{\text{Theorie}}-\text{Wert}_{\text{Messung}}\rvert}{\text{Wert}_{\text{Theorie}}}
  \label{eqn:abw}
\end{equation*}
ergibt sich somit eine relative Abweichung von $0.9 \%$.

Aus Abbildung \ref{fig:plot1} lässt sich für die Halbwertsbreite der K$\alpha$-Linie
ein Wert von etwa $\Delta \theta$= 0,7° (zwischen 39,70° und 40,40°) ablesen und für die K$\beta$-Linie ein
Wert von etwa $\Delta \theta$= 0,8° (zwischen 44,15° und 44,95°).
Hieraus lässt sich durch $\Delta$E = $\text{E}_1$ - $\text{E}_2$
eine Auflösung von $\SI{0.15}{\kilo\electronvolt}$ (aus K$\alpha$) bzw. $\SI{0.14}{\kilo\electronvolt}$
(aus K$\beta$) berechnen.
Durch die Gleichung
\begin{equation}
  \bar{x} = \frac{1}{N} \sum_{i=1}^{N} x_i \: \:
  \label{eqn:mit}
\end{equation}
\noindent lässt sich der Mittelwert bilden, wobei der dazugehörige Fehler sich durch
\begin{equation}
  \increment \bar{x} = \frac{1}{\sqrt{N}} \sqrt{ \frac{1}{N-1} \sum_{i=1}^N
  (x_i - \bar{x})^2}
  \label{eqn:mitf}
\end{equation}
ergibt.
Somit ergibt sich also insgesamt eine Auflösung von $\SI{0.145(5)}{\kilo\electronvolt}$.\\
\\



%Die Energiedifferenz zwischen der K$\alpha$-Linie und der K$\beta$-Linie beträgt etwa $\SI{0.835}{\kilo\electronvolt}$.
%Aus der Gleichung \ref{eqn:???} ergibt sich hieraus eine Abschirmkonstante von
\noindent Die K$\alpha$-Linie liegt bei etwa 40,0 ° und die K$\beta$-Linie bei etwa 44,4°.
Das entspricht Energien von $\text{E}_{k\alpha} = \SI{9.000}{\kilo\electronvolt}$ und
$\text{E}_{k\beta} = \SI{8.146}{\kilo\electronvolt}$
Aus Gleichung \ref{eqn:k} lassen sich somit Abschirmzahlen von
\begin{align*}
  \sigma_1 = 4,423 \\
  \sigma_2 = 14,11 \: . \\
\end{align*}

\subsection{Absorptionsspektren}
\subsubsection{Brom}
Die Messwerte des Absorptionsspektrums von Brom sind in Tabelle \ref{tab:tabe3} aufgeführt und
in Abbildung \ref{fig:brom} graphisch dargestellt.
\begin{table}[H]
  \centering
   \begin{tabular}{c c c}
    \toprule
     n& $\nu$/\; 1/s & $\nu_{Wechsel}$\\
    \midrule
    0,5 & 100.01& 50,0\\
    1 & 79.93 & 79.93\\
    2 & 23.93 & 47.86\\
    \bottomrule
  \end{tabular}
  \caption{Gemessene Frequenzen der Sägezahnspannung, sowie die Daraus resultierenden Frequenzen für die
  Wechselspannung.}
  \label{tab:tab3}
\end{table}

\begin{figure}[H]
  \centering
  \includegraphics{Brom.pdf}
  \caption{Absorptionsspektrum von Brom}
  \label{fig:brom}
\end{figure}
Die K-Kante liegt hierbei bei einem Winkel von ungefähr 26,05°, woraus sich nach Formel
\ref{eqn:bragg} eine Absorptionsenergie von $\SI{13.657}{\kilo\electronvolt}$ ergibt. Durch Gleichung
\ref{eqn:abw} ergibt sich somit eine prozentuale Abweichung von 1,16 \%
zum Theoriewert aus der Vorbereitung.


Aus Gleichung \ref{eqn:Ebindung} lässt sich aus dieser Absorptionsenergie eine Abschirmkonstante
von 3,317 berechnen, wobei die relative Abweichung 5,23 \% beträgt.

\subsubsection{Strontium}
Die Messwerte des Absorptionsspektrums von Strontium sind in Tabelle \ref{tab:tabe4} aufgeführt und
in Abbildung \ref{fig:Strontium} graphisch dargestellt.
\begin{table}[H]
  \centering
   \begin{tabular}{c c c c}
    \toprule
    Nummer der Oberwelle & $ U_{\text Theorie,Rechteck}\: / \si{\volt} $ &
    $ U_{\text Theorie,Dreick}\: / \si{\volt} $ & $ U_{\text Theorie,Sägezahn}\: / \si{\volt} $ \\
    \midrule
    1 & 1145 & 182 & 573 \\
    2 & 0 & 0 & 286 \\
    3 & 573 & 20 & 191 \\
    4 & 0 & 0 & 143 \\
    5 & 229 & 7 & 115 \\
    6 & 0 & 0 & 96 \\
    7 & 164 & 4 & 82 \\
    8 & 0 & 0 & 72 \\
    9 & 127 & 2 & 64 \\
    10 & 0 & 0 & 57 \\
    \bottomrule
  \end{tabular}
  \caption{Eingestellte Schwingungsamplituden.}
  \label{tab:tabe4}
\end{table}

\begin{figure}[H]
  \centering
  \includegraphics{Strontium.pdf}
  \caption{Absorptionsspektrum von Strontium}
  \label{fig:Strontium}
\end{figure}
Die K-Kante liegt hierbei bei einem Winkel von etwa 21,70°. Nach analoger Rechnung zu Brom
ergibt sich für Strontium somit eine Absorptionsenergie von $\SI{16.352}{\kilo\electronvolt}$
(relative Abweichung: 1,57 \%) und eine Abschirmkonstante von 3,332 (relative Abweichung: 7,44 \%).

\subsubsection{Zink}
Die Messwerte des Absorptionsspektrums von Zink sind in Tabelle \ref{tab:tabe5} aufgeführt und
in Abbildung \ref{fig:Zink} graphisch dargestellt.
\begin{table}[H]
  \centering
  \caption{Mechanischen Kompressorleistung zu den Zeiten $t_1$, $t_2$, $t_3$ und $t_4$.}
  \label{tab:tabe5}
    \begin{tabular}{S S}
    \toprule
    $ t  \: / \si{\second} $ & $ N_{\text{mech}} \: / \: \si{\watt}$ \\
    \midrule
    480 & 4.72 \pm 0.16 \\
    960 & 6.19 \pm 0.22 \\
    1500 & 6.67 \pm 0.26 \\
    1980 & 6.26 \pm 0.28 \\
      \bottomrule
    \end{tabular}
\end{table}

\begin{figure}[H]
  \centering
  \includegraphics{Zink.pdf}
  \caption{Absorptionsspektrum von Zink}
  \label{fig:Zink}
\end{figure}
Die K-Kante liegt hierbei bei einem Winkel von etwa 36,75°. Nach analoger Rechnung
ergibt sich für Zink somit eine Absorptionsenergie von $\SI{9.765}{\kilo\electronvolt}$
(relative Abweichung: 1,19 \%) und eine Abschirmkonstante von 3,210 (relative Abweichung: 4,75 \%).

\subsubsection{Zirkonium}
Die Messwerte des Absorptionsspektrums von Zirkonium sind in Tabelle \ref{tab:tabe6} aufgeführt und
in Abbildung \ref{fig:Zirkonium} graphisch dargestellt.
\begin{table}[H]
  \centering
  \caption{Messwerte des Absorptionsspektrums von Zirkonium}
  \label{tab:tabe6}
    \begin{tabular}{S S}
    \toprule
    $ \text{Winkel} / ° $ & $ \text{Impulse pro s}$\\
    \midrule
    18.0 & 59.0 \\
    18.2 & 59.0 \\
    18.4 & 58.0 \\
    18.6 & 56.0 \\
    18.8 & 55.0 \\
    19.0 & 57.0 \\
    19.2 & 70.0 \\
    19.4 & 88.0 \\
    19.6 & 105.0 \\
    19.8 & 105.0 \\
    20.0 & 112.0 \\
    20.2 & 115.0 \\
    20.4 & 113.0 \\
    20.6 & 116.0 \\
    20.8 & 116.0 \\
    21.0 & 114.0 \\

          \bottomrule
        \end{tabular}
    \end{table}

\begin{figure}[H]
  \centering
  \includegraphics{Zirkonium.pdf}
  \caption{Absorptionsspektrum von Zirkonium}
  \label{fig:Zirkonium}
\end{figure}
Die K-Kante liegt hierbei bei einem Winkel von etwa 19,35°. Nach analoger Rechnung
ergibt sich für Zirkonium somit eine Absorptionsenergie von $\SI{18.315}{\kilo\electronvolt}$
(relative Abweichung: 1,75 \%) und eine Abschirmkonstante von 3,309 (relative Abweichung: 8,59 \%).
\subsubsection{Moseleysches-Gesetz}
Zur Bestimmung der Rydbergkonstanten wird in Abbildung \ref{fig:plot7} die Wurzel der Energie
der K$\alpha$-Linie gegen die Kernladungszahl Z aufgetragen und eine lineare Ausgleichsrechnung
durchgeführt.
\begin{figure}[H]
  \centering
  \includegraphics{plot7.pdf}
  \caption{Lineare Ausgleichsrechnung der Wertepaare}
  \label{fig:plot7}
\end{figure}
Hieraus ergeben sich die Parameter
\begin{align*}
  a = (3,649 \pm 0,02) \frac{1}{\sqrt{eV}} \\
  b = (-10,7 \pm 0,6) \sqrt{eV} \: \: . \\
\end{align*}
Nach Gleichung \ref{eqn:Ebindung} beträgt die Steigung dieser
Geraden etwa $\sqrt{\text{R}_{\infty}}$,
wobei $\text{R}_{\infty}$ die gesuchte Rydbergenergie
bezeichnet, welche somit
\begin{align*}
  \text{R}_{\infty}= \SI{13.313}{\electronvolt}
\end{align*}
beträgt und eine relative Abweichung von 2,11 \% zum Theoriewert aufweist.
\subsubsection{Gold}
Die Messwerte des Absorptionsspektrums von Gold sind in Tabelle \ref{tab:tabe7} aufgeführt und
in Abbildung \ref{fig:Gold} graphisch dargestellt.
\begin{table}[H]
  \centering
  \caption{Werte der zweiten Messreihe für Wert 16}
  \label{tab:tabe7}
    \begin{tabular}{S S S}
    \toprule
    $ \text{R}_{2} \: / \: \si{\ohm} $ & $\text{R}_{3} \: / \: \si{\ohm} $ &
    $\text{R}_{4} \: / \: \si{\ohm} $ \\
    \midrule
    500 & 178 & 328 \\
    664 & 178 & 328 \\
    1000 & 119 & 332 \\
    \bottomrule
    \end{tabular}
\end{table}

\begin{figure}[H]
  \centering
  \includegraphics{Gold.pdf}
  \caption{Absorptionsspektrum von Gold}
  \label{fig:Gold}
\end{figure}
Die beiden L-Kanten liegen bei $\theta_1$ = 24,3° und bei $\theta_2$ = 29,65°, was Energien
von $\text{E}_{1} = \SI{14.624}{\kilo\electronvolt}$ und
$\text{E}_{2} = \SI{12.030}{\kilo\electronvolt}$ entspricht, sodass die Energiedifferenz
$\Delta E_L = \SI{2.594}{\kilo\electronvolt}$ ist.
Aus Gleichung \ref{eqn:L} lässt sich hiermit eine Abschirmkonstante von
$\sigma_L = 63,56$ berechnen.
