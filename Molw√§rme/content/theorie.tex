\section{Zielsetzung}
Ziel des Versuches ist es, die Molwärme $\symup{C_{\text{v}}}$ von Kupfer in Abhängigkeit der
Temperatur zu bestimmen und mit den Vorhersagen des klassischen Modells, des Einstein
Modells und des Debye Modells zu vergleichen. Zudem soll die Debye-Temperatur
$\Theta_{\text{D}}$ bestimmt werden.

\section{Theorie}
Zur Vorbereitung der Theorie wurde das Buch \cite{fkp} verwendet.
\subsection{Spezifische Wärme}
Als Wärmekapazität wird im Allgemeinen die Wärmemenge $\increment \text{Q}$ bezeichnet,
die benötigt wird um
die Temperatur eines
bestimmten Stoffes um $\SI{1}{\kelvin}$ zu erhöhen:
\begin{equation}
  \text{C} = \frac{\increment \text{Q}}{\increment \text{T}}
  \label{eqn:warm}
\end{equation}
Da diese Größe stets von der Menge des Stoffes abhängt, wird sie häufig im
Bezug auf $\SI{1}{\mol}$ normiert
\begin{equation}
  \text{c} = \frac{\increment \text{Q}}{\increment \text{T} \: \text{mol}} \: ,
  \label{eqn:cmol}
\end{equation}
wobei diese Größe dann als molare Wärmekapazität bezeichnet wird.  \\
Es wird zudem zwischen der Wärmekapazität bei konstantem Druck $\symup{C_{\text{p}}}$ und der Wärmekapazität bei
konstanten Volumen $\symup{C_{\text{v}}}$ unterschieden, wobei $\symup{C_{\text{p}}}$
im Allgemeinen größer ist, da hier ein Teil der zugeführten Energie zur Volumensausdehnung verwendet
wird und nicht nur zur Temperaturerhöhung.
Dies ergibt sich durch den ersten Hauptsatz der Thermodynamik
\begin{equation}
  \text{dQ} = \text{dU} + \text{pdV} \: ,
  \label{eqn:hs1}
\end{equation}
aus dem sich die Gleichungen
\begin{equation}
  \symup{C_{\text{v}}} = \frac{\partial\text{U}}{\partial\text{T}}\bigr|_{\text{v}}\; \text{;} \;
  \symup{C_{\text{p}}} = \frac{\partial\text{Q}}{\partial\text{T}}\bigr|_{\text{p}}
\end{equation}
ergeben. Die Beziehung zwischen $\symup{C_{\text{p}}}$ und $\symup{C_{\text{v}}}$
ergibt sich aus der Thermodynamik zu
\begin{equation}
  \symup{C_{\text{p}}}-\symup{C_{\text{v}}}=9 \text{TV}_0\:\alpha^2\kappa \: ,
  \label{eqn:cpcv}
\end{equation}
wobei $\alpha$ den linaren Ausdehnungskoeffizient, $\kappa$ den Kompressionsmodul und
$\text{V}_0$ das Molvolumen bezeichnet.

\subsection{Klassisches Modell}
Im Klassischen Modell setzt sich die innerer Energie U eines kristallinen Festkörpers
aus der Grundenergie $\symup{U}^{\text{eq}}$ des Gitters und der Energie der Schwingungen
des Kristallgitters zusammen.
Diese setzt sich nach dem Äquipartitionstheorem aus jeweils $\frac{1}{2}\text{k}_{\text{B}}\text{T}$
für potentielle Energie und kinetische Energie pro Freiheitsgrad zusammen. Bei einem Gitter
mit einatomiger Basis gibt es 3 Freiheitsgrade in Form von Schwingungen, je einen für jede der
drei Raumrichtungen.
Somit ergibt sich die innere Energie zu
\begin{equation}
  \symup{U} = \symup{U}^{\text{eq}} + 3\text{N}\text{k}_{\text{B}}\text{T} \: ,
  \label{eqn:Uklassisch}
\end{equation}
wobei $\text{N}=\nu \cdot \text{N}_{\text{A}}$ die Teilchenanzahl bezeichnet, mit der
Avogadrokonstante $\text{N}_{\text{A}}$ und der Molzahl $\nu$.
Somit beträgt die Wärmekapazität
\begin{equation}
  \symup{C_{\text{v}}} = 3\text{N}\text{k}_{\text{B}}
  \label{eqn:Cklassich}
\end{equation}
und die molare Wärmekapazität entsprechend
\begin{equation}
  \symup{c_{\text{v}}} = 3\text{N}_{\text{A}}\text{k}_{\text{B}} = 3\text{R}
  \label{eqn:Cklassich}
\end{equation}
mit der idealen Gaskonstante R. Dies ist auch als Dulong-Petit-Gesetz bekannt, wobei
die molare Wärmekapazität unabhängig ist vom Material und der Temperatur.
Diese Beschreibung gilt meist nur bei hohen Temperaturen und zeigt bei niedrigen
Temperaturen häufig große Abweichungen, da Nullpunktsschwingungen in diesem Modell
nicht berücksichtigt werden.

\subsection{Einstein-Modell}
Durch die Quantisierung der Gitterschwingungen, welche als Phononen bezeichnet
werden, können nur noch diskrete Energien von $ \text{E}_{\text{n}}=(\text{n}+\frac{1}{2})
\hbar \omega $ aufgenommen werden, mit der Kreisfrequenz $\omega$ der Schwingung
und der Anregungszahl n. Zur Berechnung der mittleren inneren Energie wird nun eine Summation
über alle Anregungen durchgeführt, dass heißt über alle
\textbf{q}-Vektoren, wobei jede Energie mit dem Boltzmann-Faktor
$\text{e}^{-\beta \:\text{E}_{\text{n}}}$ gewichtet wird, dabei ist
$\beta= \frac{1}{\text{k}_{\text{B}}\text{T}}$.
Hieraus ergibt sich die innere Energie zu
\begin{equation}
  \bigl<\symup{U}\bigr> = \sum_{\textbf{q}}\hbar\omega_{\textbf{q}}\Bigl(\frac{1}{2}+\bigl<\text{n}\bigr>\Bigr)
  \label{eqn:Uqm}
\end{equation}
mit der mittleren Besetzung
\begin{equation}
  \bigl<\symup{n}\bigr> =  \frac{1}{\symup{e}^{\hbar\omega\beta}-1}
  \label{eqn:boseeinstein}
\end{equation}
gemäß der Bose-Einstein Verteilung, da es sich bei Phononen um Bosonen ohne
chemisches Potential handelt.
Zur Bestimmung der mittleren inneren Energie wird bei großen N häufig der Kontinuumslimes
verwendet, dass heißt die Summe über $\textbf{q}$ wird zu einem Integral, wobei die
Dispersionsrelation $\omega(\textbf{q})$ sowie die Zustandsdichte D($\omega$)
benötigt wird.
Im Einstein-Modell wird dabei eine konstante Frequenz
für alle 3N-Schwingungsmoden angenommen, also
\begin{equation}
  \symup{D}(\omega) = 3\text{N}\delta(\omega-\omega_{\text{E}}) \: ,
  \label{eqn:zdeinst}
\end{equation}
sodass die mittlere innere Energie durch
\begin{equation}
  \bigl<\symup{U}\bigr> = 3\text{N}\omega_{\text{E}}\Bigl(\frac{1}{2}+\frac{1}{\text{e}^{\hbar\omega_{\text{E}}/
  \text{k}_{\text{B}}T}-1}\Bigr)
  \label{eqn:Uein}
\end{equation}
gegeben ist
und die Wärmekapazität sich somit zu
\begin{equation}
  \symup{C_{\text{v}}} = 3\text{N}\text{k}_{\text{B}}\Bigl(\frac{\Theta_{\text{E}}}{T}\Bigr)^2
  \frac{\text{exp}(\Theta_{\text{E}}/T)}{(\text{exp}(\Theta_{\text{E}}/T)-1)^2}
  \label{eqn:Cein}
\end{equation}
ergibt mit der Einstein-Temperatur
$\Theta_{\text{E}} = \frac{\hbar\omega_{\text{E}}}{\text{k}_{\text{B}}}$ .
Für hohe Temperaturen, also $T \gg \Theta_{\text{E}}$, ergibt sich als Grenzwert
auch hier das klassische Gesetz von Dulong-Petit. Für kleine Temperaturen
$T \ll \Theta_{\text{E}}$ ergibt sich die Näherung
\begin{equation}
  \symup{C_{\text{v}}} \approx 3\text{N}\text{k}_{\text{B}}\Bigl(\frac{\Theta_{\text{E}}}{T}\Bigr)^2 \: .
  \text{e}^{(-\Theta_{\text{E}}/T)}
  \label{eqn:ChochE}
\end{equation}
Die Beschreibung mit konstanter Frequenz passt jedoch besser zu optischen Phononen,
welche im Allgemeinen eher bei höheren Temperaturen angeregt werden und somit ist
dieses Modell zur Beschreibung des
Verhaltens für tiefe Temperaturen eher ungeeignet.

\subsection{Debye-Modell}
Im Debye-Modell wird anstelle der konstanten Frequenz eine lineare Dispersionsrelation
gewählt, also $ \omega = \text{v}_sq$ mit der Schallgeschwindigkeit
$\text{v}_s$ des Körpers. Diese Beschreibung passt insbesondere gut für
akustische Phononen, welche bei auch bei tieferen Temperaturen als die optischen Phononen
besetzt sind.
Durch Einführen der Debye-Temperatur
\begin{equation}
  \Theta_{\text{D}} = \frac{\hbar\text{v}_s\text{q}_{\text{D}}}{\text{k}_{\text{B}}}
  = \frac{\hbar\text{v}_s}{\text{k}_{\text{B}}}\Bigl(6\pi^2\frac{N}{V}\Bigr)^{1/3}
  \label{eqn:TDebye}
\end{equation}
und der Substitution $ x=\hbar\text{v}_s\text{q} /\text{k}_{\text{B}}T $
ergibt sich eine Wärmekapazität von
\begin{equation}
  \symup{C_{\text{v}}} = 9\text{N}\text{k}_{\text{B}}\Bigl(\frac{\Theta_{\text{D}}}{T}\Bigr)^3
  \int_0^{\Theta_{\text{D}}/T} \frac{x^4\text{e}^x}{(\text{e}^x-1)^2}
  \label{eqn:CDebye}
\end{equation}
Auch hier ergibt sich das Dulong-Petit Gesetz als Grenzwert für hohe Temperaturen
$T \gg \Theta_{\text{D}}$, für niedrige Temperaturen $T \ll \Theta_{\text{D}}$ ergibt sich hingegen
\begin{equation}
  \symup{C_{\text{v}}} = \frac{12\pi^4}{5}\text{N}\text{k}_{\text{B}}\Bigl(\frac{\Theta_{\text{D}}}{T}\Bigr)^3 \:.
  \label{eqn:ChochD}
\end{equation}
