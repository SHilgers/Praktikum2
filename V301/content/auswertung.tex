\section{Auswertung}
Die zu Beginn gemessene Leerlaufspannung $U_{0}$ der Monozelle und der
Innenwiderstand $R_{V}$ des Voltmeters lauten:
\begin{align*}
  U_{0} &= \SI{1,53}{\volt} \\
  R_{V} &\approx \SI{10}{\Mohm}.
\end{align*}
\noindent Um den Innenwiderstand $R_{i}$ und die Leerlaufspannung der Monozelle zu
bestimmen, wird die Klemmspannung $U_{k}$ gegen den Belastunsstrom $I$ aufgetragen.
Dies ist in Abbildung \ref{fig:plot1} zu sehen, die Steigung $a$ der Ausgleichsgeraden
entspricht dabei dem Innenwiderstand $R_{i}$ und der y-Achsenabschnitt $b$ entspricht
der Leerlaufspannung $U_{0}$.
Die Messwerte dazu finden sich in Tallele \ref{tab:tabelle1}.
\begin{table}[H]
  \centering
  \caption{Messwerte und Ergebniss der Bestimmung der Schallgeschwindigkeit}
  \label{tab:tabe1}
    \begin{tabular}{S||S S||S S||S|S}
    \toprule
    $ \text{Länge l des Zylinders [mm]} $ & $ U_{1} [\text{V}] $ &
    $ t_{1} [\mu\text{s}] $ & $ U_{2} [\text{V}] $ &
    $ t_{2} [\mu\text{s}] $ & $ \increment t [\mu\text{s}]$ &
    $ \text{c} [\text{m}/\text{s}]$\\
    \midrule
    31.0 & 1.335 \: & 24.0 & 1.096 \:  & 46.9 & 22.9 & 2707.42 \\
          \bottomrule
    \end{tabular}
  \end{table}

\begin{figure}[H]
  \centering
  \includegraphics{plot1.pdf}
  \caption{Lineare Regression der ersten Messung: Monozelle ohne Gegenspannung.}
  \label{fig:plot1}
\end{figure}

\begin{align*}
  -a &= R_{i} = \SI{0,017(000)}{\ohm} \\
  b &= U_{0} = \SI{1,576(007)}{\volt}
\end{align*}

\noindent Die Berechnung des Innenwiderstandes $R_{i}$ und der Leerlaufspannung $U_{0}$ für
die zweite Messung, die mit einer Gegenspannung an der Monozelle durchgeführt wurde,
erfolgen äquivalent.
\begin{table}[H]
  \centering
  \caption{Zählrate und Energiemaximum bei variiertem Druck, Abstand a=2cm}
  \label{tab:tab2}
    \begin{tabular}{c c c c c}
    \toprule
    Druck $\rho$/\;mbar & Energiemaximum & Zählrate $N$ & Energie $E_{\alpha}$ & effektive Länge $x$/\;cm\\
    \midrule
    0 & 796 &131382  &4          & 0.0   \\
    50 & 775 &131464 &3.89 & 0.09 \\
    100 &756 &130732 &3.79 & 0.19\\
    150 &749 &129617 &3.76  &  0.29\\
    200 &749 &130444 &3.76  & 0.39\\
    250 &727 &129600 &3.65 & 0.49\\
    300 &722 &128936 &3.63 & 0.59\\
    350 &708 &128478 &3.56 & 0.69\\
    400 &696 &128122 &3.49 & 0.79\\
    450 &687 &127415 &3.45 & 0.89\\
    500 &674 &126608 &3.39 & 0.99\\
    550 &663 &126372 &3.33 &1.09\\
    600 &651 &124989 &3.27 & 1.18\\
    650 &634 &124942 &3.19 & 1.28\\
    700 &618 &124295 &3.11 &1.38\\
    750 &602 &123299 &3.03 & 1.48\\
    800 &584 &119958 &2.93 &1.58\\
    850 &566 &120673 &2.84 &1.68\\
    900 &548 &117907 &2.75 & 1.78\\
    950 &534 &116111 &2.68&   1.88\\
    1000 &499& 108630&2.51 & 1.07\\
    \bottomrule
    \end{tabular}
  \end{table}

\begin{figure}[H]
  \centering
  \includegraphics{plot2.pdf}
  \caption{Lineare Regression der zweiten Messung: Monozelle mit Gegenspannung.}
  \label{fig:plot2}
\end{figure}

\begin{align*}
  a &= R_{i} = \SI{0,018(001)}{\ohm} \\
  b &= U_{0} = \SI{1,593(074)}{\volt}
\end{align*}

\noindent Nun folgt die Messung einer Rechteck- und Sinusschwingung eines RC-Generators.
Die Messwerte sind in Tabelle \ref{tab:tabelle3} und dargestellt.
\begin{table}[H]
  \centering
   \begin{tabular}{c c c}
    \toprule
     n& $\nu$/\; 1/s & $\nu_{Wechsel}$\\
    \midrule
    0,5 & 100.01& 50,0\\
    1 & 79.93 & 79.93\\
    2 & 23.93 & 47.86\\
    \bottomrule
  \end{tabular}
  \caption{Gemessene Frequenzen der Sägezahnspannung, sowie die Daraus resultierenden Frequenzen für die
  Wechselspannung.}
  \label{tab:tab3}
\end{table}

Der Innenwiderstand $R_{i}$ und die Leerlaufspannung $U_{0}$ werden erneut über Lineare Regression
bestimmt, dass ist in Abbildung \ref{fig:plot3} zu sehen.
\begin{figure}[H]
  \centering
  \includegraphics{plot3.pdf}
  \caption{Lineare Regression der dritten Messung: Rechteckspannung ohne Gegenspannung.}
  \label{fig:plot3}
\end{figure}
\begin{align*}
  -a &= R_{i} = \SI{0,222(007)}{\ohm} \\
  b &= U_{0} = \SI{6,610(145)}{\volt}
\end{align*}

\begin{figure}[H]
  \centering
  \includegraphics{plot4.pdf}
  \caption{Lineare Regression der vierten Messung: Sinusspannung ohne Gegenspannung.}
  \label{fig:plot4}
\end{figure}
\begin{align*}
  -a &= R_{i} = \SI{0,192(012)}{\ohm} \\
  b &= U_{0} = \SI{10,743(149)}{\volt}
\end{align*}




\label{sec:Auswertung}

%\begin{figure}
%  \centering
%  \includegraphics{plot.pdf}
%  \caption{Plot.}
%  \label{fig:plot}
%\end{figure}
