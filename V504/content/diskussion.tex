\section{Diskussion}
\label{sec:Diskussion}
Bei der Erstellung der Kennlinienschar ist auffällig, dass sich bei einer
Heizspannung von $\SI{2.5}{\ampere}$ bereits kein Sättigungsstrom mehr ablesen lässt
und auch der errechnete Wert einen großen Fehler im Vergleich zum eigentlichen Wert
aufweist. Somit lässt sich mit dieser Kennlinie in der folgenden Auswertung kein Wert
für die Austrittsarbeit mehr errechnen.

\noindent Der Theoriewert des Exponenten des Raumladungsgebiet beträgt 1,5 \cite{skript},
sodass sich hieraus eine prozentuale Abweichung von $21,87 \% $ ergibt, welche mit der Formel
\begin{equation*}
  \frac{\lvert \text{Wert}_{\text{Theorie}}-\text{Wert}_{\text{Messung}}\rvert}{\text{Wert}_{\text{Theorie}}}
\end{equation*}
errechnet wurde.
Der Wert weicht somit um etwa 65 Sigma-Umgebungen vom Theoriewert ab, sodass es sich wohl um
einen stystematischen Fehler handelt.

\noindent Der berechnete Wert der Austrittsarbeit von Wolfram weicht um etwa
$5.73 \%$ von dem Theoriewert von $\SI{4.54}{\electronvolt}$ \cite{q5} ab, was etwa
13 Sigma-Umgebungen entspricht. Auch hier scheint also ein systematischer Fehler
vorzuliegen.
