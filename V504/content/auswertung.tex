\section{Auswertung}
\subsection{Bestimmung des Sättigungsstroms aus den Kennlinien}
Zur Erstellung der 5 Kennlinien werden die jeweils gemessenen Wertepaare aus
Spannung und Anodenstrom, welche in Tabelle \ref{tab:tabe1} abzulesen sind
gegeneinander aufgetragen, wie in Abbildung \ref{fig:plot1} zu sehen ist.
Die Heizströme $I_{H}$ und die entsprechenden Spannung $U_{H}$  der 5 Messungen betragen hierbei:
\begin{align*}
  \text{Messung 1 :} I_{H1} = \SI{2.0}{\ampere} ,\: \: U_{H1} = \SI{4.3}{\volt} \\
  \text{Messung 2 :} I_{H2} = \SI{2.1}{\ampere} ,\: \: U_{H2} = \SI{4.7}{\volt} \\
  \text{Messung 3 :} I_{H3} = \SI{2.2}{\ampere} ,\: \: U_{H3} = \SI{5.1}{\volt} \\
  \text{Messung 4 :} I_{H4} = \SI{2.3}{\ampere} ,\: \: U_{H4} = \SI{5.4}{\volt} \\
  \text{Messung 5 :} I_{H5} = \SI{2.4}{\ampere} ,\: \: U_{H5} = \SI{5.9}{\volt} \\
\end{align*}

\begin{table}[H]
  \centering
  \caption{Messwerte und Ergebniss der Bestimmung der Schallgeschwindigkeit}
  \label{tab:tabe1}
    \begin{tabular}{S||S S||S S||S|S}
    \toprule
    $ \text{Länge l des Zylinders [mm]} $ & $ U_{1} [\text{V}] $ &
    $ t_{1} [\mu\text{s}] $ & $ U_{2} [\text{V}] $ &
    $ t_{2} [\mu\text{s}] $ & $ \increment t [\mu\text{s}]$ &
    $ \text{c} [\text{m}/\text{s}]$\\
    \midrule
    31.0 & 1.335 \: & 24.0 & 1.096 \:  & 46.9 & 22.9 & 2707.42 \\
          \bottomrule
    \end{tabular}
  \end{table}


\begin{figure}[H]
  \centering
  \includegraphics{plot1.pdf}
  \caption{Messwerte und Ausgleichsfunktionen der Messungen}
  \label{fig:plot1}
\end{figure}

Die Ausgleichsfunktionen werden dabei durch einen Fit mit der Funktion
\begin{equation}
  f(x) = s- \exp{(-c\cdot d)}
\end{equation}
erstellt, wobei der Parameter s den Sättigungsstrom angibt.
Aus dem Diagramm lassen sich ungfähr Heizströme von
\begin{align*}
  I_{S1} &\approx \SI{0.10}{\milli\ampere}  \\
  I_{S2} &\approx \SI{0.23}{\milli\ampere}  \\
  I_{S3} &\approx \SI{0.55}{\milli\ampere}  \\
  I_{S4} &\approx \SI{1.07}{\milli\ampere}  \\
  I_{S5} &> \SI{2}{\milli\ampere}  \\
\end{align*}
ablesen, wobei dies bei der fünften Kennlinie nicht mehr wirklich möglich ist, es lässt sich nur sagen,
dass der Sättigungsstrom deutlich größer ist als der in dem Diagramm maximal darstellbare
Strom von etwa $\SI{2}{\milli\ampere}$.
Aus der Ausgleichsrechnung ergeben sich zudem noch rechnerisch die folgenden Werte:
\begin{align*}
  I_{S1} &\approx \SI{0.112(2)}{\milli\ampere}  \\
  I_{S2} &\approx \SI{0.233(3)}{\milli\ampere}  \\
  I_{S3} &\approx \SI{0.572(10)}{\milli\ampere}  \\
  I_{S4} &\approx \SI{1.441(72)}{\milli\ampere}  \\
  I_{S5} &\approx \SI{32(23)}{\milli\ampere} \\
\end{align*}

\subsection{Bestimmung des Exponenten im Raumladungsgebiet}
Die Messwerte des maximalen Heizstroms von $\SI{2.5}{\ampere}$ bei $\SI{6.2}{\volt}$
sind in Tabelle \ref{tab:tabe2} abzulesen.
\begin{table}[H]
  \centering
  \caption{Zählrate und Energiemaximum bei variiertem Druck, Abstand a=2cm}
  \label{tab:tab2}
    \begin{tabular}{c c c c c}
    \toprule
    Druck $\rho$/\;mbar & Energiemaximum & Zählrate $N$ & Energie $E_{\alpha}$ & effektive Länge $x$/\;cm\\
    \midrule
    0 & 796 &131382  &4          & 0.0   \\
    50 & 775 &131464 &3.89 & 0.09 \\
    100 &756 &130732 &3.79 & 0.19\\
    150 &749 &129617 &3.76  &  0.29\\
    200 &749 &130444 &3.76  & 0.39\\
    250 &727 &129600 &3.65 & 0.49\\
    300 &722 &128936 &3.63 & 0.59\\
    350 &708 &128478 &3.56 & 0.69\\
    400 &696 &128122 &3.49 & 0.79\\
    450 &687 &127415 &3.45 & 0.89\\
    500 &674 &126608 &3.39 & 0.99\\
    550 &663 &126372 &3.33 &1.09\\
    600 &651 &124989 &3.27 & 1.18\\
    650 &634 &124942 &3.19 & 1.28\\
    700 &618 &124295 &3.11 &1.38\\
    750 &602 &123299 &3.03 & 1.48\\
    800 &584 &119958 &2.93 &1.58\\
    850 &566 &120673 &2.84 &1.68\\
    900 &548 &117907 &2.75 & 1.78\\
    950 &534 &116111 &2.68&   1.88\\
    1000 &499& 108630&2.51 & 1.07\\
    \bottomrule
    \end{tabular}
  \end{table}

Das Langmuir-Schottkysche Raumladungsgesetz gilt dabei im gesamten Wertebereich.
Zur Bestimmung des Exponenten der Strom-Spannungs-Beziehung werden die Messwerte
in einem logarithmischen Diagramm (\ref{fig:plot2}) gegeneinander aufgetragen und
es wird eine lineare Ausgleichsrechnung der Form
\begin{equation}
  y = a\cdot x +b
  \label{eqn:linear}
\end{equation}
durchgeführt.

\begin{figure}[H]
  \centering
  \includegraphics{plot2.pdf}
  \caption{Logaritmisches Diagramm des Raumladungsgebiets mit linearer Ausgleichsfunktion.}
  \label{fig:plot2}
\end{figure}

Hieraus ergeben sich die Parameter
\begin{align*}
  a = 1.172 \pm 0.005 \\
  b = -5.4 \pm 0.02 \: ,\\
\end{align*}
wobei $a$ den gesuchten Exponenten angibt.
\subsection{Bestimmung der Kathodentemperatur aus dem Anlaufstromgebiet}
Die Messwerte des Anlaufstromgebietes für einen Heizstrom von $\SI{2.5}{\ampere}$
befinden sich in Tabelle \ref{tab:tabe3}.
\begin{table}[H]
  \centering
   \begin{tabular}{c c c}
    \toprule
     n& $\nu$/\; 1/s & $\nu_{Wechsel}$\\
    \midrule
    0,5 & 100.01& 50,0\\
    1 & 79.93 & 79.93\\
    2 & 23.93 & 47.86\\
    \bottomrule
  \end{tabular}
  \caption{Gemessene Frequenzen der Sägezahnspannung, sowie die Daraus resultierenden Frequenzen für die
  Wechselspannung.}
  \label{tab:tab3}
\end{table}

Diese werden in einem halblogathmischen Diagramm (Diagramm \ref{fig:plot3}) gegeneinander
aufgetragen und es wird erneut eine lineare Ausgleichsrechnung mit der Funktion \ref{eqn:linear}
durchgeführt.

\begin{figure}[H]
  \centering
  \includegraphics{plot3.pdf}
  \caption{Halblogogaritmisches Diagramm des Anlaufsstroms mit linearer Ausgleichsfunktion.}
  \label{fig:plot3}
\end{figure}
Hieraus ergeben sich die Parameter
\begin{align*}
  a &= -4.47 \pm 0.05 \: \frac{1}{V}\\
  b &= 2.33 \pm 0.03  \: .\\
\end{align*}
Aus der Gleichung \ref{eqn:anlauf} folgt, dass die Kathodentemperatur T gegeben ist durch
durch die Gleichung
\begin{equation}
  T = \frac{e_0}{k \cdot a} \: ,
\end{equation}
wobei $e_0= 1,602 \cdot 10^{-19} \si{\coulomb} $ \cite{q1} die Elemantarladung bezeichnet,
und $k=1,3806 \cdot 10^{-23} \si{\joule\per\kelvin} $ \cite{q2} die Boltzmann-Konstante.
Somit ergibt sich also insgesamt eine Kathodentemperatur von
\begin{align*}
  T= \SI{2596(32)}{\kelvin} \: ,
\end{align*}
wobei sich der Fehler aus der Gauß´schen Fehlerfortpflanzung
\begin{equation}
  \increment f = \sqrt{ \sum_{i=1}^N \left( \frac{\partial f}{\partial x_i}\right)^2
  \cdot (\increment x_i)^2  }
  \label{eqn:gaus}
\end{equation}
ergibt, also in diesem Fall
\begin{equation}
  \increment T = \sqrt{(\frac{e_0}{k \cdot a^2})^2 \cdot (\increment a)^2} \: .
\end{equation}
\subsection{Bestimmung der Kathodentemperatur}
Zur Bestimmung der Kathodentemperatur wird die Formel \ref{eqn:????} verwendet, wobei
die Werte der Heizspannung und des Heizstromes aus dem ersten Auswertungsteil zu
entnehmen sind. Hieraus ergeben sich die folgenden Werte der Kathodentemperatur :
\begin{align*}
  T_1= \SI{1964.07}{\kelvin} \\
  T_2= \SI{2041.43}{\kelvin} \\
  T_3= \SI{2115.03}{\kelvin} \\
  T_4= \SI{2174.56}{\kelvin} \\
  T_5= \SI{2253.03}{\kelvin} \\
\end{align*}
\subsection{Bestimmung der Austrittsarbeit}
Zur Bestimmung der Austrittsarbeit wird Formel \ref{eqn:satt} verwendet, wobei
$m_0 = \SI{9.109e-31}{\kilo\gram} $ \cite{q3} die Elektronenmasse und $h= \SI{6.626e-34}{\joule\second}$ \cite{q4}
das Plancksche Wirkungsquantum bezeichnet.
Aus den jeweiligen Sättigungsstrom (abgelesen) und Temperatur Wertepaaren lassen sich
folgende Werte der Austrittsarbeit für Wolfram bestimmen:
\begin{align*}
  W_{A1} &= \SI{4.74}{\electronvolt} \\
  W_{A2} &= \SI{4.80}{\electronvolt} \\
  W_{A3} &= \SI{4.82}{\electronvolt} \\
  W_{A4} &= \SI{4.85}{\electronvolt} \: .\\
\end{align*}
Dabei lässt sich für die fünfte Messreihe keine Austrittsarbeit bestimmen, da sich
kein genauer Sättigungsstrom ablesen lässt.
Durch die Gleichung
\begin{equation}
  \bar{x} = \frac{1}{N} \sum_{i=1}^{N} x_i \: \:
  \label{eqn:mit}
\end{equation}
\noindent wird der Mittelwert gebildet, wobei der dazugehörige Fehler sich durch
\begin{equation}
  \increment \bar{x} = \frac{1}{\sqrt{N}} \sqrt{ \frac{1}{N-1} \sum_{i=1}^N
  (x_i - \bar{x})^2}
  \label{eqn:mitf}
\end{equation}
ergibt, sodass für die gemittelte Austrittsarbeit folgt :
\begin{align*}
  W_A = \SI{4.80(2)}{\electronvolt} \\
\end{align*}
