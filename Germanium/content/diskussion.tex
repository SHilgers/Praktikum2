\section{Diskussion}
Die lineare Regression zur Bestimmung der Energiekalibration liefert sehr geringe Ungenauigkeiten
der Fitparameter und scheint somit sehr exakt zu sein, was für ein gutes Auflösungsvermögen
des Detektors spricht. Die Parameter des Fits für die Vollenergienachweiswahrscheinlichkeit sind hingegen
mit einem großen relativen Fehler behaftet, welche teilweise sogar größer als der
eigentliche Wert sind. Dies lässt darauf schließen, dass eine Potenzfunktion
nicht vollständig zur Beschreibung der Energieabhängigkeit geeignet ist und
eine andere Funktion eventuell besser zu den Daten passen würde. \\
Die Abweichung in der gemessenen Energie des $\ce{^{137}Cs}$-Strahlers beträgt lediglich
0.0088\%, wodurch erneut die gute Energieauflösung gezeigt wird. Wie bereits erwähnt
ist die Beschreibung der Peaks durch Gaußkurve sehr genau und weißt im Verhältniss
von Halb- und Zehntelwertsbreite nur eine Abweichung von 1.64 \% auf. \\
Auch die Componkante und das Comptonkontinuum können recht genau bestimmt werden,
jedoch ergibt sich bei dem Rückstreupeak eine Abweichung von 5,14 \% , wobei der Theoriewert etwa 39,44
Fehlerintervalle von dem experimentellen Wert abweicht. Es wird somit vermutlich ein systematischer Fehler
vorliegen. \\
Auch bei dem Verhältniss des Inhalts von Comptonkontinuum und Photopeak gibt es eine
sehr große Abweichung, was vermutlich daran liegt, dass Mehrfachstreuung beim
Comptoneffekt in den theoretischen Formeln nicht berücksichtigt wird. Durch diese
Mehrfachstreuung deponiert die Gammastrahlung einen größeren Anteil oder eventuell auch die
gesamte Energie in dem Detektor, sodass der Inhalt des Comptonkontinuums sinkt und der
Inhalt des Photopeaks steigt, was den Betrag des Verhältnisses ansteigen lässt. \\
Die Aktivität lässt sich ebenfalls bis auf einen geringen
relativen Fehler bestimmen und auch die Zuordnung der Proben ist aufgrund der
guten Energieauflösung problemlos möglich. Lediglich die $\SI{92.38(1)}{\kilo\electronvolt}$-Linie
und die $\SI{92.80(1)}{\kilo\electronvolt}$-Linie von $\ce{^{234}Th}$ können nicht
getrennt aufgelöst werden, sondern erscheinen als ein einzelner Peak. Hier liegt offensichtlich
die Grenze des Auflösungsvermögens.
