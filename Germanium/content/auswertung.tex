\section{Auswertung}
\label{sec:Auswertung}
\begin{align*}
  a = -191 \pm 64 \\
  c = 2474 \pm 3467 \\
  d = -1.47 \pm 0.19\:,
\end{align*}
\subsection{Energiekalibration und Bestimmung der Vollenergienachweiswahrscheinlichkeit}
Zur Kalibration wird ein $\ce{^{152}Eu}$-Strahler verwendet, dessen Aktivität am 01.10.2000
%\begin{align*}
%  \SI{4130(60)}{\becquerel}
%\end{align*}
$\SI{4130(60)}{\becquerel} $ betrug. \\
Nach dem Gesetz des radioaktiven Zerfalls berechnet sich die Aktivität am Messtag (08.04.2019) durch

\begin{equation}
  \symup{A} (t) = \symup{A}(0)\cdot \symup{e}^{-\lambda t} \: ,
\end{equation}

wobei $\lambda=\SI{1.6244(19)e-9}{\per\second}$ \cite{lara} die Zerfallskonstante
von $\ce{^{152}Eu}$ bezeichnet.

Der Fehler ergibt sich hierbei nach der Gauß´schen Fehlerfortpflanzung
\begin{equation}
  \increment f = \sqrt{ \sum_{i=1}^N \left( \frac{\partial f}{\partial x_i}\right)^2
  \cdot (\increment x_i)^2  } \: ,
  \label{eqn:gaus}
\end{equation}
also gemäß
\begin{equation}
  \increment \symup{A} (t) = \sqrt{ (\symup{e}^{-\lambda t})^{2}\cdot (\increment \symup{A}(0))^2
   + (-t\cdot\symup{A}(0)\cdot \symup{e}^{-\lambda t})^2\cdot(\increment \lambda)^2}
\end{equation}
Die Anzahl der Tage vom 01.10.2000 bis zum 08.04.2019 beträgt 6763 Tage, was
584323200 Sekunden entspricht, sodass sich insgesamt der Wert $\SI{1599(29)}{\becquerel} $
für die Aktivität der Probe am Messtag ergibt. \\
Der agbedeckte Raumwinkel lässt sich aus dem gemessenen Abstand a der Probe
zum Detektor, wobei auch der Abstand von $\SI{1.5}{\centi\meter}$ zwischen Al-Haube und Detektor
berücksichtigt wird,
und dem angegebenen Radius r des Detektorvolumens bestimmen. Die entsprechenden Werte
betragen
\begin{align*}
  a = \SI{8.8}{\centi\meter} \\
  r = \SI{2.25}{\centi\meter} \: .
\end{align*}

Die Formel zur Berechnung des abgedeckten Raumwinkelanteils ergibt sich dabei
über geometrische Überlegungen zu
\begin{equation}
  \frac{\Omega}{4\pi}= \frac{1}{2}(1-\frac{a}{\sqrt{a^2+r^2}})   \: ,
\end{equation}
in diesem Fall also $\frac{\Omega}{4\pi}= 0.01558$.
Diese somit errechneten Werte sind später wichtig zur Bestimmung der Vollenergienachweiswahrscheinlichkeit. \\
Das gemessene Spektrum des kalibrierten $\ce{^{152}Eu}$-Strahlers ist in Abbildung
\ref{fig:plot1} dargestellt. Es sind jedoch nur die ersten 4000 Kanäle dargestellt,
da bei höheren Kanälen keine signifikanten Messwerte mehr zu sehen sind. Die Messwerte
reichen bis Kanalnummer 8191 und die Messzeit beträgt $\SI{3598}{\second}$.
\begin{figure}
  \centering
  \includegraphics[height=9cm]{Eu.pdf}
  \caption{Spektrum des $\ce{^{152}Eu}$-Strahlers}
  \label{fig:plot1}
\end{figure}

Um mit diesem die Energiekalibration durchzuführen, werden die Peaks des Spektrums
jeweils mit einer Gaußverteilung der Form
\begin{equation}
  \symup{g} (x) = a + b \cdot \symup{e}^{(\frac{x-z}{c})^2}
  \label{eqn:gausk}
\end{equation}
gefittet.

Die sich daraus ergebenen Parameter sind in Tabelle \ref{tab:tabe1} angegebenen.
\begin{table}[H]
  \centering
  \caption{Messwerte und Ergebniss der Bestimmung der Schallgeschwindigkeit}
  \label{tab:tabe1}
    \begin{tabular}{S||S S||S S||S|S}
    \toprule
    $ \text{Länge l des Zylinders [mm]} $ & $ U_{1} [\text{V}] $ &
    $ t_{1} [\mu\text{s}] $ & $ U_{2} [\text{V}] $ &
    $ t_{2} [\mu\text{s}] $ & $ \increment t [\mu\text{s}]$ &
    $ \text{c} [\text{m}/\text{s}]$\\
    \midrule
    31.0 & 1.335 \: & 24.0 & 1.096 \:  & 46.9 & 22.9 & 2707.42 \\
          \bottomrule
    \end{tabular}
  \end{table}


Die zentrale Lage der Peaks im Hinblick auf die Kanalnummer ist durch den Parameter
z gegeben. Diese werden zusammen mit der jeweiligen relativen Höhe mit den theoretischen
Emissionslinien der Datenbank \cite{lara} verglichen und es wird jedem Peak eine Linie
zugeordent. Diese Zuordnung ist zusammen mit der jeweiligen relativen Emissionswahrscheinlichkeit P
in Tabelle \ref{tab:tabe2} dargestellt.
\begin{table}[H]
  \centering
  \caption{Zählrate und Energiemaximum bei variiertem Druck, Abstand a=2cm}
  \label{tab:tab2}
    \begin{tabular}{c c c c c}
    \toprule
    Druck $\rho$/\;mbar & Energiemaximum & Zählrate $N$ & Energie $E_{\alpha}$ & effektive Länge $x$/\;cm\\
    \midrule
    0 & 796 &131382  &4          & 0.0   \\
    50 & 775 &131464 &3.89 & 0.09 \\
    100 &756 &130732 &3.79 & 0.19\\
    150 &749 &129617 &3.76  &  0.29\\
    200 &749 &130444 &3.76  & 0.39\\
    250 &727 &129600 &3.65 & 0.49\\
    300 &722 &128936 &3.63 & 0.59\\
    350 &708 &128478 &3.56 & 0.69\\
    400 &696 &128122 &3.49 & 0.79\\
    450 &687 &127415 &3.45 & 0.89\\
    500 &674 &126608 &3.39 & 0.99\\
    550 &663 &126372 &3.33 &1.09\\
    600 &651 &124989 &3.27 & 1.18\\
    650 &634 &124942 &3.19 & 1.28\\
    700 &618 &124295 &3.11 &1.38\\
    750 &602 &123299 &3.03 & 1.48\\
    800 &584 &119958 &2.93 &1.58\\
    850 &566 &120673 &2.84 &1.68\\
    900 &548 &117907 &2.75 & 1.78\\
    950 &534 &116111 &2.68&   1.88\\
    1000 &499& 108630&2.51 & 1.07\\
    \bottomrule
    \end{tabular}
  \end{table}

Mit den Wertepaaren aus Kanalnummer und Linienenergie wird nun eine lineare Ausgleichsrechnung der
Form
\begin{equation*}
  f(x) = a\cdot x +b
\end{equation*}
durchgeführt, woraus sich die Parameter
\begin{align}
  a = \SI{0.403169(29)}{\kilo\electronvolt} \\
  b = \SI{-3.034(60)}{\kilo\electronvolt}
\end{align}
ergeben. Die Wertepaare sind zusammen mit der resultierenden Gerade in Abbildung \ref{fig:plot3}
dargestellt. Die Fehler der Messwerte sind aufgrund ihrer sehr geringen relativen Größe dabei zu vernachlässigen.
\begin{figure}
  \centering
  \includegraphics[height=9cm]{plot3.pdf}
  \caption{Lineare Ausgleichsrechnung zur Energiekalibration}
  \label{fig:plot3}
\end{figure}
Die Energiekalibration erfolgt somit gemäß
\begin{equation}
  \symup{E}_{\gamma} (x) = \SI{0.403169}{\kilo\electronvolt}\cdot x
  \label{eqn:gerade}
\end{equation}
wobei x die Kanalnummer bezeichnet.
Der dazugehörige Fehler ergibt mittels Gleichung \ref{eqn:gaus} durch
\begin{equation}
   \increment \symup{E}_{\gamma} (x) =\sqrt{ (\SI{0.000029}{\kilo\electronvolt}\cdot x)^2 +
   (\SI{60}{\kilo\electronvolt})^2}
   \label{eqn:fgerade}
\end{equation}
\\
Zur Bestimmung der Vollenergienachweiswahrscheinlichkeit wird zunächst die Gleichung \ref{eqn:gausk}
integriert, um den Inhalt der Peaks zu bestimmen, wobei der Untegrund $a$ vorher abgezogen wird.
Es ergibt sich somit ein Linieninhalt von
\begin{equation}
  \symup{I} =\int_{-\infty}^{\infty} c \cdot \symup{e}^{(\frac{x-z}{b})^2} \symup{d}x
  = c \cdot b \cdot \sqrt{\pi}
  \label{eqn:inh}
\end{equation}
in Abhängigkeit der Parameter b und c.
Der Fehler ergibt sich durch Gleichung \ref{eqn:gaus} über die Gleichung
\begin{equation}
  \increment \symup{I} = \sqrt{ (\increment c \cdot b \cdot \sqrt{\pi})^{2}
   + (c \cdot \increment b \cdot \sqrt{\pi}})^{2} \: .
     \label{eqn:inhf}
\end{equation}
Mit den Werten aus Tabelle \ref{tab:tabe1} lassen sich somit die einzelnen Linieninhalte berrechen,
welche in Tabelle \ref{tab:tabe3} angegeben sind.
\begin{table}[H]
  \centering
   \begin{tabular}{c c c}
    \toprule
     n& $\nu$/\; 1/s & $\nu_{Wechsel}$\\
    \midrule
    0,5 & 100.01& 50,0\\
    1 & 79.93 & 79.93\\
    2 & 23.93 & 47.86\\
    \bottomrule
  \end{tabular}
  \caption{Gemessene Frequenzen der Sägezahnspannung, sowie die Daraus resultierenden Frequenzen für die
  Wechselspannung.}
  \label{tab:tab3}
\end{table}

Zum Vergleich werden nun die Theoriewerte berrechnet, als Produkt
der Emissionswahrscheinlichkeiten P aus Tabelle \ref{tab:tabe2},
dem abgedeckten Raumwinkelanteil $\frac{\Omega}{4\pi}= 0.01558$, der errechneten Aktivität
$A =\SI{1599(29)}{\becquerel} $
und der Messzeit von $t = \SI{3598}{\second}$
\begin{equation}
  \symup{I}_{\text{theo}} = P\cdot \frac{\Omega}{4\pi} \cdot A \cdot t
\end{equation}
mit dem Fehler über Gleichung \ref{eqn:gaus} von
\begin{equation}
  \increment \symup{I}_{\text{theo}} = \sqrt{ (\increment P\cdot \frac{\Omega}{4\pi} \cdot A \cdot t)^{2}
   + (P\cdot \frac{\Omega}{4\pi} \cdot \increment A \cdot t)^{2}} \: .
\end{equation}
Aus dem jeweiligen Quotienten
\begin{equation}
  \symup{Q} = \frac{\symup{I}}{\symup{I}_{\text{theo}}}
\end{equation}
mit dem dazugehörigen Fehler
\begin{equation}
  \increment \symup{Q} = \sqrt{ (\frac{1}{\symup{I}_{\text{theo}} \cdot \increment \symup{I})^{2}
   + (\frac{\symup{I}}{\symup{I}_{\text{theo}}})^{2}}\cdot \increment \symup{I}_{\text{theo}})^{2}}
\end{equation}
ergibt sich somit jeweils die Nachweiswahrscheinlichkeit des Peaks, wie in Tabelle
\ref{tab:tabe4} dargestellt ist.
\begin{table}[H]
  \centering
   \begin{tabular}{c c c c}
    \toprule
    Nummer der Oberwelle & $ U_{\text Theorie,Rechteck}\: / \si{\volt} $ &
    $ U_{\text Theorie,Dreick}\: / \si{\volt} $ & $ U_{\text Theorie,Sägezahn}\: / \si{\volt} $ \\
    \midrule
    1 & 1145 & 182 & 573 \\
    2 & 0 & 0 & 286 \\
    3 & 573 & 20 & 191 \\
    4 & 0 & 0 & 143 \\
    5 & 229 & 7 & 115 \\
    6 & 0 & 0 & 96 \\
    7 & 164 & 4 & 82 \\
    8 & 0 & 0 & 72 \\
    9 & 127 & 2 & 64 \\
    10 & 0 & 0 & 57 \\
    \bottomrule
  \end{tabular}
  \caption{Eingestellte Schwingungsamplituden.}
  \label{tab:tabe4}
\end{table}

Da die Nachweiswahrscheinlichkeit im Allgemeinen energieabhängig ist, wird Q in
Abhängigkeit von $ \text{E}_{\gamma} $ dargestellt und mit einer Potenzfunktion der Form
\begin{equation}
  \symup{Q}(\text{E}_{\gamma}) = c\cdot (\text{E}_{\gamma}-a)^{d}
\end{equation}
gefittet, wie in Abbildung \ref{fig:plot4} dargestellt ist.
Es ergeben sich hierbei die Parameter

\begin{align*}
  a = -191 \pm 64\\
  c = 2474 \pm 3467\\
  d = -1.47 \pm 0.19\:,
\end{align*}

welche zum Teil mit einem großen Fehler behaftet sind.
\begin{figure}
  \centering
  \includegraphics[height=9cm]{plot4.pdf}
  \caption{Werte zur Bestimmung der Vollenergienachweiswahrscheinlichkeit sowie gefittete Potenzfunktion}
  \label{fig:plot4}
\end{figure}

\section{Untersuchung eines monochromatischen Gamma-Spektrums}
Zur Untersuchung des monochromatischen Gamma-Spektrums wird das aufgenommene Spektrum
zunächst durch die Gleichung \ref{eqn:gerade} kalibriert, wobei sich der Fehler über
\ref{eqn:fgerade} ergibt. Die so erhaltenen Werte sind in Abbildung \ref{fig:plot5}
dargestellt, wobei auf Fehlerbalken aufgrund der geringen Fehler verzichtet wurde. Die
Messzeit beträgt $\SI{2593}{\second}$ und es wurden 8191 Kanäle gemessen, wobei nur die
ersten 2000 dargestellt sind, da bei höheren Kanalnummern keine signifikanten Messwerte
mehr zu erkennen sind.
\begin{figure}
  \centering
  \includegraphics[height=9cm]{Cs.pdf}
  \caption{Kalibriertes Spektrum des $\ce{^{137}Cs}$-Strahlers }
  \label{fig:plot5}
\end{figure}
Zur Bestimmung der Energie wird die Vollenergielinie mit der Gaußverteilung aus Gleichung
\ref{eqn:gausk} gefittet, wobei sich die Parameter
\begin{align*}
  a = 5.7 \pm 4.2 \\
  b = 1674 \pm 17 \\
  c = \SI{1.227(15)}{\kilo\electronvolt}\\
  z = \SI{661.5985(98)}{\kilo\electronvolt} \:
\end{align*}
ergeben.
Die Energie des Strahlers ist dabei durch den Parameter $z$ gegeben, also
\begin{align*}
  \symup{E}_{Cs} = \SI{661.5985(98)}{\kilo\electronvolt} \: .
\end{align*}
Der Bereich um die Vollenergielinie ist zusammen mit der gefitteten Gaußkurve in
Abbildung \ref{fig:plot6} dargestellt.
\begin{figure}
  \centering
  \includegraphics[height=9cm]{Plot6.pdf}
  \caption{Vollenergielinie mit Gaußkurve}
  \label{fig:plot6}
\end{figure}
Die Halbwertsbreite einer Gaußkurve gemäß Gleichung \ref{eqn:gausk} ist durch die
Formel
\begin{equation}
  \symup{E}_{1/2} = 2c*\sqrt{ln2}
\end{equation}
mit dem Fehler
\begin{equation}
  \increment \symup{E}_{1/2} = 2\increment b*\sqrt{ln2} \: ,
\end{equation}
gegeben und beträgt somit
\begin{align*}
  \symup{E}_{1/2} = \SI{1.701(21)}{\kilo\electronvolt} \: .
\end{align*}
Die Zehntelwertsbreite ergibt sich analog über die Gleichung
\begin{equation}
  \symup{E}_{1/10} = 2c*\sqrt{ln10}
\end{equation}
mit dem Fehler
\begin{equation}
  \increment \symup{E}_{1/10} = 2\increment b*\sqrt{ln10} \: ,
\end{equation}
zu
\begin{align*}
  \symup{E}_{1/10} = \SI{5.651(69)}{\kilo\electronvolt} \: .
\end{align*}
Der Inhalt der Vollenergielinie ergibt sich erneut durch Gleichungen \ref{eqn:inh}
und \ref{eqn:inhf} zu
\begin{align*}
  \symup{I}_{VEL} =  3641 \pm 58 \: .
\end{align*}
\\
Aus den Messwerten und Abbildung \ref{fig:plot5} lässt sich erkennen, dass die
Compton-Kante bei etwa $\SI{478}{\kilo\electronvolt}$ liegt, da dort das Spektrum ein lokales
Maximum animmt und anschließend abfällt.
Der Rückstreupeak wird erneut mit der Gleichung \ref{eqn:gausk} gefittet, wodurch sich
die Parameter
\begin{align*}
  a = 65.8 \pm 2.0 \\
  b = 36 \pm 28 \\
  c = \SI{0.28(28)}{\kilo\electronvolt}\\
  z = \SI{193.78(24)}{\kilo\electronvolt} \:
\end{align*}
ergeben, der Rückstreupeak liegt also bei $\SI{193.78(24)}{\kilo\electronvolt}$.
