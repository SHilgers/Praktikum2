\section{Durchführung}
\label{sec:Durchführung}
Zunächst wird eine kalibrierte $\ce{^{152}Eu}$-Probe vermessen die später zur Energiekalibration
genutzt wird. Die Probe wird in die Probenhalterung des Detektors eingespannt, um für alle
Proben den gleichen Abstand zum Detektor zu garantieren wird ein Messingstab als Abstandshalter
verwendet, dieser wird vor der Messung natürlich wieder entfernt.
Auf dem Computer wird nun die Messung gestartet, die Messzeit beträgt ca. 3600\;s $\sim$ eine Stunde.\\
Für $\ce{^{137}Cs}$ und und eine weitere Probe (entweder $\ce{^{125}Sb}$ oder $\ce{^{133}Ba}$ )
wird äquivalent vorgegangen. Zuletzt wird eine unbekannte Probe vermessen, da diese eine andere
Form besitzt und nicht in den Probenhalter passt wird sie vor der Aluminiumschutzhaube des Detektors
platziert.
