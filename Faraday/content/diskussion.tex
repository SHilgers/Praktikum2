
\section{Diskussion}
\label{sec:Diskussion}
Das experimentell bestimmte Verhältnis der effektiven Masse zur Ruhemasse ergibt sich zu
\begin{align*}
  \frac{m*}{m_e}=0,079\pm0,004 \: ,
\end{align*}
wobei der Theoriewert für dieses Verhältnis
\begin{align*}
  \frac{m*}{m_e}_{\text{theo}}=0,067 \: ,
\end{align*}
beträgt \cite{online2}.
Die Abweichung lässt sich dabei über die Formel
\begin{equation}
  \frac{\lvert \text{Wert}_{\text{Theorie}}-\text{Wert}_{\text{Messung}}\rvert}{\text{Wert}_{\text{Theorie}}}
  \label{eqn:abw}
\end{equation}
zu 17,91\% bestimmen und entspricht einem Intervall von 3 Standardabweichungen,
was auf einen systematischen Fehler in der Messung hindeutet. Dies könnte an Fehlern
bei der Justage liegen, was zum Beispiel
daran deutlich wurde, dass die Minima bei einem Abstand von 75° lagen anstatt
eines Abstandes von 90°, wie es eigentlich seien sollte.
