
\section{Auswertung}
\label{sec:Auswertung}
\subsection{Bestimmung des Magnetfeldes}

Die Messwerte zur Bestimmung des Magnetfeldes sind in Tabelle \ref{tab:tabe1} angegeben.
\begin{table}[H]
  \centering
  \caption{Messwerte und Ergebniss der Bestimmung der Schallgeschwindigkeit}
  \label{tab:tabe1}
    \begin{tabular}{S||S S||S S||S|S}
    \toprule
    $ \text{Länge l des Zylinders [mm]} $ & $ U_{1} [\text{V}] $ &
    $ t_{1} [\mu\text{s}] $ & $ U_{2} [\text{V}] $ &
    $ t_{2} [\mu\text{s}] $ & $ \increment t [\mu\text{s}]$ &
    $ \text{c} [\text{m}/\text{s}]$\\
    \midrule
    31.0 & 1.335 \: & 24.0 & 1.096 \:  & 46.9 & 22.9 & 2707.42 \\
          \bottomrule
    \end{tabular}
  \end{table}

Die Messwerte weisen dabei ein Maximum bei einer Magnetfeldstärke von $\SI{0.436}{\tesla}$ auf, welches
im Folgenden für die weitere Auswertung genutzt wird als Magnetfeldstärke am Ort der Probe.
Die Werte folgen einer Kurve der Form \cite{skript2}
\begin{equation*}
  B(z)=a\cdot(b+(z-z_0)^2)^{-3/2} \: ,
\end{equation*}
wobei sich durch einen Fit die Werte
\begin{align*}
  a &= \SI{3.85(10)}{\milli\tesla\per\milli\metre\cubic}\\
  b &= \SI{426(7)}{\milli\metre\squared}\\
  z_0 &= \SI{6.267(28)}{\milli\metre}
\end{align*}
ergeben. Die Ausgleichskurve ist zusammen mit den Messwerten in Abbildung \ref{fig:plot1}
dargestellt.\\
\begin{figure}
  \centering
  \includegraphics[width=0.7\textwidth]{plot1.pdf}
  \caption{Messwerte zur Bestimmung des Magnetfeldes zusammen mit der entsprechenden Ausgleichskurve}
  \label{fig:plot1}
\end{figure}
\subsection{Bestimmung der Faraday Rotation und der effektiven Masse}
Aus den jeweils gemessenen Winkeln $\Theta_1$ und $\Theta_2$ wird über die Formel
\begin{equation}
  \Delta \Theta =(\Theta_2-\Theta_1)/2
\end{equation}
die Differenz der Drehwinkel berechnet, wobei der Faktor $\frac{1}{2}$ verwendet wird, da
sich das Magnetfeld effektiv aufgrund der Umpolung um 2$\lvert B \rvert$ ändert.
Dieser Wert wird anschließend in Radiant umgerechnet und auf die Länge $L$ der
jeweiligen Probe normiert,
über die Formel
\begin{equation}
  \frac{\Delta \Theta \cdot 2\pi}{L\cdot360°} \: .
\end{equation}
Diese Werte sind zusammen mit den ursprünglichen Messwerten in Tabelle \ref{tab:tabe2} für hochreines
GaAs angegeben, in Tabelle \ref{tab:tabe3} für eine Dotierung von N=$1,2\cdot10^{18}$ und
in Tabelle \ref{tab:tabe4} für eine Dotierung von N=$2,8\cdot10^{18}$.
\begin{table}[H]
  \centering
  \caption{Zählrate und Energiemaximum bei variiertem Druck, Abstand a=2cm}
  \label{tab:tab2}
    \begin{tabular}{c c c c c}
    \toprule
    Druck $\rho$/\;mbar & Energiemaximum & Zählrate $N$ & Energie $E_{\alpha}$ & effektive Länge $x$/\;cm\\
    \midrule
    0 & 796 &131382  &4          & 0.0   \\
    50 & 775 &131464 &3.89 & 0.09 \\
    100 &756 &130732 &3.79 & 0.19\\
    150 &749 &129617 &3.76  &  0.29\\
    200 &749 &130444 &3.76  & 0.39\\
    250 &727 &129600 &3.65 & 0.49\\
    300 &722 &128936 &3.63 & 0.59\\
    350 &708 &128478 &3.56 & 0.69\\
    400 &696 &128122 &3.49 & 0.79\\
    450 &687 &127415 &3.45 & 0.89\\
    500 &674 &126608 &3.39 & 0.99\\
    550 &663 &126372 &3.33 &1.09\\
    600 &651 &124989 &3.27 & 1.18\\
    650 &634 &124942 &3.19 & 1.28\\
    700 &618 &124295 &3.11 &1.38\\
    750 &602 &123299 &3.03 & 1.48\\
    800 &584 &119958 &2.93 &1.58\\
    850 &566 &120673 &2.84 &1.68\\
    900 &548 &117907 &2.75 & 1.78\\
    950 &534 &116111 &2.68&   1.88\\
    1000 &499& 108630&2.51 & 1.07\\
    \bottomrule
    \end{tabular}
  \end{table}

\begin{table}[H]
  \centering
   \begin{tabular}{c c c}
    \toprule
     n& $\nu$/\; 1/s & $\nu_{Wechsel}$\\
    \midrule
    0,5 & 100.01& 50,0\\
    1 & 79.93 & 79.93\\
    2 & 23.93 & 47.86\\
    \bottomrule
  \end{tabular}
  \caption{Gemessene Frequenzen der Sägezahnspannung, sowie die Daraus resultierenden Frequenzen für die
  Wechselspannung.}
  \label{tab:tab3}
\end{table}

\begin{table}[H]
  \centering
   \begin{tabular}{c c c c}
    \toprule
    Nummer der Oberwelle & $ U_{\text Theorie,Rechteck}\: / \si{\volt} $ &
    $ U_{\text Theorie,Dreick}\: / \si{\volt} $ & $ U_{\text Theorie,Sägezahn}\: / \si{\volt} $ \\
    \midrule
    1 & 1145 & 182 & 573 \\
    2 & 0 & 0 & 286 \\
    3 & 573 & 20 & 191 \\
    4 & 0 & 0 & 143 \\
    5 & 229 & 7 & 115 \\
    6 & 0 & 0 & 96 \\
    7 & 164 & 4 & 82 \\
    8 & 0 & 0 & 72 \\
    9 & 127 & 2 & 64 \\
    10 & 0 & 0 & 57 \\
    \bottomrule
  \end{tabular}
  \caption{Eingestellte Schwingungsamplituden.}
  \label{tab:tabe4}
\end{table}

Die normierten Winkel $\Delta \Theta$ sind zudem in Abbildung \ref{fig:plot2}
dargestellt in Abhängigkeit der quadrierten Wellenlänge. \\
\begin{figure}
  \centering
  \includegraphics[width=0.7\textwidth]{plot2.pdf}
  \caption{Normierte Winkel $\Delta \Theta$ in Abhängigkeit der quadrierten Wellenlänge $\lambda^2$}
  \label{fig:plot2}
\end{figure}
Um den Einfluss des reinen GaAs aus den Messwerten der dotierten Proben herauszurechnen, wird
für jede Wellenlänge die Differenz der jeweiligen
Werte des normierten Winkels $\Delta \Theta$ von reinem und dotiertem GaAs gebildet.
Mit diesen Werten wird eine lineare Regression der Form
\begin{equation}
  \Delta \Theta(\lambda^2)=a\cdot\lambda^2
\end{equation}
durchgeführt, wobei der y-Achsenabschnitt auf 0 gesetzt wird, da eine Wellenlänge von 0
auch einer Faradayrotation von 0 entsprechen sollte.
Hieraus ergeben sich die Parameter
\begin{align}
  a_1=& (7,2\,\pm\,0,9)\,\cdot\,10^{12}\,\text{m}^{-3} \\
  a_2=& (13,8\,\pm\,1,4)\,\cdot\,10^{12}\,\text{m}^{-3} \: ,
\end{align}
wobei der Index 1 die Probe mit einer Dotierung von N=$1,2\cdot10^{18}$ bezeichnet
und der Index 2 die Probe mit einer Dotierung von N=$2,8\cdot10^{18}$.
Die linearen Ausgleichsgerade sind zusammen mit den Werten in Abbildung \ref{fig:plot3}
dargestellt. \\
\begin{figure}
  \centering
  \includegraphics[width=0.7\textwidth]{plot3.pdf}
  \caption{Differenz der normierte Winkel $\Delta \Theta$
  zwischen reiner und dotierter Probe, sowie lineare Ausgleichsgeraden}
  \label{fig:plot3}
\end{figure}
Aus der Gleichung \eqref{eqn:rot} ergibt sich, dass diese Steigung genau
\begin{equation}
  a=\frac{\text{e}_0^3\,N\,B}{8\pi^2\,\epsilon_0\,\text{c}^3\,(m^*)^2\,n}
\end{equation}
entspricht, woraus sich durch Umformungen die Gleichung
\begin{equation}
  m^*=\sqrt{\frac{\text{e}_0^3\,N\,B}{8\pi^2\,\epsilon_0\,\text{c}^3\,a\,n}}
\end{equation}
für die effektive Masse ergibt.
Der Fehler $\increment m^*$ ergibt sich dabei gemäß der Gauß´schen Fehlerfortpflanzung
\begin{equation}
  \increment f = \sqrt{ \sum_{i=1}^N \left( \frac{\partial f}{\partial x_i}\right)^2
  \cdot (\increment x_i)^2  } \: ,
  \label{eqn:gaus}
\end{equation}
über die Gleichung
\begin{equation}
  \increment m^*=\frac{1}{2}\sqrt{\frac{\text{e}_0\,N\,B}{8\pi^2\,\epsilon_0^3\,\text{c}^3\,a^3\,n}}\increment a \: .
\end{equation}
Die Werte des Brechungsindex $n$ bei der jeweiligen Wellenlänge für GaAs
stammen aus der Quelle \cite{online1} und sind zusammen mit den errechneten Werten in Tabelle
\ref{tab:tabe5}  angegeben. Die verwendeten Werte für die restlichen Naturkonstanten stammen
aus der Quelle \cite{pdg}.
\begin{table}[H]
  \centering
  \caption{Mechanischen Kompressorleistung zu den Zeiten $t_1$, $t_2$, $t_3$ und $t_4$.}
  \label{tab:tabe5}
    \begin{tabular}{S S}
    \toprule
    $ t  \: / \si{\second} $ & $ N_{\text{mech}} \: / \: \si{\watt}$ \\
    \midrule
    480 & 4.72 \pm 0.16 \\
    960 & 6.19 \pm 0.22 \\
    1500 & 6.67 \pm 0.26 \\
    1980 & 6.26 \pm 0.28 \\
      \bottomrule
    \end{tabular}
\end{table}

Hieraus lässt sich der Mittelwert über die Gleichung
\begin{equation}
  \bar{x} = \frac{1}{N} \sum_{i=1}^{N} x_i \: \:
  \label{eqn:mit}
\end{equation}
\noindent bilden, wobei der dazugehörige signifikante Fehler sich gemäß Gleichung \eqref{eqn:gaus} durch
\begin{equation}
  \increment \bar{x} = \sqrt{\sum_{i=1}^N \increment x_i^2}
  \label{eqn:mitf}
\end{equation}
ergibt, da der Gaußfehler größer ist als der Fehler durch die Mittelung
\begin{equation}
  \increment \bar{x} = \frac{1}{\sqrt{N}} \sqrt{ \frac{1}{N-1} \sum_{i=1}^N
  (x_i - \bar{x})^2} \: .
  \label{eqn:mitf}
\end{equation}
Somit ergibt sich ein Wert von
\begin{align*}
  m^*=\SI{7.2(4)e-32}{\kilo\gram} \: \,
\end{align*}
bzw. das Verhältnis
\begin{align*}
  \frac{m*}{m_e}=0,079\pm0,004
\end{align*}
zwischen der effektiven Masse und der Ruhemasse $m_e=\SI{9.109e-31}{\kilo\gram}$ der Elektronen.
