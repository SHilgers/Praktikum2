\section{Auswertung}
\label{sec:Auswertung}
Um die molare Wärmekapazität aus Gleichung \ref{eqn:cmol} zu bestimmen, muss die zugeführte
Wärmemenge ermittelt werden. Diese lässt sich mit den gemessenen Strom-Spannungs-Messwerten über
die Formel
\begin{equation}
  \Delta Q = UI\Delta t
  \label{eqn:zuführ}
\end{equation}
berechnen.
Somit folgt aus Formel \ref{eqn:cmol} und \ref{eqn:zuführ} und einer Stoffmenge $n=\sfrac{m}{M}$
für die Molwärme bei konstantem Druck der Ausdruck
\begin{equation}
  C_p=\frac{UIM\Delta t}{\Delta T m}.
  \label{eqn:Cp}
\end{equation}
Die Masse $m$ der Probe ist mit $m=\SI{0.342}{\kg}$ gegeben und die molare Masse $M$ von
Kupfer beträgt $\SI{63.55}{\g\per\mol}$ \cite{kompress}.
Um die Temperatur zu berechnen wird die monotone Temperaturfunktion der
Pt-100-Widerstände verwendet, welche in Gleichung \ref{eqn:Widerstand}
%\begin{equation}
%  T=0,00134R^2 + 2,296R - 243,02
%  \label{eqn:Pt100}
%\end{equation}
gegeben ist. %Dabei bezeichnet $R$ den Widerstand in Ohm.
Die Messwerte, als auch die berechneten Werte der Probentemperatur und die
Molwärme bei konstantem Druck sind in Tabelle \ref{tab:tab1} angegeben.\\
\\
Als Fehler für die Messgräte (Ampere-, Volt- und Ohm-Meter) wird ein Fehler von
$\pm 1$ auf die erste Nachkommastelle angenommen. Für die Ablesezeit wird ein
Fehler von 3\;s angenommen.
Dabei werden die Fehler mit der Gauß´schen Fehlerfortpflanzung:
\begin{equation}
  \delta f = \sqrt{ \sum_{i=1}^N \left( \frac{\partial f}{\partial x_i}\right)^2
  \cdot (\delta x_i)^2  } \:
  \label{eqn:gaus}
\end{equation}
berechnet. Für $C_P$ ergibt sich:
\begin{equation}
  \delta C_P=\sqrt{\Big(\frac{I\Delta t M}{\Delta T m}\Big)^2 (\delta U)^2+\Big(\frac{U\Delta t M}{\Delta T m}\Big)^2 (\delta I)^2
  +\Big(\frac{UIM}{\Delta m}\Big)^2 (\delta t)^2  +\Big (-\frac{UI\Delta t M}{m (\Delta T)^2}\Big)^2 (\delta\Delta T)^2}.
  \label{eqn:fehler}
\end{equation}
%Hier ist der Unterschied zwischen Temperatur-/Zeitdifferenzen und den jeweiligen Fehlern zu beachten, deshalb
%sind die Fehler hier mit $\delta$ benannt.
\begin{table}[H]
  \centering
  \caption{Messwerte und Ergebniss der Bestimmung der Schallgeschwindigkeit}
  \label{tab:tabe1}
    \begin{tabular}{S||S S||S S||S|S}
    \toprule
    $ \text{Länge l des Zylinders [mm]} $ & $ U_{1} [\text{V}] $ &
    $ t_{1} [\mu\text{s}] $ & $ U_{2} [\text{V}] $ &
    $ t_{2} [\mu\text{s}] $ & $ \increment t [\mu\text{s}]$ &
    $ \text{c} [\text{m}/\text{s}]$\\
    \midrule
    31.0 & 1.335 \: & 24.0 & 1.096 \:  & 46.9 & 22.9 & 2707.42 \\
          \bottomrule
    \end{tabular}
  \end{table}



Mittels der Korrekturformel aus Gleichung \ref{eqn:cpcv}
%\begin{equation}
%  C_p - C_V= 9\alpha^2\kappa V_0 T
%  \label{eqn:korrektur}
%\end{equation}
lässt sich die molare Wärmekapazität bei konstantem Volumen berechnen.
Der Fehler ergibt sich zu
\begin{equation}
  \delta C_V=\sqrt{(\delta C_p)^2 +(-9V_0 \alpha^2 \kappa)^2(\delta T)^2 +(-9TV_02\alpha\kappa)^2(\delta \alpha)^2}.
\end{equation}
Für das Kompressionsmodul wird der Wert $\kappa=\SI{137.8}{\giga\Pa}$ \cite{kompress}
und für das Molvolumen wird $V_0=\SI{7.11}{\cm^3\per\mol}$
\cite{mol} verwendet.

%Dabei bezeichnet
%$\kappa=\SI{137.8}{\giga\Pa}$ das Kompressionsmodul und $V_0=\SI{7.11}{\cm^3\per\mol}$
%das Molvolumen von Kupfer.
Die Werte für den linearen Ausdehnungskoeffizienten $\alpha$
werden mit Hilfe der Tabelle aus der Versuchsanleitung \cite{skript} ermittelt.
Die gegebenen Werte werden in Abbildung \ref{fig:alpha} gegen $\sfrac{1}{T}$
aufgetragen und mit der Formel
\begin{equation}
  \alpha(x)= \frac{m}{x}+b
\end{equation}
gefittet. Die Ausgleichsrechnung ergibt folgende Parameter:
\begin{align}
  m &= \SI{-872.2(38)e-6}{}\\
  b &= \SI{19.40(3)}{\per\K}.
\end{align}
\begin{figure}[H]
  \centering
  \includegraphics[height=9cm]{alpha.pdf}
  \caption{Linearer Zusammenhang des Ausdehnungskoeffizienten $\alpha$ mit der
  inversen Temperatur $\sfrac{1}{T}$.}
  \label{fig:alpha}
\end{figure}

Die Ergebnisse für $\alpha$ und $C_V$ sind in Tabelle \ref{tab:tab2} zu sehen.
\begin{table}[H]
  \centering
  \caption{Zählrate und Energiemaximum bei variiertem Druck, Abstand a=2cm}
  \label{tab:tab2}
    \begin{tabular}{c c c c c}
    \toprule
    Druck $\rho$/\;mbar & Energiemaximum & Zählrate $N$ & Energie $E_{\alpha}$ & effektive Länge $x$/\;cm\\
    \midrule
    0 & 796 &131382  &4          & 0.0   \\
    50 & 775 &131464 &3.89 & 0.09 \\
    100 &756 &130732 &3.79 & 0.19\\
    150 &749 &129617 &3.76  &  0.29\\
    200 &749 &130444 &3.76  & 0.39\\
    250 &727 &129600 &3.65 & 0.49\\
    300 &722 &128936 &3.63 & 0.59\\
    350 &708 &128478 &3.56 & 0.69\\
    400 &696 &128122 &3.49 & 0.79\\
    450 &687 &127415 &3.45 & 0.89\\
    500 &674 &126608 &3.39 & 0.99\\
    550 &663 &126372 &3.33 &1.09\\
    600 &651 &124989 &3.27 & 1.18\\
    650 &634 &124942 &3.19 & 1.28\\
    700 &618 &124295 &3.11 &1.38\\
    750 &602 &123299 &3.03 & 1.48\\
    800 &584 &119958 &2.93 &1.58\\
    850 &566 &120673 &2.84 &1.68\\
    900 &548 &117907 &2.75 & 1.78\\
    950 &534 &116111 &2.68&   1.88\\
    1000 &499& 108630&2.51 & 1.07\\
    \bottomrule
    \end{tabular}
  \end{table}


Der Zusammenhang zwischen der Temperatur $T$ und der Molwärme bei konstantem Volumen
$C_V$ ist in Abbildung \ref{fig:Cv} dargestellt.

\begin{figure}[H]
  \centering
  \includegraphics[height=9cm]{plot2.pdf}
  \caption{Zusammenhang zwischen der Temperatur $T$ und der Molwärme bei konstantem Volumen
  $C_V$.}
  \label{fig:Cv}
\end{figure}

Um aus den gemessenen $(C_V, T)$-Wertepaaren die Debye-Temperatur $\theta_D$ zu ermitteln
wird Tabelle 1 aus dem Skript \cite{skript} verwendet. Zunächst wird $\sfrac{\theta_D}{T}$\;
abgelesen, nach Multiplikation mit $T$ erhält man die gesuchte Debye-Tempetatur $\theta_D$.
Hier werden nur Messwerte bis $\SI{170}{\K}$ verwendet.
\begin{table}[H]
  \centering
   \begin{tabular}{c c c}
    \toprule
     n& $\nu$/\; 1/s & $\nu_{Wechsel}$\\
    \midrule
    0,5 & 100.01& 50,0\\
    1 & 79.93 & 79.93\\
    2 & 23.93 & 47.86\\
    \bottomrule
  \end{tabular}
  \caption{Gemessene Frequenzen der Sägezahnspannung, sowie die Daraus resultierenden Frequenzen für die
  Wechselspannung.}
  \label{tab:tab3}
\end{table}

Es ergibt sich ein Mittelwert von
\begin{equation}
  \theta_{D_\text{exp}}=\SI{312.5(2)}{\K}.
\end{equation}

Der theoretische Wert für die Debye-Temperatur $\theta_{D_\text{theo}}$ wird mit der Formel
\begin{equation}
  \int_{0}^{\omega_D} Z(\omega) d\omega = 3N_L
\end{equation}
berechnet. Dabei wird die Zustandsdichte für jeden Dispersionszweig mit
\begin{align}
  Z(\omega)= \frac{V}{2\pi^2}\cdot\frac{\omega^2}{v_i^2}\:\:\:\:\text{und}\:\:\:\:
  \omega_D=\frac{k_B \theta_D}{\hbar}
\end{align}
%\begin{equation}
%  Z(\omega)= \frac{V}{2\pi^2}\cdot\frac{\omega^2}{v_i^2} \;\;\text{und}
%\end{equation}
%\begin{equation}
%  \omega_D=\frac{k_B \theta_D}{\hbar}
%\end{equation}
verwendet.
Somit ergibt sich folgender Ausdruck für die Debye-Temperatur $\theta_D$:
\begin{equation}
  \theta_D=\frac{\hbar}{k_B}\sqrt[3]{\frac{18\pi^2 N_A \rho}{M}(v_l^{-3}+2v_t^{-3})^{-1}}.
\end{equation}
%Dabei kann die Teilchenzahl mit $N_L=N_A\frac{m}{M}$ berechnet werden.
Mit den Werten $v_l=\SI{4.7}{\km\per\s}$ und $v_t=\SI{2.26}{\km\per\s}$ folgt für
den theoretischen Wert der Debye-Temperatur
\begin{equation}
  \theta_{D_\text{theo}}=\SI{331.98}{\K}.
\end{equation}
Daraus ergibt sich für die Debye-Frequenz $\omega_D$
\begin{equation}
  \omega_D=\SI{4.35e13}{\Hz}.
\end{equation}
Für die Debye-Temperatur ergibt sich aus den Werten $\theta_{D_\text{exp}}$ und $\theta_{D_\text{theo}}$
eine Abweichung von
\begin{equation}
  \frac{|\theta_{D_\text{theo}}-\theta_{D_\text{exp}}|}{\theta_{D_\text{theo}}}\cdot 100=5.9\;\%.
\end{equation}
