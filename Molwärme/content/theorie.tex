\section{Zielsetzung}
Ziel des Versuches ist es, die Molwärme $\symup{C_{\text{v}}}$ in Abhängigkeit der
Temperatur zu bestimmen und mit den Vorhersagen des klassischen Modells, des Einstein
Modells und des Debye Modells zu vergleichen. Zufdem soll die Debye-Temperatur
$\Theta_{\text{D}}$ bestimmt werden.

\section{Theorie}
\subsection{Spezifische Wärme}
Als Wärmekapazität wird im Allgemeinen die Wärmemenge $\increment \text{Q}$ bezeichnet,
die benötigt wird um
die Temperatur eines
bestimmten Stoffes um $\SI{1}{\kelvin}$ zu erhöhen:
\begin{equation}
  \text{C} = \frac{\increment \text{Q}}{\increment \text{T}}
  \label{eqn:warm}
\end{equation}
Da diese Größe stets von der Menge des Stoffes abhängt, wird sie häufig im
Bezug auf $\SI{1}{\mol}$ normiert
\begin{equation}
  \text{c} = \frac{\increment \text{Q}}{\increment \text{T} \: \text{mol}} \: ,
  \label{eqn:cmol}
\end{equation}
wobei diese Größe dann als molare Wärmekapazität bezeichnet wird.  \\
Es wird zudem zwischen der Wärmekapazität bei konstantem Druck $\symup{C_{\text{p}}}$ und der Wärmekapazität bei
konstanten Volumen $\symup{C_{\text{v}}}$ unterschieden, wobei $\symup{C_{\text{p}}}$
im Allgemeinen größer ist, da hier ein Teil der zugeführten Energie zur Volumensausdehnung verwendet
wird und nicht nur zur Temperaturerhöhung.
Dies ergibt sich durch den ersten Hauptsatz der Thermodynamik
\begin{equation}
  \text{dQ} = \text{dU} + \text{pdV} \: ,
  \label{eqn:hs1}
\end{equation}
aus dem sich die Gleichungen
\begin{equation}
  \symup{C_{\text{v}}} = \frac{\partial\text{U}}{\partial\text{T}}\bigr|_{\text{v}} \\
  \symup{C_{\text{p}}} = \frac{\partial\text{Q}}{\partial\text{T}}\bigr|_{\text{p}}
  \label{eqn:cc}
\end{equation}
sowie die Beziehung zwischen beiden
\begin{equation}
  \symup{C_{\text{p}}}-\symup{C_{\text{v}}}=\text{TV}\alpha_{\text{v}}^2\text{B}
  \label{eqn:cpcv}
\end{equation}
ergeben, wobei $\alpha_{\text{v}}$ den Volumensausdehnungskoeffizient und B den Bulk-Modul
bezeichnet.

\subsection{Klassisches Modell}
Im Klassischen Modell setzt sich die innerer Energie U eines kristallinen Festkörpers
aus der Grundenergie $\symup{U}^{\text{eq}}$ des Gitters und der Energie der Schwingungen
des Kristallgitters zusammen.
Diese setzt sich nach dem Äquipartitionstheorem aus jeweils $\frac{1}{2}\text{k}_{\text{B}}\text{T}$
für potentielle Energie und kinetische Energie pro Freiheitsgrad zusammen. Bei einem Gitter
mit einatomiger Basis gibt es 3 Freiheitsgrade in Form von Schwingungen, je einen für jede der
drei Raumrichtungen.
Somit ergibt sich die innere Energie zu
\begin{equation}
  \symup{U} = \symup{U}^{\text{eq}} + 3\text{N}\text{k}_{\text{B}}\text{T} \: ,
  \label{eqn:Uklassisch}
\end{equation}
wobei $\text{N}=\nu \cdot \text{N}_{\text{A}}$ die Teilchenanzahl bezeichnet, mit der
Avogadrokonstante $\text{N}_{\text{A}}$ und der Molzahl $\nu$.
Somit beträgt die Wärmekapazität
\begin{equation}
  \symup{C_{\text{v}}} = 3\text{N}\text{k}_{\text{B}}
  \label{eqn:Cklassich}
\end{equation}
und die molare Wärmekapazität entsprechend
\begin{equation}
  \symup{c_{\text{v}}} = 3\text{N}_{\text{A}}\text{k}_{\text{B}} = 3\text{R}
  \label{eqn:Cklassich}
\end{equation}
mit der idealen Gaskonstante R. Dies ist auch als Dulong-Petit-Gesetz bekannt, wobei
die molare Wärmekapazität unabhängig ist vom Material und der Temperatur.
Diese Beschreibung gilt meist nur bei hohen Temperaturen und zeigt bei niedrigen
Temperaturen häufig große Abweichungen, da Nullpunktsschwingungen in diesem Modell
nicht berücksichtigt werden.

\subsection{Einstein-Modell}
Durch die Quantisierung der Gitterschwingungen, welche als Phononen bezeichnet
werden, können nur noch diskrete Energien von $ \text{E}_{\text{n}}=(\text{n}+\frac{1}{2})
\hbar \omega $ aufgenommen werden, mit der Kreisfrequenz $\omega$ der Schwingung
und der Anregungszahl n. Zur Berechnung der mittleren inneren Energie wird nun eine Summation
über alle Anregungen durchgeführt, dass heißt über alle
\textbfq{q}-Vektoren und alle r Polarisationen, wobei jede Energie mit dem Boltzmann-Faktor
$\text{e}^{-\beta \:\text{E}_{\text{n}}}$ gewichtet wird, dabei ist
$\beta= \frac{1}{\text{k}_{\text{B}}\text{T}}$.
Hieraus ergibt sich die innere Energie zu
\begin{equation}
  <\symup{U}> = \sum_{\textbfq{q} \: r}\hbar\omega_{\textbfq{q} \: r}(\frac{1}{2}+<\text{n}>)
  \label{eqn:Uqm}
\end{equation}
mit der mittleren Besetzung
\begin{equation}
  <\symup{n}> =  \frac{1}{\symup{e}^{\hbar\omega\beta}-1}
  \label{eqn:boseeinstein}
\end{equation}
gemäß der Bose-Einstein Verteilung, da es sich bei Phononen um Bosonen ohne
chemisches Potential handelt.




\subsection{Debye-Modell}


nam
