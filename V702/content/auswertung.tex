\section{Auswertung}
\label{sec:Auswertung}
Allgemein lässt sich aus den Werten für die Anzahl $N$ der Impulse der Fehler
über
\begin{equation}
  \Delta N= \sqrt{N}
  \label{eqn:fehler}
\end{equation}
Bei der Nullmessung wird $N_u$=215 gemessen. Unter Berücksichtigujng der Zeitspanne von
$\Delta t=900\;s$ und Gleichung \ref{eqn:fehler} eigibt sich als normierte Nullmessung:
\begin{equation}
  N_{un}=\SI{0,24(2)}.
\end{equation}

Die Ergebnisse der Messung Halbwertszeitmessung von Silber sind in Tabelle
\ref{tab:tab1} zu sehen. Diese werden zunächst in Abbildung \ref{fig:plot3}
halblogarithmisch gegen die Zeit $t$ aufgetragen.

\begin{table}[H]
  \centering
  \caption{Messwerte und Ergebniss der Bestimmung der Schallgeschwindigkeit}
  \label{tab:tabe1}
    \begin{tabular}{S||S S||S S||S|S}
    \toprule
    $ \text{Länge l des Zylinders [mm]} $ & $ U_{1} [\text{V}] $ &
    $ t_{1} [\mu\text{s}] $ & $ U_{2} [\text{V}] $ &
    $ t_{2} [\mu\text{s}] $ & $ \increment t [\mu\text{s}]$ &
    $ \text{c} [\text{m}/\text{s}]$\\
    \midrule
    31.0 & 1.335 \: & 24.0 & 1.096 \:  & 46.9 & 22.9 & 2707.42 \\
          \bottomrule
    \end{tabular}
  \end{table}


\begin{figure}[H]
  \centering
  \includegraphics[height=7cm]{plot3.pdf}
  \caption{Halblogarithmisch aufgetragene Messwerte für Ag mit Fehlerbalken.}
  \label{fig:plot3}
\end{figure}

In einem zweiten Diagramm \ref{fig:plot1} werden folgende Ausgleichsgeraden eingezeichnet:\\
1) Von $t=0$ bis $t^{*}$ eine Ausgleichsgerade für den gesamten Zerfall\\
2) Von $t^{*}$ an für den langlebigen Zerfall von $\ce{^{108}Ag}$. \\
3) Von $t=0$ bis $t^{*}$ die Differenz aus dem Gesamten Zerfall und dem langlebigen Zerfall, somit
bleibt der kurzlebige Zerfall von $\ce{^{110}Ag}$ übrig.
Außerdem wird die Zeit $t^{*}=\SI{240}{\s}$, an der das kurzlebige Isotop zerfallen ist abgelesen.

\begin{figure}[H]
  \centering
  \includegraphics[height=7cm]{plot1.pdf}
  \caption{Halblogarithmisch aufgetragene Messwerte für Ag mit Ausgleichsgeraden.}
  \label{fig:plot1}
\end{figure}

Aus Gleichung \ref{eqn:??} kann entnommen werden, dass die Steigungen der Ausgleichsgeraden
gerade der Zerfallskonstante $\lambda$ entsprechen, da $ln(N_0(1-\exp{(-\lambda\Delta t)}))$
eine Konstante ist kann auch
\begin{equation}
  ln(N_{\Delta t}(t))= -\lambda t +b
  \label{eqn:gerade}
\end{equation}
geschrieben werden.
Es werden folgende Geradensteigungen ermittelt:
\begin{align*}
  \lambda_{\text{ges}}=\SI{11,22(82)e-3}{\per\s}\\
  \lambda_{\text{lang}}=\SI{1,50(137)e-3}{\per\s}\\
  \lambda_{\text{kurz}}=\SI{9,7(16)e-3}{\per\s}\\
\end{align*}
Somit lauten die dazugehörigen Geradengleichungen:

\begin{align*}
  ln(N_{\Delta t\;\text{ges}}(t))&=\SI{11,22(82)e-3}{\per\s}\cdot t+(\SI{4,73(11)}{})\\
  ln(N_{\Delta t\;\text{lang}}(t))&=\SI{1,50(137)e-3}{\per\s}\cdot t +(\SI{2,68(47)}{})\\
  ln(N_{\Delta t\;\text{kurz}}(t))&=\SI{9,7(16)e-3}{\per\s}\cdot + (\SI{2,0(5)}{})
\end{align*}

Nach Formel \ref{eqn:gerade} entspricht $ln(N_0(1-\exp{[-\lambda\Delta t]}))$ gerade $b$, somit folgt:
\begin{align*}
  N_{0,kurz}(1-\exp{[-\lambda\Delta t]})=\SI{10(7)}{}\\
  N_{0,lang}(1-\exp{[-\lambda\Delta t]})=\SI{12(8)}{}\\
  N_{0,ges}(1-\exp{[-\lambda\Delta t]})=\SI{114(13)}{}\\
\end{align*}

Aus den Zerfallskonstanten lassen sich nach Gleichung \ref{eqn:??} die Halbwertszeiten berechnen:
\begin{align*}
  T_{ges}=\SI{62(5)}{\s}\\
  T_{lang}=\SI{500(400)}{\s}\\
  T_{kurz}=\SI{71(12)}{\s}\\
\end{align*}

Um zu zeigen, dass die Relation
\begin{equation}
  N_{\Delta t,\text{kurz}}(t_i) <<N_{\Delta t,\text{lang}}(t_i)
\end{equation}
gilt, werden die Werte an der Stelle $t^{*}$ berechnet.
\begin{align}
  N_{\Delta t,\text{kurz}}(t^{*}) &<<N_{\Delta t,\text{lang}}(t^{*})\\
  \SI{0,1(6)}{}&<<\SI{2,4(5)}{}\\
\end{align}
Da die Steigungen beide negativ sind, ist diese Ungleichung also immer erfüllt, wenn sie
am Ort $t^{*}$ erfüllt ist.

Zuletzt werden die Summenkurve aus den errechneten Werten zusammen mit den Messwerten in
einem Diagramm dargestellt, dies ist in Abbildung \ref{fig:plot4} zu sehen.
\begin{figure}[H]
  \centering
  \includegraphics[height=7cm]{plot4.pdf}
  \caption{Summenkurve zusammen mit den Messwerten.}
  \label{fig:plot4}
\end{figure}

%Indium
Für Indium, dessen Messwerte in Tabelle \ref{tab:tab2} zu sehen sind, wird ähnlich vorgegangen.
Die Messwerte werden werden mit Fehlerbalken und Ausgleichsgerade halblogarithmisch
aufgetragen, dies ist in Abbildung \ref{fig:plot2} dargestellt ist.
\begin{table}[H]
  \centering
  \caption{Zählrate und Energiemaximum bei variiertem Druck, Abstand a=2cm}
  \label{tab:tab2}
    \begin{tabular}{c c c c c}
    \toprule
    Druck $\rho$/\;mbar & Energiemaximum & Zählrate $N$ & Energie $E_{\alpha}$ & effektive Länge $x$/\;cm\\
    \midrule
    0 & 796 &131382  &4          & 0.0   \\
    50 & 775 &131464 &3.89 & 0.09 \\
    100 &756 &130732 &3.79 & 0.19\\
    150 &749 &129617 &3.76  &  0.29\\
    200 &749 &130444 &3.76  & 0.39\\
    250 &727 &129600 &3.65 & 0.49\\
    300 &722 &128936 &3.63 & 0.59\\
    350 &708 &128478 &3.56 & 0.69\\
    400 &696 &128122 &3.49 & 0.79\\
    450 &687 &127415 &3.45 & 0.89\\
    500 &674 &126608 &3.39 & 0.99\\
    550 &663 &126372 &3.33 &1.09\\
    600 &651 &124989 &3.27 & 1.18\\
    650 &634 &124942 &3.19 & 1.28\\
    700 &618 &124295 &3.11 &1.38\\
    750 &602 &123299 &3.03 & 1.48\\
    800 &584 &119958 &2.93 &1.58\\
    850 &566 &120673 &2.84 &1.68\\
    900 &548 &117907 &2.75 & 1.78\\
    950 &534 &116111 &2.68&   1.88\\
    1000 &499& 108630&2.51 & 1.07\\
    \bottomrule
    \end{tabular}
  \end{table}


\begin{figure}[H]
  \centering
  \includegraphics[height=7cm]{plot2.pdf}
  \caption{Messwerte für Indium mit Fehlerbalken und Ausgleichsgerade.}
  \label{fig:plot2}
\end{figure}

Die so ermittelte Geradengleichung lautet:

\begin{equation*}
  N_{\Delta t}(t)=\SI{23,1(94)e-5}{\per\s}\cdot t + (\SI{7,81(2)}{})
\end{equation*}

Der vergleich mit Gleichung \ref{eqn:gerade} zeigt, dass die Größe
$ln(N_0(1-\exp{[-\lambda\Delta t]}))$ gerade dem y-Achsenabschnitt entspricht.
Daraus folgt, dass
\begin{equation*}
  N_0(1-\exp{[-\lambda\Delta t]})=\SI{2,47(5)e+3}{}
\end{equation*}

Äquivalent zur vorherigen Rechnung ergibt sich aus der Zerfallskonstante $\lambda$, bzw. der
Steigung der Geraden nach Formel \ref{eqn:??} die Halbwertszeit
\begin{equation*}
  T=\SI{3,01(012)e+3}{\s}.
\end{equation*}
