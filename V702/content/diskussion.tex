\section{Diskussion}
\label{sec:Diskussion}
Die Zerfallskuve von Ag setzt sich aus einem langlebigen und einem kurzlebigen
Isotop zusammen. Die Daten für das langlebige Isotop werden durch eine Ausgleichsgerade
ab dem Zeitpunkt $t*$ bestimmt, ab dem das kurzlebige Isotop zerfallen ist.
Aus der Differenz der Gesamtkurve und der des langlebigen Isotops lassen sich die
Daten des kurzlebigen Isotops ermitteln.
Aus der über die Ausgleichsgeraden ermittelten Zerfallskonstanten $\lambda$:
\begin{align*}
  \lambda_{\text{ges}}=\SI{11,22(82)e-3}{\per\s}\\
  \lambda_{\text{lang}}=\SI{1,50(137)e-3}{\per\s}\\
  \lambda_{\text{kurz}}=\SI{9,7(16)e-3}{\per\s}\\
\end{align*}
lassen sich die
entsprechenden Halbwertszeiten $T$ bestimmen, diese werden mit den Theoriewerten verglichen:
\begin{align*}
  T_{ges}&=\SI{62(5)}{\s}\\
  T_{lang}&=\SI{500(400)}{\s}\;\;\;&T_{\text{Theorie,lang}}&=\SI{142,92(66)}{\s}\\
  \implies \text{Abweichung:}&=250\%\\
  T_{kurz}&=\SI{71(12)}{\s}\;\;\;&T_{\text{Theorie,kurz}}&=\SI{24,56(11)}{\s}\\
  \implies \text{Abweichung:}&=190\%\\
\end{align*}
\cite{silber}
Die Abweichung des langlebigen Isotops vom Theoriewert ist sehr hoch, doch das
Fehlerintervall ist ebenfalls sehr groß. Somit liegt der Fehler noch innerhalb eines Fehlerintervalls.
Auch das kurzlebige Isotop weißt einen großen Fehler auf, dieser liegt innerhalb vier
Fehlerintervalle.

Für Indium wird genauso vorgegangen, aus der Zerfallskonstante $\lambda =\SI{23,1(94)e-5}{\per\s}$
wird die Halbwertszeit
bestimmt:
\begin{align*}
  T=\SI{3,01(12)e+3}{\s}\;\;\;\;T_{Theorie}=\SI{3252}{\s}\\
  \implies \text{Abweichung:}=7,44\%\\
\end{align*}
\cite{indium}
