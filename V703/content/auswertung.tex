\section{Auswertung}
\subsection{Zählrohr-Charakteristik}
Die Messwerte der Zählrohr-Charakteristik sind in Tabelle \ref{tab:tabe1}
abzulesen.Aus den Werten für die Anzahl N der Impulse pro min lässt sich durch
\begin{equation}
  \increment\text{N}=\sqrt{\text{N}}
\end{equation}
der Fehler der Messung bestimmen.
\begin{table}[H]
  \centering
  \caption{Messwerte und Ergebniss der Bestimmung der Schallgeschwindigkeit}
  \label{tab:tabe1}
    \begin{tabular}{S||S S||S S||S|S}
    \toprule
    $ \text{Länge l des Zylinders [mm]} $ & $ U_{1} [\text{V}] $ &
    $ t_{1} [\mu\text{s}] $ & $ U_{2} [\text{V}] $ &
    $ t_{2} [\mu\text{s}] $ & $ \increment t [\mu\text{s}]$ &
    $ \text{c} [\text{m}/\text{s}]$\\
    \midrule
    31.0 & 1.335 \: & 24.0 & 1.096 \:  & 46.9 & 22.9 & 2707.42 \\
          \bottomrule
    \end{tabular}
  \end{table}

Die Werte sind, zusammen mit den entsprechenden Fehlerbalken, in Abbildung \ref{fig:plot1}
dargestellt, wobei der erste Wert bei $\SI{300}{\volt}$ zur besseren Skalierung
außer Acht gelassen wird.
\begin{figure}[H]
  \centering
  \includegraphics{plot1.pdf}
  \caption{Messwerte der Zählrohrcharakteristik mit Ausgleichsgerade}
  \label{fig:plot1}
\end{figure}
Es lässt sich erkennen, dass das Plateau in einem Bereich von etwa $\SI{400}{\volt}$
bis $\SI{650}{\volt}$ liegt. In diesem Bereich wird dann eine lineare Ausgleichsrechnung
der Form
\begin{equation}
  y = a\cdot x +b
  \label{eqn:linear}
\end{equation}
durchgeführt, woraus sich die Parameter
\begin{align*}
  a &= 1,32 \pm 0,29 \: \frac{1}{\text{V}}\\
  b &= 12154 \pm 154 \: \text{V}\: ,\\
\end{align*}
ergeben.
Da der Startwert bei $\SI{400}{\volt}$ bei 12668 Impulsen pro Minute liegt,
entspricht a einer Steigung von 0,0104\% $\pm$ 0,0023\% pro $\SI{1}{\volt}$,
also 1,04\% $\pm$ 0,23\% pro $\SI{100}{\volt}$. \\
\\
Die gemessenen Werte zum zeitlichen Abstand der Primär- und Nachentladungsimpulsen
befinden sich in Tablle \ref{tab:tabe2}.
\begin{table}[H]
  \centering
  \caption{Zählrate und Energiemaximum bei variiertem Druck, Abstand a=2cm}
  \label{tab:tab2}
    \begin{tabular}{c c c c c}
    \toprule
    Druck $\rho$/\;mbar & Energiemaximum & Zählrate $N$ & Energie $E_{\alpha}$ & effektive Länge $x$/\;cm\\
    \midrule
    0 & 796 &131382  &4          & 0.0   \\
    50 & 775 &131464 &3.89 & 0.09 \\
    100 &756 &130732 &3.79 & 0.19\\
    150 &749 &129617 &3.76  &  0.29\\
    200 &749 &130444 &3.76  & 0.39\\
    250 &727 &129600 &3.65 & 0.49\\
    300 &722 &128936 &3.63 & 0.59\\
    350 &708 &128478 &3.56 & 0.69\\
    400 &696 &128122 &3.49 & 0.79\\
    450 &687 &127415 &3.45 & 0.89\\
    500 &674 &126608 &3.39 & 0.99\\
    550 &663 &126372 &3.33 &1.09\\
    600 &651 &124989 &3.27 & 1.18\\
    650 &634 &124942 &3.19 & 1.28\\
    700 &618 &124295 &3.11 &1.38\\
    750 &602 &123299 &3.03 & 1.48\\
    800 &584 &119958 &2.93 &1.58\\
    850 &566 &120673 &2.84 &1.68\\
    900 &548 &117907 &2.75 & 1.78\\
    950 &534 &116111 &2.68&   1.88\\
    1000 &499& 108630&2.51 & 1.07\\
    \bottomrule
    \end{tabular}
  \end{table}


\subsection{Bestimmung der Totzeit}

Bei der Zwei-Quellen-Methode lauten die gemessenen Werte bei einer Spannung von
$\SI{450}{\volt}$ in einer Minute :
\begin{align*}
  N_1 &= 12835 \pm 113 \\
  N_2 &= 15937 \pm 126 \\
  N_{1+2} &= 28081 \pm 168 \\
\end{align*}
Aus Gleichung \ref{eqn:tot} ergibt sich somit eine Totzeit von
\begin{equation*}
  \text{T} \approx \SI{101(34)}{\micro\second} \: ,
\end{equation*}
wobei sich der Fehler hierbei durch die Gauß´sche Fehlerfortpflanzung
\begin{equation}
  \increment f = \sqrt{ \sum_{i=1}^N \left( \frac{\partial f}{\partial x_i}\right)^2
  \cdot (\increment x_i)^2  } \: .
  \label{eqn:gaus}
\end{equation}
ergibt, in diesem Fall also
\begin{equation}
  \increment T = \sqrt{\left( \frac{-\text{N}_{2}+\text{N}_{12}}{2\text{N}_{1}^2\text{N}_{2}} \right)^2 \cdot (\increment \text{N}_{1})^2 \\
   + \left( \frac{\text{N}_{1}+\text{N}_{12}}{2\text{N}_{1}\text{N}_{2}^2} \right)^2 \cdot (\increment \text{N}_{2})^2
  +\left( \frac{1}{2\text{N}_{1}\text{N}_{2}} \right)^2 \cdot (\increment \text{N}_{12})^2 } \: .
\end{equation}
 \\
Die mithilfe des Oszilloskops gemessenen Totzeiten sind in Tabelle \ref{tab:tabe3}
abzulesen.
\begin{table}[H]
  \centering
   \begin{tabular}{c c c}
    \toprule
     n& $\nu$/\; 1/s & $\nu_{Wechsel}$\\
    \midrule
    0,5 & 100.01& 50,0\\
    1 & 79.93 & 79.93\\
    2 & 23.93 & 47.86\\
    \bottomrule
  \end{tabular}
  \caption{Gemessene Frequenzen der Sägezahnspannung, sowie die Daraus resultierenden Frequenzen für die
  Wechselspannung.}
  \label{tab:tab3}
\end{table}

Durch die Gleichung
\begin{equation}
  \bar{x} = \frac{1}{N} \sum_{i=1}^{N} x_i \: \:
  \label{eqn:mit}
\end{equation}
\noindent lässt sich der Mittelwert bilden, wobei der dazugehörige Fehler sich durch
\begin{equation}
  \increment \bar{x} = \frac{1}{\sqrt{N}} \sqrt{ \frac{1}{N-1} \sum_{i=1}^N
  (x_i - \bar{x})^2}
  \label{eqn:mitf}
\end{equation}
ergibt. Hierdurch ergibt sich somit insgesamt eine Totzeit von
\begin{align*}
  \text{T} \approx \SI{184(9)}{\micro\second}
\end{align*}
\subsection{Freigesetzte Ladung}
Die pro Teilchen vom Zählrohr freigesetzten Ladungsmenge lässt sich durch die Formel
\begin{equation}
  \bar\text{I}= \frac{\increment \text{Q}}{\increment \text{t}} \text{N}
\end{equation}
berechen, wobei $\increment t$= 60s beträgt. Der Fehler lässt sich nach Gleichung \ref{eqn:gaus} durch
\begin{equation}
   \text{Fehler} \: \increment \text{Q} = \frac{\bar{\text{I}} \cdot \increment t}{\text{N}^2} \cdot \increment \text{N}
\end{equation}
berechnen.
Die so berechneten Werte lassen sich in Tabelle \ref{tab:tabe4} ablesen.
\begin{table}[H]
  \centering
   \begin{tabular}{c c c c}
    \toprule
    Nummer der Oberwelle & $ U_{\text Theorie,Rechteck}\: / \si{\volt} $ &
    $ U_{\text Theorie,Dreick}\: / \si{\volt} $ & $ U_{\text Theorie,Sägezahn}\: / \si{\volt} $ \\
    \midrule
    1 & 1145 & 182 & 573 \\
    2 & 0 & 0 & 286 \\
    3 & 573 & 20 & 191 \\
    4 & 0 & 0 & 143 \\
    5 & 229 & 7 & 115 \\
    6 & 0 & 0 & 96 \\
    7 & 164 & 4 & 82 \\
    8 & 0 & 0 & 72 \\
    9 & 127 & 2 & 64 \\
    10 & 0 & 0 & 57 \\
    \bottomrule
  \end{tabular}
  \caption{Eingestellte Schwingungsamplituden.}
  \label{tab:tabe4}
\end{table}

