\section{Diskussion}
\label{sec:Diskussion}
Für die Zählrohrcharakteristik hat sich durch den Versuch eine Steigung im Plateau von
1,04\% $\pm$ 0,23\% pro $\SI{100}{\volt}$ ergeben. Die Totzeit ließ sich durch die
Zwei-Quellen-Methode zu $\text{T} \approx \SI{101(34)}{\micro\second}$ bestimmen und durch
das Oszilloskop zu $\text{T} \approx \SI{184(9)}{\micro\second}$. Die freigesetzte Ladung
liegt bei dem Versuch in einem Bereich von (0.325 $\pm$ 0.004) $\cdot10^{10} / \text{e}_{0}$ und
(3.319 $\pm$ 0.028 $\cdot10^{10} / \text{e}_{0}$, die genauen Werte lassen sich in Tabelle \ref{tab:tabe4}
ablesen. \\
Da bereits die gemessenen Werte für die Anzahl der Impulse N mit einem Fehler
von $\sqrt{\text{N}}$ belegt sind, sind auch die meisten errechneten Werte aufgrund
der Gauß`schen Fehlerfortpflanzung mit einem recht hohen Fehler behaftet. \\
Bei der Bestimmung der Totzeit ist der Fehler bei der Methode mit dem Oszilloskop
geringer, da sich hier nur ein Fehler aus der Mittelwertbildung ergibt.
Die beiden Werte aus den beiden Methoden zur Totzeit Bestimmung unterscheiden sich um 45,1\%,
was sich durch die Formel
\begin{equation*}
  \frac{\lvert \text{Wert}_{\text{Osz}}-\text{Wert}_{\text{2Quel}}\rvert}{\text{Wert}_{\text{Osz}}}
  \label{eqn:abw}
\end{equation*}
ergibt. Dieser großer Unterschied deutet daurauf hin, dass bei einer der beiden Methoden
ein systematischer Fehler vorliegt.
