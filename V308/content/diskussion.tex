\section{Diskussion}
Über die Hysteresekurve konnten folgende Werte für die
Ringspule mit Luftspalt ermittelt werden:
\begin{align*}
  B_{S} &=\SI{691,5(12)}{\milli\tesla} \\
  B_{r} &=\SI{121.3(22)}{\milli\tesla}
\end{align*}
\noindent dabei bezeichnet $B_{S}$ die Sättigungsmagnetisierung
und $B_{r}$ die Remanenzfeldstärke. Außerdem wurde die
Koerzitivfeldstärke $H_{k}=\SI{1,4}{\ampere}$
ermittelt.
%Aus dieser können Rückschlüsse über das Material
%geschlossen werden.
\noindent Aus den Messdaten für die Helmholzspule geht hervor, dass
das Magnetfeld im Inneren konstant ist und für größere
Abstände quadratisch abfällt und gegen Null strebt.
Die Abweichungen der Messdaten von der Theoriekurve können
dadurch erklärt werden, dass die Hall-Sonde nicht
fest eingespannt war und dadurch nicht garantiert ist
das sie auch nach dem Verschieben immer senkreckt nach
unten zeigt. Eine weitere Erklärung ist, dass sich die
Hall-Sonde beim Verschieben verdedreht haben könnte
und dadurch nicht mehr senkrecht zum Magnetfeld stand, wodurch
kleinere Werte gemessen werden.
\noindent Da die Abweichungen für die erste Messung doch sehr hoch sind
ist von einem weitern systematischen Fehler auszugehen.

\noindent An den Daten der langen und kurzen Spule  kann erkannt
werden, dass das Magnetfeld im Mittelpunkt der Spule am
stärksten ist. Innerhalb der langen Spule ist das
Magnetfeld konstant. Außerhalb der Spulen fällt das
Magnetfeld ab und strebt auch hier für große
Abstände gegn Null.
\noindent Die Abweichungen von den Theoriekurven können dadurch begründet
werden, dass sich die Theoriekurve exakt auf die
Mittelachse der Spule bezieht, für die Messung wurde
die Mittelachse der Spule jedoch nur nach Augenmaß bestimmt.
Außerdem musste die Apperatur mit der Hall-Sonde nach der
Hälfe der Messung umgedreht werden, um auch die andere
Seite der Spule zu erreichen. Deshalb ist es möglich, dass
die Achse auf der gemessen wurde nicht erneut exakt
getroffen wurde.





\label{sec:Diskussion}
