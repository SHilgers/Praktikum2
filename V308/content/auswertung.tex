\section{Auswertung}

\subsection{Hysteresekurve}
Um die Hysteresekurve einer Ringspule mit Luftspalt zu
bestimmen, wurden die in Tabelle \ref{tab:tabelle1}
dargestellten Messwerte in einem Diagramm aufgetragen.
Dieses Diagramm ist ist Abbildung \ref{fig:plothys.pdf}
dargestellt.
\begin{table}[H]
  \centering
  \caption{Messwerte und Ergebniss der Bestimmung der Schallgeschwindigkeit}
  \label{tab:tabe1}
    \begin{tabular}{S||S S||S S||S|S}
    \toprule
    $ \text{Länge l des Zylinders [mm]} $ & $ U_{1} [\text{V}] $ &
    $ t_{1} [\mu\text{s}] $ & $ U_{2} [\text{V}] $ &
    $ t_{2} [\mu\text{s}] $ & $ \increment t [\mu\text{s}]$ &
    $ \text{c} [\text{m}/\text{s}]$\\
    \midrule
    31.0 & 1.335 \: & 24.0 & 1.096 \:  & 46.9 & 22.9 & 2707.42 \\
          \bottomrule
    \end{tabular}
  \end{table}

\begin{figure}
  \centering
  \includegraphics[height=5cm]{plothys.pdf}
  \caption{Hysteresekurve einer Ringspule mit Luftspalt.}
  \label{fig:plothys}
\end{figure}
An Abbildung \ref{fig:plothys} und auch an den Messwerten
aus Tabelle \ref{tab:tabelle1} kann die
Sättigungsmagnetisierung $B_{S}$,die Remanenz
$B_{r}$ für $I=0$ sowie die Koerzitivfeldstärke
$H_{k}$ abgelesen werden.
Die Beträge der Werte werden nach der Formel
\begin{equation}
  \bar{x}=\frac{1}{N}\sum_{1}^\N x_{i}
  \label{eqn:mittel}
\end{equation}
gemittelt, dabei bezeichnet $N$ die Anzahl der zu
mittelnden Werte.

Sättigungsmagnetisierung und Remanenz haben also folgende
Werte:
\begin{align*}
  B_{S} &=\SI{691,5(12)}{\mT}
  B_{r} &=\SI{121.3(22)}{\mT}
\end{align*}
Die Koerzitivfeldstärke $H_{k}$ kann aus dem Messwerten
der Stromsärke nach der Formel
\begin{equation}
  H=\frac{NI}{d_{L}}
  %B=\nu_{0}\frac{NI}{d_{L}}
  \label{eqn:Luftspalt}
\end{equation}
berechnet werden, wobei $d_{L}$ die Breite des Luftspalts
bezeichnet.
Werden die Beträge gemittelt, ergibt sich $I=\SI{0.7}{A}$
darauf folgt mit Formel \ref{eqn:Luftspalt}:
\begin{equation*}
  H=\SI{138833,33}{\Ampere\per\meter}.
\end{equation*}

\subsection{Helmholzspulen}
Die Messwerte für die erste Messung mit $I=2A$ sind in
Tabelle \ref{tab:tabelle3} dargestellt.
\begin{table}[H]
  \centering
   \begin{tabular}{c c c}
    \toprule
     n& $\nu$/\; 1/s & $\nu_{Wechsel}$\\
    \midrule
    0,5 & 100.01& 50,0\\
    1 & 79.93 & 79.93\\
    2 & 23.93 & 47.86\\
    \bottomrule
  \end{tabular}
  \caption{Gemessene Frequenzen der Sägezahnspannung, sowie die Daraus resultierenden Frequenzen für die
  Wechselspannung.}
  \label{tab:tab3}
\end{table}

Die Messwerte für den Innenbereich der Spulen und den
Außenbereich wurden getrennt in einem xB-Diagramm
aufgetragen und mit der Theoriekurve \ref{eqn:??}
verglichen. Die Diagramme sind in Abbildung
\ref{fig:Helmholz1} und \ref{fig:Helmholz1I}
dargestellt. Dabei muss berücksichtigt werden, dass die
Theoriekurve den Mittelpunkt zwischen den Spulen als
Nullpunkt annimmt, sich der der Nullpunkt der
Messwerte jedoch an Rand der ersten Spule befindet.
\begin{figure}
  \centering
  \includegraphics{Helmholz1.pdf}
  \caption{xB-Diagramm des Außenbereiches eines
  Helmholzspulenpaares im Vergleich zur Theoriekurve
  (I=2A)}
  \label{fig:Helmholz1}
\end{figure}
\begin{figure}
  \centering
  \includegraphics{Helmholz1I.pdf}
  \caption{xB-Diagramm des Innenbereiches eines
  Helmholzspulenpaares im Vergleich zur Theoriekurve
  (I=2A)}
  \label{fif:Helmholz1I}
\end{figure}

Für die zweite Messreihe mit einer Stromstärke von
$I=5A$ wird äquivalent zur ersten Messreihe vorgegeangen.
Im folgenden sind die Messwerte sowie die xB-Diagramme
für den Außen- und Innenbereich dargestellt.
\begin{table}[H]
  \centering
   \begin{tabular}{c c c c}
    \toprule
    Nummer der Oberwelle & $ U_{\text Theorie,Rechteck}\: / \si{\volt} $ &
    $ U_{\text Theorie,Dreick}\: / \si{\volt} $ & $ U_{\text Theorie,Sägezahn}\: / \si{\volt} $ \\
    \midrule
    1 & 1145 & 182 & 573 \\
    2 & 0 & 0 & 286 \\
    3 & 573 & 20 & 191 \\
    4 & 0 & 0 & 143 \\
    5 & 229 & 7 & 115 \\
    6 & 0 & 0 & 96 \\
    7 & 164 & 4 & 82 \\
    8 & 0 & 0 & 72 \\
    9 & 127 & 2 & 64 \\
    10 & 0 & 0 & 57 \\
    \bottomrule
  \end{tabular}
  \caption{Eingestellte Schwingungsamplituden.}
  \label{tab:tabe4}
\end{table}

\begin{figure}
  \centering
  \includegraphics{Helmholz2.pdf}
  \caption{xB-Diagramm des Außenbereiches eines
  Helmholzspulenpaares im Vergleich zur Theoriekurve
  (I=5A)}
  \label{fig:Helmholz2}
\end{figure}
\begin{figure}
  \centering
  \includegraphics{Helmholz2I.pdf}
  \caption{xB-Diagramm des Innenbereiches eines
  Helmholzspulenpaares im Vergleich zur Theoriekurve
  (I=5A)}
  \label{fif:Helmholz5I}
\end{figure}

\subsection{Magnetfeld Spulen}
Die Aufgenommenen Messwerte für die kurze und lange
Spule sind in der fplgenden Tabelle zu finden.
\begin{table}[H]
  \centering
  \caption{Zählrate und Energiemaximum bei variiertem Druck, Abstand a=2cm}
  \label{tab:tab2}
    \begin{tabular}{c c c c c}
    \toprule
    Druck $\rho$/\;mbar & Energiemaximum & Zählrate $N$ & Energie $E_{\alpha}$ & effektive Länge $x$/\;cm\\
    \midrule
    0 & 796 &131382  &4          & 0.0   \\
    50 & 775 &131464 &3.89 & 0.09 \\
    100 &756 &130732 &3.79 & 0.19\\
    150 &749 &129617 &3.76  &  0.29\\
    200 &749 &130444 &3.76  & 0.39\\
    250 &727 &129600 &3.65 & 0.49\\
    300 &722 &128936 &3.63 & 0.59\\
    350 &708 &128478 &3.56 & 0.69\\
    400 &696 &128122 &3.49 & 0.79\\
    450 &687 &127415 &3.45 & 0.89\\
    500 &674 &126608 &3.39 & 0.99\\
    550 &663 &126372 &3.33 &1.09\\
    600 &651 &124989 &3.27 & 1.18\\
    650 &634 &124942 &3.19 & 1.28\\
    700 &618 &124295 &3.11 &1.38\\
    750 &602 &123299 &3.03 & 1.48\\
    800 &584 &119958 &2.93 &1.58\\
    850 &566 &120673 &2.84 &1.68\\
    900 &548 &117907 &2.75 & 1.78\\
    950 &534 &116111 &2.68&   1.88\\
    1000 &499& 108630&2.51 & 1.07\\
    \bottomrule
    \end{tabular}
  \end{table}

Ähnlich zum Vorgehen bei den Helmhozspulen wird hier
ebenfalls ein xB-Diagramm erstellt, welches mit
der Theoriekurve verglichen wird.
\end{description}.
\begin{figure}
  \centering
  \includegraphics{LangeSpule.pdf}
  \caption{xB-Diagramm der langen Spule.}
  \label{fig:LangeSpule}
\end{figure}
\begin{figure}
  \includegraphics{KurzeSpule.pdf}
  \caption{xB-Diagramm der kurzen Spule.}
  \label{fig:KurzeSpule}
\end{figure}




\label{sec:Auswertung}

%\begin{figure}
%  \centering
%  \includegraphics{plot.pdf}
%  \caption{Plot.}
%  \label{fig:plot}
%\end{figure}
