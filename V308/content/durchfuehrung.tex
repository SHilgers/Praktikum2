\section{Durchführung}
\subsection{Messung einer kurzen und einer langen Spule}
Zunächst soll das Magnetfeld einer kurzen und einer langen Spule vermessen werden.
Dazu wird eine longitudinale Hall-Sonde verwendet, welche auf die Höhe der
Spulenachse eingestellt wird. Es muss dabei auf den maximal zulässigen Strom der Spule
geachtet werden, zudem wird der Strom erst bei fertigem Aufbau eingeschaltet.
Es sollen pro Spule mindestens je 10 Messwerte innerhalb und außerhalb der Spule
aufgenommen werden.

\subsection{Helmholtzspulen}
Im zweiten Versuchsteil werden zwei Spulen mit gleichem Radius in genau dem Abstand voneinander
aufgestellt, der ihrem Radius entspricht, sodass sie ein Helmholtz-Spulenpaar bilden.
Sie werden dann in Reihe geschaltet, damit sichergestellt ist, dass der gleiche
Strom in beiden fließt. Es ist erneut darauf zu achten, dass der maximal zulässige Strom
nicht überschritten wird.
Mit einer transversalen Hallsonde werden dann wieder je mindestens 10 Messwerte
auf der Spulenachse vor und zwischen den beiden Spulen aufgenommen.
Die Messung wird anschließend mit einer anderen Stromstärke wiederholt.

\subsection{Hysterese-Kurve}
Schlussendlich wird mit einer transersalen Hallsonde in dem Luftspalt einer Ringspule
mit einem ferromagnetischen Material als Kern das Magnetfeld in Abhängigkeit des
Stromes gemessen wird, welcher dazu in 1 Ampere Schritten erhöht bzw. erniedrigt wird.
Wichtig ist, dass vorher keine Restmagnetisiserung vorhanden ist.
Man misst nun zunächst den Bereich von 0-10 A, um eine Neukurve aufzunehmen, anschließend
veringert man den Strom wieder schrittweise bis auf 0 und anschließend bis -10 A.
Zuletzt misst man dann nochmal den Bereich von -10 bis 10 A um zusätzlich zur Neukurve eine
vollständige Hysteresekurve zu messen.
