\begin{table}[H]
  \centering
   \begin{tabular}{c c| c c}
    \hline
    \;\;\;\;   Kurze Spule & &\;\;\;\;    Lange Spule&  \\
    \toprule
    $d /\;\si{cm}$& $B/\;\si{mT}$ & $d /\;\si{mA}$ & $ B/\;\si{mT}$ \\
    \midrule
    -10 & -2.812 & 13.0 & 0.076\\
    -9 & -3.526 & 12.0 & 0.118\\
    -8 & -4.397 & 11.0 & 0.186\\
    -7 & -5.451 & 10.0 & 0.330\\
    -6 & -6.948 & 9.0 & 0.574\\
    -5 & -8.456 & 8.0 & 1.156\\
    -4 & -10.97 & 7.0 & 1.603\\
    -3 & -12.82 & 6.0 & 2.040\\
    -2 & -14.34 & 5.0 & 2.198\\
    -1 & -15.23 & 4.0 & 2.256\\
    0 & -15.48 & 3.0 & 2.297\\
    0 & 15.37 & 2.0 & 2.315\\
    1 & 15.26 & 1.0 & 2.322\\
    2 & 14.40 & 0 & 2.323\\
    3 & 12.96 & 0 & 2.367\\
    4 & 11.05 & -1.0 & -2.366\\
    5 & 9.172 & -2.0 & -2.356\\
    6 & 7.173 & -3.0 & -2.332\\
    7 & 5.315 & -4.0 & -2.291\\
    8 & 4.115 & -5.0 & -2.221\\
    9 & 5.112 & -6.0 & -2.093\\
    10 & 2.316 & -7.0 & -1.813\\
             & &-8.0 & -1.253\\
             & &-9.0 & -0.743\\
             & &-10.0& -0.427\\
             & &-11.0 &-0.264\\
             & &-12.0 &-0.172\\
             & &-13.0 &-0.134\\
    \bottomrule
  \end{tabular}
  \caption{Gemessene Daten für Kurze und Lange Spule.}
  \label{tab:tabelle2}
\end{table}
