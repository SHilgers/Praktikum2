\section{Diskussion}
Die auf verschiedene Weisen und für verschiedene Heizraten bestimmten Aktivierungsenergien lauten:
\begin{align*}
  W_\text{A,1}=\SI{0,96(8)}{\eV}\;\;\;&W_\text{A,2}=\SI{0,66(3)}{\eV}\\
  W_\text{I,1}=\SI{1,03(4)}{\eV}\;\;\;&W_\text{I,2}=\SI{0,87(2)}{\eV}.
\end{align*}

%Bei genauerer Betrachtung fällt auf, dass beide über das Integral bestimmten Werte nur
%gering voneinander abweichen.
Verglichen mit dem Literaturwert für Strontium dotiertes
Kalium-Bromid von ${W_\text{lit}=\SI{0,66}{\eV} }$~\cite{literaturwert}
fällt auf, dass die Berechnung über den Anlaufstrom mit einer Heizrate von \\
${b_2=\SI{1,06(2)}{\K\per\min}}$
exakt den Literaturwert liefert. Die größte Abweichung mit 56,1\% ergibt sich für für die Berechnung über die
Integration bei einer Heizrate von ${b_1=\SI{1,99(5)}{\K\per\min}}$. Die Abweichungen für die Werte $W_\text{A,1}$
und $W_\text{I,2}$ liegen mit 45,5\% und 31,8\% darunter. Da die exakte Messung der Aktivierungsenergie
über den Anlaufstrom wahrscheinlich nicht reproduzierbar ist, lässt sich aus den Ergebnissen nicht bestimmen,
welche der verwendeten Verfahren besser zur Bestimmung der Aktivierungsenergie geeignet ist.

%Diese Abweichung zeigt, dass die Berechnung über den Anlaufstrom nicht die bevorzugte
%Variante ist und das exakte Ergebnis von $W_\text{A,2}$ wahrscheinlich nicht reproduzierbar ist.
%Die Berehnung über das Integral weicht ebenfalls vom Literaturwert ab, ist mit 21,2 bzw. 28,8\% Abweichung aber
%genauer.

Für die Relaxationszeiten wurden folgende Werte berechnet:
\begin{align*}
  \tau_\text{01,A}&=\SI{0.7(24)e-18}{\s}\;\;\;&\tau_\text{02,A}&=\SI{7(8)e-13}{\s}\\
  \tau_\text{01,I}&=\SI{4(5)e-20}{\s}\;\;\;&\tau_\text{02,I}&=\SI{9(5)e-17}{\s},
\end{align*}
während der Literaturwert durch $W_\text{lit}=\SI{4e-14}{\s}$ \cite{literaturwert}
gegeben ist.

Da die Relaxationszeit exponentiell von der Aktivierungsenergie abhängt, pflanzen sich schon kleine Fehler
der Aktivierungsenergie stark auf die Relaxationszeit fort.
Daher ergibt sich für $\tau_\text{02,A}$ mit 1650\% der geringste Fehler.

Nach Formel \ref{eqn:taumax} wird erwartet, dass sich mit steigender Heizrate die Relaxationszeit verringert.
Dies kann durch die Messung bestätigt werden, denn sowohl für die Berechnung über den Anlaufstrom, als auch über
Integration, sind die Relaxationszeiten für die Heizrate $b_1$ kleiner als für $b_2$.

Da es sich bei der Dipolrealaxation um einen statistischen Prozess handelt, sind die berechneten Werte mit einem
statistischen Fehler behaftet. Der Fehler der Heizrate, als auch Fehler im gemessenen Relaxationsstrom, welche auf
Erschütterungen des Picoampermeters zurückzuführen wären, sind vernachlässigbar klein.
%Mögliche Fehlerquellen der Messung sind zum einen Schwankungen in der Heizrate, da diese immer nachreguliert werden
%musste und zu Beginn der Messung ein großes Temperaturgefälle vorliegt. Zum anderen ist das verwendete
%Picoampermeter sehr empfindlich gegenüber Erschütterungen, welche sich jedoch nicht vollständig vermeiden lassen.
