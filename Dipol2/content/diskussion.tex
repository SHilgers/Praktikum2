\section{Diskussion}
Die auf verschiedene Weisen und für verschiedene Heizraten bestimmten Aktivierungsenergien lauten:
\begin{align*}
  W_\text{A,1}=\SI{0,96(8)}{\eV}\;\;\;&W_\text{A,2}=\SI{0,66(3)}{\eV}\\
  W_\text{I,1}=\SI{1,03(4)}{\eV}\;\;\;&W_\text{I,2}=\SI{0,87(2)}{\eV}.
\end{align*}

%Bei genauerer Betrachtung fällt auf, dass beide über das Integral bestimmten Werte nur
%gering voneinander abweichen.
Verglichen mit dem Literaturwert für Strontium dotiertes
Kalium-Bromid von ${W_\text{lit}=\SI{0,66}{\eV} }$~\cite{literaturwert}
fällt auf, dass die Berechnung über den Anlaufstrom mit einer Heizrate von $b_2=\SI{1,06(2)}{\K\per\min}$
exakt den Literaturwert liefert. Die größte Abweichung mit 45,5\% ergibt sich für für die Berechnung über den
Anlaufstrom bei einer Heizrate von $b_1=\SI{1,99(5)}{\K\per\min}$.

Diese Abweichung zeigt, dass
die Berechnung über den Anlaufstrom nicht die bevorzugte Variante ist und das exakte Ergebnis von $W_\text{A,2}$
wahrscheinlich nicht reproduzierbar ist.
Die Berehnung über das Integral weicht ebenfalls vom Literaturwert ab, ist mit 21,2 bzw. 28,8\% Abweichung aber
genauer.
\\
Für die Relaxationszeiten wurden folgende Werte berechnet:
\begin{align*}
  \tau_\text{01,A}&=\SI{0.7(24)e-18}{\s}\;\;\;&\tau_\text{02,A}&=\SI{7(8)e-13}{\s}\\
  \tau_\text{01,I}&=\SI{4(5)e-20}{\s}\;\;\;&\tau_\text{02,I}&=\SI{9(5)e-17}{\s},
\end{align*}
während der Literaturwert durch $W_\text{lit}=\SI{4e-14}{\s}$ \cite{literaturwert}
gegeben ist.

Da die Relaxationszeit von der Aktivierungsenergie abhängt, pflanzen sich diese Fehler fort.
Daher ergibt sich für $\tau_\text{02,A}$ mit 1597,5\% der geringste Fehler.
\\
Mögliche Fehlerquellen der Messung sind zum einen Schwankungen in der Heizrate, da diese immer nachreguliert werden
musste und zu Beginn der Messung ein großes Temperaturgefälle vorliegt. Zum anderen ist das verwendete
Picoampermeter sehr empfindlich gegenüber Erschütterungen, welche sich jedoch nicht vollständig vermeiden lassen.
