\section{Zielsetzung}
In diesem Versuch wird zunächst die Zeitkonstante eines RC-Kreises bestimmt.
Anschließend wird die Amplitude der Kondensatorspannung in Abhängigkeit der Frequenz und
die Phasenverschiebung von Kondensator- und Generatorspannung ebenfalls in Abhängigkeit der
Frequenz gemessen. Zum Schluss wird gezeigt, dass der RC-Kreis Spannungen
integrieren kann.

\section{Theorie}
Ein System zeigt ein Relaxationsverhalten, wenn es aus seinem Ausgangszustand
ausgelenkt wurde und nicht-oszillatorisch in diesen zurückkerhrt. Dabei ist die
Änderungsgeschwindigkeit der Größe $A$ zum Zeitpunkt $t$ meist proportional zur
Abweichung von A zum Endzustand A(\infty)
\begin{equation}
  \frac{dA}{dt} = c[A(t)-A(\infty)].
  \label{eqn:diff1}
\end{equation}
Duch Integration diese Gleichung von 0 bis $t$ folgt
\begin{equation}
  A(t)=A(\infty)+[A(0)-A(\infty)] \cdot \exp{ct}
\end{equation}
womit A zum Zeitpunkt $t$ berechnet weden kann.
Dabei muss c<0 gelten, damit A beschränkt bleibt. \\

Ein Beipiel für Relaxationsvorgänge ist das Auf- und Endladen eines Kondensators
über einen Widerstand.
\begin{figure}[H]
  \centering
  \includegraphics[height=5cm]{RC.JPG}
  \cite{skript}
  \caption{Entladung (Stellung 1) und Aufladung (Stellung 2) eines RC-Kreises.}
  \label{fig:RC}
  \end{figure}

\textbf{Entladung eines Kondensators}\\
\label{sec:Entladung}
Für die Spannung $U_{C}$ am Kondensator mit der Kapazizät C und der Ladung Q gilt:
\begin{equation}
  U_{C} = \frac{Q}{C}.
  \label{Spannung}
\end{equation}
Mit Hilfe des ohmschen Gesetzes $U=R\cdot I$ und der Beziehung $\dot{Q}=I$
folgt aus \eqref{eqn:Spannung} eine Differentialgleichung für die Ladung des
Kondensators.
\begin{equation}
  \frac{dQ}{dt}=- \frac{1}{RC}\; Q/t)
  \label{eqn:diff2}
\end{equation}
Diese Differentialgleichung hat die gleiche Gestallt wie Gleichung \eqref{eqn:diff1}.
Mit der Annahme, dass der Kondensator nach $t \to \infty$ entladen ist, folgt
nach Integration
\begin{equation}
  Q(t)=Q(0)\exp{\frac{-1}{RC}}.
  \label{eqn:entladung}
\end{equation}

\textbf{Aufladen eines Kondensators} \\
Ähnlich wie in Abschnitt \ref{sec:Theorie} kann eine Gleichung für den
Aufladevorgang des Kondensators hergeleitet werden. Hier gelten die
Randbedingungen $Q(0)=0$ und $Q(\infty)=CU_{0}$. Somt ergibt sich
für den Aufladevorgang mit einer äußeren Spannungsquelle der Spannung $U_{0}$
die Gleichung
\begin{equation}
  Q(t)=CU_{0}(1-\exp{\frac{-t}{RC}}).
  \label{aufladen}
\end{equation}
Dabei wird der Ausdruck RC als Zeitkonstante des Relaxationsvorganges bezeichnet,
er ist ein Maß für die Geschwindigkeit, mit der das System seinem Endzusand entgegenstrebt.\\
\\

Wenn ein System von außen periodisch aus seiner Gleichgewichtslage ausgelenkt wird
treten ebenfalls Relaxationsvorgänge auf. In dem Beipiel des RC-Kreises wird
das System mit einer Wechselspannung ausgelenkt, wie in Abbildung \ref{fig:RCsin}
zu sehen.
\begin{figure}[H]
  \centering
  \includegraphics[height=5cm]{RCsin.JPG}
  \caption{Periodische Anregung eines RC-Kreises.}
  \label{fig:RCsin}
\end{figure}









\label{sec:Theorie}

%\cite{sample}
