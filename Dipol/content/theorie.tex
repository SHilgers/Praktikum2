
\label{sec:Theorie}
\section{Zielsetzung}

In diesem Versuch wird die Dipolrelaxation eines Ionenkristalls untersucht,
wobei sowohl die charakteristische Relaxationszeit der Dipole als
auch die Aktivierungsenergie des Materials experimentell bestimmt werden.

\section{Theorie}
Ein Ionenkristall ist im Allgemeinen eine periodische Anordung aus positiv geladenen
Kationen und negativ geladenen Anionen, die aufgrund der elektrostatischen Anziehung
eine Ionenbindung eingehen und somit einen Kristall bilden. Bringt man anstelle
der einfach geladenen Kationen auch doppelt geladene Kationen, wie beispielsweise
$\ce{Sr^{++}}$ ein, so kommt es aufgrund der notwendigen Ladungsneutralität des Kristalls
zu Leerstellen an den Stellen wo üblicherweise ein einfach positiv geladenes Kation
sitzt. Die Leerstellen bilden dabei mit den zweifach geladenen Kationen permanente
Dipole im Kristall aus, welche an die Symmetrie des Gitters gebunden sind und
somit nur diskrete Werte und Richtungen erlauben. Diese sind jedoch statistisch vertelit,
sodass es keine Vorzugsrichtung und somit auch kein makroskopisches Dipolmoment gibt. \\
Durch die thermische Bewegung ist es den Leerstellen, welche die Aktivierungsenergie
$W$ aufbringen um das Gitterpotential zu überwinden, möglich sich im Kristall zu bewegen,
was eine Richtungsänderung des Dipols ermöglicht. Da diese Aktivierungsenergie
durch die thermische Energie der Dipole aufgebracht wird, ist sich der Anteil dieser
beweglichen Leerstellen gemäß der Boltzmann Statistik gegeben. Daraus
lässt sich die Relaxationszeit, welche die mittlere Zeit zwischen zwei Umorientierungen
eines Dipols bezeichnet, über
\begin{equation}
  \tau(T)=\tau_0\exp{\frac{W}{\symup{k_B}}T}
\end{equation}
berechnen. Dabei bezeichnet $\tau_0$ die charakteristische Relaxationszeit, welche den Grenzwert
der Relaxationszeit bei hohen Temparaturen angibt $\tau_0=\lim_{T \to \infty}\tau(T)$. \\
Wird ein solcher Dipol in einem externen elektrischen Feld ausgesetzt, so ordet sich
ein Teil $y(T)$ der Dipole, welcher durch die Langevin-Funktion gegeben ist in Richtung des Feldes
an. Unter der Annahme $pE\leq\symup{k_B}}T$, wobei $p$ das Dipolmoment bezeichnet und $E$ die
Stärke des Feldes, lässt sich diese Funktion zu
\begin{equation}
  y(T)=\frac{pE}{3\symup{k_B}}T}
\end{equation}
nähern. Kühlt man den Kristall anschließend bei eingeschaltetem Feld zügig ab, so lässt sich
die Anzahl der Dipole "einfrieren". Wird nun eine konstante Heizrate $\frac{\symup{d}T}{}
