\section{Auswertung}
\label{sec:Auswertung}
Die Werte für die Wellenlänge $\lambda$ des Lasers, sowie der Abstand $L$ vom Optischen Element zum
Detektor betragen:
\begin{align*}
  \lambda=\SI{635}{\nm}\\
  L=\SI{1,085}{\m}.
\end{align*}

Für den Dunkelstrom wurden
\begin{equation*}
   I_{\text{Dunkel}}=\SI{0.1}{\nA}
\end{equation*}
gemessen.\\

Die Messwerte für den ersten Doppelspant sind in Tabelle \ref{tab:tabelle1} zu sehen.
Mit Hilfe einer Ausgleichsfunktion werden die Werte für die Spaltbreite $b$, den
Spaltabstand $s$ und die Verschiebung des Maximuns vom Nullpunkt $x_{0}$ bestimmt.
Diese Ausgleichsfunktion wird nach Gleichung \ref{eqn:doppelinf} berechnet und ist in
Abbildung \ref{fig:plot1} zu sehen. Für $\phi$ wurde die Relation $\phi=\frac{x-x_{0}}{L}$
verwendet.
Für den ersten Spalt sind
\begin{align*}
  b=\SI{0,15}{\mm} &\;\;\;\;\;\;\;\;\;\; s=\SI{1}{\mm}
\end{align*}
vom Hersteller angegeben.

\begin{table}[H]
  \centering
  \caption{Messwerte und Ergebniss der Bestimmung der Schallgeschwindigkeit}
  \label{tab:tabe1}
    \begin{tabular}{S||S S||S S||S|S}
    \toprule
    $ \text{Länge l des Zylinders [mm]} $ & $ U_{1} [\text{V}] $ &
    $ t_{1} [\mu\text{s}] $ & $ U_{2} [\text{V}] $ &
    $ t_{2} [\mu\text{s}] $ & $ \increment t [\mu\text{s}]$ &
    $ \text{c} [\text{m}/\text{s}]$\\
    \midrule
    31.0 & 1.335 \: & 24.0 & 1.096 \:  & 46.9 & 22.9 & 2707.42 \\
          \bottomrule
    \end{tabular}
  \end{table}

\begin{figure}
  \centering
  \includegraphics[height=7cm]{Figure_1.png}
  \caption{Messwerte und Ausgleichsfunktion für den ersten Doppelspalt.}
  \label{fig:plot1}
\end{figure}

Die Regression ergibt für den erten Spalt folgende Werte:
\begin{align*}
  b &=\SI{2.48(13631278)e-7}{\mm}\\
  s &=\SI{24.9(0)}{\mm}\\
  x_{0} &=\SI{0.12(3)}{\mm}
\end{align*}

Um die Messwerte mit den Herstellerangaben zu vergleichen wird die Abweichung nach folgender
Formel berechnet:
\begin{equation}
  \text{Abweichung}=\frac{\text{Messwert}-\text{Herstellerangabe}}{\text{Herstellerangabe}}*100
\end{equation}
Daraus ergibt sich, dass die Spaltbreite $b$ um -99\% abweicht  %(-0+/-9)e+01
und der Spaltabstand $s$ um 2390\%. Um die Abweichung der Nulllinie zu bestimmen wird
der Winkel $\phi$ nach $\phi=\frac{x-x_{0}}{L}$ berechnet. Theoretisch liegt die
Nulllinie bei $\Delta \phi=0\;\text{rad} $. Es ergibt sich
\begin{equation*}
  \Delta \phi =\SI{0.023(3)}\; \text{rad}.
\end{equation*}
%23.91+/-0.04
\\
Für den zweiten Doppelspalt wird anaglog vorgegeangen, die Messwerte sind in Tabelle
\ref{tab:tabelle2} zu finden und die Ausgleichsfunktion in Abbildung \ref{fig:plot2}.
Für diesen Doppelspalt wurden vom Hersteller
\begin{align*}
  b=\SI{0,1}{\mm}  &\;\;\;\;\;\;\;\;\;\; s=\SI{0,4}{\mm}
\end{align*}
angegeben.

\begin{table}[H]
  \centering
  \caption{Zählrate und Energiemaximum bei variiertem Druck, Abstand a=2cm}
  \label{tab:tab2}
    \begin{tabular}{c c c c c}
    \toprule
    Druck $\rho$/\;mbar & Energiemaximum & Zählrate $N$ & Energie $E_{\alpha}$ & effektive Länge $x$/\;cm\\
    \midrule
    0 & 796 &131382  &4          & 0.0   \\
    50 & 775 &131464 &3.89 & 0.09 \\
    100 &756 &130732 &3.79 & 0.19\\
    150 &749 &129617 &3.76  &  0.29\\
    200 &749 &130444 &3.76  & 0.39\\
    250 &727 &129600 &3.65 & 0.49\\
    300 &722 &128936 &3.63 & 0.59\\
    350 &708 &128478 &3.56 & 0.69\\
    400 &696 &128122 &3.49 & 0.79\\
    450 &687 &127415 &3.45 & 0.89\\
    500 &674 &126608 &3.39 & 0.99\\
    550 &663 &126372 &3.33 &1.09\\
    600 &651 &124989 &3.27 & 1.18\\
    650 &634 &124942 &3.19 & 1.28\\
    700 &618 &124295 &3.11 &1.38\\
    750 &602 &123299 &3.03 & 1.48\\
    800 &584 &119958 &2.93 &1.58\\
    850 &566 &120673 &2.84 &1.68\\
    900 &548 &117907 &2.75 & 1.78\\
    950 &534 &116111 &2.68&   1.88\\
    1000 &499& 108630&2.51 & 1.07\\
    \bottomrule
    \end{tabular}
  \end{table}

\begin{figure}
  \centering
  \includegraphics[height=7cm]{Figure_2.png}
  \caption{Ausgleichsfunktion und Messwerte des zweiten Doppelspaltes.}
  \label{fig:plot2}
\end{figure}

Hier ergibt die Regression:
\begin{align*}
  b =&\SI{3.148(0)e-2}{\mm}\\
  s =&\SI{24.699(0)}{\mm}\\
  x_{0} =&\SI{2.611(1)e-2}{\mm}.
\end{align*}
Die Abweichungen von den Herstellerangaben berechnen sich analog zum
ersten Doppelspalt, es ergibt sich für den Spaltabstand $s$ eine Abweichung von
6075\% und für die Spaltbreite eine Abweichung von -69\%. Für den Winkel der
Nulllinie ergibt sich:
\begin{equation}
  \Delta \phi =\SI{0.023(0)}\; \text{rad}.
\end{equation}
%-0.685+/-0.009
%5.17+/-0.04

Für die Ausgleichfunktion in Abblildung \ref{fig:plot3} des Einzelspaltes
wird Gleichung \ref{eqn:einzel1} verwendet.
Die dazugehörigen Messwerte stehen in Tabelle \ref{tab:tabelle3}.
Vom Hersteller wird hier die Spaltbreite mit $b=\SI{0,15}{\mm}$ angegeben.

\begin{table}[H]
  \centering
   \begin{tabular}{c c c}
    \toprule
     n& $\nu$/\; 1/s & $\nu_{Wechsel}$\\
    \midrule
    0,5 & 100.01& 50,0\\
    1 & 79.93 & 79.93\\
    2 & 23.93 & 47.86\\
    \bottomrule
  \end{tabular}
  \caption{Gemessene Frequenzen der Sägezahnspannung, sowie die Daraus resultierenden Frequenzen für die
  Wechselspannung.}
  \label{tab:tab3}
\end{table}

\begin{figure}
  \centering
  \includegraphics[height=7cm]{Figure_4.png}
  \caption{Ausgleichsfunktion und Messwerte des Einzelspaltes.}
  \label{fig:plot3}
\end{figure}

Die Regression ergibt für den Einzelspalt diese Werte:
\begin{align*}
  b =&\SI{0.15(4)}{\mm}\\
  x_{0} =&\SI{24.89(12)}{\mm}.
\end{align*}
Hier ist zu beachten, dass der Wert des Maximuns aus der Ausgleichsrechnung ausgenommen
wurde, da sich ansonsten kein numerisch zu verarbeitender Wert ergibt.
Hier beträgt die Abweichung der Spaltbreite 0\%.
Die Abweichung der Nulllinie wir äquivalent wie beim Doppelspalt berechnet.
Es ergibt sich
\begin{equation*}
  \Delta \phi= \SI{0.006(11)}\; \text{rad}.
\end{equation*}
%0.006+/-0.028
