\section{Diskussion}
\label{sec:Diskussion}
Die berechneten Werte für die Spaltbreite $b$, den Spaltabstand $s$, die Amplitude $I$
und die
Verschiebung vom Mittelpunkt $\Delta \phi$ der beiden
Doppelspalte sind hier nocheinmal zusammengefasst.\\
\\
Erster Doppelspalt:
\begin{align*}
  I =& \SI{8.1(2)}{\nA}\\
  B =& \SI{1.77(8)e-4}{\mm}\\
  S =& \SI{24.84(4)}{\mm}\\
  i =& \SI{7.8(2)}{\nA}\\
  b =& \SI{9(198564)e-7}{\mm}\\
  s =& \SI{24.9(4)}{\mm}\\
  \Delta \phi &=\SI{0.0228(9)}\; \text{rad}.
\end{align*}
Es ergeben sich die folgenden Abweichungen von den Herstellerangaben: Für die
Spaltbreite beträgt die Abweichung 18\% und für den Spaltabstand 2387\%.
Der berechnete Mittelpunkt, bzw. der Ort des Hauptmaximums weißt nur eine
geringe Abweichung von der theoretischen Nulllinie bei $\Delta \phi = 0\; \text{rad}$ auf.
Der Mittelwert der Amplitude beträgt $\SI{8.1(2)}{\nA}$.\\
\\

Zweiter Doppelspalt:
\begin{align*}
  I =& \SI{3.42(27)}{\nA}\\
  B =& \SI{0.44(4)}{\mm}\\
  S =& \SI{25.54(0)}{\mm}\\
  i =& \SI{3.7(3)}{\nA}\\
  b =& \SI{0.46(4)}{\mm}\\
  s =& \SI{25.4(0)}{\mm}\\
  \Delta \phi &=\SI{0.021(0)}\; \text{rad}.
\end{align*}
Hier ligen die Abweichungen für die Spaltbreite bei
350\% und für die Spaltabstand bei 6267.5\%.
Der Mittelwert der Amplitude beträgt $\SI{3.7(3)}{\nA}$. Wie beim ersten Doppelspalt auch
liegt das Hauptmaximum vergleichsweise nah an dem theoretischen Wert von
$ \Delta \phi= 0\;\text{rad}$.\\
\\
Für den Einzelspalt ergeben sich folgende Messwerte:
\begin{align*}
  i =& \SI{5(1)}{\nA}\\
  b =& \SI{0.08(1)}{\mm}\\
  I =& \SI{5.5(2)}{\nA}\\
  B =& \SI{0.08(1)}{\mm}\\
  \Delta \phi=& \SI{0.0005(9)}\; \text{rad}.
\end{align*}
Hier beträgt die Abweichung der Spaltbreite von der Herstellerangabe
-56.66\%. Für die Amplitude ergibt sich als Mittelwert $\SI{5.5(2)}{\nA}$.
\\
Die Nulllinie liegt bei allen Messungen sehr nah am Theoriewert, obwohl das
optische Element per Hand im Strahlengang des Lasers platziert wird und schon minimale
Positionsänderungen das Interferenzbild verändern.\\
Da bei den Doppelspalten, abgesehen von der Spaltbreite des ersten Spaltes, die Abweichungen
der Spaltbreite und des Spaltabstandes
sehr hoch sind, ist von einem systematischem Fehler auszugehen. Eine weitere
Begründung könnte eine nicht optimale Ausgleichsfunktion sein, da die Messwerte
nicht symmetrisch zur Nulllinie sind. Wie in den Abbildungen
zu erkennen, passt die Ausgleichsfunktion nicht immer ideal zu den Messwerten, obwohl
die Ausgleichsfunktionen in der Mitte aufgeteilt wurden.
