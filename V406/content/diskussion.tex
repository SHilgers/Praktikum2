\section{Diskussion}
\label{sec:Diskussion}
Die berechneten Werte für die Spaltbreite $b$, den Spaltabstand $s$ und die
Verschiebung vom Mittelpunkt $\Delta \phi$ der beiden
Doppelspalte sind hier nocheinmal zusammengefasst.\\
\\
Erster Doppelspalt:
\begin{align*}
  b &=\SI{2.48(13631278)e-7}{\mm}\\
  s &=\SI{24.9(0)}{\mm}\\
  \Delta \phi &=\SI{0.023(3)}\; \text{rad}.
\end{align*}
Hieraus ergeben sich sehr hohe Abweichungen von den Herstellerangaben. Für die
Spaltbreite beträgt die Abweichung -99\% und für den Spaltabstand 2390\%.
Der berechnete Mittelpunkt, bzw. der Ort des Hauptmaximums weißt nur eine
geringe Abweichung von der theoretischen Nulllinie bei $\Delta \phi = 0\; \text{rad}$ auf.\\
\\

Zweiter Doppelspalt:
\begin{align*}
  b =&\SI{3.148(0)e-2}{\mm}\\
  s =&\SI{24.699(0)}{\mm}\\
  \Delta \phi &=\SI{0.023(0)}\; \text{rad}.
\end{align*}
Hier sind die Abweichungen ebenfalls sehr hoch, für die Spaltbreite liegt sie bei
-69\% und für die Spaltabstand bei 6075\%. Wie beim ersten Doppelspalt auch
liegt das Hauptmaximum vergleichsweise nah an dem theoretischen Wert von
$ \Delta \phi= 0\;\text{rad}$.\\
\\
Für den Einzelspalt ergeben sich folgende Messwerte:
\begin{align*}
  b =&\SI{0.15(4)}{\mm}\\
  \Delta \phi=& \SI{0.006(11)}\; \text{rad}.
\end{align*}
Überraschenderweise stimmt hier die Spaltbreite sehr gut mit dem Theoriewert überein, auch
die Abweichung der Nulllinie ist sehr gering.
\\
Die Nulllinie liegt bei allen Messungen sehr nah am Theoriewert, obwohl das
optische Element per Hand im Strahlengang des Lasers platziert wird und schon minimale
Positionsänderungen das Interferenzbild verändern.\\
Da bei den Doppelspalten die Abweichungen der Spaltbreite und des Spaltabstandes
sehr hoch sind, ist von einem systematischem Fehler auszugehen. Eine weitere
Begründung könnte eine nicht optimale Ausgleichsfunktion sein. Wie in den Abbildungen
zu erkennen, passt die Ausgleichsfunktion nicht immer ideal zu den Messwerten, daher
kann es zu Fehlern bei der Berechnung der Parameter kommen. Die Versuche, die Ausgleichfunktion
besser anzupassen zeigten leider keinen Erfolg.
