\section{Durchführung}
\subsection{Widerstandsmessbrücke}
Zu Beginn werden mit der Widerstandsmessbrücke zwei unbekannte Widerstände $R_{x}$ mit
drei verschiedenen Widerständen $R_{2}$ ausgemessen.
Dazu wird Wechselstrom an die Schaltung angelegt, außerdem wird vor das Oszilloskop
Als Abstimmvorrischtung wird ein Potentiometer mit $1k\Omega$ Gesamtwiderstand
verwendet. Nun wird das Potentiometer so eingestellt, dass am Oszilloskop
keine Brückenspannung mehr messbar ist. Dieser Vorgang wird für zwei weitere
Widerstände $R_{2}$ wiederholt.

\subsection{Kapazitätsmessbrücke}
Nun wird mit Hilfe der Kapazitätsmessbrücke die Kapatität zweier unbekannter
Kondensatoren $C_{x}$ sowie die Daten eier RC-Kombination ausgemessen. Jede Kapatität
wird mit drei verschiedenen Kondensatoren $C_{2}$ ausgemessen.
Das diese Schaltung zwei Abstimmvorrichtungen $R_{2}$ und $R_{3}/R{4}$ besitzt werden diese
alternerend verstellt.

\subsection{Induktivitätsmessbrücke und Maxwell-Brücke}
Unter Verwendung der Induktivitätsmessbrücke wird nun die Induktivität sowie
der Verlustwiderstand einer unbekannen Spule ausgemessen. Das Vorgehen ist hier
ähnlich wie bei der Kapatitätsmessbrücke.
Die gleiche Induktivität wird nun ein zweites mal vermessen, dieses Mal unter
Verwendung der Maxwell-Brücke.

\subsection{Frequenzabhängigkeit der Brückenspannung-Wien-Robinson-Brücke}
Hier wird die Brückenspannung in Abhängigkeit der Frequenz der Speisespannung gemessen.
Dazu wird die Frequenz am Sinusgenerator langsam von 20Hz auf 30\:000Hz erhöht.

\subsection{Klirrfaktormessung}





\label{sec:Durchführung}
