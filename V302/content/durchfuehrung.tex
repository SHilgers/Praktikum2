\section{Durchführung}
\subsection{Widerstandsmessbrücke}
Zu Beginn werden mit der Widerstandsmessbrücke zwei unbekannte Widerstände $R_{x}$ mit
drei verschiedenen Widerständen $R_{2}$ ausgemessen.
Dazu wird Wechselstrom an die Schaltung angelegt. Außerdem wird vor das Oszilloskop
ein Tiefpass geschaltet, um hochfrequente Störspannungen zu unterdrücken.
Als Abstimmvorrischtung wird ein Potentiometer mit $\SI{1}{\kilo\ohm}$ Gesamtwiderstand
verwendet. Nun wird das Potentiometer so eingestellt, dass am Oszilloskop
ein Minimum der Brückenspannung zu messen ist. Dieser Vorgang wird für zwei weitere
Widerstände $R_{2}$ wiederholt.

\subsection{Kapazitätsmessbrücke}
Nun wird mit Hilfe der Kapazitätsmessbrücke die Kapatität zweier unbekannter
Kondensatoren $C_{x}$ sowie die Daten eier RC-Kombination ausgemessen. Jede Kapazität
wird mit drei verschiedenen Kondensatoren $C_{2}$ ausgemessen.
\noindent Da diese Schaltung zwei Abstimmvorrichtungen $R_{2}$ und $R_{3}/R_{4}$ besitzt werden diese
alternierend verstellt.

\subsection{Induktivitätsmessbrücke und Maxwell-Brücke}
\noindent Unter Verwendung der Induktivitätsmessbrücke wird nun die Induktivität sowie
der Verlustwiderstand einer unbekannten Spule ausgemessen. Das Vorgehen ist hier
ähnlich wie bei der Kapazitätsmessbrücke.
Die gleiche Induktivität wird nun ein zweites mal vermessen. Dieses Mal unter
Verwendung der Maxwell-Brücke.

\subsection{Frequenzabhängigkeit der Brückenspannung-Wien-Robinson-Brücke}
\noindent Hier wird die Brückenspannung in Abhängigkeit der Speisespannung und deren Frequenz gemessen.
Dazu wird die Frequenz am Sinusgenerator von $\SI{20}{\hertz}$ auf $\SI{30000}{\hertz}$ erhöht und
die dazugehörigen Werte für die Brückenspannung und die Speisespannung notiert.





\label{sec:Durchführung}
