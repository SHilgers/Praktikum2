\section{Auswertung}
\subsection{Wheatstonesche Brücke}
Die gemessenen Werte für die Widerstände $ R_2, \: R_3 \: \text{und} \: R_4 $ zur
Bestimmung des unbekannten Widerstands $\text{R}_{x10} $ (Wert 10) befinden sich in
der untenstehenden Tabelle \ref{tab:tabe1}
\begin{table}[H]
  \centering
  \caption{Messwerte und Ergebniss der Bestimmung der Schallgeschwindigkeit}
  \label{tab:tabe1}
    \begin{tabular}{S||S S||S S||S|S}
    \toprule
    $ \text{Länge l des Zylinders [mm]} $ & $ U_{1} [\text{V}] $ &
    $ t_{1} [\mu\text{s}] $ & $ U_{2} [\text{V}] $ &
    $ t_{2} [\mu\text{s}] $ & $ \increment t [\mu\text{s}]$ &
    $ \text{c} [\text{m}/\text{s}]$\\
    \midrule
    31.0 & 1.335 \: & 24.0 & 1.096 \:  & 46.9 & 22.9 & 2707.42 \\
          \bottomrule
    \end{tabular}
  \end{table}

\noindent Durch die Gleichung \ref{eqn:rx}
und Mittelung mit :
\begin{equation}
  \bar{x} = \frac{1}{N} \sum_{i=1}^{N} x_i
\end{equation}
wobei der dazugehörige Fehler sich durch
\begin{equation}
  \increment \bar{x} = \frac{1}{\sqrt{N}} \sqrt{ \frac{1}{N-1} \sum_{i=1}^N
  (x_i - \bar{x})^2}
  \label{eqn:mitf}
\end{equation}
berechnet, ergibt sich somit:
\begin{equation*}
  \text{R}_{x10} = \SI{239.69 (17)}{\ohm}
\end{equation*}
Da das Verhältniss von $ \frac{R_3}{R_4} $ laut Hersteller bereits mit einem
Fehler von $ \pm 0.5 \% $  behaftet ist und auch $ \text{R}_2 $ nur bis auf
0,2 \% genau ist, muss hier die auch die Gaußsche
Fehlerfortpflanzung mit der Formel
\begin{equation}
  \increment f = \sqrt{ \sum_{i=1}^N \left( \frac{\partial f}{\partial x_i}\right)^2
  \cdot (\increment x_i)^2  }
  \label{eqn:gaus}
\end{equation}
beachtet werden, durch die sich für $ \text{R}_{x10} $ ein Fehler von $ \pm 1.29 \si{\ohm} $
ergibt, welcher deutlich größer als der Mittelwertfehler ist und somit der signifikante
Fehler. Somit ergibt sich insgesamt
\begin{equation*}
  \text{R}_{x10} = \SI{239.7 (13)}{\ohm}
\end{equation*}


Analog lässt sich der Wert 11 berechnen, dessen Messwerte sich in Tabelle
\ref{tab:tabe2} befinden
\begin{table}[H]
  \centering
  \caption{Zählrate und Energiemaximum bei variiertem Druck, Abstand a=2cm}
  \label{tab:tab2}
    \begin{tabular}{c c c c c}
    \toprule
    Druck $\rho$/\;mbar & Energiemaximum & Zählrate $N$ & Energie $E_{\alpha}$ & effektive Länge $x$/\;cm\\
    \midrule
    0 & 796 &131382  &4          & 0.0   \\
    50 & 775 &131464 &3.89 & 0.09 \\
    100 &756 &130732 &3.79 & 0.19\\
    150 &749 &129617 &3.76  &  0.29\\
    200 &749 &130444 &3.76  & 0.39\\
    250 &727 &129600 &3.65 & 0.49\\
    300 &722 &128936 &3.63 & 0.59\\
    350 &708 &128478 &3.56 & 0.69\\
    400 &696 &128122 &3.49 & 0.79\\
    450 &687 &127415 &3.45 & 0.89\\
    500 &674 &126608 &3.39 & 0.99\\
    550 &663 &126372 &3.33 &1.09\\
    600 &651 &124989 &3.27 & 1.18\\
    650 &634 &124942 &3.19 & 1.28\\
    700 &618 &124295 &3.11 &1.38\\
    750 &602 &123299 &3.03 & 1.48\\
    800 &584 &119958 &2.93 &1.58\\
    850 &566 &120673 &2.84 &1.68\\
    900 &548 &117907 &2.75 & 1.78\\
    950 &534 &116111 &2.68&   1.88\\
    1000 &499& 108630&2.51 & 1.07\\
    \bottomrule
    \end{tabular}
  \end{table}

\noindent Hierdurch ergibt sich für Wert 11:
\begin{equation*}
  \text{R}_{x11} = \SI{487.0 (26)}{\ohm}
\end{equation*}
wobei wieder der Gauß-Fehler der größere und somit signifikante ist. Der Fehler
des Mittelwerts beträgt dabei $ \pm 0.12 \si{\ohm}$.

\subsection{Kapazitätsmessbrücke}
Im zweiten Teil sollen zwei unbekannte Kapazitäten und eine RC-Kombination
vermessen werden. Die gemessenen Werte der ersten Kapazität $ \text{C}_{x3} $ (Wert 3) befinden
sich in Tabelle \ref{tab:tabe3} und die der zweiten Kapazität $ \text{C}_{x1} $
(Wert 1) in der
darauf folgenden Tabelle \ref{tab:tabe4}
\begin{table}[H]
  \centering
   \begin{tabular}{c c c}
    \toprule
     n& $\nu$/\; 1/s & $\nu_{Wechsel}$\\
    \midrule
    0,5 & 100.01& 50,0\\
    1 & 79.93 & 79.93\\
    2 & 23.93 & 47.86\\
    \bottomrule
  \end{tabular}
  \caption{Gemessene Frequenzen der Sägezahnspannung, sowie die Daraus resultierenden Frequenzen für die
  Wechselspannung.}
  \label{tab:tab3}
\end{table}

\begin{table}[H]
  \centering
   \begin{tabular}{c c c c}
    \toprule
    Nummer der Oberwelle & $ U_{\text Theorie,Rechteck}\: / \si{\volt} $ &
    $ U_{\text Theorie,Dreick}\: / \si{\volt} $ & $ U_{\text Theorie,Sägezahn}\: / \si{\volt} $ \\
    \midrule
    1 & 1145 & 182 & 573 \\
    2 & 0 & 0 & 286 \\
    3 & 573 & 20 & 191 \\
    4 & 0 & 0 & 143 \\
    5 & 229 & 7 & 115 \\
    6 & 0 & 0 & 96 \\
    7 & 164 & 4 & 82 \\
    8 & 0 & 0 & 72 \\
    9 & 127 & 2 & 64 \\
    10 & 0 & 0 & 57 \\
    \bottomrule
  \end{tabular}
  \caption{Eingestellte Schwingungsamplituden.}
  \label{tab:tabe4}
\end{table}

\noindent Aus Gleichung \ref{eqn:cx}
ergibt sich damit für
\begin{align*}
  \text{C}_{x3} &= \SI{420.6 (29)}{\nano\farad} \\
  \text{C}_{x1} &= \SI{655.4 (35)}{\nano\farad}
\end{align*}
wobei bei $ \text{C}_{x3} $ der Fehler des Mittelwerts der größere war und bei
$ \text{C}_{x1} $ der Gauß-Fehler durch die Ungenauigkeit von $C_2$ mit $ \pm 0.2
\% $ und von dem Verhältniss von $ \frac{R_3}{R_4} $ mit $ \pm 0.5 \% $ .
Der Gauß-Fehler von $ \text{C}_{x3} $ beträgt
dabei $ \pm 2.3 \si{\nano\farad} $ und der Mittelwertfehler von $ \text{C}_{x1} $
beträgt $ \pm 2.9 \si{\nano\farad} $.


\noindent Die RC-Kombination (Wert 9) mit der Kapazität $ \text{C}_{x9} $ und dem
Widerstand $ \text{R}_{x9} $ ergibt sich
mit den Werten aus Tabelle \ref{tab:tabe5} zu
\begin{align*}
  \text{C}_{x9} &= \SI{417 (13)}{\nano\farad} \\
  \text{R}_{x9} &= \SI{475 (14)}{\ohm}
\end{align*}
Dabei wurde für die Kapazität $ \text{C}_{x9} $ der Mittelwertfehler verwendet und
für den Widerstand $ \text{R}_{x9} $ der Fehler nach Gauß mit der Formel \ref{eqn:gaus}
, welcher durch die Abweichung des variablen Widerstands $R_2$ mit $ \pm 3 \% $
recht groß ausfällt. Der entsprechende Mittelwertfehler von $ \text{R}_{x9} $
beträgt dabei $ \pm 0.54 \si{\ohm} $ und der Gauß-Fehler von $ \text{C}_{x9} $ beträgt
$ \pm 2.56 \si{\nano\farad} $.
\begin{table}[H]
  \centering
  \caption{Mechanischen Kompressorleistung zu den Zeiten $t_1$, $t_2$, $t_3$ und $t_4$.}
  \label{tab:tabe5}
    \begin{tabular}{S S}
    \toprule
    $ t  \: / \si{\second} $ & $ N_{\text{mech}} \: / \: \si{\watt}$ \\
    \midrule
    480 & 4.72 \pm 0.16 \\
    960 & 6.19 \pm 0.22 \\
    1500 & 6.67 \pm 0.26 \\
    1980 & 6.26 \pm 0.28 \\
      \bottomrule
    \end{tabular}
\end{table}


\subsection{Induktivitätsmessbrücke}

Die Werte der unbekannten Induktivität $ \text{L}_{x16} $ und dem dazugehörigen
Widerstand $ \text{R}_{x16} $(Wert 16) sind in Tabelle
\ref{tab:tabe6} abzulesen.
\begin{table}[H]
  \centering
  \caption{Messwerte des Absorptionsspektrums von Zirkonium}
  \label{tab:tabe6}
    \begin{tabular}{S S}
    \toprule
    $ \text{Winkel} / ° $ & $ \text{Impulse pro s}$\\
    \midrule
    18.0 & 59.0 \\
    18.2 & 59.0 \\
    18.4 & 58.0 \\
    18.6 & 56.0 \\
    18.8 & 55.0 \\
    19.0 & 57.0 \\
    19.2 & 70.0 \\
    19.4 & 88.0 \\
    19.6 & 105.0 \\
    19.8 & 105.0 \\
    20.0 & 112.0 \\
    20.2 & 115.0 \\
    20.4 & 113.0 \\
    20.6 & 116.0 \\
    20.8 & 116.0 \\
    21.0 & 114.0 \\

          \bottomrule
        \end{tabular}
    \end{table}

Durch die Gleichungen \ref{eqn:lx}
ergibt sich somit
\begin{align*}
  \text{R}_{x16} &= \SI{393 (12)}{\ohm} \\
  \text{L}_{x16} &= \SI{51.3 (21)}{\milli\henry}
\end{align*}
mit den Fehlern aus Gleichung \ref{eqn:gaus}, da die Mittelwertfehler aus Gleichung
\ref{eqn:mitf} mit
$ \pm 3.8 \si{\ohm} $ für $ \text{R}_{x16} $ und $ \pm 0.076 \si{\milli\henry} $
für $ \text{L}_{x16} $ jeweils kleiner sind.

\subsection{Maxwell-Brücke}

Bei der Messung wird eine Kapaziät $ \text{C}_{2} = \SI{450.0(9)}{\nano\farad} $
verwendet. Die Messwerte befinden sich in Tabelle \ref{tab:tabe7}
\begin{table}[H]
  \centering
  \caption{Werte der zweiten Messreihe für Wert 16}
  \label{tab:tabe7}
    \begin{tabular}{S S S}
    \toprule
    $ \text{R}_{2} \: / \: \si{\ohm} $ & $\text{R}_{3} \: / \: \si{\ohm} $ &
    $\text{R}_{4} \: / \: \si{\ohm} $ \\
    \midrule
    500 & 178 & 328 \\
    664 & 178 & 328 \\
    1000 & 119 & 332 \\
    \bottomrule
    \end{tabular}
\end{table}

Aus den Gleichungen \ref{eqn:mx}
ergibt sich\begin{align*}
  \text{R}_{x16} &= \SI{330 (29)}{\ohm} \\
  \text{L}_{x16} &= \SI{49 (14)}{\milli\henry}
\end{align*}
wobei der Mittelwertfehler für $ \text{R}_{x16} $ verwendet wurde (Gauß-Fehler =
$ \pm 14 \si{\ohm} $) und der Gauß-Fehler für $ \text{L}_{x16} $ (Mittelwertfehler=
$ \pm 4.4 \si{\milli\henry} $).

\subsection{Wien-Robinson-Brücke}
Die Messung der Wien-Robinson-brücke wird mit einer Kapazität C=$ 992\si{\nano\farad}$
und einem Widerstand von R= $664\si{\ohm}$ durchgeführt. Die Messwerte sind in Tabelle
\ref{tab:tabe8} dargestellt.

\begin{table}[H]
  \centering
  \caption{Werte der Messreihe die Wien-Robinson-brücke}
  \label{tab:tabe8}
    \begin{tabular}{S S S S}
    \toprule
    $ \nu \: / \: \si{\hertz} $ & $\text{U}_b \: / \: \si{\volt} $ &
    $\text{U}_s \: / \: \si{\volt} $ &
    $\frac{U_b}{U_s}$ \\
    \midrule
    20 & 0.120 & 3.08 & 0.039 \\
    50 & 0.248 & 4.56 & 0.054 \\
    100 & 0.320 & 4.64 & 0.069 \\
    150 & 0.264 & 4.56 & 0.058 \\
    200 & 0.136 & 4.50 & 0.030 \\
    220 & 0.072 & 4.48 & 0.016 \\
    230 & 0.032 & 4.48 & 0.007 \\
    240 & 0.024 & 4.48 & 0.005 \\
    242 & 0.016 & 4.48 & 0.004 \\
    250 & 0.040 & 4.48 & 0.009 \\
    265 & 0.080 & 4.48 & 0.018 \\
    300 & 0.298 & 4.56 & 0.065 \\
    500 & 0.704 & 4.56 & 0.154 \\
    1000 & 1.17 & 4.32 & 0.271 \\
    3000 & 1.41 & 4.28 & 0.330 \\
    10000 & 1.44 & 4.24 & 0.340 \\
    20000 & 1.44 & 4.24 & 0.340 \\
    30000 & 1.44 & 4.24 & 0.340 \\

    \bottomrule
    \end{tabular}
\end{table}


\noindent In Abbildung \ref{fig:plot} ist das Verhältniss $\frac{U_b}{U_s}$ gegen
$\frac{\nu}{\nu_0} $ aufgetragen.

\begin{figure}[H]
  \centering
  \includegraphics{plot.pdf}
  \caption{Diagramm zur Wien-Robinson-Brücke}
  \label{fig:plot}
\end{figure}

Der Theoriwert für $\omega_0$ ergibt sich zu
\begin{equation*}
  \omega_0 = \frac{1}{RC} = \frac{1}{664\si{\ohm } \cdot 992\si{\nano\farad}}
  = 1518.17 \si{\hertz}
\end{equation*}
und somit
\begin{equation*}
  \nu_0 = \frac{\omega_0}{2\pi} = 241.62 \si{\hertz}
\end{equation*}

\subsection{Klirrfaktorbestimmung}

Das Minimum der Frequenz liegt bei einer Frequenz von 242 $\si{\hertz}$
mit einer Brückenspannung $\text{U}_b = 0.016 \si{\volt} \: \text{und} \:
\text{U}_s = 4.48 \si{\volt} $.
Durch die Gleichung \ref{eqn:betrag}
ergibt sich f(2) zu
\begin{equation*}
  f(\Omega =2) = \sqrt{\frac{1}{9} \cdot \frac{\left(\Omega^2-1\right)^2}{\left(1-\Omega^2
  \right)^2+9\Omega^2}} = \sqrt{\frac{1}{9} \cdot \frac{(3)^2}{(-3)^2+36}} =
  \frac{1}{\sqrt{45}}
\end{equation*}
Daraus folgt, dass
\begin{equation*}
  \text{U}_2 = \frac{\text{U}_b}{f(2)} = \SI{2.385}{\milli\volt}
\end{equation*}
und schlussendlich
\begin{equation*}
  \text{k}= \frac{\text{U}_2}{\text{U}_1} = 0.149
\end{equation*}
