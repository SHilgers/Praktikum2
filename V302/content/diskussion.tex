\section{Diskussion}
Bei den Messungen ist allgemein zu erkennen, dass die Mittelwertfehler
meistens kleiner waren als die, die nach der Fehlerfortpflanzung berechnet werden und sich durch
die Ungenauigkeiten der Bauteile ergeben, was auf eine recht genaue Messung schließen lässt.

\noindent Insbesondere bei der Maxwell-Brücke treten jedoch auch teilweie
recht große Fehler auf, was auf eine eher geringe Genauigkeit hindeutet. Auch der nach Gauß
berechnete Fehler ist durch die mit $\pm 3 \% $ Ungenauigkeit behafteten Bauteile
$\text{R}_3 $ und $\text{R}_4 $ recht groß.
Zudem ist auch auffällig, dass der Widerstand $ \text{R}_{x16} $ bei der Messung
durch die Induktionsbrücke zu
\begin{equation*}
  \text{R}_{x16} = \SI{393 (12)}{\ohm}
\end{equation*}
und durch die Maxwell-Brücke zu
\begin{equation*}
  \text{R}_{x16} = \SI{330 (29)}{\ohm}
\end{equation*}
ergibt, also eine Abweichung von etwa $ 19 \% $, was zudem außerhalb der
Fehlerintervalle liegt und somit auf einen systematischen Fehler in einer der beiden
Messungen hindeuetet.\\
\noindent Auch bei der Wien-Robinson-Brücke scheint ein systematischer Fehler vorzuliegen,
da die Messwerte bei hohe Frequenzen und im Minimum zwar nahe an der Theoriekurve
liegen, bei niedrigen Frequenzen treten jedoch große Abweichungen auf.
