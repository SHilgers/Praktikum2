\section{Auswertung}
\subsection{Kontrastmessung}

Nach Formel \ref{eqn:Kontrasttheo} kann der Kontrast des Interferometers in Abhängigkeit des Polarisationswinkels $\Phi$
berechnet werden. Die dazu verwendeten Messwerte $I_{\text{max}}$, $I_{\text{min}}$, der Polarisationswinkel sowie
der berechnete Kontrast sind in Tabelle \ref{tab:tab1} dargestellt.

\begin{table}[H]
  \centering
  \caption{Messwerte und Ergebniss der Bestimmung der Schallgeschwindigkeit}
  \label{tab:tabe1}
    \begin{tabular}{S||S S||S S||S|S}
    \toprule
    $ \text{Länge l des Zylinders [mm]} $ & $ U_{1} [\text{V}] $ &
    $ t_{1} [\mu\text{s}] $ & $ U_{2} [\text{V}] $ &
    $ t_{2} [\mu\text{s}] $ & $ \increment t [\mu\text{s}]$ &
    $ \text{c} [\text{m}/\text{s}]$\\
    \midrule
    31.0 & 1.335 \: & 24.0 & 1.096 \:  & 46.9 & 22.9 & 2707.42 \\
          \bottomrule
    \end{tabular}
  \end{table}


Die berechneten Kontrastwerte sind in Abbildung \ref{fig:Kontrast} zusammen mit der Ausgleichsrechnung
gegen den Polarisationswinkel aufgetragen. Dabei wird für die Regression der nach der Theorie geltende Zusammenhang
\begin{equation}
  K\sim 2a\cdot|sin(\Phi)cos(\Phi)|
\end{equation}
verwendet, wobei hier nur die Werte von $\phi=0$ bis $\frac{\pi}{2}$ verwendet werden. Denn die Werte von
$\phi=\frac{\pi}{2}$ bis $2\pi$ passen offensichtlich nicht zum erwarteten Verlauf und werden daher bei der
Regression nicht berücksichtigt.
Für den Fitparameter $a$ ergibt sich aus der Regression folgender Wert:
\begin{equation}
  a=\SI{0,84(2)}{}.
\end{equation}

\begin{figure}[H]
  \centering
  \includegraphics[width=12cm]{Kontrast.pdf}
  \caption{Der Kontrast in Abhängigkeit der Polarisation $\Phi$, sowie die Regression.}
  \label{fig:Kontrast}
\end{figure}

Der maximale Kontrast von $K=\SI{0,84}{}$ ergibt sich bei einem Polarisationswinkel von $\Phi=\SI{50}{\degree}$,
weshalb diese Einstellung für die Weiteren Messungen verwendet wird.

\subsection{Brechungsindex von Glas}

Um aus der Anzahl der gemessenen Interferenzmaxima/-minima den Brechungsindex nach Formel \ref{eqn:pM0}
bestimmen zu können muss beachtet werden, dass in dem verwendeten Doppelglashalter zwei Glasplättchen verbaut sind,
welche um $\Theta_0=\pm\SI{10}{\degree}$ verkippt sind. Somit ergibt sich die resultierende
Phasenverschiebung nach
\begin{equation}
  \Delta\Phi(\Theta)=\Delta\Phi(\Theta+\Theta_0)+\Delta\Phi(\Theta-\Theta_0).
\end{equation}
Eingesetzt in $M=\sfrac{\Delta\Phi}{2\pi}$ ergibt sich daraus eine Formel zur Berechnung des Brechungsindex:
\begin{equation}
  \Delta\Phi(\Theta)=\frac{2T\Theta\Theta_0}{2T\Theta\Theta_0-M\lambda}.
\end{equation}

Dabei bezeichnet $M$ die Anzahl der gemessenen Maxima und $\lambda$ die Wellenlänge des
verwendeten HeNe-Lasers von $\SI{632,990}{\nm}$. Die Dicke der Platten $T$ ist mit
$T=\SI{1}{\mm}$ gegeben. Da die Messung von -4 bis 7\:$\si{\degree}$ durchgeführt wurde beträgt
$\theta=\SI{11}{\degree}$. Die so berechneten Werte für die Brechungsindizes sind zusammen mit
der Anzahl der Maxima in Tabelle \ref{tab:tab2} aufgeführt.

\begin{table}[H]
  \centering
  \caption{Zählrate und Energiemaximum bei variiertem Druck, Abstand a=2cm}
  \label{tab:tab2}
    \begin{tabular}{c c c c c}
    \toprule
    Druck $\rho$/\;mbar & Energiemaximum & Zählrate $N$ & Energie $E_{\alpha}$ & effektive Länge $x$/\;cm\\
    \midrule
    0 & 796 &131382  &4          & 0.0   \\
    50 & 775 &131464 &3.89 & 0.09 \\
    100 &756 &130732 &3.79 & 0.19\\
    150 &749 &129617 &3.76  &  0.29\\
    200 &749 &130444 &3.76  & 0.39\\
    250 &727 &129600 &3.65 & 0.49\\
    300 &722 &128936 &3.63 & 0.59\\
    350 &708 &128478 &3.56 & 0.69\\
    400 &696 &128122 &3.49 & 0.79\\
    450 &687 &127415 &3.45 & 0.89\\
    500 &674 &126608 &3.39 & 0.99\\
    550 &663 &126372 &3.33 &1.09\\
    600 &651 &124989 &3.27 & 1.18\\
    650 &634 &124942 &3.19 & 1.28\\
    700 &618 &124295 &3.11 &1.38\\
    750 &602 &123299 &3.03 & 1.48\\
    800 &584 &119958 &2.93 &1.58\\
    850 &566 &120673 &2.84 &1.68\\
    900 &548 &117907 &2.75 & 1.78\\
    950 &534 &116111 &2.68&   1.88\\
    1000 &499& 108630&2.51 & 1.07\\
    \bottomrule
    \end{tabular}
  \end{table}


Um den Mittelwert des Brechungsindizes berechnet sich über
\begin{equation}
  \bar{n}=\frac{1}{a}\sum_{i=1}^{a}n_i
  \label{eqn:Mittel}
\end{equation}
und sein Fehler mit
\begin{equation}
  \sigma_{\bar{n}}=\sqrt{\frac{1}{a(a-1)}\sum_{i=1}^{a}(\bar{n}-n_i)^2}.
  \label{eqn:Fehler}
\end{equation}
Daraus ergibt sich ein Wert von
\begin{equation}
  n_\text{Glas}=\SI{1.743(9)}{}.
\end{equation}
%Verglichen mit dem Theoriewert von $n_\text{Theorie}=\SI{1,613}{}$ ergibt sich ein Fehler von 21,2\%.

\subsection{Brechungsindex von Luft}

Um den Brechungsindex bei Atmosphärendruck über die Anzahl der Maxima zu bestimmen wird Formel \ref{eqn:pM2}
verwendet. Dabei kann $\Delta n$ durch  $\Delta n= n-1$ ersetzt werden, da das Vakuum den
Brechungsindex 1 besitzt. Eingesetzt in $M=\sfrac{\Delta\Phi}{2\pi}$ ergibt sich somit
\begin{equation}
  n=\frac{M\lambda}{L}+1.
  \label{eqn:nmax}
\end{equation}
Aus den Messwerten, welche in Tabelle \ref{tab:tab3} dargestellt sind, ergeben sich daraus folgende Werte für
den Brechungsindex von Luft bei Atmosphärendruck:

\begin{align*}
  n_\text{Luft1}&=\SI{1,000259526(26)}{}\\
  n_\text{Luft2}&=\SI{1,000253196(25)}{}\\
  n_\text{Luft3}&=\SI{1,000253196(25)}{}
\end{align*}

Nach Formel \ref{eqn:Mittel} ergibt sich daraus der Mittelwert zu:
\begin{equation*}
  n_\text{Luft}=\SI{1,000255306(26)}{}.
\end{equation*}

\begin{table}[H]
  \centering
   \begin{tabular}{c c c}
    \toprule
     n& $\nu$/\; 1/s & $\nu_{Wechsel}$\\
    \midrule
    0,5 & 100.01& 50,0\\
    1 & 79.93 & 79.93\\
    2 & 23.93 & 47.86\\
    \bottomrule
  \end{tabular}
  \caption{Gemessene Frequenzen der Sägezahnspannung, sowie die Daraus resultierenden Frequenzen für die
  Wechselspannung.}
  \label{tab:tab3}
\end{table}


Um das Lorentz-Lorentz-Gesetz zu verifizieren, wird zunächst über die Anzahl der gemessenen Maxima für jeden
Druck gemittelt. Nach Formel \ref{eqn:nmax} kann nun der Brechungsindex in Abhängigkeit des Drucks bestimmt
und in Diagramm \ref{fig:Lorentz} gegeneinander aufgetragen werden.
Die Regression erfolgt nach dem Lorentz-Lorentz-Gesetz aus Gleichung \ref{eqn:lorentz} mit
\begin{equation}
  n\sim\sqrt{1+\frac{Bp}{RT}}
\end{equation}
und liefert für den Fitparameter $B$ den Wert $B=\SI{0,001318(2)}{}$. Die allgemeine Gaskonstante wird dabei mit $R$
bezeichnet und $T$ bezeichnet die Temperatur, welche zum Messzeitpunkt $T=\SI{21,2}{\celsius}$ beträgt.
Somit ergibt sich eine druckabhängige Formel für den Brechungsindex von
\begin{equation}
  n\sim\sqrt{\frac{\SI{0,001318(2)}\cdot p}{RT}}
\end{equation}
mit $p$ in Millibar.

\begin{figure}[H]
  \centering
  \includegraphics[width=13cm]{Glasplot.pdf}
  \caption{Brechungsindizes in abhängigkeit des Drucks mit einer Regression nach dem Lorentz-Lorentz-Gesetz.}
  \label{fig:Lorentz}
\end{figure}
%Der Brechungsindex kann außerdem über das Lorentz-Lorenz Gesetz aus Formel \ref{eqn:lorentz} bestimmt
%werden.

% Mit der molaren Reflexivität $A=\frac{4\pi}{3}N_\alpha \cdot \alpha$ folgt daraus
%\begin{equation}
%  n\sim\sqrt{1+\frac{4\pi p}{RT}N_\alpha \alpha}.
%\end{equation}

%Dabei bezeichnet $R$ die Gaskonstante, $N_\alpha$ die Avogadro-Konstante
%und\\
%$\alpha=\SI{2.118(91)e-29}{\m\cubic}$ \cite{alpha} die molare Polarisierbarkeit von Luft.

%Die auf diesem Wege berechneten Werte für den Brechungsindex von Luft sind in Tabelle \ref{tab:tab3}
%dargestellt. Für den Mittelwert ergibt sich daraus $n_\text{Luft}=\SI{1.0000164(7)}{}$.
