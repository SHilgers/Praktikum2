\section{Auswertung}
\subsection{Kontrastmessung}

Nach Formel \ref{eqn:Kontrasttheo} kann der Kontrast des Interferometers in Abhängigkeit des Polarisationswinkels $\Phi$
berechnet werden. Die dazu verwendeten Messwerte $I_{\text{max}}$, $I_{\text{min}}$, der Polarisationswinkel sowie
der berechnete Kontrast sind in Tabelle \ref{tab:tab1} dargestellt.

\begin{table}[H]
  \centering
  \caption{Messwerte der Wärmepumpe}
  \label{tab:tabe1}
    \begin{tabular}{S S S S S S}
    \toprule
    $ t  \: / \si{\second} $ & $ p_a \: / \si{\bar} $ & $ p_b \: / \si{\bar} $ &
    $ T_1 \: / \si{\kelvin} $ & $ T_2 \: / \si{\kelvin} $ & $ P \: / \: \si{\watt} $\\
    \midrule
    0 & 5.0 & 5.0 & 293.65 & 293.65 & 0 \\
    60 & 4.7 & 6.0 & 294.15 & 293.55 & 115 \\
    120 & 4.4 & 6.4 & 295.15 & 293.15 & 118 \\
    180 & 4.5 & 6.9 & 296.35 & 291.95 & 122 \\
    240 & 4.6 & 7.0 & 297.55 & 290.95 & 125 \\
    300 & 4.6 & 7.0 & 298.85 & 289.95 & 125 \\
    360 & 4.5 & 7.2 & 300.05 & 289.15 & 123 \\
    420 & 4.4 & 7.4 & 301.15 & 288.45 & 123 \\
    480 & 4.3 & 7.8 & 302.35 & 287.65 & 122 \\
    540 & 4.2 & 8.0 & 303.55 & 286.95 & 122 \\
    600 & 4.2 & 8.1 & 304.65 & 286.25 & 121 \\
    660 & 4.1 & 8.3 & 305.75 & 285.55 & 121 \\
    720 & 4.0 & 8.5 & 306.75 & 284.95 & 121 \\
    780 & 4.0 & 8.8 & 307.75 & 284.35 & 121 \\
    840 & 3.9 & 9.0 & 308.75 & 283.75 & 121 \\
    900 & 3.8 & 9.1 & 309.65 & 283.15 & 121 \\
    960 & 3.8 & 9.2 & 310.55 & 282.55 & 122 \\
    1020 & 3.8 & 9.5 & 311.45 & 282.05 & 122 \\
    1080 & 3.7 & 9.8 & 312.25 & 281.55 & 122 \\
    1140 & 3.7 & 10.0 & 313.05 & 281.15 & 122 \\
    1200 & 3.7 & 10.0 & 313.9 & 280.65 & 122 \\
    1260 & 3.6 & 10.2 & 314.65 & 280.25 & 123 \\
    1320 & 3.6 & 10.3 & 315.35 & 279.85 & 123 \\
    1380 & 3.6 & 10.6 & 316.15 & 279.45 & 124 \\
    1440 & 3.6 & 10.8 & 316.85 & 279.15 & 124 \\
    1500 & 3.6 & 11.0 & 317.55 & 278.75 & 124 \\
    1560 & 3.6 & 11.1 & 318.25 & 278.55 & 124 \\
    1620 & 3.6 & 11.2 & 318.95 & 278.25 & 125 \\
    1680 & 3.5 & 11.4 & 319.55 & 277.95 & 125 \\
    1740 & 3.5 & 11.5 & 320.15 & 277.65 & 125 \\
    1800 & 3.5 & 11.7 & 320.75 & 277.45 & 125 \\
    1860 & 3.5 & 11.9 & 321.35 & 277.25 & 125 \\
    1920 & 3.5 & 12.0 & 321.95 & 277.05 & 125 \\
    1980 & 3.5 & 12.1 & 322.45 & 276.95 & 125 \\








      \bottomrule
    \end{tabular}
\end{table}


Die berechneten Kontrastwerte sind in Abbildung \ref{fig:Kontrast} zusammen mit der Ausgleichsrechnung
gegen den Polarisationswinkel aufgetragen. Dabei wird für die Regression der nach der Theorie geltende Zusammenhang
\begin{equation}
  K\sim 2a\cdot|sin(\Phi)cos(\Phi)|
\end{equation}
verwendet, wobei hier nur die Werte von $\phi=0$ bis $\frac{\pi}{2}$ verwendet werden. Denn die Werte von
$\phi=\frac{\pi}{2}$ bis $2\pi$ passen offensichtlich nicht zum erwarteten Verlauf und werden daher bei der
Regression nicht berücksichtigt.
Für den Fitparameter $a$ ergibt sich aus der Regression folgender Wert:
\begin{equation}
  a=\SI{0,84(2)}{}.
\end{equation}

\begin{figure}[H]
  \centering
  \includegraphics[width=12cm]{Kontrast.pdf}
  \caption{Der Kontrast in Abhängigkeit der Polarisation $\Phi$, sowie die Regression.}
  \label{fig:Kontrast}
\end{figure}

Der maximale Kontrast von $K=\SI{0,84}{}$ ergibt sich bei einem Polarisationswinkel von $\Phi=\SI{50}{\degree}$,
weshalb diese Einstellung für die Weiteren Messungen verwendet wird.

\subsection{Brechungsindex von Glas}

Um aus der Anzahl der gemessenen Interferenzmaxima/-minima den Brechungsindex nach Formel \ref{eqn:pM0}
bestimmen zu können muss beachtet werden, dass in dem verwendeten Doppelglashalter zwei Glasplättchen verbaut sind,
welche um $\Theta_0=\pm\SI{10}{\degree}$ verkippt sind. Somit ergibt sich die resultierende
Phasenverschiebung nach
\begin{equation}
  \Delta\Phi(\Theta)=\Delta\Phi(\Theta+\Theta_0)+\Delta\Phi(\Theta-\Theta_0).
\end{equation}
Eingesetzt in $M=\sfrac{\Delta\Phi}{2\pi}$ ergibt sich daraus eine Formel zur Berechnung des Brechungsindex:
\begin{equation}
  \Delta\Phi(\Theta)=\frac{2T\Theta\Theta_0}{2T\Theta\Theta_0-M\lambda}.
\end{equation}

Dabei bezeichnet $M$ die Anzahl der gemessenen Maxima und $\lambda$ die Wellenlänge des
verwendeten HeNe-Lasers von $\SI{632,990}{\nm}$. Die Dicke der Platten $T$ ist mit
$T=\SI{1}{\mm}$ gegeben. Da die Messung von -4 bis 7\:$\si{\degree}$ durchgeführt wurde beträgt
$\theta=\SI{11}{\degree}$. Die so berechneten Werte für die Brechungsindizes sind zusammen mit
der Anzahl der Maxima in Tabelle \ref{tab:tab2} aufgeführt.

\begin{table}[H]
  \centering
  \caption{Wertetabelle für $\alpha$ und $C_V$.}
  \label{tab:tab2}
    \begin{tabular}{S S S S S}
    \toprule
    $ T\: \text{in}\: \si{\K} $ & $ {\alpha \cdot 10^{-6} \: \text{in}\: \si {\per\K}} $ &
    $ C_V \: \text{in}\: \si{\J\per\K\mol} $\\
    \midrule %Cv, a *10-6, Cv
    %0 & 1 & 1\\
    88.60\pm0.24 & 9.56\pm0.06 & 14.17\pm8.13  \\ %&3.6 & 318.97\pm0.85\\
    93.81\pm0.24 & 10.10\pm0.06 & 17.58\pm10.03 \\ %& 4.7 & 440.90\pm1.11\\
    99.74\pm0.24 & 10.66\pm0.05 & 15.52\pm8.84 \\ %& 5.1 & 508.68\pm1.21\\
    104.74\pm0.24 & 11.07\pm0.05 & 18.44\pm10.52 \\ %& 4.6 & 481.79\pm1.09\\
    110.94\pm0.24 &  11.54\pm0.05 & 14.86\pm8.45 \\ %& 5.3 & 587.97\pm1.27\\
    115.96\pm0.24 & 11.89\pm0.05 & 18.49\pm10.52 \\ %& 4.6 & 533.41\pm1.10\\
    121.47\pm0.24 &  12.22\pm0.05 & 16.83\pm9.57 \\ %& 4.9 & 595.21\pm1.17\\
    126.99\pm0.24 & 12.53\pm0.04 & 16.79\pm9.54 \\ %& 4.9 & 622.29\pm1.18\\
    131.58\pm0.24 & 12.77\pm0.04 & 20.42\pm11.62 \\ %& 4.2 & 552.62\pm1.01\\
    136.65\pm0.24 & 13.02\pm0.04 & 18.40\pm10.47 \\ %& 4.6 & 628.57\pm1.11\\
    141.49\pm0.24 & 13.24\pm0.04 & 19.28\pm10.97 \\ %& 4.4 & 622.54\pm1.07\\
    146.34\pm0.24 & 13.44\pm0.04 & 19.24\pm10.95 \\ %& 4.4 & 643.88\pm1.07\\
    150.95\pm0.24 & 13.62\pm0.04 & 20.22\pm11.52 \\ %& 4.3 & 649.11\pm1.05\\
    155.34\pm0.24 & 13.79\pm0.04 & 21.31\pm12.14 \\ %& 4.1 & 636.88\pm0.98\\
    159.97\pm0.24 & 13.95\pm0.04 & 20.12\pm11.47 \\ %& 4.3 & 687.89\pm1.05\\
    164.62\pm0.24 & 14.10\pm0.04 & 20.18\pm11.51 \\ %& 4.3 & 707.87\pm1.06\\
    168.79\pm0.25 & 14.23\pm0.04 & 22.54\pm12.86 \\ %& 3.9 & 658.27\pm0.95\\
    173.45\pm0.25 &  14.37\pm0.04 & 20.08\pm11.46 \\ %& 4.3 & 745.84\pm1.06\\
    178.13\pm0.25 &  14.50\pm0.04 & 20.04\pm11.44 \\ %& 4.3 & 765.94\pm1.06\\
    182.56\pm0.25 &  14.62\pm0.04 & 21.11\pm12.06\\
    192.70\pm0.25 &  14.87\pm0.04 & 18.41\pm10.47\\
    200.15\pm0.25 &  15.04\pm0.04 & 25.19\pm14.28\\
    208.87\pm0.25 &  15.23\pm0.04 & 21.43\pm12.18\\
    217.12\pm0.25 &  15.38\pm0.04 & 22.65\pm12.88\\
    225.15\pm0.25 &  15.53\pm0.03 & 23.27\pm13.24\\
    232.70\pm0.25 &  15.70\pm0.03 & 24.75\pm14.08\\
    240.53\pm0.25 &  15.74\pm0.03 & 23.84\pm13.58\\
    248.39\pm0.25 &  15.89\pm0.03 & 23.74\pm13.53& \\
    256.01\pm0.25 &  15.97\pm0.03 & 24.46\pm13.94 \\
    263.41\pm0.26 &  16.01\pm0.03 & 25.22\pm14.38 \\
    271.08\pm0.26 &  16.18\pm0.03 & 24.26\pm13.86 \\
    278.52\pm0.26 &  16.27\pm0.03 & 25.03\pm14.29&\\
    285.98\pm0.26 &  16.35\pm0.03 & 24.92\pm14.25 \\
    293.21\pm0.26 &  16.42\pm0.03 & 25.74\pm14.72 \\
    300.98\pm0.26 &  16.50\pm0.03 & 23.87\pm13.68 \\
    308.51\pm0.26 &  16.57\pm0.03 & 24.63\pm14.12\\



      \bottomrule
    \end{tabular}
\end{table}


Um den Mittelwert des Brechungsindizes berechnet sich über
\begin{equation}
  \bar{n}=\frac{1}{a}\sum_{i=1}^{a}n_i
  \label{eqn:Mittel}
\end{equation}
und sein Fehler mit
\begin{equation}
  \sigma_{\bar{n}}=\sqrt{\frac{1}{a(a-1)}\sum_{i=1}^{a}(\bar{n}-n_i)^2}.
  \label{eqn:Fehler}
\end{equation}
Daraus ergibt sich ein Wert von
\begin{equation}
  n_\text{Glas}=\SI{1.743(9)}{}.
\end{equation}
%Verglichen mit dem Theoriewert von $n_\text{Theorie}=\SI{1,613}{}$ ergibt sich ein Fehler von 21,2\%.

\subsection{Brechungsindex von Luft}

Um den Brechungsindex bei Atmosphärendruck über die Anzahl der Maxima zu bestimmen wird Formel \ref{eqn:pM2}
verwendet. Dabei kann $\Delta n$ durch  $\Delta n= n-1$ ersetzt werden, da das Vakuum den
Brechungsindex 1 besitzt. Eingesetzt in $M=\sfrac{\Delta\Phi}{2\pi}$ ergibt sich somit
\begin{equation}
  n=\frac{M\lambda}{L}+1.
  \label{eqn:nmax}
\end{equation}
Aus den Messwerten, welche in Tabelle \ref{tab:tab3} dargestellt sind, ergeben sich daraus folgende Werte für
den Brechungsindex von Luft bei Atmosphärendruck:

\begin{align*}
  n_\text{Luft1}&=\SI{1,000259526(26)}{}\\
  n_\text{Luft2}&=\SI{1,000253196(25)}{}\\
  n_\text{Luft3}&=\SI{1,000253196(25)}{}
\end{align*}

Nach Formel \ref{eqn:Mittel} ergibt sich daraus der Mittelwert zu:
\begin{equation*}
  n_\text{Luft}=\SI{1,000255306(26)}{}.
\end{equation*}

\begin{table}
  \centering
  \caption{Messwerte für den ersten Doppelspalt.}
   \begin{tabular}{S S| S S | S S}
    \toprule
    $x/\; \si{\mm}$& $A/\;\si{\nA}$ &
    $x/\; \si{\mm}$& $A/\;\si{\nA}$ &
    $x/\; \si{\mm}$& $A/\;\si{\nA}$ \\
    \midrule

    15.0& 4.6& 23.0& 25.0& 29.5& 6.0\\
    15.5& 4.2& 23.5& 30.0& 30.0& 5.3\\
    16.0& 4.0& 24.0& 35.0& 30.5& 4.9\\
    16.5& 4.0& 24.25& 36.0& 31.0& 4.7\\
    17.0& 4.4& 24.5& 37.0& 31.5& 4.4\\
    17.5& 5.5& 24.75& 38.0& 32.0& 4.2\\
    18.0& 6.6& 25.00& 37.0& 32.5& 3.8\\
    18.5& 7.7& 25.25& 36.0& 33.0& 3.6\\
    19.0& 8.2& 25.5& 36.0& 33.5& 3.2\\
    19.5& 8.4& 26.0& 33.0& 34.0& 3.2\\
    20.0& 8.4& 26.5& 28.5& 34.5& 3.2\\
    20.25& 8.4& 27.0& 23.0& 35.0& 3.3\\
    20.5& 8,7& 27.5& 18.0& 35.5& 3.4\\
    21.0& 9.8& 28.0& 13.5& 36.0& 3.5\\
    21.5& 12.0& 28.5& 10.0\\
    22.0& 15.0& 29.0& 7.8\\
    22.5& 20.0& 29.25& 6.7\\


   \bottomrule
  \end{tabular}
  \label{tab:tabelle3}
\end{table}


Um das Lorentz-Lorentz-Gesetz zu verifizieren, wird zunächst über die Anzahl der gemessenen Maxima für jeden
Druck gemittelt. Nach Formel \ref{eqn:nmax} kann nun der Brechungsindex in Abhängigkeit des Drucks bestimmt
und in Diagramm \ref{fig:Lorentz} gegeneinander aufgetragen werden.
Die Regression erfolgt nach dem Lorentz-Lorentz-Gesetz aus Gleichung \ref{eqn:lorentz} mit
\begin{equation}
  n\sim\sqrt{1+\frac{Bp}{RT}}
\end{equation}
und liefert für den Fitparameter $B$ den Wert $B=\SI{0,001318(2)}{}$. Die allgemeine Gaskonstante wird dabei mit $R$
bezeichnet und $T$ bezeichnet die Temperatur, welche zum Messzeitpunkt $T=\SI{21,2}{\celsius}$ beträgt.
Somit ergibt sich eine druckabhängige Formel für den Brechungsindex von
\begin{equation}
  n\sim\sqrt{1+\frac{\SI{0,001318(2)}\cdot p}{RT}}
\end{equation}
mit $p$ in Millibar.

Mithilfe dieser Formal kann nun der Brechungsindex bei Normatmophäre bestimmt werden, indem
die definierte Temperatur von $T=\SI{15}{\celsius}$ und der definierte Druck von $p=\SI{1013,25}{\milli\bar}$
eingesetzt werden. Es ergibt sich so ein Brechungsindex von
\begin{equation}
  n_\text{Normatmophäre}=\SI{1,0002788(5)}{}.
\end{equation}

\begin{figure}[H]
  \centering
  \includegraphics[width=13cm]{Glasplot.pdf}
  \caption{Brechungsindizes in abhängigkeit des Drucks mit einer Regression nach dem Lorentz-Lorentz-Gesetz.}
  \label{fig:Lorentz}
\end{figure}
%Der Brechungsindex kann außerdem über das Lorentz-Lorenz Gesetz aus Formel \ref{eqn:lorentz} bestimmt
%werden.

% Mit der molaren Reflexivität $A=\frac{4\pi}{3}N_\alpha \cdot \alpha$ folgt daraus
%\begin{equation}
%  n\sim\sqrt{1+\frac{4\pi p}{RT}N_\alpha \alpha}.
%\end{equation}

%Dabei bezeichnet $R$ die Gaskonstante, $N_\alpha$ die Avogadro-Konstante
%und\\
%$\alpha=\SI{2.118(91)e-29}{\m\cubic}$ \cite{alpha} die molare Polarisierbarkeit von Luft.

%Die auf diesem Wege berechneten Werte für den Brechungsindex von Luft sind in Tabelle \ref{tab:tab3}
%dargestellt. Für den Mittelwert ergibt sich daraus $n_\text{Luft}=\SI{1.0000164(7)}{}$.
