\section{Diskussion}
\label{sec:Diskussion}
Die Kontrastmessung ergab einen maximalen Kontrast von $K=0,84$ für einen Polarisationswinkel von $\Phi=\SI{50}{\degree}$.
Im Vergleich zu dem theoretischen Wert von $K=1$, welcher in der Praxis jedoch nicht erreichbar ist, ergibt sich eine
Abweichung von 16\%.
\\
\\
Der Brechungsindex von Glas konnte mithilfe des Doppelglashalters auf\\
$n_\text{Glas}=\SI{1.743(9)}{}$ bestimmt werden.
Der Vergleich mit dem Theoriewert von \\
$n_\text{Theorie}=\SI{1,613}{}$ \cite{ns}
liefert eine Abweichung von 21,2\%, wobei für die Berechnung nur die signifikanten Stellen verwendet wurden.

Des Weiteren wurde der Brechungsindex von Luft bestimmt. Unter Verwendung der Anzahl der Maxima ergibt
sich bei Atmosphärendruck ein Brechungsindex von\\
$n_\text{Luft}=\SI{1,000255306(26)}{}$.
Hier zeigt der Vergleich mit dem Theoriewert für Luft von $n_\text{Luft,Theorie}=\SI{1,000272}{}$ \cite{ns}
eine Abweichung von 6,14\%, wobei hier ebenfalls nur die signifikanten Nachkommastellen
berücksichtigt wurden.
Mithilfe des Lorentz-Lorentz-Gesetzes und einer Regression konnte die druckabhängige Formel
\begin{equation}
    n\sim\sqrt{1+\frac{\SI{0,001318(2)}\cdot p}{RT}}
\end{equation}
für den Brechungsindex aufgestellt werden.
Mit dieser Formel ergibt sich für den Brechungsindex bei Normatmophäre ein Wert von $n_\text{Normatmophäre}=\SI{1,0002788(5)}{}$.
Die Abweichung zum Theoriewert von $n_\text{Normat.,Theorie}=\SI{1,00028}{}$ \cite{normath} berträgt 0,43\%.

Zusammenfassend konnten die gesuchten Brechungsindizes mithilfe des Interferometers sehr genau bestimmt werden, obwohl
einige Fehlerquellen wie Helligkeitsschwankungen oder Luftbewegungen und damit auch Dichteschwankungen der Luft auftraten.
