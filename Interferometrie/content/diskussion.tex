\section{Diskussion}
\label{sec:Diskussion}
Die Kontrastmessung ergab einen maximalen Kontrast von $K=0,84$ für einen Polarisationswinkel von $\Phi=\SI{50}{\degree}$.
Im Vergleich zu dem theoretischen Wert von $K=1$, welcher in der Praxis jedoch nicht erreichbar ist, ergibt sich eine
Abweichung von 16\%.

Der Brechungsindex von Glas konnte mithilfe des Doppelglashalters auf $n_\text{Glas}=\SI{1.641(7)}{}$ bestimmt werden.
Der Vergleich mit dem Theoriewert von $n_\text{Theorie}=\SI{1,613}{}$ \cite{ns} liefert eine Abweichung von 1,74\%.

Des Weiteren wurde der Brechungsindex von Luft auf zwei Wegen bestimmt. Unter Verwendung der Anzahl der Maxima ergibt
sich ein Brechungsindex von\\
$n_\text{Luft}=\SI{1,000255306(26)}{}$, während sich unter Verwendung des Drucks nach dem
Lorentz-Lorenz Gesetz ein Brechungsindex von $n_\text{Luft}=\SI{1.0000164(7)}{}$ ergibt.
Der Vergleich mit dem Theoriewert für Luft von $n_\text{Luft,Theorie}=\SI{1,000272}{}$ \cite{ns} liefert einen Fehler von
$\SI{1,67e-3}{}$\% für die Bestimmung über die Maxima und 0,025\% für die Bestimmung über den Druck.

Zusammenfassend konnten die gesuchten Brechungsindizes mithilfe des Interferometers sehr genau bestimmt werden, obwohl
einige Fehlerquellen wie Helligkeitsschwankungen oder Luftbewegungen und damit auch Dichteschwankungen der Luft auftraten.
