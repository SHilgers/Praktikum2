\begin{table}[H]
  \centering
  \caption{Messwerte zur Bestimmung des Brechungsindex von Luft.}
  \label{tab:tab3}
    \begin{tabular}{S S S S S}
    \toprule
    $p/\si{\milli\bar} $ & Maxima\;$M_1$  & Maxima\; $M_2$ & Maxima\; $M_3$ & $n_\text{Luft}$\\
    \midrule
    50 & 2 & 3 & 2 & 1.000001637\:(71)\\
    100 & 4 & 5 & 4 & 1.00000327\:(14)\\
    150 & 6 & 7 & 7 & 1.00000491\:(21)\\
    200 & 8 & 9 & 9 & 1.00000654\:(28)\\
    250 & 11 &  11 & 11 & 1.00000818\:(36)\\
    300 & 13 &  13 & 13 & 1.00000982\:(42)\\
    350 & 15 &  15 & 15 & 1.00001146\:(50)\\
    400 & 17 &  17 & 17 & 1.00001309\:(56)\\
    450 & 19 &  19 & 19 & 1.00001473\:(63)\\
    500 & 21 &  21 & 22 & 1.00001637\:(70)\\
    550 & 23 &  23 & 24 & 1.00001801\:(77)\\
    600 & 26 &  25 & 26 & 1.00001964\:(84)\\
    650 & 28 &  28 & 28 & 1.00002128\:(91)\\
    700 & 30 &  30 & 30 & 1.00002292\:(98)\\
    750 & 32 &  32 & 32 & 1.0000246\:(11)\\
    800 & 34 &  34 & 34 & 1.0000262\:(11)\\
    850 & 36 &  36 & 36 & 1.0000278\:(12)\\
    900 & 38 &  38 & 38 & 1.0000294\:(13)\\
    950 & 41 &  40 & 40 & 1.0000311\:(13)\\

    \bottomrule
    \end{tabular}
\end{table}
