\begin{table}[H]
  \centering
  \caption{Messwerte zur Bestimmung des Brechungsindex von Luft.}
  \label{tab:tab3}
    \begin{tabular}{S S S S S }
    \toprule
    $p/\si{\milli\bar} $ & Maxima\;$M_1$  & Maxima\;$M_2$ & Maxima\; $M_3$ &
    $n_\text{Luft}$\\
    \midrule
    50 & 2 & 3 & 2  & 1.0000147486\:(15)\\
    100 & 4 & 5 & 4 & 1.0000274084\:(27)\\
    150 & 6 & 7 & 7 & 1.0000422204\:(42)\\
    200 & 8 & 9 & 9 & 1.0000548802\:(55)\\
    250 & 11 &  11 & 11 &  1.0000696289\:(69)\\
    300 & 13 &  13 & 13 &  1.0000822887\:(82)\\
    350 & 15 &  15 & 15 &  1.0000949485\:(95)\\
    400 & 17 &  17 & 17 &  1.000107608\:(11)\\
    450 & 19 &  19 & 19 &  1.000120268\:(12)\\
    500 & 21 &  21 & 22 &  1.000135016\:(14)\\
    550 & 23 &  23 & 24 &  1.000147676\:(15)\\
    600 & 26 &  25 & 26 &  1.000160336\:(16)\\
    650 & 28 &  28 & 28 &  1.000177237\:(18)\\
    700 & 30 &  30 & 30 &  1.000189897\:(19)\\
    750 & 32 &  32 & 32 &  1.00020255\:(2)\\
    800 & 34 &  34 & 34 &  1.000215216\:(22)\\
    850 & 36 &  36 & 36 &  1.000227876\:(23)\\
    900 & 38 &  38 & 38 &  1.000240536\:(24)\\
    950 & 41 &  40 & 40 &  1.000255284\:(25)\\

    \bottomrule
    \end{tabular}
\end{table}
